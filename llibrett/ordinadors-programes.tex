\chapter{Ordinadors i programes}

\com{Llista de coses per fer en aquest capítol:
  \begin{itemize}
  \item Explicar que els mòdems interns actuals (\emph{winmodems}) i
    algunes impressores són \emph{mitjos mòdems} i \emph{mitges
      impressores} perquè no contenen tot el maquinari i programari
    necessari per a realitzar les tasques corresponents sinó que es
    recolzen en el sistema operatiu (i per això no funcionen, en
    general amb sistemes diferents de Windows).
  \item Mirar on explicar què es un DVD, quants megaoctets hi caben,
  etc. 
  \end{itemize}
}
\label{se:OiP}

Tots els sistemes informàtics\footnote{És a dir, totes les
  instal·lacions basades en ordinadors} es poden dividir en dues
parts: \emph{maquinari} i \emph{programari}.

\begin{description}
\item[Maquinari] (o \emph{hardware}): l'equipament físic que es pot
  veure i tocar. Per exemple, la pantalla, el processador central, el
  teclat, el ratolí, els xips\footnote{El xip és l'element bàsic de la
    microelectrònica i de la microinformàtica; es tracta d'un o més
    circuits integrats en una placa de silici de dimensions molt
    reduïdes, que normalment es col·loca en una capsa hermètica amb
    contactes metàl·lics.}  de memòria i les impressores.

\item[Programari] (o \emph{software}): un o més \emph{programes} (i
  les dades associades) que fan alguna funció útil per a la persona
  usuària o per a un altre \emph{programa}.  Per exemple, un
  processador de textos com LibreOffice o Microsoft Word pot estar
  compost per més d'un \emph{programa}. Un \emph{programa} és una
  seqüència (llista o conjunt ordenat) d'instruccions que són seguides
  o executades pel maquinari, de tal manera que realitzen alguna tasca
  determinada.\footnote{L'ús de la paraula \emph{programa} en
    informàtica (seqüència d'operacions o esdeveniments) és paral·lel
    a molts usos d'aquest mot en la vida quotidiana: programa \emph{de
      festes}, \emph{d'un concert}, \emph{de la llavadora}, etc.;
    encara que per a la persona usuària un programa d'ordinador és més
    similar a una espècie de caixa d'eines per a fer una tasca
    determinada, com, per exemple, editar un document de text.}
  Normalment, els ordinadors estan organitzats al voltant d'un
  \emph{processador central} (vegeu més endavant) que és capaç de
  comprendre i executar instruccions bàsiques preses d'un conjunt
  determinat (el \emph{conjunt d'instruccions} del processador).  Els
  programes poden estar %escrits a mà en un paper,
  guardats en un disc o carregats en la memòria de l'ordinador mentre
  són executats pel processador.
\end{description}

A continuació es consideren el maquinari i el programari amb més
detall.

\section{Maquinari}

Tots els sistemes informàtics tenen maquinari de les classes següents:

\begin{description}
\item[Processament:] Els dispositius de processament són els que fan
  realment el treball. La majoria dels sistemes contenen una CPU ({\em
    central processing unit}, unitat central [de processament]), o
  senzillament, un \emph{processador} que és el responsable d'executar
  totes les instruccions de programa, de processar dades, i de
  controlar el funcionament d'altres components del maquinari. En els
  ordinadors personals, la unitat central és un únic xip de silici. A
  més, la majoria dels sistemes actuals contenen també una GPU
  (\emph{graphic processing unit}, unitat de processament de gràfics),
  una CPU especialitzada en el tractament d'imatges però que també es
  pot usar per a altres tasques computacionalment molt intensives.

  La velocitat a la que una CPU executa les instruccions bàsiques d'un
  programa es mesura en megahertzs (MHz) o gigahertzs (GHz; un
  gigahertzs són 1000 megahertzs). Un megahertzs equival a un milió de
  hertzs (Hz), és a dir, un milió de cicles de processament
  d'informació per segon. Cada cicle de processament d'informació es
  correspon amb un \emph{tic} del rellotge que tots el dispositius de
  processament tenen per sincronitzar tots el circuits de
  l'ordinador. Normalment una instrucció requereix d'uns pocs cicles
  de processament per ser executada, tot i que alguns sistemes són
  capaços de processar més d'una instrucció al mateix temps. La
  velocitat típica de la CPU d'un ordinador en l'actualitat és de 3
  Ghz.
  
\item[Emmagatzematge:] els dispositius d'emmagatzematge es poden
  dividir en dos grups:
  \begin{description}
  \item[Memòria primària:] memòria ràpida de curt termini, volàtil
    (s'esborra quan s'apaga l'ordinador), que serveix per a guardar-hi
    programes i dades mentre l'ordinador està funcionant; si els
    programes i les dades no caben en la memòria primària, el sistema
    operatiu ---vegeu l'apartat~\ref{ss:programari}--- s'encarrega de
    copiar-los de la memòria al disc dur quan no s'estan usant i
    copiar-los de tornada del disc dur a la memòria quan són
    necessaris, operació que s'anomena \emph{intercanvi}.\footnote{En
      anglés \emph{swapping}. Com que el disc dur és més lent que la
      memòria primària, l'intercanvi fa que l'ordinador vaja més lent;
      per això, ampliar la memòria primària sol fer que l'ordinador
      vaja més ràpid.} La memòria primària normalment consisteix en
    xips RAM (\emph{random-access memory}, memòria d'accés
    aleatori\footnote{Noteu la diferència entre \emph{accés aleatori}
      (a voluntat) i \emph{accés seqüencial}. Un CD-ROM de música és
      d'accés aleatori perquè podem accedir a la setzena cançó
      directament; en canvi, un casset (cinta magnètica) és d'accés
      seqüencial perquè per a accedir a la setzena cançó hem de passar
      per les 15 cançons anteriors.}) de silici.
      
  \item[Memòria secundària:] memòria de llarg termini, permanent.
    Exemples: els antics disquets, discos fixos o durs interns i
    externs, memòries USB (també anomenades llapis o \emph{pendrive})
    i diverses formes de ROM (\emph{read-only memory}, memòria de
    lectura només), com els xips ROM, els CD-ROM o els DVD.  

    Els discos fixos (i els antics disquets) són dispositius
    d'emmagatzematge magnètic, poc més o menys com ho eren les
    antigues cassets. La informació s'emmagatzema fent servir les
    propietats magnètiques de determinats materials magnetitzables. En
    la actualitat la grandària típica d'un dic fix és de 500~GB o 1~TB
    (vegeu l'apartat \ref{ss:memoria} per a assabentar-vos de les
    mesures d'emmagatzemament de la informació).

    La memòria USB és un dispositiu de memòria flaix, un xip de
    memòria que manté el seu contingut en absència d'alimentació, que
    es connecta al port USB de l'ordinador. La grandària d'aquestes
    memòries pot arribar fins a 1~TB, tot i que les grandàries més
    típiques són 16, 32 i 64 MB.

    La memòria ROM sol estar feta de xips de silici. Els CD-ROM
    (\emph{compact dics read-only memory}) ---idèntics en aparença i
    similars en molts aspectes als CD de música--- emmagatzemen la
    informació òpticament.\footnote{Altres termes habituals són CD-R
      (\emph{compact disc recordable}) ---que identifica els CD en què
      es pot escriure informació només una vegada amb l'ajuda de
      dispositius coneguts com a enregistradores--- i CD-RW
      (\emph{compact disc rewritable}) ---utilitzat per als CD que
      poden ser esborrats i reescrits un nombre il·limitat de
      vegades.} La grandària d'un CR-ROM sol ser de 650 MB o de 700
    MB.

    El DVD (\emph{digital versatil dics})\footnote{Com en el cas dels
      CD, podem parlar de DVD-ROM, DVD-R i DVD-RW.} és un tipus més
    avançat de sistemes d'emmagatzematge basat en discs òptics;
    bàsicament, es tracta d'un CD més ràpid i amb més capacitat, que
    ha desplaçat quasi completament els CD-ROM. La grandària d'un DVD
    depén del tipus de DVD i sols estar entre 4,7~GB i 17~GB.
    
    La manera més comuna d'emmagatzemar les dades en memòria
    secundària és organitzar-les en {\em fitxers} o \emph{documents}
    organitzats en \emph{directoris} o \emph{carpetes}; la
    secció~\ref{se:fitxers} explica aquests conceptes amb
    detall.\label{pg:menciofitxer}
  \end{description}

\item[Entrada:] la funció primària dels dispositius d'entrada és que
  l'usuari puga interactuar amb la màquina i amb els programes que
  executa amb la finalitat d'\emph{introduir-hi} dades o
  informació. Els dispositius d'entrada més comuns són el teclat, el
  ratolí, la pantalla tàctil, la maneta de jocs, les càmeres de fotos
  i \emph{webcams} o l'escànner ---un dispositiu que llegeix una
  imatge impresa i la converteix en un fitxer \todo{potser cal fer
    referència a la part corresponent} que conté la imatge
  digitalitzada.\footnote{Quan la imatge és la d'un text imprés, un
    programa de \emph{reconeixement òptic de caràcters} (OCR,
    \emph{optical character recognition}) la pot convertir en una
    representació del text adequada per a ser manipulada amb un
    processador de textos (vegeu la secció~\ref{ss:proctext}),
    generalment amb alguns errors tipogràfics menors.}

\item[Eixida:] Aquesta és la família dels dispositius que l'ordinador
  usa per a comunicar dades o informació a l'usuari.  El monitor (la
  pantalla) n'és el més comú. Altres dispositius d'eixida són les
  impressores, els altaveus, els vibradors dels dispositius mòbils,
  etc.

\end{description}
En la figura~\ref{fg:ordinador} es resumeix esquemàticament el
maquinari d'un ordinador.
\begin{figure}
  \centering
  \includegraphics[scale=1.0]{ordinador4}
  \caption{Esquema del maquinari d'un ordinador.}
  \label{fg:ordinador}
\end{figure}


\section{Programari}
\label{ss:programari}

Hi ha tres classes bàsiques de programari:

\begin{description}
\item[Sistemes operatius i \emph{firmware}:] són els programes que
  permeten el funcionament bàsic de l'ordinador. S'anomena
  \emph{firmware} el programari del sistema que s'usa tan freqüentment
  que s'emmagatzema permanentment en xips ROM. Aquest programari
  oferix al sistema operatiu serveix bàsics d'accés als dispositius
  d'entrada i d'eixida més habituals.

  Quan connectem l'ordinador, el primer programa a executar-se és el
  \emph{firmware}, el qual s'encarrega de fer algunes comprovacions,
  com ara que hi ha un teclat enganxat a l'ordinador o que la memòria
  RAM no te defectes, i de carregar el sistema operatiu.

  El sistema operatiu, d'una banda, permet que la persona usuària hi
  execute programes i gestione els fitxers de dades, etc.; per a això,
  ofereix una \emph{interfície d'ús} (vegeu més endavant). D'altra
  banda, el sistema operatiu ofereix serveis bàsics (vegeu més avall)
  als programes d'aplicació que s'executen en l'ordinador (els quals
  poden tenir la seua pròpia interfície d'ús).

  Quant a la \emph{interfície d'ús}, és a dir, l'aparença i la forma
  d'interaccionar amb l'usuari, la majoria dels sistemes operatius són
  \emph{gràfics}, és a dir, basats en ratolí o pantalla tàctil,
  punters, finestres, etc. (GNU/Linux, Windows, MacOS, iOS, Android);
  antigament, els sistemes operatius eren de {\em línia d'ordres}, és
  a dir, basats en text (Unix primigeni, MS-DOS).

  Els sistemes més antics eren de vegades \emph{monousuari} (MS-DOS,
  Windows 3.11), és a dir, només podien donar suport a una persona
  usuària, o \emph{monotasca}, és a dir, no podien executar més d'un
  programa al mateix temps. La major part dels actuals sistemes
  operatius són \emph{multiusuari}, és a dir, poden donar accés i
  suport a més d'una persona usuària alhora, i \emph{multitasca}
  (GNU/Linux, versions recents de Windows, MacOS). La major part dels
  sistemes operatius actuals estan a més preparats per a interaccionar
  amb altres dispositius través de diferents tipus de
  \emph{xarxes}.\footnote{L'organització dels ordinadors en una xarxa
    local permet la comunicació d'informació entre ells i la
    compartició de recursos, com ara una impressora. Internet (vegeu
    el capítol~\ref{se:Internet}) no és més que una gran xarxa global
    que interconnecta moltes xarxes més locals.}

  Algunes de les operacions bàsiques que fan els sistemes operatius
  són:
  \begin{itemize}
  \item Controlar el maquinari de l'ordinador on s'executen.
  \item Copiar, moure i esborrar fitxers de dades.
  \item Crear, moure i esborrar directoris de fitxers.
  \item Establir connexions entre ordinadors.
  \item Executar programes i controlar-ne l'execució.
  \item Establir connexions amb altres ordinadors o dispositius en
    xarxa.
\end{itemize}
De fet, els programes d'aplicació solen estar escrits per a ser
executats \emph{sobre un sistema operatiu}, és a dir, els programes
d'aplicació \emph{assumeixen} que el sistema operatiu farà totes
aquestes operacions senzilles i no contenen instruccions de programa
per a fer-les, sinó només instruccions per a invocar els programes
corresponents del sistema operatiu, cosa que simplifica enormement
l'escriptura dels programes per part dels programadors.  Per això,
quan s'especifiquen les característiques d'un programa d'ordinador
s'ha de dir per a quin sistema operatiu està escrit, ja que cada
sistema operatiu ofereix serveis diferents i interacciona de manera
diferent amb els programes d'aplicació.%\footnote{En cert sentit, el
%  sistema operatiu és com el \emph{factor comú} de tots els programes
%  d'aplicació.}

\item[Programes d'aplicació:] programari dissenyat específicament per
  a satisfer les necessitats dels usuaris (de vegades s'anomenen
  simplement {\em aplicacions}). Se'n podrien fer dos grups:
        \begin{description}
        \item[Programari d'ús específic:] programari dissenyat per a
          un usuari molt concret amb unes necessitats molt concretes:
          per exemple, el programa que gestiona els préstecs, les
          quotes i les adquisicions d'un videoclub, fet a mida per a
          ell.
          \begin{description}
          \item[Programari específic per a professionals de la
            traducció:] sistemes de traducció automàtica
            (capítols~\ref{se:TiTA} a \ref{se:ASTA}), sistemes de
            traducció assistida basats en memòries de traducció
            (capítol~\ref{se:memtrad}) i bases de dades
            terminològiques (capítol~\ref{se:basesdades}).
          \end{description}
        \item[Programari d'ús general:] programari dissenyat per a fer
          tasques més genèriques, interessants per a moltes classes
          d'usuaris. Ací en teniu alguns exemples:
                \begin{description}
                \item[Editors i processadors de text] per a preparar,
                  modificar, emmagatzemar i imprimir documents de text
                  (vegeu la secció~\ref{ss:proctext}).
                  % \item[Programes d'autopublicació o autoedició,]
                  %   que
                  %   integren textos, imatges, etc. fins a produir un
                  %   document imprés amb característiques de
                  %   publicació. Quan el disseny de la publicació no
                  %   és
                  %   massa complex, és possible usar senzillament un
                  %   processador de textos.
                \item[Fulls de càlcul,] que permeten automatitzar
                  càlculs que es repeteixen sobre un conjunt més o
                  menys gran de dades (per exemple, per a calcular la
                  nota mitjana de cada estudiant d'una classe sencera
                  a partir de les notes parcials), i presentar-ne els
                  resultats de diverses maneres, per exemple, en
                  gràfics de molts tipus.
                \item[Gestors de bases de dades] \label{pg:BD} que
                  serveixen per a emmagatzemar, organitzar i gestionar
                  de diverses maneres la informació continguda en
                  \emph{bases} o bancs de dades (vegeu el
                  capítol~\ref{se:basesdades}).
                % \item[Programes de comunicacions,] que permeten
                % connectar el nostre ordinador a altres ordinadors,
                % transferir
                % fitxers, etc. Actualment, en molts casos, els programes
                % de comunicació formen part dels sistemes operatius
                % i l'usuari no s'adona que estiguen actius.
                \item[Navegadors d'Internet:] programes que permeten
                  accedir de manera senzilla als documents d'Internet
                  en màquines connectades a aquesta xarxa.\footnote{El
                    nom \emph{navegador} s'usa per l'analogia
                    ---dèbil--- existent entre els mecanismes d'accés
                    als documents de la Internet i la navegació
                    mitjançant un mapa en una zona desconeguda.}
                  Vegeu la secció \ref{ss:navegadors}.
                % \item[Programes de gràfics,] que ens ajuden a presentar
                % de manera més útil les dades de què disposem
                % (aquests programes apareixen moltes
                % voltes associats a fulls de càlcul).
                \item[Jocs] de moltes classes.
                \end{description}
        \end{description}
        Els programes d'aplicació els activa l'usuari per mitjà del
        sistema operatiu, utilitzen el sistema operatiu per a accedir
        als recursos (maquinari i altres programes) del sistema i
        interaccionen amb l'usuari mitjançant el dispositius d'entrada
        i d'eixida (vegeu la figura~\ref{fg:aplicacio-so}).
\end{description}

\begin{figure}
  \centering
  \includegraphics[scale=1.0]{aplicacio-so}
  \caption{Esquema de la interacció entre la persona usuària, el
    sistema operatiu i els programes d'aplicació.}
  \label{fg:aplicacio-so}
\end{figure}

\begin{persabermes}{programari}
Com ja s'ha dit més amunt, un programari és un conjunt de programes,
cada un dels quals consisteix en una llista d'instruccions vàlides
(executables per l'ordinador) que s'executen en l'ordre indicat, de
la primera a l'última, excepte quan s'hi presenta alguna instrucció
de \emph{salt} que indica quina és la següent instrucció que s'ha
d'executar.

Per exemple, un programa que suma tots els nombres enters del 1 al 10
podria ser el següent, el qual usa dues posicions de memòria RAM per a
guardar valors necessaris per al càlcul. Cada una de les ordres es
correspon amb una instrucció bàsica de les que pot entendre qualsevol
processador.
\begin{enumerate}
\item Fes que l'acumulador (un registre de la memòria interna del
  processador) valga 1.
\item Guarda el valor de l'acumulador en una posició de memòria que
  anomenarem \emph{índex}.
\item Fes que l'acumulador valga 0.
\item Guarda el valor de l'acumulador en una posició de memòria que
  anomenarem \emph{suma}, la qual contindrà la suma total.
\item \label{en:z} Carrega el valor de \emph{suma} en l'acumulador.
\item Suma el valor d'\emph{índex} a l'acumulador.
\item Guarda el valor de l'acumulador en \emph{suma}.
\item Carrega el valor d'\emph{índex} en l'acumulador.
\item Compara el valor de l'acumulador amb 10.
\item Si és igual, salta a la instrucció~\ref{en:q}
\item Incrementa en 1 el valor de l'acumulador.
\item Guarda el valor de l'acumulador en \emph{índex}.
\item Salta a la instrucció~\ref{en:z}.
\item \label{en:q} Para.
\end{enumerate}

Moltes voltes s'usen noms curts (en anglés \emph{mnemonics}) per a les
instruccions del processador i també noms elegits pel programador per
a referir-se a posicions del programa (aquesta notació se sol anomenar
\emph{llenguatge assemblador}). El programa de dalt tindria l'aparença
següent:
\begin{verbatim}
         mov #1,A
         mov A,index
         mov #0,A
         mov A,suma
 altre:  mov suma,A
         add A,index
         mov A,suma
         mov index,A
         cmp A,#10
         jeq final
         inc A
         mov A,index
         jmp altre
 final:  hlt
\end{verbatim}

\paragraph{Processadors de llenguatges de programació:} les
instruccions que executa el processador central d'un ordinador són
massa senzilles perquè un programador humà en faça programes útils;
seria llarg i enutjós, com hem vist en l'exemple de programa que
sumava els enters de l'1 al 10.  Els programadors normalment escriuen
els seus programes en {\em llenguatges de programació} basats en
instruccions més potents (com ara BASIC, Java, C, C++, Pascal, Perl o
Python) i usen programes especials ---els processadors de
llenguatges--- per a traduir-los a les instruccions senzilles que
entén la màquina.\footnote{Hi ha dos famílies bàsiques de processadors
  de llenguatges: els \emph{compiladors}, que tradueixen tot el
  programa al llenguatge de la màquina abans d'executar-lo, i els
  \emph{intèrprets}, que lligen el programa línia a línia i executen
  petits programes ja escrits en el llenguatge de la màquina i que
  corresponen a les sentències del llenguatge de programació.} Quasi
tots els programes que s'executen en un ordinador han estat escrits en
algun llenguatge de programació. El programa que suma els nombres de
l'1 al 10 quedaria així en el llenguatge Pascal:
\begin{verbatim}
program SUMA;
var
   index, suma: integer;
begin
   suma:=0;
   for index:=1 to 10
      suma:=suma+index;
end.
\end{verbatim}

% \paragraph{Programes i algorismes:} Moltes voltes, un programa és la
% realització (entre informàtics se'n diu {\em implementació}) d'un
% \emph{algorisme}. Un algorisme (també \emph{algoritme}, per
% interferència amb \emph{aritmètica}) és una seqüència finita
% d'operacions executables i no ambigües\footnote{Aquestes operacions no
%   necessàriament han de ser comprensibles per a un ordinador.} que
% defineix un procediment que sempre es deté. El nom \emph{algorisme} ve
% del nom d'un matemàtic que treballava a Bagdad vora l'any 825 de la
% nostra era, Muhammad ibn Musa al Khwarizmi, anomenat així perquè
% sembla que era d'una ciutat al sud del mar d'Aral anomenada Khwarizm
% (ara Kheva). Aquest matemàtic va introduir el sistema decimal hindú en
% el món àrab. El seu llibre, \emph{Kitab al-jabr wa al-muqabalah}, o
% {\em Llibre de la integració i l'equació}, va donar nom també a
% l'\emph{àlgebra} quan es va traduir al llatí en el segle XII. El
% llibre és bàsicament una compilació de procediments algebraics i
% geomètrics, és a dir, d'\emph{algorismes}.

% Normalment els algorismes especifiquen un mètode general per a obtenir
% la resposta (correcta o incorrecta) a qualsevol cas particular d'un
% problema o pregunta, com ara ``quin és el quocient de la divisió de
% dos nombres enters''?  Els algorismes es poden convertir fàcilment en
% programes d'ordinador si es canvien les instruccions bàsiques de
% l'algorisme per instruccions en un llenguatge de programació que es
% puguen traduir a les instruccions senzilles que entén la màquina.

% Un exemple d'algorisme/programa és el següent, que diu si dues
% seqüències de caràcters (també anomenades \emph{cadenes} de caràcters)
% són iguals o no. Aquest algorisme està relacionat amb la recerca d'un
% mot en un diccionari, per exemple.  Els algorismes s'enuncien \emph{en
%   imperatiu}, com si fossen ordres que es donen a algú perquè les
% execute seqüencialment, excepte quan es trenca l'ordre amb una
% instrucció ``Vés a''.  \vspace{0.4cm}

% {\sf
% \noindent ALGORISME COMPARA\newline
% \noindent Entrada: Dues seqüències de caràcters $A$ i $B$.\newline
% Eixida:  \emph{Sí}, si són iguals; \emph{no} si no ho són.

% \begin{itemize}
% \item[1.] Fes que $p$ valga 1 ($p$ indica la posició actual dins
%   de les seqüències que estem comparant: $p=1$, per al primer
%   caràcter, $p=2$, per al segon, etc.)
% \item[2.] Si no existeix la posició $p$ de $A$, vés a 7.
% \item[3.] Si no existeix la posició $p$ de $B$, vés a 8.
% \item[4.] Si el caràcter en la posició  $p$ de $A$ no és
%           igual al caràcter en la posició $p$ de $B$, vés
%           a 8.
% \item[5.] Suma 1 a $p$.
% \item[6.] Vés al pas 2.
% \item[7.] Si no existeix la posició $p$ de $B$, digues \emph{sí} i
% para.
% \item[8.] Digues \emph{no} i para.
% \end{itemize} 
% } %
% Si es pot suposar que a les cadenes de caràcters s'afegeix un símbol
% especial de final de cadena (per exemple, `\$'), l'algorisme se
% simplifica: \vspace{0.4cm}

% {\sf
% \noindent ALGORISME COMPARA2\newline
% \noindent Entrada: Dues seqüències de caràcters $A$ i $B$,
%          acabades en `\$'.\newline
% Eixida:  \emph{Sí}, si són iguals; \emph{no} si no ho són.

% \begin{itemize}
% \item[1.] Fes que $p$ valga 1.
% \item[2.] Si el caràcter en la posició  $p$ de $A$ no és
%           igual al caràcter en la posició $p$ de $B$, digues
%           \emph{no} i para.
% \item[3.] Si el caràcter en la posició  $p$ de $A$ és `\$',
%           digues \emph{sí} i para.
% \item[4.] Suma 1 a p.
% \item[5.] Vés al pas 2.
% \end{itemize}
% } 
% Aquest algorisme es pot convertir a un programa d'ordinador.  Per
% exemple, en Pascal tindria una forma com aquesta:
% \begin{alltt}
% function compare(A, B : string);
%   var p : int ;
%   labels : 2 ;
%   begin
%     p:=1;
%  2: if A[p]<>B[p] then return "no";
%     if A[p]='\$' then return "yes"
%     p:=p+1;
%     goto 2;
%   end;
% \end{alltt}
% \noindent on \texttt{A[p]} representa el caràcter que hi ha en la
% posició \texttt{p} de la seqüència \texttt{A}.
\end{persabermes}


\section{Memòria}
\label{ss:memoria}

Tota la informació ---instruccions de programa o dades--- que
s'emmagatzema en la memòria d'un ordinador s'hi guarda en forma
binària; és a dir, cada dada és una cadena de dígits binaris o
\emph{bits}. Un bit pot tenir dos valors: 0 (apagat, inactiu) o 1
(encés, actiu); això és perquè el dispositiu electrònic corresponent
pot estar en dos estats. Si necessitem guardar objectes o unitats
d'informació que tenen més de dos valors possibles, no tindrem prou
amb 1~bit; haurem de combinar més d'un bit. Per exemple, si tenim una
unitat d'informació que pot presentar-se en 778 formes
diferents,\footnote{Com, per exemple, els signes d'algun sistema
  d'escriptura no alfabètic} necessitarem 10 bits, perquè amb 9 bits
només podem fer $$2 \times 2 \times 2 \times 2 \times 2 \times 2
\times 2 \times 2 \times 2 = 2^9 = 512$$ combinacions diferents, però
amb 10, ja en podem fer suficients, perquè $2^{10}=1~024$ (en
quedarien $1~024-778=246$ combinacions sense usar).

Els bits s'agrupen normalment en grups de vuit, anomenats {\em octets}
o \emph{bytes}. Un octet pot estar, per tant, en $2^8=256$ estats
diferents. Per exemple, els caràcters i símbols més comunament usats
en textos se guardaven històricament cada un en un octet, usant el
codi ASCII\label{pg:ASCII} (\emph{American Standard Code for
  Information Interchange}), on el codi de la ``{\tt A}'' és ``{\tt
  01000001}'' o el de la ``{\tt z}'' és ``{\tt 01111010}'' (vegeu
l'epígraf~\ref{ss:formats}).  El codi ASCII va ser el primer codi
estàndard per a emmagatzemar textos; quan els textos són més rics i
contenen informació sobre tipus i grandàries de lletra, diagramació,
notes a peu de pàgina, etc., s'usen formats més avançats que
s'expliquen en l'epígraf~\ref{ss:formats}.  Un octet pot contenir, per
tant, molt poca informació (un caràcter, una instrucció senzilla del
processador central, un nombre de 0 (``{\tt 00000000}'') a 255 (``{\tt
  11111111}''), etc.).  Per exemple, un document de text com aquest té
desenes de milers de caràcters, i una enciclopèdia, centenars de
milions. En les imatges en blanc i negre, cada punt és un bit; una
pantalla d'ordinador en conté més o menys un milió. Si són de colors,
cal més d'un bit per a cada punt. Les instruccions dels programes que
executa el processador central també s'emmagatzemen en
octets.\footnote{En l'exemple de la secció anterior, la instrucció
  {\tt inc A}, que incrementa el valor emmagatzemat l'acumulador en 1,
  podria ser l'octet {\tt 11010110}}

Com que un octet pot contenir poca informació, normalment es parla de:
\begin{itemize}
\item \emph{kilooctets} o \emph{kilobytes} (kB), o milers d'octets.
  De fet, per fidelitat al sistema binari, un kilooctet no té 1.000,
  sinó 1.024 octets ($2^{10}$ és $1.024$), és a dir
  1.024$\times$8=8.192 bits.
\item \emph{megaoctets} o \emph{megabytes} (MB), o milions d'octets.
  De fet, com en el cas dels kilooctets, no exactament:
$$1~\mbox{MB} = 1.024 \times 1.024~\mbox{octets}= 1.048.576~\mbox{octets}.$$
\item \emph{gigaoctets} o \emph{gigabytes} (GB), o milers de milions
  ---una mica més--- d'octets:
$$1~\mbox{GB}=1.024~\mbox{MB}=1.048.576~\mbox{kB}=1.073.741.824~\mbox{octets}.$$
\item \emph{teraoctets} o \emph{terabytes} (TB), o bilions (milions de
  milions) ---de nou, una mica més--- d'octets:
\[
\begin{array}{c}
$$1~\mbox{TB}=1.024~\mbox{GB}=1.048.576~\mbox{MB}=1.073.741.824~\mbox{kB}=\\
=1.099.511.627.776~\mbox{octets}.$$
\end{array}
\]
\end{itemize}
Com que els prefixos \emph{k}, \emph{M}, \emph{G} i \emph{T} s'usen en
la resta de les disciplines científiques per expressar a potències
exactes de 10 (de 1.000), hi ha qui prefereix parlar de
\emph{kibioctets} o \emph{kibibytes} (kiB), \emph{mibioctets} o
\emph{mibibytes} (MiB), \emph{gibioctets} o \emph{gibibytes} (GiB) i
\emph{tebibytes} o \emph{tebioctets} (TiB) per a referir-se a les
unitats de capacitat d'emmagatzematge basades en múltiples de 1.024.

%Per exemple, actualment (any 2015) és comú que un PC de taula
%tinga 8~GB de memòria RAM i un disc dur d'1~TB.

%Quant a l'emmagatzematge òptic, un CD-ROM conté uns 700~MB, tot i
%que també n'hi ha varietats que arriben als 900~MB. Els DVD actuals
%contenen uns 4,7~GB, però ja s'han desenvolupat tècniques que n'amplien
%considerablement la capacitat.

%Les memòries USB típiques (i també les targetes de memòria SD, micro
%SD, etc. que s'usen de memòria suplementària en mòbils, tauletes i
%càmeres) solen tenir 16, 32 o 64~GB.

\section{Fitxers i directoris}
\label{se:fitxers}
\label{pg:fitxer}

Com ja s'ha dit en la pàgina~\pageref{pg:menciofitxer}, és comú que
les dades ---de qualsevol classe: textos, instruccions de programa,
dades gràfiques,de so, de vídeo, etc.--- emmagatzemades en memòria
secundària estiguen organitzades en {\em fitxers}, també anomenats
\emph{documents} o \emph{arxius}. Els fitxers són conjunts de dades
amb un nom que els identifica i que es manipulen ---s'obrin, es
tanquen, es copien, s'esborren--- com un tot. En discos grans, seria
molt incòmode tenir tots els fitxers un darrere l'altre; per això, és
comú que els fitxers estiguen organitzats en {\em directoris}, també
anomenats \emph{carpetes}. Els directoris són fitxers especials que
agrupen els noms i les característiques d'altres fitxers; de fet, els
directoris poden contenir zero o més fitxers o també zero o més
directoris (sense restriccions de quantitat), i així successivament,
de manera que la persona usuària pot establir una estructuració
jeràrquica o arbòria dels seus fitxers en el disc.

Normalment, cada disc té un {\em directori principal} o
\emph{directori arrel} (el més elevat en la jerarquia de directoris),
dins del qual es troba tota la resta de directoris. Dos fitxers
---també dos directoris--- només poden tenir el mateix nom si es
troben en directoris diferents. Per raons històriques, els noms de
fitxers solen tenir dues parts: el \emph{nom} pròpiament dit i
l'\emph{extensió}, separades per un punt (per exemple,
\texttt{alacant.txt}). El nom sol ser normalment lliure, però
l'extensió sol ser curta (entre una i quatre lletres) i el sistema
operatiu la sol usar per a identificar el programa que s'ha d'usar per
a processar-lo o el format en què es troben les dades que conté (per
exemple, l'extensió \texttt{.txt} identifica normalment un fitxer de
text pla, vegeu l'apartat~\ref{ss:formats}; l'extensió \texttt{.exe}
s'usa per als programes d'ordinador, etc.).


La seqüència dels noms de les carpetes que cal anar obrint fins que
arribem a un fitxer s'anomena la \emph{trajectòria} o la \emph{ruta}
del fitxer.  De fet, convé considerar la trajectòria com a part del
nom del fitxer, cosa que ens permetria dir, senzillament, que en un
disc no pot haver-hi dos fitxers amb el mateix nom.  

\begin{figure}
\centering
\begin{parsetree}
    ( .{$\Box$}.
      (.{\carpeta{prac1}}.
         (.{\carpeta{tt}}.
               `\fitxer{tt1}{txt}'
               `\fitxer{tt2}{txt}'
      )
         (.{\carpeta{reserva}}.
         (.{\carpeta{tt}}.
              `\fitxer{tt1}{txt}'
         )
         )
      )
    )
\end{parsetree}
\caption{Exemple d'estructura de fitxers i directoris en un dispositiu
  d'emmagatzemament. El directori principal o arrel està
  representat pel símbol $\Box$.}\label{fg:fitxersdirs}
\end{figure}

Tots aquests conceptes es veuen potser més clars amb l'exemple de la
figura \ref{fg:fitxersdirs} en què es mostra l'estructura de fitxers i
directoris en un dispositiu d'emmagatzemament qualsevol. En aquest
dispositiu, el directori principal o arrel (representat amb el símbol
$\Box$) conté un únic [sub]directori \carp{prac1}; aquest directori
conté dos [sub]directoris, \carp{tt} (que conté els fitxers
\texttt{tt1.txt} i \texttt{tt2.txt}) i \carp{reserva}.  El directori
\carp{reserva} conté un subdirectori \carp{tt} (que conté l'arxiu
\texttt{tt1.txt}). Fixeu-vos que dues carpetes diferents contenen
arxius amb el mateix nom \texttt{tt1.txt}; això no és problema si
considerem la trajectòria completa com a nom del fitxer. Si el disc es
diu \texttt{C:} (típic en el cas del disc dur d'un PC amb sistema
operatiu Windows), les trajectòries d'aquests dos fitxers serien
\texttt{C:}\barra\carp{prac1}\barra\carp{tt}\barra\texttt{tt1.txt} i
\texttt{C:}\barra\carp{prac1}\barra\carp{reserva}\barra\carp{tt}\barra\texttt{tt1.txt},
i, per tant, serien diferents. En el cas d'el sistema operatiu
GNU/Linux les trajectòries d'aquests dos fitxers serien
\texttt{/}\carp{prac1/}\carp{tt/}\texttt{tt1.txt} i
\texttt{/}\carp{prac1/}\carp{reserva/}\carp{tt/}\texttt{tt1.txt}. Fixeu-vos
que cada sistema operatiu fa servir un símbol diferent per al
directori principal o arrel i per a separa els noms dels directoris i
arxius dins de la ruta o trajectòria.

\section{Tipus d'ordinadors}
Una classificació no gaire exhaustiva dels diferents tipus
d'ordinadors que podem trobar-nos avui dia és la següent:

\begin{description}
\item[De sobretaula] (en anglés \emph{desktop}): Estan formats per una
  \emph{caixa} amb dispositius de processament i emmagatzematge i un
  conjunt de perifèrics (dispositius d'entrada o d'eixida) com ara el
  teclat, el ratolí o la pantalla.  Són, amb diferència, els
  ordinadors més habituals.
\item[Portàtils] (en anglés \emph{laptop}, encara que quan són menuts
  se'n diu de vegades \emph{notebook} o \emph{netbook}): Tenen una
  grandària menor que la d'un maletí i un pes lleuger que permet
  dur-los sense massa esforç d'un lloc a un altre. Però, el seu volum
  reduït limita les possibilitats de fer-hi ampliacions i, per tant,
  el seu temps de vida pot ser més curt que el dels ordinadors de
  sobretaula. 
%Tot i que fa uns anys el seu preu era molt superior al
%  d'un ordinador de sobretaula, avui la diferència és cada vegada més
%  reduïda.
\item[Tauletes i \emph{smartphones}]: les tauletes (en anglés
  \emph{tablets}) i els telèfons mòbils més moderns, anomenats
  \emph{smartphones}, són veritables ordinadors por\-tà\-tils, amb una
  pantalla tàctil i sense teclat, amb càmera, connectivitat Wi-Fi i
  Bluetooth, receptor GPS, etc. Solen venir amb un sistema operatiu
  gràfic: el més comú és Android, però els de la marca Apple usen
  iOS.\footnote{Aquests dispositius han desplaçat els antics
    \emph{handhelds} o dispositius de mà, que eren una evolució de les
    antigues agendes electròniques i se solien anomenar \emph{PDA} per
    \emph{Personal Digital Assistant}, `assistent digital personal'.}

\item[Servidors:] Els servidors són ordinadors que contenen i
  gestionen informació que s'utilitzarà en d'altres ordinadors
  (``clients'') connectats a ells a través d'una xarxa interna o a
  través d'Internet; no són massa diferents d'un ordinador de
  sobretaula, típicament més potents quant a memòria, disc i
  processador, però, com que ningú ha de seure davant d'ells no solen
  tenir pantalles, teclats o ratolins i sovint estan pensats per a ser
  disposats horitzontalment en armaris especials anomenats
  \emph{racks}.  Aquests ordinadors es poden presentar en grups
  connectats entre sí per a oferir major potència i capacitat.
\end{description}

Quant als ordinadors de taula i els portàtils, sovint es fa la
distinció entre els ordinadors de tipus PC i els Macintosh (sovint
anomenats Mac). Els PC són l'evolució dels primers ordinadors
personals desenvolupats per IBM, tot i que actualment són fabricats
per un nombre molt gran d'empreses. Els Mac, a hores d'ara també són
ordinadors de tipus PC, però antigament eren ordinadors tipus PowerPC
fabricats exclusivament per l'empresa Apple. Els Mac fan servir un
sistema operatiu propi (MacOS) i tenen una quota de mercat més reduïda
entre el públic general però més gran en determinades aplicacions
especialitzades (per exemple, el disseny gràfic).

% Una classificació que pràcticament ha passat de moda és la següent:
% \begin{description}
% \item[``Mainframes'':] Ordinadors que normalment contenen més d'un
%   processador central i són capa\c{c}os de donar servei a més d'un
%   usuari alhora mitjan\c{c}ant estratègies de compartició del
%   temps. La noció de ``mainframe'' està associada a la de ``centre
%   de processament de dades'' (d'una universitat, d'un ministeri,
%   d'una gran empresa, etc.).  N'hi ha alguns que es diuen
%   ``supercomputadors'' i s'usen en enginyeria i en ciència.
% \item[Miniordinadors:] Versió redu\"{\i}da dels ``mainframes''
%   (normalment basats en un processador únic) que s'usen molt
%   comunament en ambients de recerca i de manufacturació. També poden
%   ser usats per més d'un usuari i executar més d'una tasca alhora.
% \item[Microordinadors:] Ara se'ls anomena més comunament
%   \emph{ordinadors personals}, i estan dissenyats per a ser usats
%   per un únic usuari. Els PC i els Macintosh en són exemples.
% \end{description}

% La classificació no pot ser molt rígida. En concret, els PC són cada
% volta més potents i, amb un sistema operatiu adequat ---multitasca i
% multiusuari--- poden fer les funcions que fa uns cinc anys només
% feien els miniordinadors i en fa uns deu només feien els
% ``mainframes''.

\section{Configuració típica d'un ordinador personal}
La configuració clàssica d'un ordinador personal de sobretaula model
2015 sol ser més o menys com segueix:
\begin{itemize}
\item La unitat base (la ``caixa'' o la ``torre'') conté:
\begin{itemize}
\item Un processador compost de quatre nuclis o processadors
  individuals (\emph{quad-core}) o més, com ara un \emph{Intel Core
    i5} o més o un processador equivalent de la marca AMD (vegeu el
  glossari, apartat~\ref{ss:OiPgloss}) a 3~GHz.
\item La memòria RAM (per exemple, de 8~GB).
\item Una bona targeta gràfica, amb la seua pròpia unitat independent
  de processament gràfic o GPU (\emph{graphics processing units}),
\item Un disc fix amb una capacitat de l'ordre d'1~TB
\item Una unitat enregistradora de DVD i de CD-ROM\footnote{En les
    unitats de CD-ROM és important la velocitat màxima de
    transferència de dades, que es dóna com a múltiple de l'estàndard
    (la d'un CD de música, de l'ordre d'uns 150~kilooctets per segon):
    quàdrupla (4$\times$), sèxtupla (6$\times$), etc.  Actualment no
    és estrany que una unitat de CD-ROM tinga una velocitat punta de
    lectura i d'escriptura de 52$\times$ o més.  De tota manera, les
    velocitats \emph{mitjanes} de tot un procés de lectura i
    escriptura solen ser més baixes.}
\item Una targeta de so amb altaveus i micròfon.
\item Una càmera (de vegades anomenada \emph{webcam})
\item Una o més targetes de comunicacions incorporades (amb fils o
  sense fils, vegeu el glossari, apartat~\ref{ss:OiPgloss}).
\end{itemize}
\item Un monitor o pantalla, normalment una pantalla
  LCD\footnote{\emph{liquid-crystal display} o pantalla de cristall
    líquid} de 17 o més polzades,
\item Un teclat separat i un ratolí.
\item Una impressora (d'injecció o de raig de tinta ---la més
  típica---, o làser\footnote{Impressores matricials o d'agulles ja no
    se'n fan.}).
\end{itemize}
Les especificacions dels portàtils (memòria, processador) solen ser
similars, normalment una miqueta més reduïdes. Els telèfons mòbils
intel·ligents o \emph{smartphones} i les tauletes no solen tenir disc,
sinó una memòria flaix no volàtil, per exemple, de 8~GB, i una memòria
RAM de l'ordre d'1~GB.

\section{Un petit glossari}
\label{ss:OiPgloss}
\comprof{Comprovar si cal afegir alguna cosa més}

Aquest glossari arreplega alguns termes d'ús comú en la descripció
d'ordinadors i programes que no han estat definits més amunt.


\begin{description}
\item[adaptador de vídeo] (també anomenada targeta gràfica o
  controlador de vídeo): Dispositiu (targeta independent, o integrada
  en la placa base) que permet connectar un monitor a l'ordinador. Hi
  ha molts tipus d'adaptadors de vídeo. Se n'ha de considerar la
  \emph{resolució}, és a dir, el nombre de punts, elements d'imatge
  (\emph{píxels}) que caben en una imatge, per exemple $1024 \times
  768$ (horitzontal $\times$ vertical), la \emph{profunditat de color}
  (en bits: per exemple 24 bits permeten $2^{24}=16~777~216$ colors
  diferents) i altres paràmetres com la {\em freqüència de
    refrescament} (que es mesura en hertzs o cicles per segon; vegeu
  ``megahertz''). Actualment no és estrany tenir en 1ordinadors de
  taula o portàtils resolucions com l'anomenada \emph{HD 1080} (1920
  $\times$ 1080) o fins i tot més grans.
  
\item[ADSL] (de l'anglés \emph{asymmetric digital subscriber line},
  línia d'abonat digital asimètrica): Versió asimètrica de DSL (vegeu
  DSL). L'asimetria fa referència al fet que la velocitat de
  transmissió de dades de la central cap a l'abonat és superior que la
  velocitat de transmissió de dades de l'abonat cap a la central (per
  exemple, 8 Mb/s cap a l'abonat i 512 kb/s cap a la central).

\item[cache] o \emph{memòria cau}: Memòria RAM intermèdia, d'accés més
  ràpid per part del processador, on es copia de quan en quan un bloc
  (també ``pàgina'') complet de posicions consecutives de la memòria
  RAM general per a simplificar accessos repetits a posicions en la
  mateixa zona. Per exemple, en un ordinador amb 512~kilooctets
  (524.288 octets) de \emph{memòria cau}, després d'accedir a la
  posició 2.000.000 és molt probable que el processador vulga accedir
  a la posició 2.000.003. Si quan s'ha demanat la 2.000.000 es copien
  en la \emph{memòria cau} les 524.288 posicions que van de la
  1.572.864 a la 2.097.151, l'accés a la posició 2.000.003 serà més
  ràpida.

  % \item[densitat alta:] Quan els disquets de 3 polzades i mitja (uns
  %   90~mm) són de densitat alta solen estar marcats amb les lletres
  %   HD (\emph{high density}) i poden emmagatzemar 1,44~MB de
  %   dades. Els anomenats de \emph{densitat doble} (DD) ---molt poc
  %   usats actualment--- poden contenir 720~kB, és a dir, la
  %   meitat. Es distingeixen perquè els primers tenen un petit
  %   foradet quadrat addicional (a la part inferior dreta si posem la
  %   finestra dalt).
  
\item[DSL] (de l'anglés \emph{digital subscriber line}, línia d'abonat
  digital), tecnologia de connexió que permet aprofitar les línies
  telefòniques i elèctriques per a fer connexions d'alta velocitat
  (fins a uns 10 Mb/s). En el cas d'usar les línies elèctriques, la
  tecnologia rep també el nom de PLC (\emph{power line communications}
  o comunicacions a través de les línies de força), però a Espanya no
  s'usa per a proveir serveis d'Internet a les llars.

\item[GHz:] vegeu gigahertz.

\item[gigahertz:] un gigahertz són 1000 megahertz (vegeu
  \emph{megahertz} en aquest glossari).
  
\item[GNU/Linux:] un sistema operatiu multitasca i multiusuari
  gratu\"{\i}t, de l'estil de l'Unix que es podia trobar en els
  anomenats \emph{miniordinadors} dels anys 70 i 80, desenvolupat de
  manera col·laborativa per milers de voluntaris independents i per
  empreses arreu del món i que és \emph{programari lliure} (vegeu
  l'entrada en aquest glossari): es pot copiar lliurement si es
  compleixen certes condicions. Es pot instal$\cdot$lar GNU/Linux (que
  es presenta en moltes \emph{distribucions} diferents com ara
  \emph{Ubuntu}, \emph{Mint}, \emph{Fedora}, etc.) en un PC amb
  processador de la família x86 (vegeu \emph{Pentium}) o superior i en
  molts altres tipus d'ordinador.
  
\item[Macintosh] o \emph{Mac}: nom genèric (i comercial) d'una família
  d'ordinadors construïts per Apple Computer i que són bàsicament
  equivalents als PC. Aquests ordinadors, llançats al mercat el 1984,
  popularitzaren la interfície gràfica d'usuari, tota una revolució
  per a l'època. Fa uns anys havia diferències significatives entre
  els PC i els \emph{Mac} de manera que no eren compatibles, es a dir,
  que els programes d'un no funcionaven en l'altre, s'havien d'adaptar
  a les característiques particulars de cadascun. Aquestes diferències
  eren degudes al fet que el processador del \emph{Mac} no era de la
  família x86 (vegeu \emph{Pentium}), sinó d'una altra (antigament la
  família 68000 de Motorola, i després l'anomenat PowerPC). En
  l'actualitat esta diferència no existeix i tant uns com altres
  empren processadors de la família x86. En el cas dels \emph{Mac} des
  de l'any 2006 incorporen processadors Intel, així que podem
  instal·lar-hi Microsoft Windows o GNU/Linux amb tots els seus
  programes, encara que també podem usar el sistema operatiu propi
  dels Mac, anomenat \emph{MacOS}.
  
\item[megahertz:] Un megahertz (MHz) és un milió de hertzs (Hz),
  és a dir, un milió de cicles per segon. La velocitat de les
  unitats centrals dels ordinadors es mesura en MHz i més recentment
  en GHz, és a dir, en milions o milers de milions de cicles
  bàsics de processament d'informació ---corresponents als \emph{tics}
  o impulsos del rellotge que sincronitza tots els circuits de
  l'ordinador--- per segon.  L'execució d'una instrucció per part del
  processador sol consumir un nombre menut de cicles, quasi sempre
  més d'un. Els models actuals poden executar, en determinades
  circumstàncies, més d'una instrucció al mateix temps, el que fa que de
  vegades s'execute una instrucció per cicle de rellotge o fins i tot
  més d'una.  Una velocitat típica en l'actualitat és 3~GHz, és a
  dir, 3000~MHz. Una velocitat més gran implica una velocitat
  d'execució més gran, sempre que no hi haja altres circumstàncies
  limitants (per exemple, una falta de memòria).  Altres components,
  com ara la memòria RAM, també funcionen a una determinada velocitat,
  independent de la del processador, que es mesura també en MHz.


\item[MHz:] vegeu megahertz.
  
\item[mòdem:] abreviatura de modulador-desmodulador. En el cas del
  mòdem més comú a finals del segle passat, el \emph{mòdem telefònic},
  es tractava d'un dispositiu (normalment una placa interna, encara
  que també pot ser extern) que permetia usar la línia telefònica
  (senyals analògics) per a comunicacions informàtiques (digitals)
  entre dos ordinadors, establint una telefonada; era aleshores la
  manera estàndard d'accedir a Internet des de casa. Un dels
  paràmetres més interessants d'un mòdem és la
  \emph{velocitat} de transmissió de dades, que es mesura en b/s (bits
  per segon). Una velocitat clàssica en mòdems domèstics era 33.600
  b/s (més recentment, 57.600 b/s; les línies telefòniques actuals
  poden admetre potser velocitats al voltant dels 100.000 b/s).  Això
  permetia enviar una carta d'una pàgina en unes dècimes de segon però
  no seria suficient per a la major part dels usos actuals d'Internet.

  La paraula \emph{mòdem} es pot usar també per a altres tipus de
  mòdems, normalment més ràpids: els \emph{mòdems de cable}, que
  permeten connectar l'ordinador a Internet a través dels cables
  d'empreses especialitzares que ofereixen televisió, telèfon i
  Internet, els \emph{mòdems ADSL} (vegeu \emph{ADSL} en aquest
  glossari), etc.
\comprof{Posem alguna cosa sobre fibra?}

% \item[PCMCIA] (de l'anglés \emph{Personal Computer Memory Card
%   International Association}, associació internacional de targetes
%   de memòria per a ordinadors personals): nom d'un estàndard de
%   targetes de memòria, mòdems, discos, unitats de disquets, targetes
%   de xarxa i altres dispositius perifèrics per a ordinadors
%   portàtils.  Aquestes targetes no són massa diferents en grandària
%   d'una targeta de crèdit i s'insereixen en el port corresponent de
%   l'ordinador, amb la peculiaritat que en permeten l'inserció i
%   l'extracció ``en calent'', és a dir, sense haver d'apagar
%   l'ordinador.

\item[Programari lliure:] (\emph{free software}, també anomenat
  \emph{programari de codi font obert} o \emph{open-source software})
  és el programari que es distribueix amb llicències que donen una
  sèrie de llibertats a qui rep el programari: la llibertat d'usar-lo
  per a qualsevol propòsit sense restricció, la llibertat
  d'examinar-lo per veure com funciona i modificar-lo per adaptar-lo a
  un nou ús, i la llibertat de distribuir còpies ---originals o
  modificades--- lliurement a qui desitgem. Per a poder modificar el
  programari, no hi ha prou amb tenir accés a la versió executable en
  l'ordinador: hem de tenir accés a l'anomenat \emph{codi font}, és a
  dir, a la versió del programari que escriuen i modifiquen les
  persones que programen (d'ací el nom \emph{de codi font obert}) i
  que després es converteix automàticament en la versió
  executable. Exemples de programari lliure són: el sistema operatiu
  \emph{GNU/Linux}, el navegador \emph{Firefox}, o el processador de
  textos \emph{LibreOffice}. No s'ha de confondre \emph{programari
    lliure} amb \emph{programari gratuït} (o \emph{freeware}). Hi ha
  programari gratuït que no és lliure perquè no atorga totes les
  llibertats (per exemple, el lector de PDF \emph{Adobe Acrobat}, o el
  programa de telefonia per Internet \emph{Skype}: per exemple, tot i
  tenir el programari executable, no tenim accés al seu codi font).
  
\item[Pentium:] nom genèric d'una família actual de processadors
  centrals de la companyia Intel, els més recents de la sèrie ``x86''
  de processadors que començà amb el 8086 a principis dels anys 80,
  passant pel 80286, el (80)386 i el (80)486.\footnote{El nom
    \emph{Pentium} es va triar perquè Intel no podia registrar ``586''
    com a marca.}. Els nous processadors tenien jocs d'instruccions
  més complexos i eren capaços d'executar els programes que executaven
  els anteriors (per exemple, un Pentium pot executar qualsevol
  programa escrit per a un 386) però introduïen millores que permetien
  ordinadors més ràpids, amb capacitat més gran de càlcul, capaços de
  processar més dades en cada instrucció (8, 16, 32 ---a partir del
  386---, i actualment 64 bits) i de gestionar més memòria. Els
  Pentium més recents tenen més d'un \emph{nucli} o sub-processador, i
  poden, per tant, executar instruccions de programa en paral·lel.

\item[placa de so:] En els ordinadors més antics, s'havia de comprar a
  banda una placa (o targeta) de so si es volia usar l'ordinador per a
  processar, enregistrar, reproduir, i manipular sons
  digitalitzats. En l'actualitat tots els ordinadors porten aquestes
  capacitats incorporades.

\item[targeta de xarxa:] En els ordinadors més antics, per a connectar
  ordinadors i formar una xarxa (normalment local) per a compartir
  recursos, calia dotar a cada ordinador d'una placa o targeta de
  xarxa. Hi ha diversos estàndards de connexió en xarxa; els més
  anomenats són Ethernet (per a connexions amb fils) i \emph{Wi-Fi}
  (vegeu \emph{Wi-Fi} en aquest glossari).

\item[USB] (de l'anglés \emph{universal serial bus}, bus sèrie
  universal): Estàndard o norma de connexió de dispositius perifèrics
  (impressores, mòdems, reproductors digitals de música, càmeres
  digitals, unitats de memòria) que transmet les dades en sèrie (és a
  dir, un bit darrere de l'altre) a velocitats que en les versions més
  modernes de l'estàndard poden arribar als Gb/s, i que permet la
  connexió i desconnexió de dispositius de moltes classes ``en
  calent'', és a dir, sense haver d'apagar l'ordinador.

\item[Wi-Fi] (probablement de l'anglés \emph{wireless fidelity},
  fidelitat sense fils): tecnologia de connexió sense fils (via
  ràdio), principalment per a formar xarxes locals, i que en
  l'actualitat (estàndard 802.11n; octubre de 2009) permet velocitats
  de transmissió de fins a 600~Mb/s. \todo{Hay versiones más modernas:
   802.11ac, hasta 6.77 Gb/s}
\end{description}


\section{Qüestions i exercicis}
\todo{Revisar preguntas}
\begin{enumerate}
\item Quants \emph{kilobytes} (kilooctets) hi ha en un \emph{gigabyte}
  (gigaoctet)?
        \begin{enumerate}
        \item 1.024
        \item 1.073.741.824
        \item 1.048.576
        \end{enumerate}
        
      \item Si una memòria USB té 6~\emph{gigabytes}
        (gigaoctets) i una pàgina de text (europeu occidental) típica
        té 50 línies de 60 caràcters (contant els blancs), quantes
        pàgines caben aproximadament en la memòria?
       \begin{enumerate}
       \item 200
       \item 20000
       \item 2000000 
       \end{enumerate}
       

\item Una persona connectada a Internet per telèfon observa que les
  velocitats de transferència que li indica el seu navegador (vegeu el
  capítol~\ref{se:Internet}) varien al
  voltant dels 300~kilooctets (\emph{kilobytes}) per segon. Una
  d'aquestes tres \emph{no} pot ser la velocitat del seu servei d'ADSL:
     \begin{enumerate}
     \item 1 megabit per segon
     \item 6 megabits per segon
     \item 4 megabits per segon
     \end{enumerate}
 
% \item Quina d'aquestes condicions impedeix
%       que un procediment o mètode siga un \emph{algorisme}?
%         \begin{enumerate}
%         \item Que la resposta siga incorrecta.
%         \item Que s'execute indefinidament.
%         \item Que continga instruccions innecesàries però que no
%           destorben el seu funcionament.
%         \end{enumerate}

\item Quina d'aquestes afirmacions és incorrecta?
      \begin{enumerate}
      \item Els mòdems converteixen informació digital en senyals
            analògics però no al revés.
          \item Les velocitats típiques de connexió a Internet via
            ADSL són d'uns quants Mb/s.
          \item El servei ADSL aprofita les línies de telefonia
            convencional per a oferir connexió a Internet.
      \end{enumerate}
      
    \item Es podria enregistrar (guardar) en un CD-ROM tota la
      informació continguda en un instant determinat en la memòria RAM
      d'un vell ordinador que en té 256 MB?
      \begin{enumerate} 
      \item Sí.
      \item No, perquè no hi cap.
      \item No, perquè un suport és electrònic i l'altre òptic.
      \end{enumerate}
\item Quina d'aquestes afirmacions es certa?
        \begin{enumerate}
        \item En qualsevol disc (disquet, dur, CD-ROM) sempre hi ha un directori especial que
          es diu arrel.
        \item Un disc no pot contenir més de dos nivells de jerarquia
              de carpetes.
        \item Una carpeta no pot contenir només una altra carpeta.
        \end{enumerate}

\item Pot haver-hi dos carpetes amb el mateix nom una dins de l'altra?
        \begin{enumerate}
        \item No.
        \item Sí, si tenen data i hora diferents.
        \item Sí.
        \end{enumerate}


\item Quants valors possibles pot prendre un octet o \emph{byte}?
 \begin{enumerate}
 \item 2
 \item 256
 \item 8
 \end{enumerate}

\item Quin dels tres mitjans d'emmagatzemament següents no és òptic:
        \begin{enumerate}
        \item Un CD-ROM
        \item Un DVD
        \item Un disc fix.
        \end{enumerate}

\item On resideix un programa d'ordinador mentre  l'estem executant?
        \begin{enumerate}
        \item En el disc dur o en un disquet.
        \item En la memòria RAM (almenys parcialment).
        \item En el CD-ROM.
        \end{enumerate} 

\item Quina d'aquestes definicions de fitxer és més
        correcta?
        \begin{enumerate}
        \item Un conjunt de dades que es manipula com un tot, resideix
              en algun mitjà d'emmagatzemament i  
              té un nom.
        \item Una estructura que conté els noms d'altres fitxers.
        \item Una estructura de dades que representa el text generat
              per un processador de textos i que té un nom associat.
        \end{enumerate} 

\item Quines són les característiques de la memòria
      RAM d'un ordinador de taula?
       \begin{enumerate}
       \item és lenta, volàtil i d'accés aleatori.
       \item és ràpida, volàtil i d'accés aleatori.
       \item és ràpida, permanent i d'accés seqüencial.
       \end{enumerate}

     \item Es pot fer que diversos ordinadors compartisquen un recurs
       connectat a un d'ells com, per exemple, una impressora?
\begin{enumerate}
\item Sí, si els ordinadors estan connectats formant una xarxa local.
\item Només si la impressora està connectada a Internet.
\item Sí, instal·lant-li un mòdem ADSL a la impressora.
\end{enumerate} 

\item 
   Cada punt d'una pantalla pot tenir 256 colors: quants octets
   (\emph{bytes}) de
   memòria ocupa cada punt?
   
\begin{enumerate}
\item 1
\item 256
\item 8
\end{enumerate}

\item 
   Quant es tarda aproximadament en enviar a través d'una connexió
   via mòdem de 28800 bits per segon el contingut d'un disquet
   de densitat alta de 3 polzades i mitja completament ple?
   
\begin{enumerate}
\item Un minut aproximadament
\item Uns 7 minuts
\item Uns set segons
\end{enumerate}

\item Quan el processador central està executant un programa, on espera
  trobar la següent instrucció?
  
\begin{enumerate}
\item en el CD-ROM
\item en el disc fix
\item en la memòria RAM
\end{enumerate}

\item És possible posar un fitxer de text en el directori (la
  carpeta) arrel?
  
\begin{enumerate}
\item Com en qualsevol directori.
\item Només si és un fitxer propi del sistema operatiu.
\item No, primer s'hi ha de crear una carpeta (un directori).
\end{enumerate}

\item En la Universitat d'Alacant hi havia 32050 alumnes durant el
   curs 2001--2002. Si assignem un número a cada alumne, quants octets
   (\emph{bytes}) fan falta per a guardar el número de cada alumne?
   
\begin{enumerate}
\item 15
\item 2
\item 3
\end{enumerate}

\item Quin d'aquests tres mitjans d'emmagatzematge és el més ràpid?
   
\begin{enumerate}
\item La memòria RAM
\item Un CD-ROM
\item Un disc dur.
\end{enumerate}


\item Tenim dues unitats lectores de CD-ROM. La velocitat de la
   primera és 40\begin{math}\times\end{math} i la de la segona
   4\begin{math}\times\end{math}. Si en la segona unitat un programa tarda a
   carregar-se 10 segons, quant tarda en la primera?
   
\begin{enumerate}
\item 1 segon aproximadament
\item 100 segons aproximadament
\item el mateix temps
\end{enumerate}

\item Pràcticament tots els programes necessiten fer operacions
   bàsiques com ara obrir i tancar arxius o gestionar el ratolí i la
   pantalla. Vol dir això que tant un navegador com un processador de
   textos com una memòria de traducció contenen instruccions de
   programa per a executar aquestes operacions bàsiques?
   
\begin{enumerate}
\item No, només instruccions per a invocar els corresponents
      programes del sistema operatiu.
\item Sí, perquè formen part del processador central.
\item Sí, perquè, si no, no les podrien executar.
\end{enumerate}

\item Dins d'una carpeta (directori) podem posar carpetes i
   documents (fitxers)
   mesclats? 
   
\begin{enumerate}
\item No. Si una carpeta està dividida en subcarpetes no pot
      contenir documents; els documents haurien d'anar dins de les subcarpetes
\item Només en la carpeta arrel.
\item Sí. 
\end{enumerate}

\item 
   Quanta memòria ocupa una imatge de 1024 per 1024 punts on cada punt
   pot tenir 8 colors?
   
\begin{enumerate}
\item 1 megaoctet (\emph{megabyte})
\item 384 kilooctets (\emph{kilobytes})
\item 8 megaoctets (\emph{megabytes})
\end{enumerate}


\item Alguns ordinadors portàtils estan dissenyats de manera que, quan
  les bateries estan a punt d'esgotar-se (o l'ordinador no s'està
  usant), copien \emph{tota} la memòria RAM al disc dur i
  s'apaguen. Si tornem a carregar les bateries i encenem l'ordinador,
  fan l'operació inversa. Podem esperar que l'execució dels programes
  continue en el mateix punt on es trobava quan les bateries van
  fallar?
  
\begin{enumerate}
\item No, perquè la memòria RAM s'esborra quan falta l'alimentació
     elèctrica.
\item No, perquè només s'hi ha guardat el sistema operatiu.
\item Sí, perquè els programes en execució i les seues dades
     estaven tots en la memòria RAM (si no eren ja al disc).
\end{enumerate}

\item Pot un fitxer contenir les instruccions d'un programa executable?
\begin{enumerate}
\item No.
\item Només si està escrit en un llenguatge de programació d'alt nivell,
perquè només així serà un text i podrà guardar-se en un fitxer.
\item Sí.
\end{enumerate}
\item Si les imatges enviades per una vella càmera digital sense
  colors (blanc i negre) tenen 100$\times$100 píxels, quantes
  d'aquestes imatges podriem emmagatzemar en una memòria USB d'1~GB?
   
\begin{enumerate}
\item Depenent de la codificació escollida per als caràcters de la imatge, 
   entre 400~000 i 800~000.
\item Unes 10~000.
\item Unes 800~000.
\end{enumerate}

\item Quina característica és comuna a tots els tipus de programari?
  
\begin{enumerate}
\item Que comencen a executar-se en connectar l'ordinador.
\item Que consisteixen en una llista d'instruccions executables.
\item Que s'encarreguen de la gestió de tots els recursos del 
      maquinari de l'ordinador on s'executen.
\end{enumerate}

\item És possible que un fitxer de text i la carpeta en què està
  inclòs tinguen el mateix nom?
  
\begin{enumerate}
\item Només si el fitxer ha estat creat pel sistema operatiu.
\item Només si es tracta del directori arrel.
\item Sí, no importa el nom de la carpeta.
\end{enumerate}

\item Quants bits necessitem per codificar un número de telèfon de 9
xifres suposant que codifiquem els dígits un a un?
\begin{enumerate}
\item 27
\item 36
\item 9
\end{enumerate}

\item La tarjeta Compact Flash on s'emmagatzemen les fotografies d'una
   càmera digital té 2~GB. Si suposem que hem triat una
  resolució i un format d'imatge que fa que cada fotografia necessite
  un espai de 2048 kB, quantes tarjetes d'aquestes hem de comprar si
  volem fer 2500 fotos al llarg d'un viatge?
\begin{enumerate}
\item 1
\item 3
\item 4
\end{enumerate}

\item El \emph{sistema operatiu} d'un ordinador és{\ldots}
   
\begin{enumerate}
\item {\ldots} maquinari (\emph{hardware}).
\item {\ldots} programari (\emph{software}).
\item {\ldots} una manera d'especificar el format dels textos.
\end{enumerate}
\item Si reduïm de 3000~MHz a 1500~MHz la freqüència del rellotge
   d'un ordinador i encara funciona{\ldots}
   
\begin{enumerate}
\item {\ldots} executarà els programes a la mateixa velocitat.
\item {\ldots} executarà els programes més lentament.
\item {\ldots} tardarà menys a executar els programes.
\end{enumerate}
\item Són les tres de la matinada i ja és hora d'anar a
   dormir. Abans d'apagar l'ordinador, on es guarda el
   treball que s'ha fet per a continuar-lo demà?
   
\begin{enumerate}
\item En l'acumulador del processador central.
\item En la memòria RAM de l'ordinador.
\item En un suport magnètic, normalment.
\end{enumerate}

\item Windows usa les \emph{extensions} dels noms de fitxers per
   a{\ldots}
   
\begin{enumerate}
\item {\ldots} associar-los el programa que els obrirà quan
      fem doble clic sobre la icona del fitxer.
\item {\ldots} estalviar espai quan es guarden els fitxers.
\item {\ldots} saber si estan buits o contenen text.
\end{enumerate}
\item Un mòdem és un dispositiu que{\ldots}
   
\begin{enumerate}
\item {\ldots} converteix la informació digital en senyals analògics.
\item {\ldots} converteix senyals analògics en informació digital.
\item {\ldots} fa les dues coses.
\end{enumerate}

% < QT2

\end{enumerate}
\section{Solucions}\todo{Comprovar les respostes}
\begin{enumerate}
\item (c). Un gigaoctet té 1024 megaoctets, i un megaoctet, 1024
  kilooctets: $1024 \times 1024 = 1048576$.
\item (c). Una memòria USB de 6 gigaoctets conté aproximadament 6.000.000.000
  octets. Un carácter, en les codificacions usades comunament en
  Europa occidental, ocupa un octet; per tant, la pàgina de $50\times
  60$ ocupa 3.000 octets. Caben $6.000.000.000/3.000=2.000.000$
  pàgines en un disc.
\item (a). Una velocitat de 300 kilooctets per segon equival a uns
$300000\times 8=2400000$ bits per segon.
\item (a). Els mòdems modulen (converteixen senyals
  digitals a analògics) i desmodulen (converteixen senyals analògics
  en digitals) per a enviar i rebre dades a través d'un determinat mitjà. Les
  línies ADSL domèstiques actuals admeten connexions via mòdem
  telefònic d'uns quants megabits
  per segon (vegeu la secció~\ref{ss:OiPgloss}).
\item (a): Un CD-ROM  pot emmagatzemar 650~MB.
\item (a)
\item (c)
\item (b): $2^8=2\times 2\times 2\times 2\times 2\times 2\times
  2\times 2=256$.
\item (c): Els discos fixos són generalment magnètics. 
\item (b): Almenys la porció del programa que s'està executant ha de
  residir en la RAM.
\item (a) 
\item (b)
\item (a)
\item (a): Cada punt pot prendre un de 256 colors. Per a poder
  emmagatzemar el color cal un nombre de bits suficient per a fer 256
  combinacions. Amb 8 bits podem fer \(2^8=2\times2\times 2\times
  2\times 2\times 2\times 2\times 2=256\) combinacions.Per tant, es necessiten 8 bits, és a dir, un octet.
\item (c)
\item (a)
\item (b). El nombre de bits necessari per a poder generar 32050
  combinacions és el nombre de vegades que cal multiplicar \(2\times
  2\times\ldots\) just fins al punt en què se supera 32050. Cal fer-ho
  15 vegades; per tant, necessitem 15 bits. Com que cada octet son 8
  bits, són necessaris 2 octets.
\item (a)
\item (a)
\item (a)
\item (c)
\item (b). Cada punt pot presentar-se en 8 colors diferents. Amb 3
  bits podem emmagatzemar \(2\times2\times2=8\)~colors. Per tant, la
  pantalla ocupa 1024?1024?3=3.145.728 bits, que són
  3.145.728/8=393.216 octets, que són 393.216/1024=384 kilooctets. Si
  els càlculs es fan amb números redons (1.000 en comptes de 1.024,
  etc.) la resposta més aproximada (375 kB) és la correcta.
\item (c)
\item (c)
\item (c)
\item (c)
\item (b)
\item (c)
\item (b). Cada dígit decimal pot prendre, per separat, 10 valors
  diferents. Tres bits per dígit decimal no són suficients (permeten només 8
  combinacions); quatre, sí. Així, cada dígit decimal ocupa quatre
  bits; si n'hi ha 9, necessitem 36 bits.
\item (c)
\item (b)
\item (b)
\item (c)
\item (a)
\item (c)

% < SL2
\end{enumerate}
