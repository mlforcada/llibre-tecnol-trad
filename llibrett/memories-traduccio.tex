\chapter{Memòries de traducció}
\label{se:memtrad}

\begin{quote}
  \textsl{Existing translations contain more solutions to more
    translation problems than any other existing resource}
  \citep{isabelle93p}
\end{quote}

\section{Introducció}

Una aproximació a la traducció humana assistida per ordinador (és a
dir, semiautomàtica) que està molt relacionada amb la traducció
directa és la que s'usa en les anomenades \emph{memòries de
  traducció}.\footnote{Com veurem més avall, en comptes de traduir mot
  a mot fent una simple substitució de cada mot origen pel(s) mot(s)
  meta corresponents, les memòries de traducció fan substitucions de
  fragments de més d'un mot, i, en lloc d'usar un diccionari bilingüe,
  usen una base de dades de fragments de més d'un mot prèviament
  traduïts.}  La noció bàsica \citep{somers96b,samuelson-brown96b} és
la utilitat de tenir a mà, quan s'està traduint un text nou, exemples
de frases similars i de les traduccions corresponents, provinents de
traduccions realitzades anteriorment. De fet, certs tipus de textos
com ara documents tècnics, informes anuals o manuals d'instruccions,
els quals se solen revisar freqüentment, sovint tenen moltes
repeticions. En aquests casos, la comparació de versions diferents del
que és essencialment el mateix text i la traducció repetitiva de
textos similars és innecessàriament laboriosa. A més, moltes vegades
el treball de traducció comporta un esforç creatiu considerable, com
ara quan es tracta de trobar una equivalència adequada a alguna
expressió especialment difícil de traduir; les memòries de traducció
permetrien no haver de repetir aquest esforç en el futur.

L'\textbf{objectiu} és, per tant, aprofitar traduccions anteriors per
a no repetir l'esforç quan s'han de fer traduccions noves. Ha de
quedar clar que, per a poder fer-ho, els textos originals i les
traduccions han d'estar en format informatitzat (fitxers de text).

\section{Bitextos}
\label{ss:bitextos}

Suposarem que, com a resultat del treball anterior de traducció, tenim
parells de textos $(E,D)$ on $E$ és el \emph{text esquerre} (en la
\emph{llengua esquerra}) i $D$ és el \emph{text dret} (en la
\emph{llengua dreta}), i volem traduir de la llengua esquerra a la
llengua dreta.\footnote{Parlem d'\emph{esquerra} i \emph{dreta} perquè
  pot ser que no siga important (o que no se sàpia) quin dels dos
  textos és l'original i quin és la traducció.}  De fet, quan dos
textos $E$ i $D$ són equivalents (és a dir, traducció un de l'altre)
direm que el parell $(E,D)$ és un \emph{bitext} o \emph{text
  paral·lel}.

Vegeu aquest exemple on la llengua esquerra és el català i la llengua
dreta, l'espanyol:
\begin{quote}
  $E$: ``Tenim textos equivalents i els volem aprofitar per a fer
  noves traduccions. Quan dos textos són equivalents diem que formen
  un bitext.''

  $D$: ``Tenemos textos equivalentes y los queremos aprovechar para
  hacer nuevas traducciones. Cuando dos textos son equivalentes
  decimos que forman un bitexto.''
\end{quote}

Però els bitextos complets no es poden aprofitar tal com estan perquè
és molt improbable que ens encarreguen de nou exactament la mateixa
tasca de traducció, és a dir, que ens encarreguen traduir de nou un
text $E'$ quan tenim ja el bitext $(E',D')$, però el que sí que és
probable és que algunes parts del text nou $E'$ apareguen també en la
part esquerra d'alguns dels bitextos que ja tenim. Per això,
necessitem obtenir a partir d'ells bitextos més menuts, amb parts esquerres que tinguen possibilitat de aparéixer de nou en el futur.

\subsection{Segmentació de bitextos}
La primera operació consisteix a \emph{segmentar} o \emph{dividir}
automàticament cada un dels dos textos en unitats més menudes
\emph{segments}, usant algun criteri programable (vegeu més avall).
La figura~\ref{fg:segmentat} mostra el resultat de la segmentació dels textos $E$ i $D$. Com s'hi veu, pot ser que inicialment el nombre de segments del text esquerre $E$ siga diferent del nombre de segments del text dret $D$.
\begin{figure}
  \begin{center}
    \begin{tabular}{p{3cm}p{3cm}}
           $E$ & $D$      
    \end{tabular}
    \begin{tabular}{|p{3cm}|p{3cm}|}
\hline
        $e_1$ & $d_1$ \\\hline
        $e_2$ & $d_2$ \\\hline
        $e_3$ & $d_3$ \\\hline
        \ldots & \ldots \\\hline
        \ldots  & $d_M$ \\\hline
        \ldots \\\cline{1-1}
        $e_L$ \\\cline{1-1}
    \end{tabular}
  \end{center}
  \caption{Els bitextos $E$ i $D$, segmentats, respectivament, en $L$
    i $M$ segments: $E=e_1e_2\ldots e_L$ i $D=d_1d_2\ldots d_M$. En
    l'exemple, el text esquerre té dos segments més (és a dir, $L=M+2$).}
  \label{fg:segmentat}  
\end{figure}


\subsection{Alineament de bitextos. Unitats de traducció}

El resultat de la segmentació encara no és útil: no queda clara la
correspondència entre els segments esquerres i els segments
drets. Necessitem \emph{revisar la segmentació}, és a dir, juntar
segments o partir segments, en un costat o en altre, fins que tinguem
el mateix nombre de segments i siguen traducció mútua. D'aquesta
operació se'n diu \emph{alinear} el bitext.

Direm que un bitext $(E,D)$ està \emph{alineat} si les dues parts
tenen el mateix nombre $N$ de segments, $E= e_1 e_2 e_3 \ldots e_N$ i
$D= d_1 d_2 d_3 \ldots d_N$ i els segments es corresponen: $e_1$
correspon a $d_1$, $e_2$ a $d_2$, etc. És a dir, el text alineat ens
proporiciona $N$ bitextos $(e_1,d_1)$, $(e_2,d_2)$, $(e_3,d_3)$, etc.,
més menuts (per això els escrivim amb minúscula); aquests bitextos
s'anomenen normalment \emph{unitats de traducció} (UT): vegeu la
figura~\ref{fg:alineat}.
\begin{figure}
  \begin{center}
    \begin{tabular}{p{3cm}p{3cm}}
           $E$ & $D$      
    \end{tabular}
    \begin{tabular}{|p{3cm}|p{3cm}|}
\hline
        $e_1$ & $d_1$ \\\hline
        $e_2$ & $d_2$ \\\hline
        $e_3$ & $d_3$ \\\hline
        \ldots & \ldots \\\hline
        $e_N$ & $d_N$ \\\hline
    \end{tabular}
  \end{center}
  \caption{Els bitextos $E$ i $D$, segmentats i alineats, en $N$ unitats de traducció $(e_1,d_1), (e_2,d_2), \ldots (e_N,d_N)$}
  \label{fg:alineat}
\end{figure}



Així és més probable que quan ens donen un text nou
$E'$ tinguem traduccions per a alguns dels seus fragments. El bitext
de més amunt es podria alinear per a formar les unitats de traducció 
\begin{center}
\pair{"Tenemos"}{"Tenim"}\\ 
\pair{"textos equivalents"}{"textos  equivalents"}\\ 
\pair{"{}i els volem aprofitar"}{"y los queremos  aprovechar"}\\ 
\ldots\\ 
\pair{"formen un bitext"}{"forman un bitexto"}
\end{center}

Com ja s'ha dit més amunt, una operació molt important per a la
reutilització o el \emph{reciclatge} de traduccions antigues és la
d'\emph{alinear} els textos i les traduccions existents per a
identificar fragments o \emph{unitats de traducció} que es puguen
reutilitzar posteriorment.  Una \emph{memòria de traducció} és una
base de dades en la qual cada fitxa (registre) conté una unitat de
traducció, i té, per tant, com a mínim dos camps: el text esquerre i
el text dret.

L'operació d'alineament de bitextos existents és una de les tasques
que es pot realitzar amb l'ajuda d'un programa de memòries de
traducció. La figura~\ref{fg:aliMT} mostra un esquema del procés.
\begin{figure}
{\small
$$
\mbox{\textsf{bitext}}\;\;\left\{\begin{array}{rcl}
\mbox{\textsf{text esquerre $E$}} &\to \mbox{\framebox{\parbox{1.25cm}{\textsf{segmen-
      tació}}}} 
\to \\
\mbox{\textsf{text dret $D$}} &\to \mbox{\framebox{\parbox{1.25cm}{\textsf{segmen- tació}}}} 
\to \\
\end{array}
\right.
\mbox{\framebox{\parbox{1.6cm}{\textsf{alineador- corrector assistit}}}}
\to \mbox{\parbox{1cm}{\textsf{UTs} $(e,d)$}} \to \mbox{\framebox{\parbox{1.4cm}{\textsf{Memòria de traducció}}}}
$$
}
\caption{Esquema del procés d'\emph{alineament} d'un bitext existent per a alimentar
  una memòria de traducció.}
\label{fg:aliMT}
\end{figure}



\subsubsection{L'alineament ideal}

Però quin és l'alineament òptim d'un bitext? Quina és la millor manera
de dividir-lo en unitats de traducció?  La longitud de les unitats de
traducció pot anar des dels mots fins a les oracions senceres. La
probabilitat que un fragment esquerre (procedent dels bitextos ja
existents) $e$ aparega en un nou text $E'$ és tant més gran com més
menut és el fragment. Però si el fragment és massa menut és més
probable que la traducció present en la memòria de traducció siga més
imprecisa per ambigua (poden aparéixer correspondències múltiples
entre les quals s'hauria d'elegir: per exemple a una part esquerra $e$
li poden correspondre dues parts dretes $d$ i $d'$ diferents en
unitats de traducció diferents).  D'altra banda, si els fragments són
massa llargs és més improbable que siguen ambigus, però és molt menys probable que es repetisquen exactament en
textos futurs.\footnote{Per
  exemple, el fragment espanyol \emph{las decepciones} es pot
  correspondre en català amb \emph{les decebis} o \emph{les
    decepcions} però el fragment \emph{las decepciones sufridas} només
  pot aparéixer alineat amb \emph{les decepcions patides}.} El
\emph{fragment ideal} seria el que és suficientment menut per a
aparéixer sovint però suficientment complet com per a tenir una
traducció constant. És a dir, per una banda, es dóna un compromís
entre la \emph{cobertura} (fracció d'un text nou que podria ser
traduït usant els fragments alineats) i la \emph{precisió} (correcció
de les traduccions resultants).  La grandària ideal és, per tant, un
compromís entre la cobertura dels fragments menuts i la precisió dels
fragments més grans.

\subsubsection{L'alineament real}

L'alineament automàtic de textos traduïts no és una tasca senzilla;
són necessaris coneixements previs sobre les llengües involucrades
(per exemple, correspondències entre mots o alineaments prèviament
validats per una persona experta\footnote{O obtingut mitjançant
  mètodes estadístics con els descrits en l'apartat~\ref{ss:induc}.});
per això, la majoria dels sistemes de memòria de traducció usen
mecanismes molt senzills per a segmentar els textos, tant per a
alinear bitextos com per a dividir un text esquerre nou en fragments
per a cercar-los en la base de dades. Per tant, els fragments
obtinguts no són en general els \emph{ideals}.  Els mecanismes de
segmentació usuals intenten identificar unitats similars a l'oració
usant la puntuació i el format com a indicadors, amb la idea que el
nombre d'oracions serà bàsicament el mateix en el text dret i
l'esquerre (això pot fallar). Però és que, al contrari del que podria
semblar, no és gens senzill distingir quan un punt representa el final
d'una oració:
\begin{verbatim}
     [...] molt tard.   Després van anar [...] 
     [...] per CC.OO.   Els d'UGT no van [...] 
     [...] per CC.OO.   i per UGT [...]          
     [...] per al Sr.   Martínez [...]             
\end{verbatim}
En els dos primers casos el punt és el final de l'oració però en els
dos últims no. Cal definir clarament les regles de segmentació; els
programes de memòries de traducció solen permetre que la persona
usuària modifique o refine aquestes regles. De fet, hi ha un format
XML estándar per a especificar i intercanviar regles de segmentació,
anomenat SRX\footnote{\url{http://www.unicode.org/uli/pas/srx/srx20.html}} (\emph{segmentation rules exchange}).

Convé, a més, esmentar un altre aspecte addicional important. Els
documents de text, a més de caràcters i paraules, contenen informació
de \emph{format} que cal considerar en el procés de segmentació i
alineament.  Quan es tracta de traduir documents de text, la persona
usuària vol alinear \emph{text} i en molts casos probablement no
necessita veure els codis de format per a validar o corregir un
alineament.\footnote{La situació és diferent quan s'estan traduïnt els
  missatges o textos inclosos en \emph{programes} d'ordinador, com a
  part de les tasques anomenades genèricament de \emph{localització}
  (adaptació de programes d'ordinador als usuaris d'una regió i idioma
  concrets); en aquest cas, pot molt útil en alguns casos veure tot el
  programa a més dels textos.} A més, les unitats de traducció que se
n'obtinguen de l'alineament haurien de ser independents del format del
document. Els programes més recents resolen això amb \emph{filtres} o
\emph{conversors} per als formats de text més freqüents, filtres que
tracten d'ocultar al màxim possible a la persona usuària les
característiques de format dels documents perquè puguen concentrar-se
en les textuals.\footnote{I això que, a voltes, qui tradueix ha de gestionar exolícitament les etiquetes de format per assegurar una bona traducció: tots els programes ofereixen la possibilitat d'\emph{editar etiquetes} manualment: les etiquetes s'agrupen i simplifiquen per fer més fàcil la faena.} Els professionals valoren molt la capacitat de
gestionar el format de manera senzilla i eficient, ja que la
preservació del format de les traduccions és una de les tasques que
els fa perdre més temps.


\subsubsection{Alineament assistit}

És possible que els dos textos no tinguen el mateix nombre
d'``oracions'' o que l'estratègia per a fragmentar-los falle per
alguna causa i l'alineament no siga perfecte. La majoria dels
programes de memòria de traducció ofereixen a la persona usuària la
possibilitat de validar o modificar (usant una representació gràfica,
comandes senzilles i el ratolí, per a unir o dividir fragments en el
text esquerre o en el text dret) l'alineament automàtic inicial usant
una interfície senzilla i intuïtiva, abans d'incorporar els fragments
resultants a la memòria de traducció.


\subsection{La memòria de traducció com a base de dades}

Una vegada alineats els bitextos les unitats de traducció s'organitzen
perquè tant el programa com la persona usuària hi puguen accedir
eficientment; per exemple, com una base de dades. S'ha de tenir en
compte que la utilitat de les memòries de traducció millora
considerablement amb la grandària del corpus de traduccions usades per
a omplir-les; per tant, no és estrany que una memòria de traducció
haja de gestionar una gran quantitat de fragments. Molts programes
marquen els fragments amb un codi que indica la temàtica o la
naturalesa o el nom del document del qual s'ha extret el fragment, de manera
que la temàtica del nou document servisca per a localitzar el fragment
més adequat en cada cas.

\subsection{Pretraducció i traducció interactiva}

L'organització de les UT en una base de dades permet, a més,
\emph{recuperar} de la memòria de traducció, quan s'està traduint un
text nou $E'$, els segments esquerres, $e'_1, e'_2, \ldots$ i
\emph{construir}, partint dels segments drets corresponents, la
traducció desitjada $D'$.

La memòria de traducció pot contenir unitats de traducció amb segments
esquerres idèntics o similars. En cas de trobar un segment idèntic
(\emph{concordància exacta}, en anglés \emph{exact match}) per al qual
només hi haja una traducció disponible, només cal inserir-ne la
traducció directament.


Però, com que això succeix poques vegades, ja que els segments són
normalment oracions i és difícil que es repetisca \emph{exactament},
la major part dels sistemes comercials usen estratègies per a no
desaprofitar unitats de traducció que continguen parts esquerres
\emph{similars} a la nova (les anomenades \emph{concordances
  parcials}, o, en anglés, \emph{fuzzy matches}). 


Alguns programes, si no troben una UT que té la part esquerra idèntica
a l'observada en el nou text ($e'$), però troben una, $(e,d)$, la part
esquerra $e$ de la qual es diferencia en un mot o en un cert
percentatge dels mots de $e'$, presenten com a \emph{traducció
  aproximada} la part dreta ($d$) corresponent a $e$.  Normalment, els
sistemes, quan troben un fragment similar però no idèntic, destaquen
gràficament les diferències (per exemple, amb colors); així, la
persona usuària pot fer-hi les modificacions necessària perquè la
traducció resultant siga correcta. Normalment, se sol establir un
\emph{llindar} (en anglés \emph{fuzzy match threshold}) de manera que
no es presenten propostes per davall d'una certa \emph{puntuació de
  concordança parcial} mínima (en anglés \emph{fuzzy match
  score}). Les puntuacions de concordança parcial se solen expressar
com a percentatges, on 0\% indica falta total de concordança i el
100\% una concordança exacta.


Alguns sistemes, fins i tot, són
capaços d'usar les bases de dades lèxiques o terminològiques de la
persona usuària per a proposar traduccions per als mots discordants.
Per exemple, si a la memòria hi ha la UT \pair{"{}Connecteu
  l'ordinador a la impressora."}{"{}Conecte el ordenador a la
  impresora"} però el nou text conté la frase \texttt{"{}Connecteu
  l'ordinador a la \emph{xarxa}"}, el programa pot trobar les
correspondències 
\pair{"xarxa"}{"red"} i \pair{"{}impressora"}{"{}impresora"}  en una base de dades lèxica i usar-les per a
proposar la traducció correcta (\texttt{"{}Conecte el ordenador a la
  red}).  D'altres programes usen estratègies pròpies (no descrites)
per a ``assemblar'' o ``construir'' traduccions usant fragments de
parts dretes de més d'una UT.  En general, la utilitat d'una memòria
de traducció depén en gran part de la capacitat del sistema per a
proposar traduccions per a segments {\em similars} (i per a això s'han
de definir i usar criteris adequats de \emph{similitud}).

Hi ha dues modalitats d'ús de les memòries de traducció:
\begin{description}
\item[Interactiva:] Qui està traduint rep diverses propostes per a
  cada nou segment $e'$, entre les quals elegeix la més adequada per a
  produir la traducció corresponent $d'$. Aquesta modalitat comporta
  l'accés per part de qui tradueix a la memòria de traducció.
\item[Pretraducció:] Qui està traduint rep com a molt una única
  proposta per a cada nou segment $e'$, elegida automàticament. En
  aquesta modalitat, qui tradueix no té accés a la memòria de
  traducció.
\end{description}
\begin{figure}
  \begin{center}
    \begin{tabular}{p{3cm}p{4cm}}
           $E'$ & $D'$ en construcció      
    \end{tabular}
    \begin{tabular}{|p{3cm}|p{4cm}|}
\hline
        $e'_1=e_{234}$ (100\%)& $d'_1=d_{234}$ \\\hline
        $e'_2\simeq e_{112}$ (87\%) & posteditar $d_{112} \to d'_2$ \\\hline
        $e'_3\simeq e_{47}$ (93\%) & posteditar $d_{47} \to d'_3$ \\\hline
        $e'_4$ & generar $d_4$ desde zero \\\hline
        $e'_5=e_{51}$ (100\%) & $d'_5=d_{51}$ \\\hline
        \ldots & \ldots \\\hline
    \end{tabular}
  \end{center}
  \caption{Traducció d'un nou bitext $E'$ i els tipus de situacions de concordança que s'hi poden donar: concordança exacta per a $e'_1$ i $e'_5$, concordança parcial per a $e'_2$ i $e'_3$, i absència de concordança raonable per a $e'_4$.}
  \label{fg:pretrad}
\end{figure}
Tant en un cas com en l'altre, poden passar tres coses, tal com es veu en l'exemple de la figura~\ref{fg:pretrad}
\begin{itemize}
\item Que es done una concordança exacta, com en el cas dels segments $e'_1$ i
  $e'_5$ i per tant, només calga comprovar que la proposta de la
  memòria es pot usar com la seua traducció.
\item Que es done una concordança parcial, com en el cas dels segments
  $e'_2$ (similar a $e_{112}$) i $e'_3$ (similar a $e_{47}$), i calga
  posteditar, respectivament, les propostes $d_{112}$ i $d_{47}$ de la
  memòria de traducció, per a produir $d'_2$ i $d'_3$.
\item Que no es done cap concordança raonable, com en el cas del segment $e'_4$, i calga generar la traducció $d'_4$ des de zero.
\end{itemize}

Tant en el cas interactiu com en el de la pretraducció, qui tradueix ha d'acabar la traducció. El
procés i es mostra en la figura~\ref{fg:preMT}.
\begin{figure}
{\small
$$
\begin{array}{rcl}
\mbox{\textsf{text esquerre}} \to
\mbox{\framebox{\textsf{segmentació}}} \to &
\mbox{\framebox{\parbox{1.9cm}{\textsf{concordança i selecció de propostes}}}} & \to \mbox{\framebox{\textsf{postedició}}} \to \mbox{\parbox{1.4cm}{\textsf{text dret
    traduït i segmentat}}} \\[0.8cm]
& \uparrow \downarrow \mbox{\textsf{UTs}}& \\[0.5cm]
& \mbox{\framebox{\parbox{1.9cm}{\textsf{Memòria de traducció}}}} & \\
\end{array}
$$
}
\caption{Esquema del procés de traducció d'un nou text esquerre usant
  una memòria de traducció.}
\label{fg:preMT}
\end{figure}



\subsection{Ampliació de la memòria}

Una vegada feta la pretraducció o la selecció interactiva de propostes
de traducció para cada un dels segments del text esquerre $E'$, el
programa les mostra alineades amb els segments del text esquerre
perquè la persona usuària les puga posteditar i produir el text dret
correcte $D'$. Les noves unitats de traducció $(e',d')$ que s'hagen
produït en el procés es poden enviar a la memòria de traducció.
%El procés
%s'esquematitza en la figura~\ref{fg:corMT}; <---- ELIMINADA
La repetició del cicle pretraducció--correcció--enviament de noves UT
a la memòria enriqueix la memòria de traducció i permet obtenir cada
vegada traduir amb menys esforç, ja que la cobertura de la memòria
augmenta.
% FIGURA ELIMINADA POR CONFUSA
% \begin{figure}
% $$
% \begin{array}{r}
% \mbox{\textsf{text esquerre segmentat}} \to \\
% \mbox{\textsf{text dret pretraduït i segmentat}} \to \\ \end{array}
% \mbox{\framebox{\parbox{1.6cm}{\textsf{alineador- corrector assistit}}}}
% \begin{array}{l}\to \mbox{\textsf{UTs}} \to
%   \mbox{\framebox{\parbox{2.1cm}{\textsf{Memòria de traducció}}}} \\
%   \to \mbox{\textsf{text dret corregit}}\end{array}
% $$
% \caption{Esquema del procés de \emph{correcció} del text pretraduït i
%   \emph{actualització} de la memòria de traducció.}
% \label{fg:corMT}


% \end{figure}


% TEXT que n'he eliminat...
%  També es pot donar el cas que una frase nova es
% puga traduir de més d'una manera perquè hi haja més d'una combinació
% de fragments pretraduïts que la cobrisca; en aquest cas, es pot usar
% un sistema d'avaluació (per exemple, estadístic) per a elegir la
% millor fragmentació o donar a la persona usuària la possibilitat de
% triar entre les possibles fragmentacions.


\section{Productes}

En l'actualitat, hi ha molts programes de traducció assistida per
ordinador basats en memòries de traducció disponibles comercialment
(\emph{SDL Trados} de SDL International, \emph{Déjà Vu} d'Atril,
\emph{IBM Translation Manager}, \emph{Transit} de Star, etc., per
nomenar algunes de les més conegudes). També hi ha una alternativa
interessant de codi obert i distribució gratuïta, anomenada
OmegaT.\footnote{\url{http://www.omegat.org}} Aquests programes de
traducció assistida són molt populars entre les persones que es
dediquen professionalment a la traducció, molt més populars que els
programes de traducció automàtica. Això pot ser, d'una banda, perquè
els professionals tenen així la impressió que el programa no tradueix
i els relega a un mer paper de correctors, sinó que \emph{organitza} i
fa més eficient el treball de traducció professional i, d'altra banda,
perquè les memòries de traducció \emph{conserven l'estil} i \emph{les
  decisions terminològiques} de traduccions anteriors, que poden
variar d'un equip a altre, mentres que els sistemes de traducció
automàtica solen basar-se en seleccions terminològiques i d'estil de
propòsit general, encara que siga dins d'una temàtica concreta.

S'ha de tenir en compte que la traducció automàtica pot constituir una
alternativa acceptable en aquells segments on la memòria de traducció
no pot fer una proposta raonable, i, de fet, molts programes de
traducció assistida per ordinador combinen l'accés a memòries de
traducció i l'accés a la traducció automàtica. El punt en el que les
propostes de la memòria de traducció començen a ser menys útils que la
traducció automàtica depén de la cobertura de la memòria de traducció
per al tipus de text, de la llengua origen i la llengua meta, i, per
descomptat, del sistema concret de traducció automàtica: per a
llengües molt similars com ara l'espanyol i el català, podria passar
que només les concordançes millors que, diguem-ne, el 95\% foren
acceptables, mentre que si les llengües són més distants, com
l'espanyol i el turc, podria passar que la traducció automàtica només
fóra útil per a puntuacions de concordança molt baixes.

Hi ha tecnologies de traducció automàtica que s'assemblen molt
al funcionament de les memòries de traducció, com ara la traducció
automàtica estadística~\ref{ss:induc}. Un exemple clàssic és l'edició
bilingüe (espanyol--català) d'\emph{El Periódico de Catalunya} (vegeu
l'epígraf~\ref{ss:ePdC}), que es prepara diàriament amb un mètode
completament automàtic que funciona en molts aspectes de manera
similar a una memòria de traducció.

% no són completament automàtiques, però no és imposible una
%automatització total del procés, especialment quan es disposa d'una
%gran quantitat de documents pretraduïts.  


\section{L'intercanvi de memòries de traducció}

\subsection{El format d'intercanvi TMX}


Freqüentment, els traductors formen equips que col·laboren a l'hora de
produir les traduccions; quan usen memòries de traducció, és possible
que hi haja traductors que preferisquen un programa i uns altres que
en preferisquen un altre. Vol dir això que no podran compartir les
memòries de traducció que hagen anat construint? Per sort, no. En
agost de 1998 es va aprovar la versió 1.1 d'un format estàndard
anomenat TMX (\emph{Translation Memory eXchange}, ``intercanvi de
memòries de traducció''); quasi tots els programes gestors de memòries
de traducció poden escriure i llegir memòries en aquest format.  El
format TMX segueix les especificacions XML (vegeu
l'apartat~\ref{s3:SGML}); és a dir, les memòries TMX són un tipus de
document XML, definit, per tant, per una DTD concreta.\footnote{\url{http://www.ttt.org/oscarstandards/tmx/tmx14b.html}}

Heus ací part d'una memòria de traducció en TMX (s'hi mostren només
\emph{dues} UTs):
\begin{verbatim}
<?xml version='1.0' encoding='ISO-8859-1' ?>
<!DOCTYPE tmx SYSTEM 'tmx13.dtd'>
<tmx version='1.3'>
 <header
  creationtool='Waikoloa'
  creationtoolversion='1.00'
  datatype='plaintext'
  segtype='paragraph'
  adminlang='EN-US'
  srclang='EN-US'
  o-tmf='okLiteTM'
 >
 </header>
 <body>
 <!-- ... -->
   <tu tuid='511'>
   <prop type='tmkey'>a thesaurus error occurred word 
   is ending the current session</prop>
   <tuv xml:lang='EN-US'> 
    <seg>A thesaurus error occurred. Word is ending 
    the current session.</seg>
   </tuv>
   <tuv xml:lang='FR-FR'> 
    <seg>Une erreur s'est produite pendant l'exécution 
    du dictionnaire des synonymes. Word met fin à la 
    session en cours.</seg>
   </tuv>
  </tu>
  <tu tuid='512'>
   <prop type='tmkey'>a thumbnail preview is not 
    available for this file</prop>
   <tuv xml:lang='EN-US'>
    <seg>A thumbnail preview is not available 
         for this file.</seg>
   </tuv>
   <tuv xml:lang='FR-FR'>
    <seg>Il n'y a pas d'aperçu disponible 
         pour cette image.</seg>
   </tuv>
  </tu>
  <!-- ... -->
</body>
</tmx>
\end{verbatim}
En l'exemple, les unitats de traducció (\texttt{tu}) del cos
(\texttt{body}), cada una amb el seu identificador únic
(\texttt{tuid}) contenen dues variants (\texttt{tuv}), cada una en una
llengua (\texttt{xml:lang="}\ldots\texttt{"}), a més d'un element
\texttt{prop} que conté la clau (\texttt{key}) per a cercar la UT en
la base de dades. Abans del cos, l'element arrel \texttt{tmx} conté
una capçalera (\texttt{header}) amb informació sobre la creació i les
característiques de la memòria.

\subsection{Altres problemes}

Però, fins i tot quan ja s'ha resolt aquest problema tècnic,
l'intercanvi de memòries de traducció entre traductors o equips de
traducció diferents no està exempt de problemes. D'una banda, es poden
produir incoherències terminològiques i d'estil entre els fragments
procedents de grups diferents; les decisions en cas de conflicte
comporten mecanismes complexos de reconeixement d'autoritat o de
prestigi, que poden ser difícils de consensuar. D'altra banda,
l'organització, el manteniment i la explotació de grans memòries de
traducció distribuïdes (en les diverses màquines d'una xarxa) està
lluny de ser trivial. Per exemple, en el cas de l'espanyol i el català,
una gran memòria de traducció alimentada amb les traduccions fetes
només en l'àmbit de les administracions autonòmiques i locals
estalviaria grans quantitats de temps i diners a l'hora de mantenir la
documentació bilingüe d'aquestes institucions, però encara no s'ha
substanciat un recurs d'aquesta mena, malgrat la gran quantitat de
veus que n'han expressat la necessitat i la conveniència.


\section{Qüestions i exercicis}

\begin{enumerate}
\item Indica quina d'aquestes afirmacions és certa.
\begin{enumerate}
\item Les memòries de traducció són bàsicament sistemes de
     traducció directa i, per tant, la unitat bàsica de traducció que
     usen és el mot.
\item Les memòries de traducció usen informació sobre les
     categories lèxiques dels mots per a decidir els alineaments.
\item Per a traduir un text nou amb una memòria de traducció
     és necessari que hi haja textos
     originals i traduïts que hagen estat alineats.
\end{enumerate}

\item Una memòria de traducció es pot veure com una base de dades 
   
\begin{enumerate}
\item on cada registre és una llengua i cada camp una unitat de
      traducció
\item on cada registre és una oració i cada camp un mot
\item on cada registre és una unitat de traducció i la variant en
      cada llengua es guarda en un camp diferent
\end{enumerate}

\item Quina característica dels textos que s'han de traduir fa que
   l'ús de memòries de traducció siga la solució adequada?
   
\begin{enumerate}
\item El fet que els textos estiguen escrits amb un lèxic
      monosèmic, és a dir, precís i no gens ambigu.
\item La repetitivitat
\item La similitud entre les llengües origen i meta
\end{enumerate}

\item Quant al funcionament, 
a quins sistemes de traducció automàtica s'assemblen més les
memòries de traducció?
\begin{enumerate}
\item Als sistemes de traducció automàtica directes o mot per mot.
\item Als sistemes de traducció automàtica per interlingua.
\item Als sistemes de traducció automàtica per transferència.
\end{enumerate}
\item Les memòries de traducció comercials actuals segmenten i alineen els textos
{\ldots}
\begin{enumerate}
\item {\ldots}fent un balanç òptim entre cobertura i precisió.
\item {\ldots}usant la informació sintàctica com a pista.
\item {\ldots}usant regles que analitzen la puntuació i el format.
\end{enumerate}

\item Indica quina d'aquestes afirmacions és falsa.
\begin{enumerate}
\item Els resultats produïts per una memòria de traducció no
     necessiten revisió, ja que es basen en traduccions correctes realizades
     anteriorment per professionals.
   \item Les memòries de traducció poden organitzar els fragments i
     les traduccions corresponents en bases de dades similars a les
     bases de dades lèxiques o terminològiques.
\item Una memòria de traducció és un sistema de traducció
     directa amb equivalències entre segments de text 
     observades en textos anteriorment traduïts.
\end{enumerate}

\item Les memòries de traducció usen els signes de puntuació per a 
     segmentar les oracions abans d'alinear els textos. En particular,
     l'aparició d'un punt (``\texttt{.}'') és moltes vegades un bon  
     indicador del final d'una oració, però no sempre. Quan no? Doneu
     almenys \emph{quatre} excepcions diferents a aquesta regla i   
     descrigueu, en cada cas, una regla senzilla que permeta decidir
     amb seguretat raonable quan ens trobem en cada una d'aquestes
     excepcions, usant el mínim possible d'anàlisi lingüística del
     text.

\item La majoria de les memòries de traducció comercials divideixen
      els bitextos en unitats de traducció{\ldots}
   
\begin{enumerate}
\item {\ldots} aproximadament equivalents a una oració, usant
      regles senzilles i relativament independents de la llengua per a
      segmentar (dividir) cada text en oracions.
\item {\ldots} en mots i petites unitats multimot (entre dos i
      quatre mots) de gran repetitivitat.
\item {\ldots} equivalents a una oració, usant una anàlisi
      lingüística detallada del
      text per a determinar l'extensió de cada oració.
\end{enumerate}
\item Que hem de fer amb els bitextos existents per a poder reutilitzar la
   informació que contenen per a fer noves traduccions amb una memòria
   de traducció?
   
\begin{enumerate}
\item Passar-los a XML.
\item Segmentar cada un dels dos textos en oracions.
\item Segmentar els dos textos i alinear-los.
\end{enumerate}

\item Estem traduint un segment en un programa de traducció assistida
  (com ara OmegaT) i el programa ens dóna una sèrie de propostes que
  vénen de la memòria de traducció. Quin d'aquests indicadors indica
  millor l'esforç que haurem de fer per acabar de traduir el segment?
  \begin{enumerate}
\item El percentatge de concordança parcial de la millor
    proposta
\item El nombre de propostes
\item La longitud en
    paraules de la millor proposta (com més llarga, millor)
  \end{enumerate}


\item La majoria dels programes de traducció assistida basats en
  memòries de traducció no donen una de les tres informacions
  següents: 
  \begin{enumerate}
  \item El percentatge de coincidència entre el nou
    segment a traduir i el segment origen de la unitat de traducció
    proposada. 
  \item Els mots que cal canviar en el segment
    meta de la unitat de traducció proposada.
  \item Els mots del
    segment origen de la unitat de traducció proposada que es
    diferencien de les del nou segment a traduir.
  \end{enumerate}


\item Indiqueu quina d'aquestes afirmacions és certa. 
  \begin{enumerate}
  \item Les
    memòries de traducció són bàsicament sistemes de traducció directa
    i, per tant, la unitat bàsica de traducció que usen és el mot.
  \item Les memòries de traducció usen informació sobre les
    categories lèxiques dels mots per a decidir els alineaments.
  \item Per a no haver de traduir un text nou des de zero amb un
    sistema de ajuda a la traducció basat en memòries de traducció, és
    necessari que hi haja textos originals i traduïts que hagen estat
    alineats.
  \end{enumerate}

\item Quina d'aquestes característiques d'un treball de traducció el fa més tractable amb memòries de traducció? 
  \begin{enumerate}
  \item La repetitivitat dels textos.
  \item La proximitat entre les llengües origen i meta.
  \item Que els textos origen i meta estiguen codificats en ISO-8859-1 (\emph{Latin-1}).
  \end{enumerate}



% < QT10

\end{enumerate}

\section{Solucions}
\begin{enumerate}
\item (c)
\item (c)
\item (b)
\item (a)
\item (c)
\item (a)
\item El punt apareix en moltes construccions que no indiquen el final de
   una oració. Heus ací uns exemples i com detectar-los per a no
   segmentar.

   \begin{enumerate}
   \item Punts de milers (1.259), milions (1.032.200), decimals saxons
     (3.4), números de telèfon francesos (01.10.23.87.49), dates
     (20.01.1999), etc. \emph{Solució:} 
     Si detectem [xifra] ``.'' [xifra] (sense blancs), no
     segmentem.
     
    \item Punts en mig de sigles: CC(.)OO., EE(.)UU.,
      etc. \emph{Solució:} 
      si detectem [majúscula] ``.'' [majúscula] (sense blancs),
     no segmentem.
     
    \item Punts al final d'abreviatura de cortesia: Sr(.), Dra(.),
     Excm(.), etc. o d'altres que precedeixen noms propis (Avda.,
     Pça.) o números (Tel.). \emph{Estratègia de solució:}
     si detectem ``.'' [blanc] [minúscula], no segmentem;  si detectem
     ``.'' [blanc] [número], no segmentem;    
     si detectem ``.'' [blanc] [majúscula], què fem?: depén de
     l'abreviatura (la decisió és impracticable sense vocabularis de
     noms propis).
  
     \item Punts al final de sigles: CC.OO(.), O.N.U(.). \emph{Solució:}
      si detectem [majúscules] ``.'' [majúscules] ``.'', no segmentar.
     
     \item Punts en URIs i adreces de correu electrònic.
       \emph{Solució:} la millor seria una regla general (patró o expressió
     regular) que detectara aquestes entitats i evitara segmentar-les.
     Possibles escapatòries:  
     si detectem [minúscula] ``.'' [minúscula] (sense blancs), no
     segmentem.
     
     \item Punts suspensius (``...''). \emph{Solució:}
      no segmentar mai ``...'' si es troba.
     
 \end{enumerate}

\item (a)
\item (c)
\item (a)
\item (b)
\item (c)
\item (a)

% < SL10

\end{enumerate}

