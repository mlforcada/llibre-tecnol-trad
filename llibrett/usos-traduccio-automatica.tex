\chapter[Traducció automàtica i aplicacions]{La traducció automàtica i les seues aplicacions}
\label{se:UTA} \label{se:TiTA} % dos etiquetes per si cal agarrar-ho tot.

Una de les aplicacions més importants de la informàtica a la traducció
és la \emph{traducció automàtica} (TA). Però abans de considerar
l'automatització de la traducció i les seues aplicacions fóra bo que
ens paràrem un poc per a discutir què vol dir exactament el mot
\emph{traducció}. Més endavant, en la secció~\ref{ss:humaut},
discutirem sobre la relació entre traducció humana i traducció
automàtica.

\section{Què és la traducció?}
\label{ss:trad}

Per començar, s'ha de tenir en compte que el mot \emph{traducció} és
ambigu\footnote{Com molts altres substantius acabats en \emph{-ció}}
perquè es pot referir al \emph{procés} de traduir o al \emph{producte}
(resultat) d'aquest procés.

\citet{sager93b}\footnote{Els conceptes d'aquesta secció estan presos
  quasi íntegrament d'aquesta obra.} comença la seua definició dient
que, com a procés, es pot anomenar traducció ``un rang d'activitats
humanes deliberades, que es fan com a resultat d'instruccions rebudes
d'un tercer, i que consisteixen en la producció de textos en una
llengua meta (LM), basada, entre altres coses, en la modificació d'un
text en una llengua origen (LO) per a fer-lo adequat a un propòsit
nou'', però encara no descriu la naturalesa de la modificació.

Com a producte, una \emph{traducció} es pot identificar com a tal
perquè és un document (en LM) derivat d'un altre document en un altre
idioma (LO), i que manté una certa similitud de contingut amb aquest.

Es poden dir encara més coses sobre la traducció:
\begin{itemize}
\item les traduccions solen estar escrites en un subllenguatge
  particular (registre, especialitat, etc.) de la comunitat
  lingüística de la LM, basat en un subllenguatge paral·lel de la LO;
\item els documents i les traduccions corresponents es poden
  classificar en tipus\footnote{Per exemple, \emph{carta comercial},
    \emph{edicte municipal}, \emph{comentari editorial}, \emph{manual
      tècnic informàtic} o \emph{recull de poemes}. } i aquesta
  tipologia afecta la traducció;
\item la traducció es veu afectada per elements extralingüístics
  perquè, normalment, els documents són entitats que uneixen
  l'expressió lingüística amb l'expressió no lingüística;
\item les traduccions tenen un \emph{receptor} o \emph{lector}; una
  traducció, com a acte comunicatiu ha de considerar, a més de la
  intenció de la traducció, les expectatives dels lectors, que
  resulten del seu rerefons cultural i de les seues necessitats
  comunicatives, i que influeixen en la recepció del text traduït;
\item la traducció sempre té una motivació: la superació de barreres
  comunicatives; per això se n'ha creat una professió.
\end{itemize}

Es pot aprofundir un poc més en la definició de traducció que hem
considerat més amunt, revisant definicions existents (algunes preses
de \citealt{sager93b}):
\begin{itemize}
\item \citet[p.~19]{nida59b}: ``La traducció consisteix a produir en
  la llengua meta l'equivalent natural més proper del missatge en la
  llengua origen, primerament quant al significat i segonament quant a
  l'estil.'' Sager diu que, més que \emph{natural} (en sentit absolut)
  caldria dir \emph{adequat} (a la tasca concreta). Aquesta definició
  introdueix dues de les tres dimensions bàsiques d'un document escrit
  (original o traduït): el \emph{contingut} (significat) i la
  \emph{forma} (estil), però oblida el \emph{propòsit}.
\item \citet{flamand83b}: traduir és representar amb precisió
  (fidelitat a l'autor) un missatge en LO en una forma autèntica i
  correcta de la LM, adaptada al contingut i al receptor (fidelitat al
  lector)''. El problema d'aquesta definició és la indefinició del
  concepte de \emph{fidelitat}.
\item \citet{jakobson66b}: ``Traducció és la interpretació de signes
  verbals per mitjà d'una altra llengua''. Aquesta definició evita el
  concepte d'\emph{equi\-va\-lèn\-cia} i introdueix el
  d'\emph{interpretació} com a conjunt de processos cognitius que
  tenen lloc en la ment del traductor.
\item En el \emph{Diccionari de la Llengua
    Catalana}\footnote{Editorial Enciclopèdia Catalana, 7a ed., 1987-}
  es defineix {\em traducció} com la ``reproducció del contingut d'un
  text o d'un enunciat oral, formulat en una llengua, en formes
  pròpies d'una altra llengua'' (i \emph{traduir} com ``escriure o dir
  en una llengua allò que ha estat escrit o dit en una altra''). La
  definició inclou, per tant, el tractament i la producció de
  missatges no textuals (orals).
\item \citet{alcaraz97b} defineixen traducció com ``expresión de un
  enunciado en la lengua de llegada [lengua meta] que sea equivalente
  al de la lengua de partida [lengua origen]''; queda per definir la
  noció d'\emph{equivalència}, que els mateixos autors defineixen
  així: ``la posesión del mismo valor por parte de los enunciados de
  la lengua de partida y de la de llegada''; l'equivalència pot ser
  {\em semàntica}, \emph{estilística} i \emph{textual}.
\end{itemize}

Per a acabar aquest apartat, convé esmentar alguns processos que no
s'anomenaran \emph{traducció} en el context d'aquest llibre:
\begin{itemize}
\item l'adaptació de textos antics a la forma moderna d'un idioma;
\item la traducció de mots i frases quan s'ensenya un nou idioma;
\item la interpretació (de missatges parlats);
\item la codificació (en Morse, etc.).
\end{itemize}

\section{Traducció automàtica}
\label{ss:TA}

%\subsection{Definició}

La traducció automàtica (TA) es pot definir com el procés (o el
producte) de traduir un text informatitzat\footnote{Anomenarem
  \emph{text informatitzat} un fitxer que conté un text codificat en
  un format conegut (vegeu el capítol~\ref{se:EPT}), i que pot ser
  editat amb un editor o amb un processador de textos adequat.} en una
llengua origen a un text informatitzat en una llengua meta mitjançant
l'ús d'un programa d'ordinador. Normalment es reserva la denominació
\emph{traducció automàtica} per a la completament automàtica; quan
s'hi produeix intervenció humana es parla de {\em traducció assistida
  per l'ordinador} o de \emph{traducció semi-automàtica}.  El resultat
de la traducció automàtica és normalment un producte bastant diferent
a la traducció professional, i en la majoria dels casos no es pot usar
en el seu lloc tal com està; per això, el capítol~\ref{se:UTA} està
dedicat a analitzar les diverses modalitats d'interacció entre
persones i màquines en traducció.

Un aclariment és necessari sobre el tractament dels textos
informatitzats. Quan els programes de traducció automàtica i
semiautomàtica han de tractar documents estructurats (com els
discutits en els epígrafs~\ref{s3:SGML} i \ref{s3:RTF}) han de ser
capaços d'identificar les parts dels documents que corresponen als
textos que s'han de traduir, destriant-les de les etiquetes.
Normalment, els programes tenen un mòdul inicial que podríem anomenar
\emph{desformatador} i un mòdul final que podríem anomenar
\emph{reformatador} i que restitueix les etiquetes de manera que el
format i l'estructura del document es conserven tant com siga
possible. En general, aquestes operacions es poden considerar
bàsicament independents del procés de traducció mateix ---com farem en
aquest llibre---, però hi ha programes més avançats que són fins i tot
capaços d'usar la informació de les etiquetes com a context per a
elegir una traducció on hi ha més d'una alternativa.

Les referències que s'han fet en l'epígraf~\ref{ss:trad} al propòsit o
motivació de la traducció i a la tipologia dels documents que han de
ser traduïts són també molt importants a l'hora d'analitzar la
traducció automàtica.


\paragraph{Sobre el nom en altres llengües.}
En anglés, la traducció automàtica s'anomena \emph{machine
  translation} i s'abreuja MT, paral·lelament a l'alemany, que usa la
denominació \emph{maschinelle übersetzung}; en aquestes dues llengües
s'expressa la noció d'automatisme mitjançant la referència a una {\em
  màquina}. En canvi, en francés es parla, com en català o en
espanyol, de {\em traduction automatique}. Altres llengües, com ara el
neerlandés, usen un mot compost amb la paraula \emph{ordinador}:
\emph{Computervertaling}.

\begin{persabermes}{la història de la traducció automàtica}
  La major part del discutit d'aquest apartat està pres dels treballs
  de John Hutchins, especialment de \cite{hutchins1995} i
  \cite{hutchins2001}. El Dr.\ John Hutchins és considerat
  l'historiador de la traducció automàtica, i fins fa poc mantenia
  activament l'arxiu \url{www.mt-archive.info}, on es poden trobar
  reproduccions facsímils de molts articles dels inicis d'aquesta
  disciplina.

  \paragraph{Els pioners, fins 1954:} La traducció mitjançant màquines
  és una ambició humana des de fa segles que no es va fer realitat
  fins al segle XX. No feia molt que s'havia creat el primer
  ordinador, quan ja es va començar a pensar en la possibilitat
  d'usar-los per a traduir llenguatges humans.

  Tot i que en els decennis dels 1930 i 1940 hi va haver alguns
  treballs precursors, és als primers cinquanta quan comença realment
  la recerca en TA en moltes universitats arreu del món, especialment
  als Estats Units. Els recursos de maquinari, programari i
  llenguatges de programació eren massa reduïts i la primera
  aproximació va ser la traducció mot per mot basada en diccionari amb
  algunes regles senzilles de reordenament (de vegades erròniament
  anomenada \emph{traducció directa}, i similar als sistemes de
  traducció indirecta per transferència morfològica avançada, vegeu
  l'apartat~\ref{s3:STMorf}). Aquesta manca de recursos va fer que els
  primers objectius foren molt modests i, així, els primers
  investigadors van concentrar-se en el desenvolupament de llenguatges
  controlats (vegeu~\ref{ss:llecon}) i en l'ajuda humana en tasques de
  preedició i postedició (vegeu~\ref{ss:preedposted}); era prou clar
  que els sistemes reals no podrien produir més que traduccions de
  molt baixa qualitat.  El 1952 es va celebrar als Estats Units el
  primer congrés sobre TA on es van definir les línies fonamentals a
  seguir.

  \paragraph{El decenni de l'optimisme, 1954--1966:}
  La primera demostració pública d'un sistema de TA va ser
  desenvolupada per IBM i la Universitat Georgetown el 1954.  Es va
  traduir a l'anglés un conjunt de 49 frases en rus usant un
  diccionari de només 250 mots i 6 regles gramaticals; aquestes
  llengües van ser les elegides per raons geopolítiques per als
  primers sistemes de TA.  Tot i que els resultats no eren massa bons,
  el públic i la indústria van creure que en uns anys es podrien
  aconseguir traduccions automàtiques de qualitat de documents
  científics i tècnics.  Aquesta idea es va reforçar pel fet que van
  començar a aparèixer millores significatives en el maquinari, els
  primers llenguatges de programació i moltes millores en la
  lingüística formal (especialment en l'àrea de la
  sintaxi). L'entusiasme va fer que es finançaren un gran nombre de
  projectes entre la meitat dels 50 i la meitat dels 60, projectes
  dins els quals van nàixer la major part de les tècniques actuals,
  com ara la traducció indirecta per transferència o la traducció per
  interlingua (vegeu el capítol~\ref{se:TdTA}).

  L'objectiu era el desenvolupament de sistemes perfectes. Calia
  reduir al mínim la intervenció humana en el procés de TA, fins
  assolir la independència total i una qualitat comparable a la dels
  humans. Pràcticament ningú va considerar com es podria traure profit
  d'un sistema imperfecte ---amb excepcions comptades:
  \cite{masterman67b} va estudiar la utilitat de la traducció
  \emph{mot per mot} com a \emph{pidgin}, és a dir, com a llengua de
  contacte, en comparació amb una traducció \emph{nativa}. Per què
  pensar-hi si molt prompte es disposaria de sistemes perfectes? Els
  traductors es van sentir amenaçats. No obstant això, algunes veus es
  pronunciaren en contra del perfeccionisme dominant i defensaren una
  aproximació més a llarg termini al problema i la construcció de
  sistemes que feren un ús efectiu de la interacció persona--màquina.

  Un decenni després, i com que les expectatives eren tan altes, els
  avanços eren escassos i el futur pròxim no semblava poder millorar
  la situació.  Molts investigadors començaven a trobar barreres de
  tot tipus, especialment semàntiques, que semblaven massa difícils de
  superar i que exigien mètodes més complexos. La Acadèmia Nacional de
  les Ciències dels Estats Units va publicar el 1966 l'informe ALPAC
  (Automatic Language Processing Advisory Committee) en el qual es
  recomanava que els nombrosos recursos que es dedicaven a la recerca
  en TA s'utilitzaren per a tasques menys ambicioses i més bàsiques
  relacionades amb el processament del llenguatge natural i amb el
  desenvolupament d'eines de suport per als traductors com ara
  diccionaris automàtics.  La conclusió era que només després de
  conéixer les arrels del problema, podria estudiar-se la realització
  d'un sistema de TA real. L'informe assegurava que la TA era més
  lenta i menys exacta que la feta pels humans, a més de ser el doble
  de cara, i que no hi havia cap indici de l'obtenció en el futur més
  o menys inmediat d'un sistema de TA útil.  L'informe va fer que es
  reduira significativament el nombre de persones que es dedicaven a
  la TA i que els laboratoris començaren a treballar en el que es va
  conéixer com a lingüística computacional.

  \paragraph{Des de l'informe ALPAC (1966) fins als vuitanta}:
  L'informe va acabar quasi virtualment amb la recerca en TA als
  Estats Units (també va tenir un impacte negatiu en els projectes
  desenvolupats a la resta del món) i durant molts anys la TA va ser
  percebuda com un autèntic fracàs. Tot i això, alguns grups van
  continuar treballant a Canadà i a Europa i van aparéixer els primers
  sistemes que funcionaven; el 1970 el sistema Systran va començar a
  ser usat per la USAF (United States Air Force) i el 1976 per la
  Comissió de la Comunitat Europea.  També el 1976 apareix Metéo,
  desenvolupat per la Universitat de Montréal, que tradueix al francés
  els informes meteorològics.  Per aquesta època, a més, els sistemes
  de TA comencen a ser demanats per empreses i administracions i no
  sols per traduir textos científics i tècnics.

  Des de l'informe ALPAC el camp va patir una redefinició progressiva
  vers una concepció de la TA com un procés en el qual els traductors
  humans juguen un paper bàsic, i comencen a desenvolupar-se eines de
  traducció pensant en aquesta intervenció.

  Els principals corrents dins la TA des dels 70 són, per tant: eines
  de suport a la traducció per a traductors, sistemes de TA amb
  intervenció humana i recerca teòrica vers un sistema completament
  automàtic de traducció.

  \paragraph{Els primers vuitanta:} Als 1980 apareixen nous sistemes
  de TA arreu del món amb expectatives més reals i l'interés en la TA
  resorgeix.  Són especialment importants els resultats obtinguts a
  diverses empreses com Xerox on s'elimina quasi completament la
  postedició (vegeu la p.~\pageref{pg:homografia}) gràcies al control
  de la llengua origen (vegeu l'apartat~\ref{ss:llecon}); això permet
  la traducció senzilla dels manuals tècnics en anglés de la companyia
  a un gran nombre d'idiomes (francés, alemany, italià, espanyol,
  portugués i llengües escandinaves).

  Durant aquest decenni els esforços es dirigeixen vers la traducció
  indirecta amb representacions intermèdies o sense (com la
  interlingua; vegeu l'apartat~\ref{ss:interlingua}) mitjançant
  anàlisis morfològiques i sintàctiques i, de vegades, coneixements
  semàntics.  Els projectes més notables són GETA-Ariane (Grenoble),
  SUSY (Saarbrücken), Mu (Kyoto), DLT (Utrecht), Rosetta (Eindhoven),
  el projecte de la Universitat Carnegie-Mellon (Pittsburgh) i dos
  projectes internacionals: Eurotra, finançat per la Comunitat Europea
  i el projecte japonés CICC amb participants a la Xina, Indonèsia i
  Tailàndia.

  Eurotra és un dels projectes de traducció més coneguts del decenni
  dels 80. El seu objectiu era la construcció d'un sistema de
  transferència multilingüe que permetera la traducció entre totes les
  llengües de la Comunitat Europea.  Tot i que s'esperava que la
  traducció resultant seria de gran qualitat, encara es preveia una
  gran quantitat de postedició. El projecte, que va ser abandonat el
  1992, no va ser capaç d'entregar un sistema de TA funcional, però va
  estimular la investigació sobre tecnologies lingüístiques a tot
  Europa.

  En aquests anys es consolida la idea que els sistemes de TA no són
  per a traductors; un traductor necessita eines que li faciliten el
  treball: diccionaris, bases de dades terminològiques (vegeu el
  capítol~\ref{se:basesdades}), sistemes de comunicació, memòries de
  traducció (vegeu el capítol~\ref{se:memtrad}), etc.  De fet,
  actualment, la postedició no s'encarrega sempre a traductors (molts
  dels quals no no consideren açò com a part del seu treball), sinó a
  persones que es presumeix preparades específicament.

  Tothom accepta ja en aquest decenni la importància dels llenguatges
  controlats i els subllenguatges en la TA, com ja havien defensat els
  precursors de la TA durant el decenni dels cinquanta.

  El sistema comercial més sofisticat dels 1980 és Metal (1988),
  finançat per Siemens i que tradueix de l'alemany a l'anglés.  Es
  tracta bàsicament d'un sistema per transferència, indicat per a la
  traducció de documents relacionats amb el processament de dades i
  les telecomunicacions.

  Al final dels 1980 comença l'aplicació de tècniques d'intel·ligència
  artificial al processament del llenguatge humà (sistemes experts i
  sistemes basats en el coneixement dissenyats per entendre els
  textos).

  \paragraph{Els primers noranta:} Tots els sistemes de TA dels
  vuitanta, tant els de transferència com els d'interlingua, funcionen
  bàsicament a partir de regles lingüístiques. Als 1990, però,
  apareixen noves estratègies conegudes com a mètodes basats en
  corpus. Els mètodes basats en corpus es poden dividir en dos grups:
  estadístics i basats en exemples.

  Els mètodes estadístics ja van ser considerats als anys seixanta,
  però prompte van ser descartats perquè els resultats obtinguts no
  eren acceptables. Ara, però, el descobriment de noves tècniques va
  fer possible projectes com Candide a IBM. Candide usa mètodes
  estadístics per a l'anàlisi i la generació, però cap regla
  lingüística. Els treballs a IBM van utilitzar el corpus de textos en
  anglés i francés resultants de les sessions del Parlament de Canadà.
  El mètode consisteix a alinear en primer lloc les frases, els grups
  de mots i els mots en els dos textos i calcular després la
  probabilitat que un mot del text origen corresponga a un o més mots
  del text meta amb el qual ha estat alineat.

  Els mètodes basats en exemples (vegeu el capítol~\ref{se:memtrad})
  s'aprofiten també de l'existència de grans corpora de textos
  traduïts (per això també s'en diu basats en memòria). La idea
  fonamental es que el procés de traducció es pot fer sovint
  consultant traduccions anteriors i identificant frases o grups de
  mots en el corpus ja traduït. Per poder dur a terme la traducció és
  necessari que els textos del corpus hagen estat alineats prèviament
  (mitjançant mètodes estadístics o mètodes basats en regles).

  Tot i que la gran innovació dels noranta van ser els mètodes
  descrits adés, la recerca i el desenvolupament dels sistemes
  clàssics també va continuar: per exemple, el projecte Eurolang basat
  en el sistema de transferència Metal pot traduir de l'anglès al
  francés, alemany, italià i espanyol, i viceversa.  Ens els darrers
  10 anys, un dels camps amb més investigacions ha estat el de
  traducció de la parla, una idea que evidentment ha estat present des
  de fa dècades, però que només ara es pot materialitzar parcialment.
  L'objectiu no és obtenir un sistema de traducció perfecta, sinó un
  sistema adequat per a aplicacions amb llenguatges, dominis i usuaris
  restringits. El principals són els desenvolupats a ATR, CMU i el
  projecte Verbmobil.

  Una característica important dels primers 1990 és l'aparició de les
  primeres aplicacions pràctiques per a traductors: eines de suport a
  la traducció, diccionaris i bases de dades terminològiques,
  processadors de text multilingües, accés a glossaris i terminologies
  electròniques, eines de comunicació (escàners, OCRs, Internet; vegeu
  els capítols~\ref{se:Internet} i~\ref{se:EPT}) o eines per a entorns
  restringits.  La combinació d'algunes d'aquestes eines en un
  programari concret és el que es coneix com \emph{estacions de
    treball per a traductors}; per exemple, el Translation Manager
  d'IBM, recentment alliberat com a programari lliure/de codi font
  obert amb el nom OpenTM2, \url{http://www.opentm2.org}, o el
  Translator Workbench de Trados, ara anomenat SDL Trados Studio. La
  major part d'aquestes estacions de treball estan disponibles per a
  ordinadors personals.

  \paragraph{Dels darrers noranta a l'actualitat:}
  La TA i les eines de suport a la traducció son cada vegada més
  usades per les grans empreses i per les administracions,
  principalment per a la traducció de documentació tècnica.

  Al llarg dels darrers anys, amb la generalització de l'ús
  d'Internet, s'han desenvolupat serveis de traducció disponibles en
  línia, com ara Google Translate, \url{http://translate.google.com} o
  Bing Translator, \url{http://translator.bing.com}, d'ús molt comú
  per part del públic en general per a l'\emph{assimilació} (vegeu
  l'apartat~\ref{s3:assim}) de continguts web escrits en altres
  llengües i fins i tot per a la traducció de cartells i textos
  fotografiats amb la càmera del telèfon mòbil.

  Des dels seus inicis, quasi tota la recerca i quasi tots els
  sistemes comercials de TA s'han centrat en els principals idiomes
  internacionals: anglés, francés, espanyol, japonés, rus, etc.
  Encara resta molt a fer amb les altres llengües del món; amb
  excepcions com ara el projecte Apertium
  (\url{http://www.apertium.org}), que ofereix traducció automàtica
  per a llengües menys centrals, com ara el gallec, l'occità, el
  bretó.

  En el moment d'escriure aquestes línies (desembre de 2015), la major
  part dels sistemes de traducció automàtica es basen en una evolució
  de la \emph{traducció automàtica estadística} iniciada en IBM durant
  la dècada dels 1990; se sol parlar de \emph{traducció automàtica
    estadística basada en frases}, en anglés \emph{phrase-based
    statistical machine translation}. Aquesta hegemonia es deu en gran
  part a la disponibilitat de programari lliure/de codi font obert per
  a \emph{entrenar} i aplicar sistemes de traducció automàtica, com
  ara Moses (\url{http://statmt.org/moses}). Fins i tot hi ha empreses
  com ara KantanMT (\url{http://kantanmt.com}) que construeixen
  sistemes a mida per als seus clients usant simplement un navegador.

  En els últims anys s'està investigant una nova modalitat de
  traducció basada en l'anomenat \emph{aprenentatge profund}, en
  anglés \emph{deep learning}, que usa mètodes d'un camp de la
  intel·ligència artificial anomenat \emph{xarxes neurals}, i els
  resultats comencen a ser, en proves de laboratori, comparables als
  millors disponibles.

%\mbox{}% per evitar un problema de formatatge en el "persabermes".
\end{persabermes}


\section{Utilitat de la traducció automàtica}
\label{ss:UTA}
La traducció automàtica produeix resultats que normalment no poden
substituir directament els produïts per professionals de la traducció
(vegeu el capítol~\ref{se:ASTA}).  Per exemple, en molts casos és
difícil aconseguir que l'ordinador sàpia elegir la interpretació
correcta entre les possibles interpretacions d'un enunciat ambigu com
\begin{center}
  \emph{Els soldats van disparar als xiquets. Els vaig veure caure.}
\end{center} 
ja que això requereix l'ús de quantitats enormes de coneixement
enciclopèdic sobre el funcionament del ``món real''. En aquest cas, el
sistema ha de saber: que els trets fereixen greument o maten les
persones que els reben i que la condició de ferit greu o mort és
incompatible amb mantenir-se dret, i que, per tot això, la
interpretació més probable és que van caure els xiquets, no els
soldats.  

% Com a conseqüència de problemes com aquest i d'altres de
% naturalesa diversa, en moltes aplicacions, la traducció produïda per
% un bon programa s'ha de considerar com un esborrany que ha de ser
% revisat abans de la publicació.


% Però, per posar-nos en l'altre extrem, ha de quedar clar que la
% traducció automàtica pot ser molt útil en aquelles situacions en què
% la traducció completa per part de professionals de la traducció siga
% impracticable o impossible econòmicament. Algunes d'aquestes
% situacions són:
% \begin{itemize}
% \item La traducció automàtica de correu electrònic entre les persones
%   d'un grup de treball internacional amb la finalitat d'agilitzar les
%   comunicacions; la traducció immediata de documents durant la
%   \emph{navegació} per Internet (de fet, hi ha programes especialment
%   dissenyats per a aquesta finalitat, com ara \emph{Google Translate}
%   o \emph{Bing Translator}), o la traducció automàtica de converses
%   electròniques interactives (teclat i pantalla, \emph{chat}).
% \item La traducció i el manteniment de totes les versions de tots els
%   manuals tècnics d'una família de productes (per exemple, els manuals
%   de manteniment d'una gamma d'automòbils).
% \end{itemize}


Moltes de les aplicacions de la traducció automàtica es poden dividir
en dos grans grups: l'\emph{assimilació} d'informació (quan una
persona usa la traducció automàtica per a obtenir informació a partir
d'un document escrit en una altra llengua) i la \emph{disseminació}
---també anomenada \emph{difusió}--- d'informació (quan una persona
usa la traducció automàtica per a produir documents que han de ser
distribuïts a més d'un usuari).  La traducció automàtica, tot i ser
molt diferent de les traduccions fetes per professionals competents,
pot ser una eina molt útil en aquests dos grups d'aplicacions.

\subsection{Assimilació} 
\label{s3:assim}
En situacions d'\emph{assimilació} de la informació no sembla
necessària una traducció gramaticalment correcta i similar al text que
produiria una persona nativa, sinó més aïna una traducció ràpida i
raonablement intel·ligible. S'ha de tenir en compte que hi ha
característiques dels textos natius que poden no ser necessàries per a
la comprensió. Per exemple, un text pot ser intel·ligible encara que
no concorden els adjectius amb els noms o fins i tot encara que se
n'hagen eliminat els articles (\emph{A amic meu li agraden xiques
  vell}), o l'ordre dels mots no siga gramatical (\emph{La guerra
  evitar no podrem}).\footnote{De fet, hom podria dissenyar els
  sistemes de traducció automàtica perquè no es \emph{preocuparen}
  d'aquests assumptes menors.}

Una de les primeres aplicacions de la traducció automàtica als EUA va
ser l'anomenat \emph{screening} o exploració de documents per a
decidir quins eren rellevants i mereixien una atenció més detallada:
es volia tenir accés a la informació tecnològica present en documents
de la Unió Soviètica. Els usos civils de l'\emph{screening} han
superat actualment l'ús tradicional, el militar. En el cas de
l'\emph{screening}, fins i tot una traducció incompleta a més
d'incorrecta (per exemple, només dels mots terminològics) pot ser de
gran utilitat.  Altres exemples d'ús de la traducció automàtica per a
l'\emph{assimilació} són:
\begin{itemize}
\item La traducció automàtica de correu electrònic entre les persones
  d'un grup de treball internacional amb la finalitat d'agilitzar les
  comunicacions.
\item La traducció immediata de documents durant la \emph{navegació}
  per Internet (de fet, hi ha programes especialment dissenyats per a
  aquesta finalitat, com ara \emph{Google Translate} o \emph{Bing
    Translator}).
\item La traducció automàtica de \emph{converses electròniques}
  interactives (usant el teclat i la pantalla d'ordinadors connectats
  entre si; \emph{xat}) entre persones que parlen dos idiomes
  diferents. Les mancances de la traducció automàtica es poden
  compensar amb preguntes o dient les coses d'una altra manera fins
  que els dos interlocutors s'entenguen (és a dir, mitjançant una
  \emph{negociació}).
\item La traducció de despatxos de premsa en altres idiomes.
\end{itemize}

És important indicar que en quasi totes les situacions d'assimilació
el paper del traductor professional és inexistent, ja que el treball
és de naturalesa molt diferent, i l'ús d'un traductor professional
seria molt car i molt lent.  

Per rudimentari que siga un sistema de traducció automàtica, pot ser
molt útil en tasques d'assimilació. Una de les aproximacions més
simples a la TA és l'anomenada \emph{traducció mot per mot}, en què el
programa identifica cada mot, el busca en un diccionari bilingüe i el
substitueix per una traducció aproximada (vegeu també la
pàg.~\ref{pg:mpm}). A tall d'exemple, considereu el següent text en
tok pisin\footnote{Llengua de contacte que es parla a Papua Nova
  Guinea i que té 50.000 parlants que la parlen com a primera llengua
  i més de dos milions de parlants que la parlen com a segona
  llengua.} (el text està pres de \citealt{lyovin97b}):
\begin{quote}{\sl Long taim bifo, wanpela ailan, draipela pik i save
    stap ya, na em i save kaikai ol man. Em i save kaikai ol man nau;
    wanpela taim, wanpela taim nau ol man go tokim bikpela man bilong
    ol, bos bilong ol, ol i go tokim em nau, em i tok: ``Orait yumi
    mas painim nupela ailan''. }
\end{quote} 
Si prenem un diccionari i traduïm el text mot per mot, prenent la
primera traducció possible en cada cas ---pot haver-n'hi més d'una---,
s'obté el text següent:\footnote{És possible que ja us hàgeu adonat
  que el tok pisin té molt vocabulari pres de l'anglés, com a llengua
  de contacte que és.}
\begin{quote}{\sl En temps passat, un illa, enorme porc - soler viure
    esmentat i ell - soler menjar més-d'un home. Ell - soler menjar
    més-d'un home aleshores; un temps, un temps aleshores, més-d'un
    home anar parlar gran home en més-d'un, cap en més-d'un, més-d'un
    - anar parlar ell aleshores, ell - dir: ``Molt-bé, vosaltres-i-jo
    haver-de trobar nou illa''.}
\end{quote}
I ara, veritat que s'entén una miqueta més? Una traducció
més idiomàtica podria ser:
\begin{quote}{\sl Fa molt temps, en una certa illa, vivia un gran porc
    i se solia menjar la gent. Se solia menjar la gent, i una vegada,
    la gent va anar i va dir al seu gran home, al seu cap, va anar i
    van parlar amb ell. Ell va dir: ``Molt bé, hem de trobar una nova
    illa''.}
\end{quote}
L'ordre dels mots no és molt diferent en tok pisin i en català i això
fa que la traducció mot per mot siga prou llegidora. En canvi, si el
text original està en basc, les coses no són tan senzilles. El text,
pràcticament inintel·ligible per a qui no sàpia basc:
\begin{quote}{\sl Bazkaria bukatu ondoren Koldo egunkarira joan zen
    eta Teoren foto bat hartu zuen. Gero, egunkariaren ale zaharrak
    irakurri zituen, boxeo txapelketako berriak
    aztertzeko. Boxealarien izenak apuntatu zituen.}
\end{quote}
es pot traduir mot per mot com:
\begin{quote}{\sl El-dinar acabat després Koldo al-diari anat era i
    de-Teo foto una pres l'havia. Després, del-diari número els-vells
    llegit els-havia, boxa del-campionat les-notícies
    per-a-examinar. Dels-boxadors els noms apuntat els-havia.}
\end{quote}
que és molt més difícil de llegir que el resultat de traduir el text
en tok pisin mot per mot. Una traducció idiomàtica possible és:
\begin{quote}{\sl Després de dinar Koldo va anar al diari i va prendre
    una foto de Teo. Després, va llegir [els] números vells del diari
    per a examinar les notícies del campionat de boxa. Va apuntar els
    noms dels boxadors.}
\end{quote}   
Fixeu-vos que fins i tot en aquest cas tan desfavorable el text
traduït mot per mot dóna bastants pistes sobre el significat del text
original.

En els últims anys, especialment des que s'ha generalitzat l'accés
públic a Internet, s'observa una tendència a incorporar sistemes de
traducció automàtica com un dels components de sistemes més grans de
comunicació.  Aquesta aplicació de la TA per a l'assimilació es pot
veure en \emph{xats} bilingües, o en els sistemes que tradueixen les
pàgines \emph{web} segons anem visitant-les seguint enllaços; en
aquests sistemes, la TA no s'invoca explícitament, sinó implícitament
quan usem el servei.

\subsection{Disseminació} 
En situacions de \emph{disseminació} de la informació cal revisar
l'esborrany de traducció produït pel traductor automàtic i fer les
modificacions oportunes per convertir-la en una traducció adequada al
propòsit de les traduccions.  Per tal de minimitzar les modificacions
a fer a la traducció automàtica, pot ser útil restringir la llengua
d'origen (no permetre'n totes les realitzacions possibles, ni tot el
lèxic, ni tots els registres) a un llenguatge que puga ser traduït
automàticament amb el mínim possible de problemes, és a dir, amb el
mínim esforç de postedició, o almenys, amb un esforç acceptable per un
revisor.\footnote{És a dir, quan la revisió no és més costosa que
  refer tota la traducció a mà.} Açò és especialment important quan es
tracta de traduir manuals tècnics a diversos idiomes. Les restriccions
es poden expressar sota la forma de missatges interactius dirigits a
la persona que prepara el document original.\footnote{Vegeu
  l'apartat~\ref{ss:llecon}, on es discuteix un concepte molt
  relacionat, el de \emph{llenguatge controlat}.}

La traducció automàtica per a la disseminació és especialment eficient
quan només es tradueixen textos pertanyents a una part molt reduïda i
ben regulada de l'idioma en qüestió (un \emph{subllenguatge}).  Un
exemple n'és Méteo, el sistema que des del 1982 fins al 2001 produïa
informes meteorològics simultanis en francés i en anglés al Canadà.

%\section{CAT, HAMT i MAHT} 
\section{Traducció semiautomàtica} 
\label{se:cat}
Moltes situacions de traducció automàtica es poden classificar com a
situacions de traducció assistida per ordinador (en anglés
\emph{computer-aided translation}; CAT), també anomenada de vegades
\emph{traducció semiautomàtica}.  El terme \emph{computer-aided
  translation} s'usa normalment per a referir-se a l'entorn de
programari que permet la traducció professional amb el suport de bases
de dades lèxiques (vegeu l'epígraf~\ref{ss:bdterm}), i dels
suggeriments de traducció provinents de memòries de traducció (vegeu
el capítol~\ref{se:memtrad}), i fins i tot, de la traducció
automàtica.

Per precisar millor què volem dir amb això d'``assistida per
ordinador'', es fa necessari considerar les nocions de traducció
humana assistida per una màquina (en anglés \emph{machine-aided human
  translation}; MAHT), i traducció automàtica assistida per un humà
(en anglés \emph{human-aided machine translation}; HAMT), que
estableixen les dues situacions bàsiques d'interacció entre una
persona i un ordinador a l'hora de fer la traducció. Els paràgrafs
següents en donen alguns exemples.

\paragraph{MAHT:} L'usuari (un traductor competent o un professional
independent) utilitza diccionaris bilingües, tesaurus o \emph{thesauri},
conjugadors i declinadors, correctors ortogràfics, sintàctics i
d'estil, i formularis o models de documents, com a ajuda mentre
produeix una traducció de manera manual usant un processador de
textos. Altres eines ---d'ús comú entre diversos traductors, i
accessibles normalment com a recursos remots--- poden ser les bases de
dades terminològiques i les bases de dades lèxiques multilingües
(vegeu l'epígraf~\ref{ss:bdterm}), o les memòries de traducció (vegeu
el capítol~\ref{se:memtrad}). 

\paragraph{HAMT:} Un programa de traducció automàtica pregunta a
l'usuari quan té més d'una possible traducció per a un mot o per a una
frase.  Aquesta i altres situacions de \emph{negociació} del text
d'origen amb l'usuari del sistema impliquen una interacció que també
pot ajudar a preparar un text més correcte, és a dir, a
\emph{preeditar-lo} (vegeu l'apartat \ref{ss:preedposted}) perquè puga
ser traduït automàticament. Altres voltes, el programa pot analitzar
l'estructura profunda de la frase i presentar-ne les possibles
interpretacions a l'autor, per tal que resolga alguna possible
ambigüitat. En aquests sistemes interactius, cal tenir en compte dos
factors: el primer, que un sistema que pregunta massa no és còmode
d'usar (no és \emph{ergonòmic}) i el segon, que pot passar que
l'usuari siga monolingüe, circumstància que canvia molt la naturalesa
de la interacció entre el programa i l'usuari.  Els usuaris d'aquest
tipus de sistemes es podrien classificar en tres grans grups:
traductors ocasionals, traductors professionals individuals i
traductors professionals que treballen per a empreses de traducció.

\section{Automatització del procés de traducció}
\label{ss:preedposted}
A l'hora d'abordar l'automatització del procés de traducció cal fer
una anàlisi dels costos de traducció per tal d'estimar l'estalvi en
recursos (com ara temps i diners) que es produirà amb la introducció
de la traducció automàtica. El capítol \ref{se:ASTA} es centra en la
avaluació dels sistemes de traducció automàtica i l'anàlisi de costos
de traducció; en aquest apartat discutirem les diferents tasques i
opcions per automatitzar el procés de traducció.

\subsection{Postedició}
\label{ss:postedicio}
La \emph{postedició} és la modificació \emph{mínima} d'una traducció
generada per ordinador per a \emph{fer-la adequada a un
  \textbf{propòsit} ben definit}: el text meta produït pel sistema es
refina o revisa ({\em postedita}) perquè siga gramaticalment correcte
o estiga escrit d'acord amb un registre determinat.

A l'hora de posteditar hem d'evitar fer canvis \emph{preferencials}
(aquesta solució adequada ``m'agrada més'' que aquesta altra que també
és adequada). Els canvis estilístics s'han de fer estrictament quan,
si no es feren, la traducció resultant no compliria amb el propòsit
per al qual va ser encarregada.  Les modificacions poden ser:
\emph{esborrats} d'un mot que sobra, \emph{substitucions} d'un mot per
un altre, o \emph{insercions} d'un mot que falta. Han de ser les
\emph{mínimes} necessàries: si hi ha més d'una edició possible, cal
elegir la que es faça amb el mínim de modificacions necessàries.

Hem de tenir en compte que la persona posteditora, a més de conéixer
la llengua meta i ser capaç de convertir el text en brut a una forma
genuïna en aquesta llengua (és a dir, a més de ser professional de la
traducció), ha de ser una veritable especialista en postedició, que
coneix el sistema de traducció automàtica i quins en són els errors
més típics. Així, la tasca de postedició és molt més eficient, ja que
en conéixer l'origen i la causa dels errors se'n fa més fàcil i ràpida
la correcció.

En primera aproximació, la postedició serà convenient quan
$$\textbf{cost}\left(\mbox{\begin{tabular}{c}traducció automàtica\\ +\\
      postedició\end{tabular}}\right) <
\textbf{cost}(\mbox{traducció professional}).
$$
\label{pg:cost}
Cal comprovar que la fórmula anterior es compleix, encara que siga a
llarg termini, abans de triar una estratègia de traducció per a la
disseminació basada en la postedició.\footnote{Per a una anàlisi de
  costos més detallada, vegeu l'apartat~\ref{ss:costdetall}.}

\subsection{Preedició} 
\label{ss:preedicio}
La \emph{preedició} consisteix a preparar o adaptar (\emph{preeditar})
el text origen per a facilitar la seua traducció i millorar el
comportament del sistema de traducció automàtica, reduint-ne la
necessitat de postedició de la traducció en brut. Això s'aconsegueix,
per exemple, eliminant l'ambigüitat del text,\footnote{Per exemple, en
  anglés tècnic, el mot \emph{replace} presenta una \emph{ambigüitat
    lèxica} (vegeu l'apartat~\ref{ss:amblex}), ja que pot voler dir
  \emph{exchange} (reemplaçar) o \emph{put back} (tornar a
  col·locar).}, evitant l'ús de la veu passiva, reduint l'ús
d'oracions subordinades o usant frases curtes i completes sintàcticament i
semànticament.\footnote{\citet{kohl08} ofereix indicacions
  per a escriure textos en anglés per a una audiència global, de
  manera que els textos siguen més fàcils d'entendre per als no natius
  i més fàcils de traduir manualment i automàticament.} La preedició
del text origen es pot fer també per a marcar parts del text que no
han de ser traduïdes, com ara una citació, o que han de ser tractades
de manera especial per no ser frases completes, com un títol.

La preedició sol ser tant més convenient quant a més llengües es
traduïsca el text preeditat perquè un canvi al text origen pot
estalviar tantes postedicions com llengües d'arribada tinguem.

En resum, hi ha tres modalitats bàsiques d'interacció entre les
persones i els programes de traducció automàtica:
\begin{itemize}
\item la preedició (preparació del text \emph{abans} de la traducció
  automàtica),
\item la postedició (correcció del text \emph{després} de la traducció
  automàtica) i
\item la interacció de la persona amb el sistema de traducció
  automàtica durant el procés de traducció.
\end{itemize}

\com{Tinc en la llibreta un esquema en què apareix tot el procés de
  HAMT: preedició, postedició, interacció, etc. i vindria bé
  col·locar-lo ací.}

\subsection{Llenguatges controlats}
\label{ss:llecon}
Quan la traducció automàtica s'usa per a la disseminació de documents
tècnics de temàtica homogènia, pot ser interessant fer que els
documents originals estiguen escrits usant un lèxic estàndard sense
ambigüitats semàn\-ti\-ques i seguint unes regles sintàctiques i
d'estil ben determinades, és a dir, en un \emph{llenguatge controlat}
\citep{wojcik96u,arnold94b,o2003controlling} dissenyat de manera que
el resultat de la traducció automàtica puga ser usat directament per a
publicar-lo amb el mínim possible de postedició.

Un \emph{llenguatge controlat} és ``un subconjunt del llenguatge
natural definit amb precisió, d'una banda restringit quant al lèxic, a
la gramàtica i a l'estil, i d'una altra, possiblement estés amb
terminologia i construccions gramaticals específiques d'un domini''
\citep{huijsen98u}.

Un llenguatge controlat té sovint associat un conjunt de programes de
suport que ajuden a avaluar i escriure documents que en complisquen
les restriccions. L'escriptor de llenguatges controlats usa normalment
un editor de textos intel·ligent que fa les següents tasques:
\begin{itemize}
\item Comprovar el compliment de les restriccions:
  \begin{itemize}
  \item terminològiques (com el cas del mot \emph{replace} esmentat
    més amunt; per a això, pot ser útil accedir a una base de dades
    terminològica, com les esmentades en el
    capítol~\ref{se:basesdades});
  \item sintàctiques (per exemple, fent l'anàlisi sintàctica de les
    oracions i detectant les ambigüitats estructurals, vegeu
    l'apartat~\ref{ss:ambest}), i
  \item d'estil (per exemple, especificant quin ha de ser el format de
    les dates o de les hores).
  \end{itemize}
\item Emetre un missatge d'error com més informatiu millor quan es
  detecte una violació de les especificacions del llenguatge.
\item Proposar a la persona usuària formes alternatives vàlides al
  text erroni.
\end{itemize}
Com es pot veure, els desenvolupaments tècnics fets al voltant del
disseny d'un llenguatge controlat es relacionen amb molts conceptes
que es tracten en aquest llibre.

Un exemple històric de llenguatge controlat que va ser usat per a
millorar els resultats de la traducció automàtica ---en concret, els
obtinguts amb un sistema també històric anomenat Weidner MicroCat---
és PACE (\emph{Perkins Approved Clear English}), el llenguatge
controlat usat durant els anys vuitanta i part dels noranta per
l'empresa d'enginyeria Perkins Engines per a facilitar la traducció
automàtica dels manuals que descriuen les característiques i el
manteniment d'aquests motors \citep{newton92b,douglas96p}. Un dels
principis de PACE és ``un mot, un significat'', és a dir, s'hi
estableixen restriccions lèxiques clares a través d'un diccionari,
cosa que simplifica el disseny dels diccionaris del sistema de
traducció automàtica. A més del lèxic, PACE també especifica la
sintaxi \cite[secció~8.3]{arnold94b}. Altres exemples de llenguatges
controlats són l'\emph{ScaniaSwedish} usat per la firma de camions i
autobusos Scania \citep{almqvist96p}, o el {\em Caterpillar Technical
  English} de la companyia de maquinària d'excavació Caterpillar.

També hi ha llenguatges controlats no específicament dissenyats per a
la traducció automàtica, com ara l'anglés simplificat ({\em Simplified
  English}) de l'AECMA (Associació Europea d'Indústries
Aeroespacials), que es caracteritza per ``una sintaxi senzilla, un
nombre limitat de mots, un nombre limitat de significats ben definits
per mot (normalment un), i un nombre limitat de categories
lèxiques\footnote{Les \emph{categories lèxiques} (o simplement
  \emph{categories}) són conjunts de mots que tenen la mateixa funció
  sintàctica; hi ha categories \emph{majors}, \emph{lèxiques} o
  \emph{de classe oberta} (substantiu, adjectiu, verb, etc.) que
  creixen quan s'afig nou lèxic a la llengua i categories
  \emph{menors}, \emph{gramaticals} o \emph{de classe tancada}
  (articles, conjuncions, etc.), que no creixen i contenen mots amb
  funció gramatical. La sintaxi es defineix normalment, no en termes
  de mots, sinó en termes de categories lèxiques.}\label{pg:catgra}
per mot (normalment una)'', amb ``l'objectiu de produir textos breus i
no ambigus'' \citep{AECMA07u}.

Alguns dels avantatges de l'ús de llenguatges controlats
\citep{schwitten07u} es poden resumir com segueix:
\begin{itemize}
\item els textos són més senzills i intel·ligibles;
\item el manteniment dels documents es facilita;
\item se simplifica el tractament computacional dels documents, en
  particular la traducció automàtica.
\end{itemize}
Quant als desavantatges, podem dir que:
\begin{itemize}
\item el disseny d'un llenguatge controlat no és gens trivial: cal
  estudiar amb profunditat corpus de textos pertanyents al domini i
  prendre decisions difícils;
\item el poder d'expressió d'un llenguatge controlat és sempre més
  restringit;
\item l'escriptura de textos en llenguatge controlat és més lenta;
\item és necessària una inversió addicional de temps en l'aprenentatge
  del llenguatge controlat per part dels autors.
\end{itemize}
Els dos últims desavantatges es poden reduir si es dota els autors
d'eines informàtiques, com ara d'un editor de textos intel·ligent que
els ajude a escriure en el llenguatge controlat.

Per últim, cal deixar clar que l'ús d'un llenguatge controlat és una
alternativa a la preedició dels textos, però que no elimina per
complet la necessitat de postedició o, almenys, de revisió de les
traduccions en brut.

\section{Qüestions i exercicis}
\begin{enumerate}
\item(*) Elegiu un idioma qualsevol que conegueu bé, $L$. És ben segur
  que $L$ té mots polisèmics que en una altra llengua $L'$ tenen més
  d'una traducció, segons el sentit que se'n prenga. Elegiu tres mots
  de $L$ que tinguen aquest problema i descrigueu com els tractaríeu
  en un llenguatge controlat basat en $L$. Les regles que formuleu per
  als autors que escriguen en el llenguatge controlat han de estar
  escrites en $L$ i no han de contenir referències a altres llengües.

\item En els sistemes de traducció automàtica, la preedició...
  \begin{enumerate}
  \item ... redueix la quantitat de postedició.
  \item ... és una alternativa a la postedició, que elimina
    completament aquesta última fase.
  \item ... impossibilita l'ús del sistema per a tasques de
    disseminació d'informació.
 \end{enumerate}

\item Indica en quina d'aquestes situacions de traducció automàtica
  són menys crucials la gramaticalitat o naturalitat lingüística de la
  traducció.
  \begin{enumerate}
  \item Joan usa el Web Translator mentre navega per les pàgines
    d'Internet de la Universität Mainz per a saber quina assignatura
    dóna el professor Karl-Hans Lehninger i quins són els seus
    interessos investigadors.
  \item Joan usa el Web Translator per a fer una versió en alemany de
    la seua pàgina Web.
  \item El personal d'IBM tradueix patents europees per a detectar
    possibles avanços en correcció d'errors de comunicacions digitals.
  \end{enumerate}

\item Imagineu que podem elegir entre dos sistemes de traducció
  automàtica diferent $t_A$ i $t_B$ per a traduir manuals de
  televisors de l'anglés al francés, i que s'ha de dissenyar un anglés
  controlat per a minimitzar la postedició. Les regles de l'anglés
  controlat, poden dependre del sistema de TA elegit?
  \begin{enumerate}
  \item No, perquè els llenguatges controlats s'han de dissenyar
    independentment dels sistemes de TA.
  \item Sí, perquè en cada cas s'han d'evitar problemes diferents.
  \item No, perquè la llengua meta dels dos sistemes és la mateixa.
  \end{enumerate}

\item Indica quina d'aquestes situacions de traducció automàtica és
  d'\emph{assimilació} d'informació:
  \begin{enumerate}
  \item Narcís usa el programa traductor de l'anglés a l'espanyol
    Spanish Assistant per a llegir els documents electrònics que troba
    en Internet sobre la influència de l'èuscar sobre el gascó.
  \item Joan usa el Web Translator per a fer una versió en alemany de
    la seua pàgina Web abans de publicar-la en Internet.
  \item L'empresa Into the Wind tradueix automàticament el seu catàleg
    de milotxes i catxerulos a diverses llengües.
  \end{enumerate}

\item Moltes voltes, la preedició la fa l'autor quan interacciona amb
  el programa de traducció automàtica. És possible dissenyar un
  sistema de preedició interactiva per a autors monolingües?
  \begin{enumerate}
  \item Sí.
  \item No. Per a preeditar correctament cal conéixer l'idioma de
    destinació.
  \item Només per a certs idiomes amb estructura gramatical senzilla
    com l'anglés.
  \end{enumerate}

\item Quin dels següents \emph{no} és un avantatge dels llenguatges
  controlats?
  \begin{enumerate}
  \item S'evita la necessitat que una persona interaccione amb el
    programa de traducció automàtica per a resoldre ambigüitats durant
    la traducció.
  \item Els textos meta resultants són molt més curts.
  \item Els textos origen es fan més intel·ligibles.
\end{enumerate}

\item Per què és necessària la preedició en els sistemes de traducció
  automàtica?
  \begin{enumerate}
  \item Per a evitar construccions o frases difícils de traduir.
  \item Perquè el format quede més agradable a la vista.
  \item és una alternativa a la postedició.
  \end{enumerate}

\item Imagineu que un traductor professional cobra 0,05 euros per mot
  de text traduït i que un corrector de textos cobra 0,10 euros per
  mot de text corregit. Imagineu que tenim un sistema de traducció
  automàtica que ens costa uns 0,03 euros per mot traduït i que
  produeix un 10\% de mots incorrectes en les traduccions. Convé
  adoptar-lo i contractar el corrector o és millor contractar el
  traductor professional? (si no sabeu calcular-ho en general, feu els
  càlculs amb un text de, per exemple, 1000 mots).

\item La traducció automàtica instantània de pàgines \emph{web} durant
  la navegació és un cas de traducció automàtica{\ldots}
  \begin{enumerate}
  \item {\ldots} amb preedició.
  \item {\ldots} per a la disseminació.
  \item {\ldots} per a l'assimilació.
\end{enumerate}

\item El ``control'' dels llenguatges controlats{\ldots}
  \begin{enumerate}
  \item {\ldots} es refereix tant a la terminologia com a la sintaxi.
  \item {\ldots} només pot referir-se a la sintaxi.
  \item {\ldots} només pot referir-se a la terminologia.
  \end{enumerate}

\item Quan s'usen per a la traducció, els llenguatges controlats
  restringeixen directament{\ldots}
  \begin{enumerate}
  \item {\ldots} la llengua meta.
  \item {\ldots} la llengua origen.
  \item {\ldots} tant la llengua origen com la llengua meta.
  \end{enumerate}

\item Si un angloparlant usa el traductor automàtic portugués--anglés
  de \verb|babelfish.altavista.com| per a llegir en línia el diari
  brasiler \emph{O Globo}, està usant la traducció automàtica per a
  un propòsit\ldots
  \begin{enumerate}
  \item \ldots d'assimilació d'informació.
  \item \ldots de disseminació.
  \item \ldots per al qual no està pensada.
  \end{enumerate}

\item Quina és l'alternativa estàndard a la preedició en un entorn de
  producció massiva de documentació multilingüe?
  \begin{enumerate}
  \item L'ús d'un llenguatge controlat
  \item L'ús d'un sistema d'interlingua.
  \item La postedició sistemàtica
  \end{enumerate}

\item Fran consulta a través d'Internet la base de dades terminològica
  IATE (vegeu l'apartat~\ref{ss:bdterm}) quan tradueix dossiers
  antiglobalització de l'anglés al neerlandés. En quina de les tres
  situacions següents es troba?
  \begin{enumerate}
  \item Traducció automàtica assistida per la persona
  \item Traducció humana assistida per la màquina
  \item Usa un llenguatge controlat
\end{enumerate}

\item Si enviem un document HTML a un servidor de traducció automàtica
  i després posteditem el resultat perquè siga una traducció
  acceptable de l'original abans de publicar-la, estem usant la
  traducció automàtica{\ldots}   
  \begin{enumerate}
  \item {\ldots} amb memòria de traducció.
  \item {\ldots} per a la disseminació.
  \item {\ldots} per a l'assimilació.
  \end{enumerate}

\item L'adopció d'un llenguatge controlat en una situació de traducció
  de documents d'una llengua a moltes llengües per a la disseminació
  és, en el procés complet, una alternativa a{\ldots}
  \begin{enumerate}
  \item {\ldots}la postedició repetitiva dels documents meta.
  \item {\ldots}la preedició repetitiva dels documents origen.
  \item {\ldots}la traducció de fragments ja traduïts anteriorment.
  \end{enumerate}

\item Un sistema que suggereix millores a l'estil d'un document es pot
  considerar com {\ldots}
  \begin{enumerate}
  \item {\ldots} HAMT.
  \item {\ldots} MAHT.
  \item {\ldots} un sistema de traducció automàtica ergonòmic.
\end{enumerate}

\item Una inventora monolingüe consulta documents web traduïts a la
  seua llengua per tal de descobrir si el seu nou invent ha estat
  patentat abans. Si la traducció es fa mitjançant un sistema
  automàtic, quin ús n'està fent?
  \begin{enumerate}
  \item Assimilació; més concretament per a allò que es diu
    \emph{screening}.
  \item Disseminació.
  \item Postedició, ja que l'idioma del document canvia per què puga
    ser entés.
  \end{enumerate}

\item El programa de la Generalitat Valenciana SALT 4.0 tradueix
  textos de l'espanyol a la varietat valenciana del català i pregunta
  esporàdicament a la persona usuària quin equivalent és més adequat
  per a alguns mots ambigus difícils.  Aquesta és una situació
  de{\ldots}
  \begin{enumerate}
  \item {\ldots} postedició.
  \item {\ldots} traducció automàtica assistida per la persona.
  \item {\ldots} traducció humana assistida per l'ordinador.
  \end{enumerate}

\item Volem posteditar un text traduït automàticament mirant tan poc
  com siga possible el text original. Ens ajuda conéixer quins són els
  mots homògrafs (vegeu la p.~\pageref{pg:homografia}) més comuns de
  la llengua origen?
  \begin{enumerate}
  \item No, perquè els homògrafs del text origen no afecten el text
    meta en brut.
  \item No, perquè només estem mirant el text meta.
  \item Sí, perquè són una font molt important d'errors especialment
    difícils de corregir si no es coneix què ha passat.
  \end{enumerate}

\item Una persona està escrivint un document en llengua origen que
  després serà traduït automàticament a més d'una llengua meta i el
  sistema que usa per a escriure l'avisa quan tecleja un mot que
  donarà problemes de traducció ---i li suggereix alternatives--- o
  quan escriu una estructura que serà difícil de traduir. Aquesta és
  una situació\ldots
  \begin{enumerate}
  \item \ldots de preedició.
  \item \ldots d'aplicació d'un llenguatge controlat.
  \item \ldots de postedició.
  \end{enumerate}

\item Quina de les següents situacions és absurda en traducció
  automàtica?
  \begin{enumerate}
  \item La postedició en una aplicació d'assimilació.
  \item La postedició en una aplicació de disseminació.
  \item L'ús d'un llenguatge controlat en una aplicació de
    disseminació.
  \end{enumerate}

\item Només una d'aquestes tres afirmacions és certa. Quina?
  \begin{enumerate}
  \item Els llenguatges controlats defineixen regles de postedició.
  \item L'ús d'un llenguatge controlat elimina completament la
    necessitat de postedició.
  \item Quan s'apliquen les regles d'un llenguatge controlat, el text
    resultant és gramaticalment acceptable però s'hi eviten
    construccions i mots que donen problemes.
  \end{enumerate}

\item Un sistema de traducció automàtica hipotètic del rus al català
  produeix text que és bàsicament correcte excepte pel fet que no
  genera ni articles determinats (\emph{el}, \emph{la}, \emph{l'},
  \emph{els}, \emph{les}) ni indeterminats (\emph{un}, \emph{una},
  \emph{uns}, \emph{unes}). Què diríeu d'aquest sistema?
  \begin{enumerate}
  \item Que és especialment adequat per a l'assimilació, però no tant
    per a la disseminació en vista que els articles són més del 10\%
    del text.
  \item Que és especialment adequat per a la disseminació, perquè els
    articles són paraules molt poc freqüents en el text i per tant no
    serà necessària molta postedició.
  \item Que no és útil ni per a l'assimilació ni per a la
    disseminació.
  \end{enumerate}

\item Es pot posteditar sense mirar el text original?
  \begin{enumerate}
  \item Sí.
  \item En general, no. Qui postedita produeix una traducció. Per
    tant, ha d'estar segur que el resultat és traducció del text
    original.
  \item Si el text és tècnic, es pot fer sense mirar. En altre cas, cal
    mirar sempre el text original.
  \end{enumerate}

\item L'ús d'un llenguatge controlat fa que \ldots
  \begin{enumerate}
  \item \ldots l'escriptura siga més ràpida.
  \item \ldots l'estil del document resultant siga més homogeni.
  \item \ldots el poder d'expressió de l'idioma siga més gran.
  \end{enumerate}

\item Quan posteditem un text trobem una paraula que el traductor
  automàtic no ha sabut traduir, i ens l'ofereix en la llengua
  origen. No obstant això, aquest error no ha afectat a la traducció
  de la resta de l'oració. Què hem de fer?
  \begin{enumerate}
  \item Preeditar el text original complet substituint la paraula per
    un sinònim que sí que reconega el traductor automàtic i tornar a
    traduir tot el text.
  \item Provar a traduir tot el text amb un altre traductor automàtic.
  \item Corregir-la i seguir posteditant.
  \end{enumerate}

\item Si en una fàbrica de frigorífics s'usen sistemes de traducció
  automàtica per a traduir a moltes altres llengües els manuals dels
  nombrosos models que s'hi fabriquen (i que són molt similars entre
  ells), la solució més eficient per a evitar errors de traducció
  és{\ldots}
  \begin{enumerate}
  \item \ldots regular la manera en què els autors escriuen els
    manuals.
  \item \ldots posteditar totes les traduccions.
  \item \ldots preeditar els manuals abans de traduir-los.
  \end{enumerate}

\item A l'hora de preeditar un text per a traduir-lo automàticament
  convé \ldots
  \begin{enumerate}
  \item \ldots usar frases curtes.
  \item \ldots usar la forma passiva.
  \item \ldots usar oracions subordinades.
  \end{enumerate}

\item La postedició de la traducció realitzada per un traductor
  automàtic és sempre necessària \ldots
  \begin{enumerate}
  \item \ldots per a usar-la amb finalitats de disseminació.
  \item \ldots per a usar-la amb finalitats d'assimilació.
  \item\ldots quan s'ha realitzat també preedició.
  \end{enumerate}
\end{enumerate}

\section{Solucions}
\begin{enumerate}
\item(*) \label{pr:escondite} Per exemple, si $L$ és l'espanyol, mots
  com {\em escondite} poden referir-se a un lloc on amagar-se (1) o a
  un joc (2) (en $L'$=català, \emph{amagatall} (1) i \emph{fet}, {\em
    amagar}, \emph{fet a amagar} o \emph{conillets a amagar} (2)). En
  el llenguatge controlat, es podria evitar el primer significat
  proposant els autors que feren servir el mot alternatiu {\em
    escondrijo}. Les regles es podrien formular com segueix en
  espanyol:
  {\sl
  \begin{description}
  \item[escondite] úsese sólo en el sentido de ``juego del
    escondite''; úsese \emph{escondrijo} si se quiere indicar el lugar
    donde se esconde alguna persona o cosa.
  \item[registro] úsese sólo en el sentido de ``transcripción'',
    ``inscripción'' u ``oficina de registro''; úsese \emph{inspección}
    cuando se refiera, por ejemplo a la investigación detallada de un
    local por parte de la policía.
  \item[explotar] úsese sólo en el sentido de ``aprovechar
    económicamente''; úsese \emph{estallar} en el sentido de
    ``deflagrar'' (una bomba, etc.) o ``reventar'' (un globo, etc.).
  \end{description}
}
\item (a)
\item (a)
\item (b)
\item (a)
% 5
\item (a). Les preguntes es poden plantejar com en el
  problema~\ref{pr:escondite}.
\item (b)
\item (a) 
\item
  \begin{description}
  \item [Solució 1 ($n$ mots):] El traductor professional tradueix un
    text de $n$ mots per 0,05$\times n$ euros. El sistema de traducció
    automàtica el tradueix per 0,03$\times n$ i corregir-lo costa
    0,10$\times($10$/$100$)\times n$, és a dir, 0,01$\times n$ euros
    perquè només 10 de cada 100 mots són incorrectes. Per tant, el
    sistema semiautomàtic costa només $($0,03$+$0,01$)\times n =
    $0,04$\times n$ euros, davant dels 0,05$\times n$ euros del
    traductor professional.
 
  \item[Solució 2 (1000 mots):] El traductor professional tradueix un
    text de 1000 mots per 0,05$\times$1000$=$50 euros. El sistema de
    traducció automàtica el tradueix per 0,03$\times$1000$=$30 euros,
    i corregir-lo costa 10 euros, perquè en 1000 paraules hi ha
    1000$\times $10$/$100$=$100 mots incorrectes i corregir cada un
    costa 0,10 euros: 0,10$\times$100$=$10.  Per tant, el sistema
    semiautomàtic costa només 40 euros, davant dels 50 del traductor
    professional.
  \end{description}
\item (c)
% 10
\item (a)
\item (b)
\item (a)
\item (a)
\item (b)
% 15
\item (b)
\item (b)
\item (b)
\item (a)
\item (b)
% 20
\item (c)
\item (b)
\item (a)
\item (c) 
\item (a)
% 25
\item (b)
\item (b)
\item (c)
\item (a)
\item (a)
% 30
\item (a)
\end{enumerate}

