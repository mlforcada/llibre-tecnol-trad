\chapter[Per què és difícil la TA? Ambigüitat]{Per què és difícil la
  traducció automàtica? Ambigüitat}
\label{se:ambig}


\section{Els quatre problemes de la traducció automàtica}
Per què és difícil per a un sistema informàtic traduir com un
professional?  Una classificació interessant dels problemes de la
traducció automàtica la dóna \cite{arnold03p}. Segons aquest autor, la
traducció automàtica té quatre grans problemes:
\begin{enumerate}
\item \emph{La forma no determina completament el contingut} (és a
  dir, la \emph{interpretació}): no sempre és fàcil determinar la
  interpretació que es volia que tinguera el que s'ha escrit. Aquest
  és el \emph{problema de l'anàlisi}, també anomenat
  \emph{ambigüitat}. Exemples: \emph{Portaven notícies de Grècia}
  (tema o procedència?), \emph{Ha venut les taronges que ha comprat a
    Joan} (Joan ven taronges o les compra?), \emph{Treballa en
    l'estudi que li han encarregat} (prepara un document o està
  dissenyant un espai de treball?), etc.

\item \emph{El contingut no determina completament la forma}. És a
  dir, és difícil determinar com s'ha d'expressar una interpretació
  concreta perquè hi ha més d'una manera de dir el mateix en qualsevol
  llengua. Aquest és el \emph{problema de la síntesi}. Exemples: com
  es diu en quin moment del dia ens trobem?  Cada idioma ho fa
  diferentment: català: \emph{Quina hora és?}; portugués: \emph{Que
    horas são?} (\emph{Quines hores són?}); alemany: \emph{Wie spät
    ist es?} (\emph{Com és de tard?}); alemany: \emph{Wieviel Uhr ist
    es?} (\emph{Quantes del rellotge són?}), etc.\footnote{Vegeu
    l'exemple \emph{m'agrada nadar} en la
    p.~\pageref{pg:magradanadar}}

\item \emph{Les llengües divergeixen.} És a dir, hi ha diferències
  irreductibles en la manera que el mateix contingut s'expressa en
  llengües diferents. Aquest és el \emph{problema de la
    transferència}, perquè es manifesta típicament en els sistemes de
  traducció automàtica per transferència
  (vegeu~\ref{ss:classtrans}). Per exemple, l'ordre estàndard de les
  oracions en català és \emph{subjecte}--\emph{verb}--\emph{objecte},
  mentres que en basc o en turc és
  \emph{subjecte}--\emph{objecte}--\emph{verb}, en irlandés és
  \emph{verb}--\emph{subjecte}--\emph{objecte} i en malgaix és
  \emph{verb}--\emph{objecte}--\emph{subjecte}.  O per exemple, els
  idiomes difereixen en la manera en la qual expressen les relacions
  entre dos noms: on en català diem \emph{president \textbf{de}
    Kazakhstan}, el rus diu \emph{prezident Kazakhstan\textbf{a}}, el
  basc diu \emph{Kazakstan\textbf{go} presidente}, o el Kazakh diu
  \emph{Qazaqstan prezident\textbf{i}}.

\item Construir un sistema de traducció automàtica comporta la gestió
  d'una gran quantitat de coneixement, que s'ha d'aplegar, descriure,
  i representar en una forma útil per al processament per
  ordinador. Aquest és el \emph{problema de la descripció}.
\end{enumerate}

D'aquests quatre, dedicarem la resta del capítol a descriure amb més
detall el més important per a la traducció automàtica (i en general,
per a qualsevol programa que haja de processar textos en llenguatge
natural): l'ambigüitat inherent al llenguatge humà.

\section{Ambigüitat}

Podem dir que un enunciat (una oració, un text) és ambigu quan és
susceptible de dues o més interpretacions
\citep{alcaraz97b}.\footnote{\citet{don96u} ho expressen dient que
  l'ambigüitat és ``el fenomen pel qual una expressió té més d'un
  significat''.} Per tant, un enunciat ambigu pot tenir més d'una
traducció a un altre idioma, tot i que de vegades pot tenir només una
traducció a un altre idioma perquè aquesta traducció conserva
l'ambigüitat de la frase original;\footnote{Per exemple, l'oració
  espanyola \emph{Aprendió a afeitarse en dos minutos} es pot traduir
  al català \emph{Va aprendre a afaitar-se en dos minuts} sense
  resoldre l'ambigüitat següent: és el temps que va tardar a aprendre
  o el temps que empra per afaitar-se?}  d'això, se'n sol dir
\emph{free ride} (``passi gratuït'') i és tant més freqüent com més
properes siguen les llengües involucrades en la traducció. En aquest
capítol ens fixarem molt especialment en l'ambigüitat de les oracions.

Una de les perspectives més interessants per a analitzar i ordenar els
tipus d'ambigüitat descrits més amunt ens la proporciona l'anomenat
\emph{principi de composicionalitat} \citep[cap.~23]{radford09b}:
\begin{quote} {\sl La interpretació d'una oració està determinada per
    la interpretació dels mots que apareixen en l'oració i per
    l'estructura sintàctica de l'oració.}
\end{quote}
Aquest principi explica per què la interpretació de l'oració
\begin{exemple}
\label{eq:pare}
El pare escura plats
\end{exemple}
és diferent de la de l'oració
\begin{exemple} 
  La mare llegeix llibres
\label{eq:mare}
\end{exemple}
Les oracions (\ref{eq:pare}) i (\ref{eq:mare}) tenen la mateixa
sintaxi però diferent interpretació perquè contenen mots diferents amb
interpretacions diferents. També explica per què l'oració
\begin{exemple}
  El gos va mossegar l'home
\end{exemple}
no té la mateixa interpretació que la frase
\begin{exemple}
  L'home va mossegar el gos
\end{exemple} 
Aquestes oracions no volen dir el mateix perquè, malgrat tenir els
mateixos mots, l'estructura sintàctica no és la mateixa.

Per això, no és possible assignar una interpretació clara a oracions
sin\-tàc\-ti\-ca\-ment incorrectes, com ara
\begin{exemple}
  *Llegeix mare llibres la
\end{exemple} 
encara que els mots tinguen interpretació independentment, ni tampoc a
una oració sintàcticament correcta que continga algun mot al qual no
podem assignar cap interpretació:
\begin{exemple}
  La mare *ingurpleix llibres
\end{exemple}
Com veurem més endavant, trobem una complicació addicional: en algunes
oracions, hi ha parts de l'estructura sintàctica que no es
reflecteixen en cap mot, perquè generen \emph{categories buides} que
no tenen una representació fonètica o gràfica explícita.  Per exemple,
l'oració
\begin{exemple}
  Té molts amics
\end{exemple} 
té un subjecte buit (també anomenat el·líptic). En aquests casos podem
considerar que les categories buides són mots ``de zero lletres'' que
tenen una interpretació.

Si una oració és ambigua quan té més d'una interpretació possible,
això pot tenir dues causes bàsiques:
\begin{itemize}
\item un o més mots de l'oració tenen més d'una interpretació possible
  (és a dir, són \emph{lèxicament ambigus}).
\item l'oració té més d'una estructura sintàctica possible (és a dir,
  és \emph{estructuralment ambigua} o \emph{sintàcticament ambigua}).
\end{itemize}
Les dues causes poden concórrer. De fet, estudiarem tres casos:
l'ambigüitat purament deguda a l'ambigüitat dels mots (explícits o
nuls); l'ambigüitat purament deguda a l'existència de més d'una
estructura sintàctica, i l'ambigüitat deguda a les dues causes alhora.

\subsection{Ambigüitat  deguda a l'ambigüitat lèxica}
\label{ss:amblex}
En moltes llengües, els mots es flexionen i prenen formes diferents.
Un mot (i, en general, una unitat lèxica de més d'un mot) es pot veure
des de dues perspectives; d'una banda, la \emph{forma superficial} del
mot és la forma concreta que apareix en el text: \emph{cantàvem};
d'altra banda, hi ha la \emph{forma lèxica}, que consisteix en
\begin{itemize}
\item un \emph{lema} o \emph{forma canònica} (\emph{cantar}),
\item una \emph{categoria lèxica},\label{pg:catlex}\footnote{Vegeu la
    nota al peu de la pàg.~\pageref{pg:catgra}} classe de mot, o part
  de l'oració (verb) i
\item uns \emph{indicadors de flexió} que expressen les
  característiques morfològiques o flexives (primera persona, nombre
  plural, temps pretèrit imperfet, mode indicatiu).
\end{itemize}
Quan dues formes lèxiques diferents tenen la mateixa forma
superficial; és a dir, quan s'escriuen de la mateixa manera, se sol
dir que són \emph{homògrafes}\label{pg:homografia}; a més, s'anomena
simplement \emph{homògraf} a la forma superficial a què correspon més
d'una forma lèxica; el fenomen s'anomena \emph{homografia}. Per
exemple, el mot \emph{riu} és homògraf perquè té tres formes lèxiques:
\emph{riu}, substantiu, masculí singular; \emph{riure}, verb, 3a.\
persona del singular, present d'indicatiu, i \emph{riure}, verb,
2a. persona del singular, imperatiu. Podem diferenciar tres tipus
d'ambigüitat per homografia:
\begin{enumerate}
\item \emph{Ambigüitat entre categories lèxiques diferents}. Per
  exemple, el mot \emph{moc} té dues formes lèxiques possibles,
  cadascuna amb una categoria lèxica diferent: \emph{moc}, substantiu,
  masculí, singular; \emph{moure}, verb, 1a.\ persona del singular,
  present d'indicatiu.
\item \emph{Ambigüitat dins de la mateixa categoria lèxica sense canvi
    de lema}. Per exemple, el mot \emph{canta} té dues formes lèxiques
  amb el mateix lema, la mateixa categoria lèxica, però distinta
  informació de flexió: \emph{cantar} verb, 3a.\ personal del
  singular, present d'indicatiu; \emph{cantar}, verb, 2a.\ personal
  del singular, imperatiu.
\item \emph{Ambigüitat dins de la mateixa categoria lèxica amb canvi
    de lema}. Per exemple, el mot \emph{poden} té, entre d'altres,
  dues formes lèxiques amb la mateixa categoria lèxica, la mateixa
  informació de flexió, però distint lema: \emph{poder}, verb, 3a.\
  personal del plural, present d'indicatiu; \emph{podar} verb, 3a.\
  personal del plural, present d'indicatiu.
\end{enumerate}

Però l'homografia no és l'única causa possible d'ambigüitat lèxica; hi
ha mots que són ambigus tot i tenir la mateixa forma lèxica, perquè el
que és ambigu és la interpretació del lema. Aquests mots s'anomenen
habitualment \emph{polisèmics} i aquest tipus d'ambigüitat,
\emph{polisèmia}. Per exemple, el mot \emph{estació} (forma lèxica:
\emph{estació}, substantiu, femení singular) és polisèmic perquè el
lema corresponent té més d'una interpretació: indret on s'aturen
temporalment els trens, part de l'any compresa entre un solstici i un
equinocci, conjunt d'instal·lacions per a un propòsit determinat (per
exemple, l'esquí), etc. La polisèmia afecta totes les formes
flexionades d'un determinat mot de la mateixa manera (\emph{estacions}
té exactament la mateixa ambigüitat que \emph{estació}), ja que és una
propietat del \emph{lema}.\footnote{Hi ha casos que no són tan
  senzills. Per exemple, en anglés, el mot \emph{case}, un substantiu
  singular, pot referir-se a un tipus de contenidor (\emph{a case of
    wine}) o a un exemple o situació particular (\emph{It does not
    apply in this case}). Cada un dels mots ve d'un mot llatí
  diferent: el primer del mot femení \emph{capsa}, i el segon de
  \emph{casus}, el participi de \emph{cado} `caure'. Els diccionaris
  anglesos, que solen agrupar els mots polisèmics en una entrada,
  típicament en fan dues entrades diferents, i, de fet, en
  lexicografia no és estrany referir-se a \emph{case} com un
  homògraf.}

L'ambigüitat d'una oració pot ser causada per diversos tipus bàsics
d'ambigüitat lèxica:
\begin{enumerate}
\item L'oració conté una o més unitats lèxiques (per exemple mots)
  polisèmiques: si diem que algú
\begin{exemple}
  Treballa en l'estudi que li van encarregar
\end{exemple} 
podem referir-nos a un investigador o a un decorador, depenent de
quina interpretació assignem al mot polisèmic \emph{estudi}. De
l'ambigüitat d'aquestes unitats lèxiques, també se'n sol dir {\em
  ambigüitat lèxica pura}.  Aquesta oració l'hem de desambiguar si la
volem traduir, per exemple, a l'anglés, perquè en el primer cas
hauríem de dir \emph{study} i en el segon, \emph{studio}; per això
l'efecte de la polisèmia en traducció pot causar en molts casos
l'anomenada {\em ambigüitat de transferència}.  L'ambigüitat lèxica de
transferència és especialment perillosa quan afecta un mot de la
llengua origen no percebut com a ambigu; per exemple, el mot espanyol
\emph{destino} es pot traduir al català com a {\em destí} (sort
futura) o \emph{destinació} (punt d'arribada).
  
Un altre exemple: l'oració
\begin{exemple}
  Han posat un banc nou a la plaça
\end{exemple}
pot tenir dues interpretacions, segons la interpretació que s'assigne
al mot polisèmic \emph{banc} (``seient estret i llarg'' o bé
``institució financera'').
  
Una ambigüitat que és molt semblant a l'ambigüitat lèxica pura es
produeix quan una expressió idiomàtica es pren bé com a tal o bé en
sentit literal. Per exemple, la interpretació d'\emph{enviar algú a
  pastar fang} pot ser la idiomàtica de dir a algú que deixe de
molestar (espanyol \emph{mandar a freír espárragos}) però podria ser
també la literal en un taller de terrisseria.
  
\item L'oració conté un homògraf que té dues o més interpretacions
  però la mateixa categoria lèxica, i no afecta, per tant,
  l'estructura de l'oració. 
 Hi ha tres situacions possibles:
   \begin{itemize}
   \item canvia només el lema però no els indicadors de flexió: el
     mot espanyol \emph{creo} pot ser la 1a.\ persona del singular del
     present d'indicatiu del verb \emph{creer} o del verb \emph{crear}.
   \item no canvia el lema però sí els indicadors de flexió: el mot
     espanyol \emph{cantamos} pot ser la 1a.\ persona del plural del
     present d'indicatiu o del pretèrit indefinit (perfet simple) del
     verb \emph{cantar}.
   \item canvien el lema i els indicadors de flexió: el mot espanyol
     \emph{salen} pot ser la 3a.\ persona del plural del present
     d'indicatiu del verb salir o del present de subjuntiu del verb
     \emph{salar}.
   \end{itemize}
  
\item L'oració conté una \emph{expressió anafòrica}, com ara un
  pronom, adjectiu possessiu, etc., la qual pot tenir, en principi,
  més d'una possible interpretació, però aquesta interpretació està
  determinada per la relació de \emph{coreferència} entre l'expressió
  i el seu \emph{antecedent} (un sintagma que es pot trobar en la
  mateixa oració o en una altra oració del text) o perquè es refereix
  a algun objecte o concepte exterior al text. La relació que assigna
  una interpretació a una expressió anafòrica s'anomena \emph{deïxi} o
  \emph{dixi}: quan la interpretació és per \emph{coreferència} amb un
  antecedent que apareix anteriorment en el text s'anomena
  \emph{anàfora}\label{pg:anafora}, i \emph{catàfora} si l'antecedent
  és posterior.  En la frase
\begin{exemple}
  Vaig obrir [la porta]$_i$ a [la cuinera]$_j$ i la$_{i/j?}$ vaig fer
  passar
\end{exemple}
els índexs ($i$, $j$, $i/j?$) indiquen que el pronom feble \emph{la}
es pot referir a la mateixa persona a la qual ens hem referit amb el
sintagma nominal \emph{la cuinera}, però no hi ha cap raó sintàctica
perquè el referent no siga el mateix que el del sintagma \emph{la
  porta}: aquesta pot ser una possible causa d'ambigüitat en la segona
oració coordinada.

\item L'oració té constituents que no es reflecteixen com a mots però
  als quals cal assignar una interpretació. En algunes llengües
  romàniques (en italià, espanyol i català però no en francés) és
  comuna l'absència del subjecte quan és de tercera persona. En aquest
  cas, la posició on hauria d'anar el subjecte es pot suposar ocupada
  per un pronom sense forma superficial que dóna lloc a ambigüitat
  mitjançant mecanismes molt similars als de l'anàfora i per tant,
  se'ls pot considerar mecanismes lèxics.  En el fragment
  \begin{exemple}
    Anna va apunyalar Marta. Joan va veure com queia redolant
  \end{exemple}
  qui va caure redolant, Anna o Marta?  O alguna altra persona?  El
  problema és que falta el subjecte de l'oració subordinada \emph{com
    queia rodolant}. Aquesta omissió dóna lloc a una ambigüitat. Quan
  es tracta de l'omissió del subjecte, se sol postular en lingüística
  l'existència d'un pronom especial anomenat PRO, sense forma
  superficial, que fa de subjecte nul,
  \begin{exemple}
    Joan va veure com PRO queia redolant
  \end{exemple}
  i al qual s'assigna interpretació mitjançant processos
  deïctics\footnote{També anomenats \emph{díctics}, relacionats amb la
    \emph{deïxi} o \emph{dixi}.} o anafòrics com els descrits per a
  altres expressions anafòriques.
  
  Aquesta classe d'ambigüitats se sol incloure dins d'un grup de
  fenòmens més generals anomenats \emph{ambigüitats per el·lipsi}.
  \citet{alcaraz97b} defineixen l'\emph{el·lipsi}\label{pg:ellipsi}
  com l'omissió o l'absència d'alguna part d'una oració. Com veurem
  més avall, de vegades l'el·lipsi dóna lloc a l'existència de més
  d'un arbre d'anàlisi sintàctica per a l'oració i per tant aquests
  tipus d'el·lipsi no es poden incloure pròpiament en aquest apartat
  dedicat a l'ambigüitat purament lèxica.
\end{enumerate} 

\subsection{Ambigüitat estructural pura}
\label{ss:ambest}
L'ambigüitat d'una oració també pot estar deguda al simple fet que
tinga més d'un arbre d'anàlisi sintàctica. S'hi poden distingir
diversos casos:
\begin{enumerate}
\item \emph{Ambigüitat estructural d'origen coordinatiu}: Per exemple,
  si diem
  \begin{exemple}
    Posa els llençols i els cobertors nets a l'armari
  \end{exemple} 
  hi ha dues possibles interpretacions; en una els llençols no estan
  nets, en l'altra sí, segons que es considere que l'adjectiu
  \emph{nets} modifica els dos substantius coordinats o només l'últim
  (vegeu la fig.~\ref{fg:cobertors}). L'ambigüitat estructural
  associada a les conjuncions coordinatives se sol anomenar
  \emph{d'origen coordinatiu}.

\begin{figure}
\begin{center}
\begin{parsetree}
    ( .SN.
      (.SN.
        (.SN. ~ `els llençols' )
        `i' 
        (.SN. ~ `els cobertors' )
      )
      (.SA. ~ `nets' )
    )
\end{parsetree}
\end{center}
\begin{center}
\begin{parsetree}
    ( .SN.
        (.SN. ~ `els llençols' )
        `i' 
        (.SN. 
        (.SN. ~ `els cobertors' )
        (.SA. ~ `nets' )
        )
    )
\end{parsetree}
\end{center}
\caption{Dos arbres per a la frase ``Posa els llençols i els cobertors
  nets a l'armari'' (SN = sintagma nominal, SA = sintagma
  adjectival).}
\label{fg:cobertors}
\end{figure}
 
\item \emph{Ambigüitat estructural d'adjunció} (en anglés
  \emph{attachment ambiguity}): es tracta d'un cas típic d'ambigüitat
  estructural que es manifesta quan hi ha un {\em adjunt}\footnote{Un
    \emph{adjunt} és un sintagma o constituent que, conjuntament amb
    un altre sintagma o constituent, forma un constituent del mateix
    tipus que aquest últim (per exemple, un sintagma nominal més un
    sintagma preposicional formen un sintagma nominal); en un cert
    sentit, l'adjunt no és necessari sinó opcional.} (típicament un
  sintagma preposicional) que es pot inserir de diverses maneres en
  l'arbre d'anàlisi sintàctica de la frase.  Per exemple, la frase
  \begin{exemple}
    Joan va portar notícies de Grècia
  \end{exemple} 
  es pot interpretar de dues maneres: en una, el sintagma
  preposicional \emph{de Grècia} modifica \emph{notícies}; en l'altra,
  modifica \emph{portar} (vegeu els arbres de la
  figura~\ref{fg:noticies}).  Més exemples:
  \begin{exemple}
    Va parlar amb l'encarregat de la neteja de la seua casa
  \end{exemple}
  \begin{exemple}
    Hi ha una bossa de roba perduda en la Secretaria de l'Escola
  \label{ex:bossa}
  \end{exemple}
  \citet{tuson99b} explica que aquesta última oració pot tenir fins a
  12 interpretacions possibles.

\begin{figure}
\begin{center}
\begin{parsetree}
(.SI. (.SN. ~ `Joan')
      (.{\={I}}. 

          (.I. `va' ) 
          (.SV. 
                (.V. `portar' )
                (.SN. 
                     (.N. `notícies' ) 
                     (.SP. ~ `de Grècia' )
                )
          ) 
      ) 
) 
\end{parsetree}
\end{center}
\begin{center}
\begin{parsetree}
(.SI. (.SN. ~ `Joan') (.{\={I}}. (.I. `va' ) (.SV. (.SV. (.V. `portar' )
(.N. `notícies' ) )  (.SP. ~ `de Grècia' ) ) ) )
\end{parsetree}
\end{center}
\caption{Dos arbres per a la frase ``Joan va portar notícies de
  Grècia'' (SI = sintagma inflexional, \={I} = projecció intermèdia de
  la inflexió, I = inflexió, SV = sintagma verbal, V = verb, N = nom,
  SP = sintagma preposicional). }
\label{fg:noticies}
\end{figure}
\end{enumerate}

\begin{persabermes}{ambigüitat estructural}
  Altres tipus d'ambigüitat estructural:
  \begin{enumerate}
  \item \emph{Ambigüitat estructural deguda a l'el·lipsi} d'un o més
    constituents de l'oració, especialment quan aquesta oració hauria
    de tenir, si s'hagués produït en forma explícita, una estructura
    paral·lela a la d'una oració anterior (per exemple, en
    coordinacions, comparacions, etc.).  \com{Dir que en aquest cas
      l'esborrat de les parts repetides és obligatori en la major part
      de les llengües} Considerem l'exemple següent, tret de
    \citet[p.~399]{radford09b}:
    \begin{exemple}
      \label{eq:escocesos}
      Els escocesos aprecien el whisky més que els gal·lesos
    \end{exemple} L'oració té dues interpretacions:
    \begin{exemple}
    \item[(a)] Els escocesos aprecien el whisky més que els gal·lesos
      (aprecien el whisky)
    \item[(b)] Els escocesos aprecien el whisky més que (els escocesos
      aprecien) els gal·lesos\footnote{De fet, per a evitar aquesta
        ambigüitat, es considera convenient \emph{però no obligatòria}
        en català la solució alternativa amb preposició \emph{als
          gal·lesos} per a la segona interpretació.}
    \end{exemple}
    En aquests dos casos, l'ambigüitat està causada pel fet que són
    possibles dues estructures sintàctiques per a la segona oració
    coordinada: en la primera estructura, el sintagma \emph{els
      gal·lesos} és el subjecte mentre que en la segona estructura és
    l'objecte (vegeu els arbres de la fig.~\ref{fg:whisky}).

  \item \emph{Ambigüitat estructural per moviment de Qu}. De vegades,
    l'anàlisi de la sintaxi d'una oració es complica per la presència
    de fenòmens de moviment de constituents. Considerem l'oració
    \begin{example}
      Qui diu que vindrà?
    \end{example}
    Aquesta oració té, fonamentalment, dues interpretacions. Una és
    \begin{example}
      Qui diu que PRO vindrà?
    \end{example}
    i l'altra
  \begin{example}
    * PRO diu que qui vindrà?
  \end{example}
  És a dir, en la primera, el pronom interrogatiu \emph{qui} és el
  subjecte de l'oració principal; en la segona, és el subjecte de
  l'oració subordinada, el qual ha experimentat el \emph{moviment de
    Qu} (en anglés \emph{Wh-movement}) al principi de l'oració, el
  qual obligatori en molts idiomes ---no en tots: el xinés o el turc
  no ho fan, per exemple--- per als mots amb funció interrogativa. En
  aquest cas, com en l'exemple~\ref{eq:escocesos}, l'el·lipsi permet
  dos posicionaments diferents del pronom \emph{qui} abans del
  moviment de Qu, però les ambigüitats causades pel moviment de Qu
  poden produir-se també sense el·lipsi, com en l'exemple
  \begin{exemple}
    Com dius que Jordi ha explicat que vindria?
  \end{exemple}
  on la posició inicial de l'adverbi interrogatiu \emph{Com} pot
  resultar de la transformació per moviment de Qu de tres estructures
  hipotètiques diferents; en cada una d'elles, l'adverbi és adjunt
  d'un sintagma verbal diferent:
  \begin{exemple}
  \item[(a)] *Dius \emph{com} que Jordi ha explicat que vindria?
  \item[(b)] *Dius que Jordi ha explicat \emph{com} que vindria?
  \item[(c)] *Dius que Jordi ha explicat que vindria \emph{com}?
  \end{exemple}
  En la primera interpretació es pregunta per la manera de dir-ho, en
  la segona per la manera d'explicar-ho i en la tercera per la manera
  de venir.
  
  Es produeixen també moviments similars amb els relatius; per
  exemple, aquests es mouen \emph{cap a fora}, és a dir, cap a l'arrel
  de l'arbre d'anàlisi sintàctica, des de les subordinades
  substantives completives amb verbs del tipus de {\em dir},
  \emph{explicar}, etc. En l'oració
  \begin{example}
    No m'agrada la manera com vas dir que vivia encara.
  \end{example}
  el primer \emph{com} és un relatiu que pot modificar \emph{dir} en
  l'oració \emph{vas dir que vivia encara} (``no m'agrada la manera de
  dir-ho'') o pot modificar {\em vivia} però ha estat mogut fora de la
  subordinada completiva \emph{vivia encara}, que modifica {\em la
    manera} (``no m'agrada la manera com vivia, segons que vas dir'').
\end{enumerate}
\end{persabermes}

\begin{figure}
\begin{center}
\begin{parsetree}
(.SI. (.SD. (.D. `els' ) (.N. `gal·lesos' ) ) (.{\={I}}. (.I.  `$\emptyset$' )
(.SV. (.V. `[aprecien]' ) (.SD. (.D. `[el]' ) (.N. `[whisky]' ) ) ) ) )
\end{parsetree}
\end{center}
\begin{center}
\begin{parsetree}
(.SI. (.SD. (.D. `[els]' ) (.N. `[escocesos]' ) ) (.{\={I}}. (.I. `$\emptyset$' )
(.SV. (.V. `[aprecien]' ) (.SD. (.D. `els' ) (.N. `gal·lesos' ) ) ) ) )
\end{parsetree}
\end{center}
\caption{Dos arbres per a la segona part (``els gal·lesos'') de la
  comparació ``Els escocesos aprecien més el whisky que els
  gal·lesos'' (SD =sintagma determinant, D = determinant).}
\label{fg:whisky}
\end{figure}

\subsection{Ambigüitats mixtes}
Hi ha oracions que són ambigües tant perquè contenen mots ambigus com
perquè tenen més d'una estructura sintàctica possible. N'estudiarem
dos casos:
\begin{enumerate}
\item L'oració conté mots afectats d'ambigüitat lèxica categorial amb
  canvi de categoria (vegeu la pàg.~\pageref{pg:catlex}).  Per
  exemple, el mot \emph{deu} pot voler dir ``nou més un'' (numeral) o
  ``ha de donar o pagar'' (verb).  O el mot \emph{cap} que pot ser un
  substantiu (``part superior del cos''), un verb (forma del verb
  ``cabre''), un adjectiu o pronom (``no n'hi ha cap''), o part de la
  preposició composta ``cap a''.
  
  Aquest tipus d'ambigüitat lèxica pot provocar de vegades ambigüitat
  estructural, causada per la presència de més d'una anàlisi
  sintàctica acceptable (si, tot i els homògrafs, només n'hi ha una
  anàlisi acceptable, l'ambigüitat passa desapercebuda per al
  receptor; això és així perquè habitualment només es consideren
  estructures acceptables quan es vol assignar interpretació a una
  oració).  Per exemple, la frase anglesa
  \begin{exemple}
    Time flies like an arrow
  \end{exemple}
  vol dir normalment \emph{El temps vola (com una fletxa)} però també
  són possibles altres dues interpretacions (semànticament
  destrellatades però sintàcticament impecables): \emph{A les mosques
    del temps els agrada una fletxa} o \emph{Cronometra les mosques
    com una fletxa}.  Aquesta varietat d'interpretacions es deu al fet
  que hi ha tres mots en la frase que poden pertànyer a dues
  categories lèxiques diferents: \emph{time} pot ser verb
  (\emph{cronometrar}) i substantiu (\emph{temps}), \emph{flies} pot
  ser verb (\emph{vola}) i substantiu (\emph{mosques}) i \emph{like}
  pot ser verb (\emph{agradar}) i preposició (\emph{com}).  De les 8
  ($2\times 2\times 2$) anàlisis morfològiques possibles de la frase,
  tres en resulten sintàcticament acceptables, amb interpretacions
  molt diferents. Aquest tipus d'ambigüitat se sol anomenar
  \emph{ambigüitat estructural d'origen categorial}. En català ---i en
  general en les llengües romàniques--- són molt comunes les
  ambigüitats degudes a la combinació d'un mot que pot ser pronom
  feble de tercera persona o article (\emph{el}, \emph{la}, \emph{l'},
  \emph{els}, \emph{les}) i un altre mot que pot ser substantiu o verb
  conjugat. Per exemple, l'oració
    \begin{exemple}
      La mata el vol
    \end{exemple} 
    pot voler dir dues coses, segons l'elecció de categories lèxiques
    (``l'acte de volar li provoca la mort'' o ``la planta sent estima
    per ell'').
  
  \item Un altre tipus d'ambigüitat mixta succeeix quan l'ambigüitat
    lèxica categorial d'alguns mots es combina amb mecanismes
    d'el·lipsi com els descrits més amunt per a construccions
    coordinatives o comparatives.  Per exemple, l'oració
    \begin{exemple}
      Les gallines han destrossat el sembrat, però no les mates
    \end{exemple}
    té dues interpretacions:
    \begin{exemple}
      \label{eq:gallines}
    \item[(a)] Les gallines han destrossat el sembrat, però (les
      gallines) no (han destrossat) les mates.
    \item[(b)] [Les gallines]$_i$ han destrossat el sembrat, però (tu)
      no les$_i$ mates.
    \end{exemple}
    En (\ref{eq:gallines}a) \emph{les mates} és un sintagma
    determinant compost d'un article i un substantiu, que fa d'objecte
    del verb el·líptic \emph{destrossat}, mentre que en
    (\ref{eq:gallines}b) \emph{les mates} és un sintagma verbal
    compost d'un pronom ({\em les}) que es refereix a \emph{Les
      gallines} i un verb ({\em mates}), sintagma que constitueix un
    sintagma verbal en la segona oració coordinada (vegeu la
    fig.~\ref{fg:mates}).\footnote{Dir que l'anàfora és només un
      procés lèxic és una simplificació. Hi ha involucrats aspectes
      sintàctics. Per exemple, en l'oració \emph{Maria va parlar amb
        ella}, el pronom \emph{ella} no es pot mai referir a
      \emph{Maria}, però en l'oració \emph{Maria va parlar amb una
        amiga d'ella}, sí que pot, i això és degut al fet que en
      l'estructura sintàctica de la primera oració hi ha una
      \emph{barrera} a la co-referència que en la segona no existeix.}
\end{enumerate}

\begin{figure}
\begin{center}
\begin{parsetree}
(.SI. (.SD. ~ `[les gallines]' ) (.{\={I}}. (.NEG. `no' )
(.{\={I}}. (.I. `[han]' ) (.SV. (.V. `[destrossat]' )  (.SD. ~ `les
mates' ) ) ) ) ) 
\end{parsetree}
\end{center}
\begin{center}
\begin{parsetree}
(.SI. (.SD. `[tu]' ) (.{\={I}}. (.NEG. `no' ) (.{\={I}}. ~ `les mates'
) ) )
\end{parsetree}
\end{center}
\caption{Dos arbres per a la segona oració coordinada en ``Les
  gallines han destrossat el sembrat però no les mates'' (NEG =
  negació). Els triangles s'usen per a no haver d'indicar tots els
  detalls d'un subarbre concret.}
\label{fg:mates}
\end{figure}

\begin{persabermes}{ambigüitats més complexes}
  Hi ha també certs tipus d'ambigüitat que no es poden explicar de
  manera senzilla amb el principi de composicionalitat, com ara {\em
    l'ambigüitat en l'abast dels quantificadors}. Els quantificadors
  són mots com \emph{algun}, \emph{tot}, \emph{cada}.  Quan l'abast
  d'un quantificador (és a dir, els mots que afecta) és imprecís, una
  oració pot tenir més d'una interpretació. Considerem l'exemple de
  \cite{hutchins92b}
  \begin{exemple}
    Totes les dones no s'estimen els abrics de pell.
    \label{eq:abric}
  \end{exemple}
  Aquest exemple pot tenir dues interpretacions
  \begin{exemple}
  \item[(a)] No totes les dones s'estimen els abrics de pell.
  \item[(b)] No hi ha cap dona que s'estime els abrics de pell.
  \end{exemple}
  a pesar de no tenir cap ambigüitat lèxica ni estructural aparent.
  Aquest tipus d'ambigüitat es pot explicar pel fet que el principi de
  composicionalitat per si sol no és suficient per a especificar
  completament l'assignació d'interpretació a una oració. En paraules
  de \citet[p.~364]{radford99b} ``hem de reconéixer [l'existència]
  d'un buit inacceptable entre el que proporciona la sintaxi i el que
  la semàntica necessita en el cas d'oracions que continguen sintagmes
  nominals quantificats''. La interpretació de les oracions se sol
  explicar de vegades en termes de {\em formes lògiques}
  \citep[cap.~23]{radford09b}; en el cas de les oracions amb
  quantificadors, aquestes formes lògiques contenen d'una banda,
  \emph{variables} que poden referir-se a un rang d'objectes que cal
  considerar i, d'altra, operacions sobre aquestes variables.  Doncs
  bé, en aquestos casos, es pot assignar més d'una forma lògica a una
  oració.

\mbox{}

\end{persabermes}

\subsection{Estratègies de resolució de l'ambigüitat}
En general, els humans usem els nostres coneixements, les nostres
expectatives i les nostres creences sobre el funcionament del món real
(o d'un món fictici concret, com en una novel·la) per a elegir una de
les interpretacions com a més versemblant (és a dir, per a
\emph{resoldre l'ambigüitat}); quan els coneixements, les creences i
les expectatives són compartides entre l'emissor i el receptor, es pot
usar l'ambigüitat com un mecanisme molt eficient per a produir
missatges més curts.

Com hem vist, les causes de l'ambigüitat són molt diverses; per això,
també són molt diverses les estratègies de resolució. Aquest epígraf
recull unes notes ---no exhaus\-tives--- sobre les estratègies de
resolució d'alguns tipus d'ambigüitat en sistemes automàtics de
tractament del llenguatge humà.

Les estratègies de resolució de l'ambigüitat solen basar-se en
\emph{restriccions} i \emph{preferències}. Com veurem, les
restriccions són normalment de naturalesa lingüística ---per tant,
requereixen un cert nivell d'anàlisi del text--- i permeten descartar
certes interpretacions, però no eliminen completament l'ambigüitat.
Per a acabar de resoldre l'ambigüitat, s'usen preferències: s'assigna
algun tipus de puntuació o valor a cada interpretació per elegir la
millor. Les preferències se solen basar freqüentment en mètodes
estadístics, basats en observacions obtingudes en grans quantitats de
text.

\subsubsection{Resolució de l'ambigüitat lèxica categorial}
\label{s3:reshom}
La resolució de l'ambigüitat lèxica categorial dels mots homògrafs,
altrament coneguda com etiquetatge (dels mots) amb parts de l'oració
(en anglés, \emph{part-of-speech (PoS) tagging}) està molt ben
estudiada.  L'ambigüitat es redueix normalment usant restriccions
basades en el coneixement lingüístic, i, com que normalment això no
sol ser suficient, s'hi estableixen preferències basades en l'estudi
estadístic de la freqüència d'aparició conjunta en els textos de
determinades seqüències curtes de categories lèxiques.

En alguns casos, les restriccions poden ser suficients. Per exemple,
el mot espanyol \emph{ahorro} pot ser substantiu o verb. Si apareix
entre un article i un adjectiu com ara en \emph{el ahorro doméstico}
no hi ha cap dubte que es tracta d'un substantiu: la seqüència
determinant--verb personal no és permesa.

Però de vegades, les restriccions només redueixen l'ambigüitat sense
eliminar-la completament.  Per exemple, la paraula \emph{sobre} pot
ser un substantiu masculí singular, una preposició, i tres formes del
verb \emph{sobrar} (present de subjuntiu, 1a.\ i 3a.\ persona del
singular, i un imperatiu cortés de 3a.\ persona del singular). En el
context \emph{Porta'm aquell sobre de la caixa}, una vegada feta
l'anàlisi morfològica dels mots de l'oració, es podrien aplicar
restriccions lingüístiques basades en seqüències de dues categories
lèxiques per a reduir l'ambigüitat. Per exemple, una preposició no pot
anar seguida d'una altra preposició. Com que \emph{sobre} va seguit de
\emph{de} que només pot ser preposició, podem descartar que
\emph{sobre} siga una preposició. Però no es poden descartar la resta
de formes lèxiques: si \emph{aquell} és un determinant, \emph{sobre}
pot ser un nom (com és el cas); si \emph{aquell} és un pronom,
\emph{sobre} podria ser un verb (com en les oracions \emph{Ara ens han
  sobrat dos cotxes, però pot ser que aquell sobre també més
  endavant.}).

Per tant, cal considerar l'ús de preferències. Per exemple, podem usar
l'aproximació estadística.  Si prenem un corpus (conjunt)
suficientment gran de textos on un expert ha indicat la categoria
lèxica de cada mot i comptem quantes voltes apareixen totes les
seqüències possibles de dues categories lèxiques, podem usar aquestes
freqüències per a assignar la categoria d'un mot ambigu: de totes les
seqüències de tres mots possibles que es puguen formar amb aquest mot,
en prendrem la més freqüent.

\begin{persabermes}{estratègies de resolució de l'ambigüitat}
\paragraph{Resolució de la polisèmia.}
La resolució de la polisèmia (en anglés \emph{word sense
  disambiguation}) consisteix a assignar a un mot polisèmic, en un
text o discurs, una interpretació concreta, possiblement diferent de
les que podria tenir en altres textos (o contextos). La desambiguació
s'efectua usant informació procedent de tres fonts: el \emph{cotext}
(intern al text o discurs) i el \emph{context} (extern al text o
discurs però relacionat amb ell) i fonts de coneixement addicionals.
En traducció automàtica, estem interessats a elegir una de les
interpretacions possibles, perquè és comú que els mots polisèmics
tinguen diverses traduccions (l'\emph{ambigüitat de transferència}
esmentada en l'apartat~\ref{ss:amblex}).

S'accepta comunament que la major part dels mots polisèmics d'un text
(o d'un fragment del text) solen tenir una única interpretació en un
text donat, però aquest principi s'ha de concretar en un mètode
concret per a resoldre la polisèmia.

La resolució de la polisèmia s'ha abordat des de perspectives molt
diverses (vegeu \citet{ide98j}).  És possible aplicar restriccions per
resoldre la polisèmia però s'han de basar en una anàlisi de naturalesa
bastant profunda. Per exemple, podem decidir que el mot espanyol
\emph{gato} és un animal i no una ferramenta per alçar vehicles en la
frase \emph{El gato me miró desde debajo del coche}, perquè
\emph{gato} és el subjecte de \emph{miró}, i el verb \emph{mirar}
requereix un subjecte animat, però com es pot veure, això requereix
que s'hi haja fet una anàlisi sintàctica i semàntica.
 
Per això, en general s'usen mètodes basats en preferències. Heus ací
dos exemples:
\begin{itemize}
\item L'ús de \emph{xarxes semàntiques} on els conceptes se situen en
  els nodes (nusos) de la xarxa i s'agrupen jeràrquicament en
  superconceptes cada vegada més generals (per exemple, els conceptes
  \emph{poma}, \emph{pera}, \emph{taronja} s'agruparien sota el
  concepte \emph{fruita}): un exemple de xarxes semàntiques és
    \emph{Wordnet}, \url{http://wordnet.princeton.edu}, que s'està
    generalitzant a altres llengües d'Europa
    (\url{http://www.illc.uva.nl/EuroWordNet/}).  Quan tenim un mot
  polisèmic, li podem associar més d'un concepte o \emph{sentit}. Per
  a elegir-ne un, podem, per exemple, prendre tots els possibles
  sentits del mot ambigu i assignar-los el sentit associat al concepte
  que està més prop dels conceptes representats pels mots veïns en el
  text.  La informació present en diccionaris electrònics preexistents
  pot servir per a construir aquestes xarxes o ser usada directament
  per a la resolució de la polisèmia.
\item L'estadística d'aparició conjunta de mots en corpus bilingües de
  textos pot ajudar a resoldre directament l'ambigüitat de
  transferència quan es disposa de diccionaris de transferència o quan
  els textos estan alineats. Per exemple, si en un corpus bilingüe
  espanyol--català l'aparició de \emph{destino} prop d'\emph{incierto}
  en espanyol coincideix amb l'aparició de \emph{destí} en català,
  podem dir que el mot \emph{destino} té en aquest cas la
  interpretació de ``sort futura''; en canvi, si l'aparició de {\em
    destino} prop d'\emph{estación} o \emph{aeropuerto} en espanyol
  coincideix amb l'aparició \emph{destinació} en català, podem elegir
  el sentit de ``punt d'arribada''. Aquesta informació podria servir
  per a traduir després de l'espanyol a l'anglés i elegir {\em
    destiny} o \emph{destination} en cada cas amb molta probabilitat
  d'èxit. %\todo{Potser caldria esmentar ací els sistemes de TA
          %estadística basat en frases}
  \end{itemize}

  \paragraph{Resolució de l'anàfora.}
  La resolució de l'anàfora ---és a dir, la determinació de l'{\em
    antecedent} d'un pronom o d'una altra expressió anafòrica--- es
  pot basar també en restriccions i preferències.

  Les \emph{restriccions} es poden basar en informació morfològica,
  sintàctica, o fins i tot semàntica; tot depén del nivell d'anàlisi
  que estiga disponible:
  \begin{itemize}
  \item Un pronom masculí no pot tenir un antecedent femení
    (restricció morfològica): \emph{Maria} no pot ser l'antecedent de
    \emph{ell} en l'oració \emph{Maria es va passar tot el dia parlant
      d'ell}.
  \item La informació sintàctica pot ser més rellevant que no ho
    sembla: si diem
    \begin{exemple}
      Marta la va veure
    \end{exemple} 
    l'antecedent d'\emph{la} no pot ser \emph{Marta}, per causa de les anomenades \emph{barreres}, 
    restriccions associades a determinades característiques de
    l'estructura sintàctica de l'oració.
    En canvi, si diem
    \begin{exemple}
      Marta va parlar amb qui la va veure
    \end{exemple} 
    no es pot descartar completament que l'antecedent de \emph{la}
    siga \emph{Marta}.
  \item Hi ha vegades que només podem recórrer a una anàlisi
    semàntica; en l'exemple (ja discutit en la secció~\ref{ss:UTA})
    \begin{exemple} 
      [Els soldats]$_i$ van disparar [als xiquets]$_j$. Els$_{i/j?}$
      vaig veure caure
    \end{exemple}
    s'ha d'usar informació semàntica per a saber quin és l'antecedent
    d'\emph{els} en la segona oració (\emph{els soldats} o \emph{els
      xiquets}).
  \end{itemize}

  Les restriccions no solen ser suficients, i sol ser necessari
  l'establiment de \emph{preferències}. Per exemple, es poden preferir
  \begin{itemize}
  \item els antecedents més recents,
  \item els antecedents que fan de subjecte als que fan d'objecte, o
  \item els antecedents que han estat introduïts explícitament com
    l'assumpte del discurs o de la conversa: \emph{Doncs, pel que fa a
      \underline{Joan}\ldots}.
  \end{itemize}
  Açò se sol instrumentar a través d'un sistema que assigna
  \emph{puntuacions} per cada una de les característiques: se sumen
  les puntuacions per a tots els antecedents possibles i s'elegeix el
  que obté la puntuació més alta \citep{lappin94j}.

  \paragraph{Resolució de l'ambigüitat estructural.}
  \label{s3:resambest}
  En principi, es podria dir que les persones resolem l'ambigüitat
  estructural ---pura o d'origen categorial--- elegint, usant les
  interpretacions assignades a cada una de les estructures possibles
  (principi de composicionalitat), quines són \emph{acceptables} i,
  entre les acceptables, quina és la més versemblant i per tant
  preferida en una situació comunicativa determinada. Segons aquest
  model, les persones consideraríem sempre \emph{totes} les
  estructures sintàctiques. Es podria argumentar fàcilment en contra
  dient que en frases complexes (per exemple, l'oració~\ref{ex:bossa})
  hi ha massa estructures a considerar. De fet, hi ha experiments
  psicolingüístics que indiquen que de vegades usem estratègies
  fonamentalment sintàctiques, elegint entre les possibles estructures
  fins i tot quan no hem sentit o llegit tota l'oració, potser per
  evitar un esforç intel·lectual excessiu, ja que hi pot haver
  moltíssimes interpretacions parcials. A canvi, hem de fer l'esforç
  (presumiblement més lleuger) de predir una entre les possibles
  continuacions (sintàctiques) del que hem llegit; segons arriben
  mots, els anem encaixant en l'estructura predita i usem la sintaxi i
  la interpretació dels mots per a anar construint a poc a poc la
  interpretació de l'oració completa.  L'experiència ens ajuda a fer
  prediccions que en general tenen èxit, però de vegades hi ha
  oracions ``enganyoses'' que ``ens porten a l'hort'' (anomenades, per
  això, en anglés \emph{garden-path sentences}, de l'anglés \emph{lead
    up the garden path}) ja que en cert punt del procés ens obliguen a
  descartar la predicció feta i reinterpretar el que havíem llegit
  fins a aquell punt (l'estudi dels moviments oculars, en
    anglés \emph{eyetracking}) durant la lectura donen pistes molt
  rellevants sobre l'existència d'aquests processos).  Heus ací alguns
  exemples d'oracions que ``ens porten a l'hort'', amb una continuació
  inesperada en les notes a peu de pàgina del final d'aquesta secció:
  \begin{exemple}
    Joan besà Maria i la seua germana\ldots\footnote{\ldots el va
      recriminar per haver-ho fet.}
  \end{exemple}
  \begin{exemple}
    Com que Joan sempre corre un parell de
    quilòmetres\ldots\footnote{\ldots li semblen poc.}
  \end{exemple}
  \begin{exemple}
    En el otro accidente murieron sesenta y
    cinco\ldots\footnote{\ldots resultaron heridos.}
  \end{exemple}
  \begin{exemple}
    The horse raced by the barn\ldots\footnote{\ldots fell down.}
  \end{exemple}
  Aquests processos de selecció purament sintàctica donen com a
  resultat que hi ha certes estructures finals que són preferides a
  altres, potser perquè simplifiquen la comprensió.  Per exemple, si
  llegim
  \begin{exemple}
    Va aprendre a afaitar-se en dos minuts
  \end{exemple}
  podríem considerar la interpretació que s'hi parla de la durada de
  l'afaitat (el que \emph{va aprendre} és \emph{afaitar-se en dos
    minuts}) com a més probable que la que interpreta que s'hi parla
  de la durada de l'aprenentatge (\emph{aprendre a afaitar-se} li va
  costar \emph{dos minuts}), ja que en el segon cas potser hauria
  estat més natural dir
  \begin{exemple}
    Va aprendre en dos minuts a afaitar-se
  \end{exemple}
  La regla que afavoreix que els adjunts s'associen a l'últim sintagma
  que els admeta ---i que permet, per tant, anar construint l'arbre
  d'anàlisi sintàctica gradualment sense haver de fer-hi grans
  reorganitzacions--- se sol anomenar regla de \emph{clausura tardana}
  ---en anglés \emph{late closure}---; per exemple, aquesta regla
  afavoreix el primer dels arbres de la figura~\ref{fg:noticies}. Una
  altra regla que se sol usar és la d'\emph{adjunció mínima} ---en
  anglés \emph{minimal attachment}--- que afavoreix l'arbre sintàctic
  amb el mínim de nodes (punts de ramificació).  Aquestes estratègies
  són d'utilitat en els sistemes de traducció automàtica per
  transferència sintàctica pura (vegeu la secció~\ref{ss:classtrans}),
  ja que no s'hi fa cap processament semàntic.

  El punt de vista purament sintàctic és pot considerar una
  simplificació excessiva; moltes vegades, les persones resolem
  l'ambigüitat estructural usant restriccions semàntiques o fins i tot
  lèxico-semàntiques:
  \begin{itemize}
  \item Per exemple, el verb \emph{vendre} admet un objecte directe i
    un d'indirecte, però el verb \emph{menjar} només el directe, de
    manera que si diem ``Va presentar l'home que venia taronges a
    Joan'' es pot interpretar de dues maneres degut a l'ambigüitat
    estructural, però si diem la frase estructuralment idèntica ---i
    per tant idènticament ambigua--- ``Va presentar l'home que menjava
    taronges a Joan'' no hi ha més que una interpretació possible.
  \item Considereu aquestes dues frases estructuralment idèntiques
    afectades per la mateixa ambigüitat estructural pura d'adjunció:
    \begin{exemple}
      \label{eq:armari}
      Porta'm les claus de l'armari gran
    \end{exemple}
    \begin{exemple}
      \label{eq:cadira}
      Porta'm les claus de la cadira verda
    \end{exemple}
    En l'oració~\ref{eq:armari}, podem dubtar, ja que no sabem si les
    claus són les que obrin l'armari o les que estan allà
    guardades. En canvi, en l'oració~\ref{eq:cadira} no considerem la
    primera interpretació (encara que siga la preferida sintàcticament
    segons la regla de clausura tardana), perquè no és gens
    versemblant que les cadires tinguen pany (hem usat informació
    semàntica basada en les nostres creences sobre el món). Si el
    sistema que resol l'ambigüitat és capaç d'usar informació
    semàntica, podria elegir correctament en aquest cas.
  \end{itemize}
\end{persabermes}

\section{Qüestions i exercicis}
Per a poder respondre a les preguntes marcades amb (*) cal que us
llegiu els quadres \emph{Per saber més}.

\begin{enumerate}
\item Indiqueu quina classe d'ambigüitat presenten aquestes frases
  (justifiqueu molt breument la vostra resposta): 
  \begin{enumerate}
  \item \emph{Expulsaran l'alcalde de la ciutat} (1: ``L'alcalde de la
    ciutat serà expulsat.''  2: ``L'alcalde serà expulsat de la
    ciutat'').
  \item \emph{Hi havia un gat sota l'automòbil} (1: ``...perquè
    acabaven de reparar una roda''; 2: ``...i va eixir corrents quan
    el vaig posar en marxa'')
  \item \emph{Maria va entrar amb una bossa gran. Jo la vaig posar
      damunt de la taula} (1: ``Vaig posar Maria damunt de la taula'';
    2: ``Vaig posar la bossa damunt de la taula'')
  \item \emph{Què vols, galetes o pa de la tia Pepa?} (Les galetes són
    també de la tia Pepa?)
  \item \emph{Posa una mà de paper en la impressora i connecta-la.}
    (Ha de connectar la mà de paper o la impressora?)
  \item \emph{Us han dit que vaja?} (Qui ha d'anar?).
  \item \emph{Vale más que las comas} (1: ``...que els signes de
    puntuació''; 2:``...que les menges'')
  \item \emph{El mecánico revisó la suspensión del auto de Garzón} (1:
    ``Aquest mecànic és un expert en legislació i s'ha llegit la
    resolució judicial sencera''; 2: ``Els amortidors del cotxe de
    Garzón ja necessitaven una revisió'').
  %\item ``Mi vecino aseguró que había bajado la bolsa'' (1: La bossa
  %  d'escombraries ja és al contenidor; 2: Les meues accions de borsa
  %  no deixen de donar-me disgustos).
  %\item \emph{A qui has dit que telefonarien?}  (1: ``Per a qui seria la
  %  telefonada que has dit que farien?'' 2: ``A qui has dit que es
  %  produiria una telefonada?'')
  \item \emph{A pesar de haber sido soldado, salió despedido del
      avión} (1: ``El sistema fotogràfic estava fortament fixat al
    fuselatge però es va soltar de l'aparell quan l'avió va girar en
    ple vol''; 2: ``Malgrat el seu passat militar gloriós, el
    president el va destituir abans d'arribar a l'aeroport de
    destinació'').
  \item \emph{Els lladres van ser atrapats a una fàbrica incendiada per un
    policia} (1: ``Els lladres van ser capturats pel comissari a una
    fàbrica abandonada''; 2: ``La fàbrica on van ser capturats va ser
    l'objectiu d'un agent piròman'').
  \item \emph{Coto privado de caza} (1: ``Aquesta àrea de caça no és
    pública''; 2: ``Aquesta és una àrea sense caça'').
  \item \emph{Vull ballar i cantar cançons de bressol} (1: ``Ballarem
    durant una estona i després et cantaré perquè et dormes''; 2:
    ``M'estime tant ballar cançons de bressol com cantar-ne'').
  \item \emph{El manifestant es refà de la pallissa que li van donar en
    l'hospital} (1: ``La manifestació va ser una mica violenta i
    algunes persones han hagut de ser traslladades a l'hospital''; 2:
    ``Quina pallissa va donar-li l'infermera al quiròfan!'').
  \item \emph{Serviran polp a la gallega} (1: ``Serviran polp a una senyora
    de Galícia''; 2: ``Serviran polp preparat a l'estil gallec'').
  \item \emph{No puc veure bé la foto que m'has enviat per correu
      electrònic perquè no puc tancar totes les finestres} (1:
    ``Encara hi entra sol i es reflecteix en la pantalla''; 2: ``Tinc
    l'escriptori ple de documents oberts'').
    % \item ``Ens va explicar com deien que vindria'' (1: Ens va
    %   explicar de quina manera ho deien; 2: Ens va explicar de quina
    %   manera vindria segons deien'').
  \item \emph{--Hem rebut notícies que diuen que, per causa de
      l'humitat i la calor en l'interior del temple, els bancs i els
      altars de fusta han rebrotat i els han crescut branques i
      fulles.  --I les creus?} (1: ``Creus aquestes notícies?'' 2:
    ``Els crucifixos també han rebrotat?'').
  \item \emph{S'han de repassar les entrades i les despeses que
      s'hagen fet en euros} (1: ``Les entrades s'han de repassar
    totes; les despeses només si s'han fet en euros''; 2: ``De les
    entrades, se n'han de repassar només les fetes en euros'').
  \item \emph{Després que la venedora acabà la descripció dels
      avantatges de la urbanització projectada, el comentari unànime
      dels inversors va ser que la trobaven molt interessant}. (1:
    ``Els inversors, la veritat, prestaven més atenció a la venedora
    que al producte''; 2: ``Els va agradar l'estil de la descripció'';
    3: ``La venedora no parlava clar, la descripció era incompleta,
    però a pesar de tot, la urbanització era una inversió
    prometedora'').
  %\item Els professors han renyat els alumnes. És que, quan es
  %  barallen, se'ls nota. (1: Els professors estan barallats i al
  %  final els alumnes ho paguen; 2: La baralla d'alumnes havia acabat,
  %  però els professors ho van detectar i els van recriminar).
  \item \emph{A la trapezista, últimament, no li eixien els números}
    (1: ``Sempre acabava caent a la xarxa''; 2: ``tenia més despeses
    que ingressos'').
    % \item ``Van fer una manifestació pel parc científic'' (1: Van
    %   desfilar a través del parc científic; 2: La manifestació era a
    %   favor del parc científic)
    % \item ``Els assassins són policies o milicians rebels'' (Els
    %   policies eren també rebels?)
    % \item ``Com penses que han arribat al cim?'' (1: Com és possible
    %   que penses això? 2: Saps de quina manera hi han arribat?)
    % \item ``Ocultó el rollo para que nadie conociese su verdadera
    %   orientación sexual'' (1: Les fotos, si les revelaven, podien
    %   descobrir les seues preferències sexuals; 2: aquella relació
    %   deixaria clares les seues orientacions sexuals)
    % \item ``Els professors acompanyaven els xiquets quan els
    %   policies els van cridar'' (A qui van cridar?)
    % \item ``És difícil però no impossible'' (Què és difícil?)
    % \item ``Será mejor que verse sobre la nueva construcción'' (1:
    %   Més val que tracte sobre el mateix tema; 2: és millor citar-se
    %   en aquell altre lloc).
    % \item ``Aureli explicarà les històries que ha escoltat a Martí''
    %   (1: li les explicarà a Martí; 2: li les ha escoltades a
    %   Martí);
    % \item ``Com dius que l'has matat?'' (1: Com l'has matat?; 2: Com
    %   dius això?)
    % \item ``Un problema de llengua li impedeix parlar bé'' (1: el
    %   problema és físic, anatòmic; 2: el problema és lingüístic,
    %   psicològic)
    % \item ``Li he dit que vindria més tard'' (qui?)
    % \item ``Connecteu el ratolí a l'ordinador i activeu-lo'' (1:
    %   activeu el ratolí; 2: activeu l'ordinador).
    % \item ``Han trobat gerres i àmfores ibèriques'' (les gerres
    %   també són ibèriques?).
    % \item ``Les gallines han destrossat el sembrat, però no les
    %   mates'' (1: no les sacrifiques; 2: no les plantes).
  \end{enumerate}

\item (*) Hi ha ambigüitats de tipus lèxic que poden ser sempre
  correctament resoltes després de fer una anàlisi morfològica. No
  obstant això, hi ha d'altres que només poden ser tractades si es fa
  una anàlisi sintàctica (tot i que l'ambigüitat siga de tipus lèxic)
  i fins i tot n'hi ha que requeririen una anàlisi semàntica per a
  resoldre-les.
    
  Elegiu una llengua origen i una llengua meta (francés, anglés,
  alemany, català o espanyol) i poseu un exemple d'oració per a cada
  un dels tres casos anteriors, on siga necessari un determinat nivell
  d'anàlisi per tal de resoldre una ambigüitat i produir-ne la
  traducció correcta. Expliqueu quina informació usa el sistema en
  cada cas per a prendre una decisió.
      
\item (*) Indiqueu breument quines estratègies es podrien usar per a
  resoldre l'ambigüitat sintàctica d'adjunció.  Per a inspirar-vos,
  fixeu-vos en els següents exemples:
  \begin{itemize}
  \item \emph{Va aprendre en dos minuts a afaitar-se}
  \item \emph{Va aprendre a afaitar-se en dos minuts} 
  \item \emph{Porta'm les claus de l'armari gran}
  \item \emph{Porta'm de l'armari gran les claus}
  \item \emph{Porta'm les claus de la cadira verda}
  \item \emph{Toni comprarà les taronges que ha de vendre a Reme}
  \item \emph{Toni comprarà a Reme les taronges que ha de vendre} 
  \end{itemize}

\item Si una oració té només una ambigüitat lèxica pura\ldots
  \begin{enumerate}
  \item \ldots té un únic arbre d'anàlisi sintàctica, però més d'una
    anàlisi morfològica.
  \item \ldots té un únic arbre d'anàlisi sintàctica i una única
    anàlisi morfològica, però dues interpretacions se\-màn\-ti\-ques
    diferents.
  \item \ldots té més d'un arbre d'anàlisi sintàctica.
  \end{enumerate}

\item La frase \emph{M'agrada més que la bata} pot tenir dues
  interpretacions; en la primera es parla d'una peça de vestir; en la
  segona, d'una preferència a l'hora de preparar, per exemple, una
  salsa. Indiqueu de quina classe d'ambigüitat es tracta.
  \begin{enumerate}
  \item Estructural d'adjunció.
  \item Lèxica categorial.
  \item Estructural d'origen categorial.
  \end{enumerate}

\item La frase \emph{Baixa i puja amb ascensor} pot voler dir ``(baixa) i
  (puja amb ascensor)'' o ``(baixa i puja) amb ascensor''.  De quin
  tipus d'ambigüitat es tracta?
  \begin{enumerate}
  \item Lèxica categorial.
  \item Estructural d'origen categorial.
  \item Estructural d'origen coordinatiu.
  \end{enumerate}

\item En l'oració \emph{El cotxe s'ha cremat amb el garatge i
    l'assegurança no el cobreix} no se sap quina de les dues coses
  està coberta per l'assegurança, el garatge o el
  cotxe. L'ambigüitat...
  \begin{enumerate}
  \item ... es deu a l'el·lipsi.
  \item ... es deu a l'anàfora.
  \item ... és estructural d'origen coordinatiu.
  \end{enumerate}

\item De quina classe és l'ambigüitat de l'oració \emph{Va vendre les
  taronges que havia comprat a Maria}?
  \begin{enumerate}
  \item Estructural d'origen coordinatiu.
  \item Estructural d'adjunció.
  \item Extrasentencial per anàfora.
\end{enumerate}

\item (*) Considereu l'homògraf espanyol \emph{vendo} (``Te vendo un
  coche'' ``Y yo, ¿para qué quiero un coche vendado?''). Es pot
  resoldre l'ambigüitat lèxica a què dóna lloc usant només informació
  sintàctica (és a dir, sobre les categories lèxiques que l'acompanyen
  en l'oració)?
  \begin{enumerate}
  \item No, perquè les dues formes \emph{vendo} s'escriuen exactament
    igual.
  \item No, perquè les dues formes \emph{vendo} tenen la mateixa
    categoria lèxica i la mateixa anàlisi morfològica, tret del lema,
    i, per tant, poden fer exactament les mateixes funcions
    sintàctiques.
  \item Sí, només mirant la categoria lèxica dels mots anteriors i la
    dels posteriors ja hi ha prou per a saber en quin dels dos casos
    ens trobem.
  \end{enumerate}

% \item Imagineu que teniu un text de 1.000.000 de mots en l'idioma
%   \emph{séverla} i un diccionari de correspondències
%   espanyol--\emph{séverla}. El diccionari estableix (entre d'altres)
%   les correspondències següents (s'hi indica la freqüència del mot en
%   el text en \emph{séverla}).\todo{Relacionar amb els models de
%     llengua quan parlem de TA estadística?}
%   \begin{center}
%     \begin{tabular}{l|l|l}
%       \hline\hline
%       \textsc{Espanyol} & \textsc{séverla} & \textsc{freqüència} \\
%       \hline
%       \textsf{asistido, -a} & \textsf{aduya} & 123 \\
%       \textsf{aspecto} & \textsf{atnip} & 44 \\ 
%       \textsf{dirección} & \textsf{obmur} & 150  \\
%       \textsf{dirección} & \textsf{odnam} & 55 \\
%       \textsf{dirección} & \textsf{ragul} & 128 \\
%       \textsf{diseño} & \textsf{atnip} & 43 \\
%       \textsf{diseño} & \textsf{raerc} & 100 \\
%       \textsf{electrónico, -a} & \textsf{dered} & 100 \\
%       \textsf{general} & \textsf{latot}  & 43 \\
%       \textsf{moderno, -a} & \textsf{aídla} & 56 \\
%       \textsf{postal} & \textsf{atrac} & 188 \\
%       \hline
%     \end{tabular}
%   \end{center}
%   En el text en \emph{séverla} trobem, a més, les freqüències següents
%   per a grups de dos mots (només es llisten els grups que apareixen
%   alguna vegada):
%   \begin{center}
%     \begin{tabular}{l|l}
%       \hline\hline 
%       \textsc{Grup} & \textsc{freqüència} \\\hline
%       \textsf{atrac ragul} & 25 \\
%       \textsf{atrac odnam} & 3 \\
%       \textsf{dered ragul} & 12 \\
%       \textsf{dered odnam} & 2 \\
%       \textsf{latot odnam} & 10 \\
%       \textsf{latot atnip} & 8 \\
%       \textsf{aduya odnam} & 10 \\
%       \textsf{aídla raerc} & 3 \\
%       \textsf{aídla atnip} & 15 \\
%       \textsf{aduya raerc} & 12 \\
%       \hline
%     \end{tabular}
%   \end{center}
%   Indiqueu com traduïríeu les següents expressions espanyoles al
%   séverla i perquè heu elegit aqueixa traducció i no una altra:
%   \begin{enumerate}
%   \item \textsf{dirección postal}
%   \item \textsf{dirección electrónica}
%   \item \textsf{dirección asistida}
%   \item \textsf{dirección general}
%   \item \textsf{diseño moderno}
%   \item \textsf{diseño asistido}
%   \end{enumerate}

% \item Imaginem que tenim textos en la llengua $L_1$ i les traduccions
%   (correctes) corresponents en la llengua $L_2$, fetes per una persona
%   de manera que s'hi respecte tant com siga possible l'estructura de
%   les frases del text original. Indiqueu com es podria usar la
%   informació present en aquestes traduccions per a resoldre les
%   ambigüitats que es troba un sistema de traducció automàtica quan
%   tradueix \emph{els mateixos textos} a una altra llengua $L_3$
%   (Martin Kay, de Xerox Palo Alto, Califòrnia, anomena aquest procés
%   \emph{triangulació en la traducció}). Si voleu, fixeu-vos només en
%   les ambigüitats lèxiques.

\item (*) Moltes vegades, l'ambigüitat lèxica no és ni polisèmia
  (\emph{estació}, \emph{bomba}), ni ambigüitat lèxica amb canvi de
  categoria gramatical (\emph{sobre} [preposició, substantiu i, en la
  varietat valenciana, verb], \emph{riu} [substantiu i verb]) sinó que
  succeeix perquè dues formes \emph{de la mateixa categoria lèxica}
  són homògrafes: \emph{volem} pot ser una forma del verb \emph{volar}
  i del verb \emph{voler}; \emph{podeu} pot ser una forma del verb
  \emph{poder} i del verb \emph{podar}; en espanyol, \emph{creo} és
  una forma de \emph{creer} o de \emph{crear}, \emph{fui} és una forma
  de \emph{ir} o de \emph{ser}, etc. Per a resoldre l'ambigüitat d'un
  mot polisèmic s'ha d'usar informació semàntica; per a resoldre
  l'ambigüitat lèxica categorial sol ser suficient usar informació
  sintàctica (per exemple, la categoria gramatical del mots anterior i
  posterior); però, és possible resoldre l'ambigüitat deguda a
  l'homografia de mots de la mateixa categoria usant només la sintaxi
  o és necessari l'ús d'informació semàntica?

\item Els sistemes de traducció mot per mot poden cometre, per
  exemple, errors deguts a l'elecció incorrecta de la categoria
  gramatical d'un mot lèxicament ambigu. Elegiu dues llengües, $L_1$ i
  $L_2$ i poseu dos exemples de traduccions errònies de $L_1$ a $L_2$,
  indicant la frase original, la frase mal traduïda i la frase
  correcta.

\item Si una forma superficial és ambigua però té només una forma
  lèxica{\ldots}   
  \begin{enumerate}
  \item {\ldots} es tracta d'un mot homògraf.
  \item {\ldots} hi ha algun error en l'anàlisi morfològica.
  \item {\ldots} podem dir que el lema és polisèmic.
  \end{enumerate}

\item Pot una oració tenir més d'una traducció a un altre idioma
  malgrat estar formada completament per mots que no són ni homògrafs
  ni polisèmics en la llengua original?
  \begin{enumerate}
  \item No: ho prohibeix el principi de composicionalitat semàntica.
  \item Sí, encara que no continga pronoms o altres expressions
    anafòriques susceptibles de tenir més d'un antecedent possible.
  \item Sí, però només si conté pronoms o altres expressions
    anafòriques susceptibles de tenir més d'un antecedent possible.
  \end{enumerate}

\item (*) En absència d'informació lèxica o semàntica, l'ambigüitat
  estructural{\ldots}
  \begin{enumerate}
  \item {\ldots} és impossible de resoldre.
  \item {\ldots} es pot resoldre usant regles derivades d'un estudi de
    les preferències sintàctiques observades en experiments
    psicolingüístics.
  \item {\ldots} no pot afectar mai el resultat de la traducció
    automàtica.
  \end{enumerate}

\item (*) És possible resoldre en alguns casos l'ambigüitat deguda a un
  mot homògraf utilitzant exclusivament informació morfològica?
  \begin{enumerate}
  \item No, aquest tipus d'ambigüitat exigeix un tractament semàntic
    com a mínim.
  \item No, sempre cal utilitzar informació de caràcter sintàctic per
    resoldre-la.
  \item Sí, usant la informació morfològica dels mots adjacents.
  \end{enumerate}

\item Si un mot té només una forma lèxica i una única traducció a una
  determinada llengua, pot ser encara ambigu?
  \begin{enumerate}
  \item No.
  \item Sí, pot ser polisèmic encara que la traducció a aquesta
    llengua de totes les interpretacions siga la mateixa.
  \item Sí; pot ser homògraf i tractar-se d'un passi gratuït.
  \end{enumerate}

\item Si traduim automàticament la frase espanyola \emph{Ayer cantamos
    las mismas canciones} i obtenim en anglés \emph{Yesterday we sing
    the same songs} o en francés \emph{Hier nous chantons les mêmes
    chansons}, quin tipus d'ambigüitat ha estat mal resolta?
  \begin{enumerate}
  \item Una ambigüitat lèxica per homografia d'un mot.
  \item Una ambigüitat lèxica pura per polisèmia d'un mot.
  \item Una anàfora.
  \end{enumerate}

\item Si traduim automàticament la frase catalana \emph{Hilari no coneix
  bé la Mariona: cada dia troba sorprenent el que fa} i obtenim en
  anglés \emph{Hilari does not know Mariona well: every day he finds what
  he does astonishing} o en francés \emph{Hilari ne connaît pas bien
  Mariona: tous les jours elle trouve ce qu'il fait etonnant}, quin
  tipus d'ambigüitat ha estat mal resolta?
  \begin{enumerate}
  \item L'anàfora d'un pronom buit.
  \item L'anàfora del pronom \emph{que}.
  \item Una ambigüitat sintàctica de l'oració subordinada ``el que
    fa''.
  \end{enumerate}

\item El principi de composicionalitat diu que la interpretació d'una
  oració està determinada per les interpretacions dels mots i per la
  sintaxi. Quan es produeix una ambigüitat perquè no queda clara
  l'adscripció d'un sintagma preposicional, com en \emph{porta la clau
    de l'armari gran}, quina n'és la raó?  
  \begin{enumerate}
  \item La existència de més d'una estructura sintàctica possible.
  \item L'ambigüitat categorial de la preposició.
  \item La polisèmia de la preposició.
  \end{enumerate}

% \item En les frases comparatives de l'estil de
%   \begin{center}\emph{Els processadors processen més ràpidament les
%       dades que els dispositius perifèrics}\end{center} què fa que
%   siga possible la existència de dues interpretacions (els
%   \emph{dispositius perifèrics} són processats o processen?).
%   \begin{enumerate}
%   \item L'ambigüitat d'adjunció del sintagma que comença per
%     \emph{que}.
%   \item L'elipsi d'un sintagma nominal, que queda substituït per un
%     pronom nul PRO.
%   \item L'elipsi simultània d'un sintagma nominal i un sintagma
%     verbal, que fa dubtosa l'adscripció del sintagma nominal restant.
%   \end{enumerate}

\item L'ambigüitat lèxica categorial d'un mot{\ldots}
  \begin{enumerate}
  \item {\ldots}no es pot resoldre mai si no s'usa informació
    semàntica sobre el text o sobre els mots contigus.
  \item {\ldots}no es pot resoldre si no es fa l'anàlisi sintàctica
    completa de la frase, ja aquesta és l'única manera d'elegir
    l'anàlisi morfològica correcta.
  \item {\ldots} s'intenta resoldre normalment amb regles basades en
    les categories lèxiques dels mots que l'acompanyen en la frase.
  \end{enumerate}

\item Quin tipus d'ambigüitat es produeix en el pronom feble \emph{li}
  de la frase \emph{Vaig veure Mario i la seua mare; li vaig dir que
    m'agradava molt el seu fill}?
  \begin{enumerate}
  \item Homografia, perquè el pronom pot, en principi, estar
    substituint un nom o un altre.
  \item Una anàfora.
  \item Polisèmia.
  \end{enumerate}

\item Una d'aquestes tres no és un tipus d'ambigüitat lèxica:
  \begin{enumerate}
  \item L'homografia.
  \item L'ambigüitat d'adjunció.
  \item La polisèmia.
  \end{enumerate}

\item De quina classe és l'ambigüitat que presenta l'oració
  \emph{Expulsaran el portaveu del partit}?
  \begin{enumerate}
  \item Lèxica, deguda al fet que el mot \emph{expulsar} és polisèmic.
  \item Estructural d'origen coordinatiu: no sabem si el sintagma
    preposicional \emph{del partit} modifica només al segon element
    \emph{el portaveu} o a tot el sintagma \emph{Expulsaran el
      portaveu}.
  \item Estructural d'adjunció: el sintagma preposicional \emph{del
      partit} pot ser un adjunt del sintagma verbal \emph{Expulsaran
      el portaveu} o del sintagma nominal \emph{el portaveu}.
  \end{enumerate}

\item Només una d'aquestes tres afirmacions és certa. Quina?
  \begin{enumerate}
  \item Una oració pot ser ambigua sense que cap dels seus mots siga
    ambigu per ell mateix.
  \item Una oració només pot ser ambigua si almenys un dels seus mots
    és ambigu.
  \item El fet que una oració siga ambigua implica necessàriament que
    les traduccions de les diverses interpretacions a una altra
    llengua han de ser diferents.
  \end{enumerate}

\item Quin tipus d'ambigüitat es dóna en l'oració \emph{La mata
    l'enveja} (1: ``l'enveja l'està matant''; 2: ``la planta li té
  enveja a ella'')?
  \begin{enumerate}
  \item Ambigüitat lèxica per polisèmia.
  \item Ambigüitat lèxica deguda a l'anàfora.
  \item Ambigüitat estructural deguda a l'ambigüitat lèxica
    categorial.
  \end{enumerate}

\item Quan un adjectiu presenta la mateixa ambigüitat (per exemple,
  pot tenir més d'una traducció), independentment de com es trobe
  flexionat en gènere i nombre, direm que l'adjectiu és{\ldots}
  \begin{enumerate}
  \item {\ldots} anafòric.
  \item {\ldots} homògraf.
  \item {\ldots} polisèmic.
  \end{enumerate}
 
\item Segons Arnold (2003) els problemes als quals s'enfronta la
  traducció automàtica són quatre. Indiqueu quina de les afirmacions
  següents és falsa:
  \begin{enumerate}
  \item El problema de l'anàlisi es refereix a la dificultat per
    resoldre l'ambigüitat d'un enunciat.
  \item El problema de la síntesi es refereix a l'ambigüitat dels
    textos traduïts automàticament.
  \item El problema de la descripció consisteix en el fet que és
    impracticable descriure de forma suficient i computacionalment
    eficient tot el coneixement necessari per traduir.
  \end{enumerate}

\item Un traductor automàtic per transferència morfològica avançada
  \ldots 
  \begin{enumerate}
  \item \ldots resol la polisèmia mitjançant l'ús d'un analitzador
    morfològic.
  \item \ldots resol la polisèmia mitjançant l'ús d'un desambiguador
    lèxic categorial.
  \item \ldots no pot resoldre la polisèmia amb cap dels programes
    esmentats en les altres dues opcions.
  \end{enumerate}

\item Quin tipus d'ambigüitat es dóna en l'oració ``\emph{Aston Family
    Man era el baix de The Wailers}'' (Aston era el més baix del grup;
  Aston tocava el baix en el grup)? 
  \begin{enumerate}
  \item Ambigüitat lèxica per polisèmia.
  \item Ambigüitat lèxica categorial dins de la mateixa categoria
    lèxica.
  \item Ambigüitat lèxica categorial entre categories lèxiques
    diferents.
  \end{enumerate}

\item Indiqueu quina de les afirmacions següents és falsa:
  \begin{enumerate}
  \item Hi ha casos en els quals no fa falta resoldre l'ambigüitat per
    produir una traducció adequada en la llengua meta.
  \item L'ambigüitat sempre representa un problema a l'hora de
    traduir entre dues llengües.
  \item L'ambigüitat és un dels problemes als quals ha d'enfrontar-se
    un traductor automàtic.
  \end{enumerate}
\end{enumerate}

\section{Solucions}

\begin{enumerate}
\item \begin{enumerate}
  \item Ambigüitat sintàctica o estructural (pura) d'adjunció: el
    sintagma preposicional \emph{de la ciutat} es pot inserir en dues
    posicions diferents de l'oració: pot modificar \emph{alcalde} o
    \emph{expulsaran [l'alcalde]}.
  \item Ambigüitat lèxica pura (polisèmia) del mot \emph{gat}.
  \item Ambigüitat lèxica per anàfora: el pronom \emph{la} pot tenir
    dos antecedents: \emph{Maria} i la \emph{bossa}.
  \item Ambigüitat sintàctica o estructural (pura) d'origen
    coordinatiu: el sintagma \emph{de la tia Pepa} pot modificar als
    dos sintagmes nominals coordinats (\emph{galetes i pa}) o només al
    segon (\emph{pa}).
  \item Ambigüitat lèxica per anàfora: el pronom \emph{la} pot tenir
    dos antecedents: \emph{mà [de paper]} i la \emph{impressora}.
  \item Ambigüitat per el·lipsi: el subjecte de \emph{vaja} pot ser
    \emph{jo}, \emph{ell}, \emph{ella}, etc. En el cas dels pronoms de
    tercera persona, la interpretació estarà determinada pels
    antecedents que se'ls assignen (ambigüitat lèxica per anàfora).
  \item Ambigüitat sintàctica o estructural d'origen categorial deguda
    al fet que els mots \emph{las} (article o pronom) i \emph{comas}
    (substantiu o verb) són homògrafs afectats d'ambigüitat lèxica
    categorial. De les quatre combinacions possibles, dues són
    sintàcticament acceptables.
  \item L'oració és ambigua perquè dos dels seus mots presenten
    ambigüitat lèxica pura (polisèmia): \emph{auto} pot ser un
    automòbil o un tipus de resolució judicial; \emph{suspensión} pot
    ser l'acció de suspendre (la resolució) o el sistema d'amortidors
    de l'automòbil. De les quatre combinacions possibles, dues tenen
    un cert sentit.  
%  \item Ambigüitat estructural amb el·lipsi i moviment: en una
%    interpretació \emph{la bolsa} pot ser l'objecte de l'acció
%    (transitiva) de baixar si el subjecte (nul) és el mateix que
%    \emph{mi vecino} (``Mi vecino aseguró que (él mismo) había bajado
%    la bolsa''); en l'altra, \emph{la bolsa} és el subjecte (mogut a
%    posició postverbal) de l'acció intransitiva de baixar.

%  \item Ambigüitat estructural pura per moviment de Qu. El SP \emph{a
%      qui} pot ser l'objecte indirecte del verb telefonar en l'oració
%    subordinada (\emph{Has dit [que [PRO telefonarien a qui]]}) o
%    l'objecte indirecte del verb dir en l'oració principal (\emph{Has
%      dit [que [PRO telefonarien]] a qui}).
  \item L'oració és ambigua per l'ambigüitat lèxica (homografia) de
    \emph{soldado}. En la primera interpretació és un participi en la
    forma passiva \emph{haber sido soldado}; en la segona és un
    substantiu masculí singular. També hi intervé l'ambigüitat lèxica
    pura (polisèmia) de \emph{despedir} (en la primera \emph{llançar};
    en la segona, \emph{deixar sense treball}). Dues de les quatre
    combinacions tenen sentit.
  \item Ambigüitat estructural pura d'adjunció. El sintagma
    preposicional \emph{per un policia} pot modificar el sintagma
    verbal \emph{atrapats a una fàbrica incendiada} o només el
    sintagma verbal \emph{incendiada}.
  \item Ambigüitat mixta. D'una banda, lèxica: el mot \emph{privado}
    pot ser un adjectiu (interpretació 1) o un participi
    (interpretació 2). D'altra banda, estructural: en la primera
    interpretació, el sintagma preposicional \emph{de caza} modifica
    el sintagma nominal \emph{coto privado} (\emph{[[coto privado] [de
      caza]]}); en el segon, només el participi \emph{privado}
    (\emph{[[coto] [[privado] [de caza]]]}).
  \item Ambigüitat estructural d'origen coordinatiu. El sintagma
    nominal \emph{cançons de bressol} pot modificar només el segon
    sintagma verbal \emph{cantar} o el sintagma verbal complet
    \emph{ballar i cantar} (és a dir, \emph{cançons de bressol} pot
    ser objecte directe només del segon verb o dels dos).
      \item Ambigüitat estructural pura d'adjunció. El sintagma
      preposicional \emph{a l'hospital} pot modificar el sintagma
      verbal \emph{li van donar}  o el sintagma verbal \emph{es refà de
      la pallisa que li van donar}.
  \item Ambigüitat estructural d'adjunció: el sintagma preposicional
    \emph{a la gallega} pot modificar el nom \emph{polp} per a formar
    sintagma nominal \emph{polp a la gallega} o modificar el sintagma
    verbal \emph{serviran polp} (amb nucli \emph{serviran}) i formar
    el sintagma verbal \emph{serviran polp a la gallega}.
  \item L'oració és ambigua per polisèmia (ambigüitat lèxica) del
    substantiu \emph{finestra} (de la paret/ del sistema operatiu)
  % \item Ambigüitat estructural per moviment de constituents (en
  %   concret, per moviment de l'element Qu \emph{com}). Aquest element
  %   pot modificar els verbs \emph{deien} i \emph{vindria}.  Les
  %   estructures, abans del moviment, i de la formació de l'estructura
  %   de relatiu, es podrien representar així:
  %   \begin{itemize}
  %   \item Ens va explicar \emph{de quina manera}. Deien
  %     \emph{d'aquella manera} que vindria.
  %   \item Ens va explicar \emph{de quina manera}. Deien que vindria
  %     \emph{d'aquella manera}.
  %    \end{itemize}
  %    Noteu que l'el·lipsi dels subjectes no hi està involucrada perquè
  %    l'element Qu que es mou no és un sintagma nominal.
   \item Ambigüitat estructural d'origen categorial. En la primera
     interpretació \emph{les} és un pronom i \emph{creus} és un verb,
     i formen junts un sintagma verbal; en la segona \emph{les} és un
     article i \emph{creus} és un substantiu i formen junts un
     sintagma nominal.
   \item Ambigüitat estructural d'adjunció. El sintagma (oració
     subordinada de relatiu) \emph{que s'hagen fet en euros} pot
     modificar al segon sintagma nominal \emph{les despeses} o al
     sintagma nominal complet \emph{les entrades i les despeses}.
   \item Ambigüitat de l'oració deguda a l'ambigüitat lèxica per
     anàfora.  El pronom \emph{la} pot tenir tres antecedents:
     \emph{la venedora}, \emph{la descripció} o \emph{la urbanització
       projectada} i, per tant, tres interpretacions diferents.
   %\item Ambigüitat de l'oració deguda a l'ambigüitat lèxica per
   %  anàfora: el pronom \emph{'ls} i el pronom nul PRO que fa de
   %  subjecte de \emph{es barallen} poden tenir els antecedents
   %  \emph{els professors} i \emph{els alumnes} i, per tant, poden
   %  tenir interpretacions diferents.
\item Ambigüitat de l'oració per polisèmia (ambigüitat lèxica) del mot
  \emph{número} (part d'una actuació / comptes econòmics)
  % \item ``Van fer una manifestació pel parc científic'' (1: Van
  %   desfilar a través del parc científic; 2: La manifestació era a
  %   favor del parc científic)
  % \item ``Els assassins són policies o milicians rebels'' (Els
  %   policies eren també rebels?)
  % \item ``Com penses que han arribat al cim?'' (1: Com és possible
  %   que penses això? 2: Saps de quina manera hi han arribat?)
  % \item ``Ocultó el rollo para que nadie conociese su verdadera
  %   orientación sexual'' (1: Les fotos, si les revelaven, podien
  %   descobrir les seues preferències sexuals; 2: aquella relació
  %   deixaria clares les seues orientacions sexuals)
  % \item ``Els professors acompanyaven els xiquets quan els policies
  %   els van cridar'' (A qui van cridar?)
  % \item ``És difícil però no impossible'' (Què és difícil?)
  % \item ``Será mejor que verse sobre la nueva construcción'' (1: Més
  %   val que tracte sobre el mateix tema; 2: és millor citar-se en
  %   aquell altre lloc).
  % \item Ambigüitat estructural pura d'adjunció: el sintagma
  %   preposicional ``a Martí'' pot modificar a ``explicarà'' (oració
  %   principal) o a ``ha escoltat'' (oració subordinada).
  % \item Ambigüitat estructural pura (d'adjunció): l'adverbi
  %   interrogatiu ``com'' pot pertànyer a l'oració principal o pot
  %   pertànyer a la subordinada i haver estat desplaçat
  %   obligatòriament (``moviment de Qu'') a la posició inicial.
  % \item Ambigüitat lèxica del mot ``llengua''.
  % \item Ambigüitat deguda a l'el·lipsi: el subjecte del verb
  %   ``vindria'' pot ser ``jo'', ``ell'', ``ella''.
  % \item Ambigüitat extrasentencial deguda a l'anàfora: el pronom
  %   ``lo'' pot tenir dos antecedents diferents: ``l'ordinador'' i
  %   ``el ratolí''.
  % \item Ambigüitat estructural pura d'origen coordinatiu:
  %   ``ibèriques'' pot modificar tot el sintagma nominal coordinat o
  %   només el segon.
  % \item Ambigüitat estructural d'origen coordinatiu, deguda a dues
  %   ambigüitats lèxiques categorials: ``les'' (pronom o article) i
  %   ``mates'' (substantiu i verb); de les quatre combinacions, dues
  %   són sintàcticament acceptables. 
\end{enumerate}

\item (*) Exemples de l'espanyol al català:
  \begin{itemize}
  \item Ambigüitat que es pot resoldre després de fer una anàlisi
    morfològica de l'oració: en \emph{mi trabajo}, l'homògraf
    \emph{trabajo} pot ser substantiu (català \emph{treballo}) o verb
    (català \emph{treball}), però la presència de \emph{mi}
    (determinant possessiu) desambigua l'homògraf perfectament.
  \item Ambigüitat lèxica que necessita una anàlisi sintàctica per a
    ser resolta: l'expressió multimot \emph{sesenta y cinco} pot ser
    un únic numeral (català \emph{seixanta-cinc}) o dos numerals
    coordinats (català \emph{seixanta i cinc}). L'anàlisi morfològica
    no és suficient per a detectar que en la frase \emph{En el lugar
      donde murieron sesenta y cinco quedaron restos} és el primer cas
    i en la frase \emph{Murieron sesenta y cinco quedaron malheridos}
    és el segon.
  \item Ambigüitat lèxica que necessita una anàlisi semàntica per a
    resoldre-la: el mot polisèmic \emph{destino} en l'oració \emph{El
      destino estaba escrito en el pasaje arrugado que encontraron} es
    traduiria pel català \emph{destinació} i en canvi en l'oració
    \emph{El destino estaba escrito en el libro sagrado que
      encontraron} es traduiria pel català \emph{destí}; l'elecció
    exigeix identificar relacions semàntiques entre les
    interpretacions dels mots.
  \end{itemize}

\item (*) Vegeu el quadre \emph{Per saber més} de la
  secció~\ref{s3:resambest}.  En el cas de les frases ``Toni
  comprarà...'', sembla lògic usar la regla de {\em clausura tardana},
  ja que es correspon prou bé amb les interpretacions preferides per
  les persones.

\item (b)
\item (c). Els mots \emph{la} i \emph{bata} poden pertànyer cada un a
  dues categories lèxiques diferents. De les quatre combinacions
  resultants, dues són sintàcticament acceptables.
\item (c)
\item (b). El pronom feble \emph{el} pot referir-se a \emph{garatge} o
  a \emph{cotxe}.
\item (b). El sintagma preposicional ``a Maria'' pot ser l'objecte
  indirecte de l'oració principal i de la subordinada.
\item (b)
% \item Cada mot en espanyol es correspon, de vegades, amb més d'un mot
%   en séverla: es tracta d'un cas típic d'ambigüitat lèxica de
%   transferència. Buscarem, entre les possibles traduccions dels grups
%   de dos mots, les que apareguen més freqüentment en el corpus i les
%   elegirem.
%   \begin{description}
%   \item[dirección postal:] \emph{dirección} pot ser \emph{obmur},
%     \emph{odnam} o \emph{ragul}; \emph{postal} només és
%     \emph{atrac}. Les úniques combinacions (col·locacions) que
%     apareixen en els textos són \emph{atrac odnam} i \emph{atrac
%       ragul}; la segona és la més freqüent i, per tant, l'elegim.
%   \item[dirección electrónica:] \emph{electrónica} és \emph{dered}; la
%     combinació \emph{dered obmur} no s'hi dóna, i \emph{dered odnam}
%     és menys freqüent que \emph{dered ragul}. Així doncs, elegim la
%     segona.
%   \item[dirección asistida:] \emph{asistida} és \emph{aduya} i només
%     trobem \emph{aduya odnam}: per tant, l'elegim.
%   \item[dirección general:] \emph{general} és \emph{latot} i només
%     trobem \emph{latot odnam}. Per tant, l'elegim.
%   \item[diseño moderno:] \emph{diseño} pot ser \emph{raerc} o
%     \emph{atnip}; \emph{moderno} és \emph{aídla}; \emph{aídla atnip}
%     és molt més freqüent que \emph{aídla raerc}. Per tant, elegim la
%     primera.
%   \item[diseño asistido:] \emph{asistido} és \emph{aduya}. Només
%     trobem \emph{aduya raerc}; per tant, l'elegim.
%   \end{description}
%   Nota: \emph{séverla} és \emph{alrevés} al revés. Si llegiu els mots
%   al revés, veureu per què s'han elegit.

% \item La traducció (humana) de les frases en la llengua $L_1$ a la
%   llengua $L_2$ ens pot servir per a determinar quina de les anàlisis
%   (interpretacions) de les frases en $L_1$ s'hauria d'elegir i usar
%   aquesta elecció abans de traduir automàticament a $L_3$.
      
%   Fixem-nos en l'ambigüitat lèxica deguda als mots polisèmics, i
%   imaginem que tenim $L_1$=espanyol, $L_2$=anglés i $L_3$=català.  Si
%   la traducció del mot \emph{destino} a l'anglés és
%   \emph{destination}, la traducció correcta al català és
%   \emph{destinació}, ja que s'ha elegit la interpretació ``punt final
%   o finalitat d'un procés o desplaçament''; si és \emph{destiny}, la
%   traducció correcta al català és \emph{destí}, ja que s'ha elegit la
%   interpretació ``sort futura que ens reserva l'atzar''. Es podria fer
%   similarment en el cas dels mots homògrafs.
      
%   Aquesta \emph{triangulació} també pot servir per a tractar altres
%   tipus d'ambigüitat, com ara la sintàctica.
      
\item La solució semàntica és més potent i general però exigeix una
  anàlisi molt profunda de la frase. En alguns casos, la sintaxi
  podria donar pistes que permetrien una desambiguació molt
  aproximada. Per exemple, si {\em fui\/} va seguit de la preposició
  {\em a}, és molt probable que es tracte del verb {\em ir}; d'altra
  banda, si va seguit d'un participi passat, és molt probable que es
  tracte del verb {\em ser\/}: es podria fer una categoria gramatical
  especial per al verb ser i usar tècniques de desambiguació
  categorial. Si trobem {\em podem\/} ({\em volem\/}) seguit
  d'infinitiu, és molt més probable que es tracte del verb {\em
    poder\/} ({\em voler\/}) que del verb {\em podar} ({\em volar\/});
  de nou, caldria usar una categoria gramatical especial, en aquest
  cas per als verbs modals.
     
\item Es poden trobar molts exemples; per exemple, entre $L_1 =$
  espanyol i $L_2 =$ català, tenim:
  \begin{itemize}
  \item Ayer por la mañana vino tarde $\rightarrow$ *Ahir pel demà vi
    vesprada (Ahir de matí va venir tard).
  \item Río porque no llegó a la meta $\rightarrow$ *Riu perquè no va
    arribar a la fique (Ric perquè no va arribar a la meta).
  \end{itemize}
      
\item (c)
\item (b). Pot tenir més d'un arbre d'anàlisi sintàctica (ambigüitat
  estructural).
\item (b)
\item (c)
\item (b)
\item (a). \emph{Cantamos} pot ser present o passat.
\item (a). El pronom buit que fa de subjecte de \emph{fa}.
\item (a)
%\item (c). Mireu l'exemple~\ref{eq:escocesos}
\item (c)
\item (b). El pronom \emph{li} pot tenir els antecedents \emph{Mario}
  i \emph{la seua mare}
\item (b)
\item (c)
\item (a)
\item (c)
\item (c)
\item (b)
\item (c)
\item (c)
\item (b)
\end{enumerate}

