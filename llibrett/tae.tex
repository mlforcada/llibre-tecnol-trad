% \todo{Millorar aquest apartat (ho deixe per a Felipe): aprofitar les
% transparències de Felipe, potser el que Mikel va fer a Castelló,
% col·locant material avançat en \emph{per saber més}; deixar clar que
% SMT es SotA; connectar amb exercicis com el del \emph{séverla}}

%\section{Sistemes de traducció automàtica estadística}

Totes les tècniques de traducció automàtica descrites fins ara són de
naturalesa \emph{deductiva}, és a dir, estan basades en teories i
coneixements lingüístics sobre la traducció. Però recentment (sobretot
en els primers anys del tercer mil·lenni) s'està produint un
creixement espectacular de tècniques de traducció automàtica
\emph{inductives}, en les quals el sistema \emph{aprén} automàticament
a traduir entre dues llengües $L_1$ i $L_2$ a partir d'una mostra o
corpus paral·lel suficientment gran de frases en $L_1$ acompanyades de
la traducció a $L_2$ (vegeu l'apartat~\ref{ss:bitextos}). Aquestes
aproximacions inductives també reben el nom de \emph{traducció
  automàtica basada en corpus}.

La principal tècnica de traducció automàtica basada en corpus es la
\emph{traducció automàtica estadística} (en anglés \emph{statistical
  machine trnaslation}; SMT), la qual va ser inventada cap a finals
dels vuitanta per un grup d'investigadors d'IBM \citep{brown90j}; el
sistemes actuals, com veurem més endavant, són una evolució d'aquests
sistemes.

A l'hora de traduir hi ha una diferència fonamental entre els sistemes
basats en regles o coneixement i el sistemes estadístics: mentres que
el sistemes basat en regles produeixen una única traducció, el
sistemes estadístics generen una gran quantitat d'\emph{hipòtesis de
  traducció} (idealment totes les possibles) i utilitzen models
estadístics per \emph{puntuar} les hipòtesis generades i escollir la
millor de totes. Els principals models estadístics que s'usen per
puntuar les hipòtesis de traducció son el \emph{model de traducció} i
el \emph{model de llengua}.

\begin{table}[tb]
\begin{tabular}{p{5cm}|p{5cm}}
  \textbf{Anglés} & \textbf{Espanyol}\\
  It has been exciting in many ways . &
  Ha sido un trabajo apasionante en varios sentidos . \\
  \hline
  As the shadow rapporteurs know , this has been my first report
  during my time in Parliament and it has been a good learning
  experience . &
  Como bien saben los ponentes alternativos , éste ha sido el primer
  informe en el que he trabajado durante mi mandato parlamentario , y
  me ha venido muy bien como experiencia formativa . \\
  \hline
  It has also been very challenging to work on three reports and
  therefore also with other rapporteurs . &
  También ha sido un gran desafío trabajar en tres informes , y por lo
  tanto con otros ponentes . \\
  \hline
  It has been exciting ! &&
  ¡ Ha sido emocionante \\
  \hline
\end{tabular}
\caption{Oracions paral·leles extretes del  corpus paral·lel amb les actes del
  Parlament Europeu.}
\label{fg:alinora}
\end{table}

El \textbf{model de traducció} s'aprén a partir d'un corpus paral·lel
alineat a nivell d'oració com el que es mostra a la figura
\ref{fg:alinora}. Primerament, s'han d'obtenir els alineaments a
nivell paraula (vegeu la figura \ref{fg:alinpar}) per a després
estimar el model de traducció a partir d'aquestos. 












Els primer sistemes de traducció automàtica estadística ja vn
construeix models probabilístics que indiquen, per exemple, la
probabilitat que els mots d'una oració en llengua meta s'ordenen d'una
determinada manera, la probabilitat que la traducció d'un determinat
mot en una llengua siga traducció d'un determinat mot en l'altra, la
probabilitat que la traducció d'un mot siga una expressió de més d'un
mot o la probabilitat de determinats reordenaments, i, usa aquests
models per a obtenir la \emph{traducció més probable} de noves
oracions que no ha vist anteriorment.






, i més
concretament la traducció automàtica estadística basada en segments
bilingües (en anglés \emph{phrase-based statistical machine
  translation}).

Els primer sistemes




Hi ha diverses aproximacions a la traducció automàtica basada en
corpus. En mencionarem molt breument dues:
\begin{itemize}
\item 
\item La \emph{traducció automàtica basada en exemples} que intenta
construir \emph{plantilles} de traducció a partir dels exemples
observats i \emph{generalitzar-les} perquè servisquen en noves
situacions. Per exemple, si sabem que el substantiu anglés \emph{ski}
es tradueix per \emph{esquí} i que la locució substantiva \emph{ski
  station} es tradueix per \emph{estació d'esquí} podem generalitzar
aquesta última locució substituint \emph{ski} per qualsevol altres
substantiu $N$, de manera que la traducció de ``$N$ \emph{station}''
és ``estació de $N$''; així, si la traducció de \emph{train} és
\emph{tren}, la traducció de \emph{train station} és \emph{estació de
  tren}, etc. (exemple pres de \citealt{carl01j}). Fixeu-vos que la traducció automàtica basada en exemples
pot necessitar que la mostra de frases i traduccions estiga, a més,
anotada lingüísticament (en l'exemple, indicant quins mots o
estructures funcionen com un nom).
\end{itemize}
