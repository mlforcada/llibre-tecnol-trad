\todo{Millorar aquest apartat (ho deixe per a Felipe): aprofitar les transparències de Felipe,
  potser el que Mikel va fer a Castelló, col·locant material avançat en
  \emph{per saber més}; deixar clar que SMT es SotA; connectar amb
  exercicis com el del \emph{séverla}}
\label{ss:induc}

Totes les aproximacions a la traducció automàtica descrites fins ara
són de naturalesa \emph{deductiva}, és a dir, estan programades a
partir de teories i coneixements lingüístics sobre la traducció. Però
recentment (sobretot en els primers anys del tercer mil·leni) s'està
produint un creixement espectacular de les aproximacions
\emph{inductives} a la traducció automàtica, en les quals el sistema
\emph{aprén} a traduir entre dues llengües $L_1$ i $L_2$ a partir
d'una mostra o corpus suficientment gran de frases en $L_1$
acompanyades de la traducció a $L_2$ (vegeu
l'apartat~\ref{ss:bitextos}). Aquestes aproximacions inductives també
reben el nom de \emph{traducció automàtica basada en corpus}.

Hi ha diverses aproximacions a la traducció automàtica basada en
corpus. En mencionarem molt breument dues:
\begin{itemize}
\item La \emph{traducció automàtica estadística},
inventada cap a finals dels vuitanta per un grup d'investigadors d'IBM
\citep{brown90j} construeix models probabilístics que indiquen, per
exemple, la probabilitat que els mots d'una oració en llengua meta
s'ordenen d'una determinada manera, la probabilitat que la traducció
d'un determinat mot en una llengua siga traducció d'un determinat mot
en l'altra, la probabilitat que la traducció d'un mot siga una
expressió de més d'un mot o la probabilitat de determinats
reordenaments, i, usa aquests models per a obtenir la \emph{traducció
  més probable} de noves oracions que no ha vist anteriorment. 
\item La \emph{traducció automàtica basada en exemples} que intenta
construir \emph{plantilles} de traducció a partir dels exemples
observats i \emph{generalitzar-les} perquè servisquen en noves
situacions. Per exemple, si sabem que el substantiu anglés \emph{ski}
es tradueix per \emph{esquí} i que la locució substantiva \emph{ski
  station} es tradueix per \emph{estació d'esquí} podem generalitzar
aquesta última locució substituint \emph{ski} per qualsevol altres
substantiu $N$, de manera que la traducció de ``$N$ \emph{station}''
és ``estació de $N$''; així, si la traducció de \emph{train} és
\emph{tren}, la traducció de \emph{train station} és \emph{estació de
  tren}, etc. (exemple pres de \citealt{carl01j}). Fixeu-vos que la traducció automàtica basada en exemples
pot necessitar que la mostra de frases i traduccions estiga, a més,
anotada lingüísticament (en l'exemple, indicant quins mots o
estructures funcionen com un nom).
\end{itemize}
