\chapter{Textos i formats}
\label{se:EPT}

El tipus de fitxer bàsic amb què treballen els professionals de la
traducció sol ser un fitxer amb text, és a dir, un text informatitzat,
també anomenat \emph{document de text}. Aquest fitxer pot contenir, a
més del text mateix, informació sobre la presentació (el format dels
paràgrafs i de les pàgines, els tipus i les grandàries de lletra que
s'usen amb cada mot, etc.) o sobre l'organització del contingut
(indicacions de que una determinada part del text és un títol de
capítol, el títol d'una secció o una nota a peu de pàgina, etc.).

Un text informatitzat pot tenir orígens diversos:
\begin{itemize}
\item Pot haver estat generat per un altre programa d'ordinador, per
  exemple a partir de les dades contingudes en alguna base de dades
  (vegeu el capítol~\ref{se:basesdades}).
\item El podem haver rebut annex a un missatge de correu electrònic
  (vegeu l'apartat~\ref{ss:correue}) o per missatgeria instantània
  (vegeu l'apartat~\ref{ss:missinst}).
\item El podem haver descarregat (copiat) d'algun servidor d'Internet
  (vegeu la pàg.~\pageref{pg:ftp}).
\item El podem haver generat, potser a partir d'un altre
  text, usant un \emph{processador de textos} (vegeu
  l'apartat~\ref{ss:proctext}).
\item El pot haver generat un \emph{sistema de reconeixement de la
    parla} a partir de la veu de la persona que l'ha dictat.
\item El pot haver generat un \emph{sistema de reconeixement de textos
    escrits} a partir d'un text tipografiat o manuscrit.
\end{itemize}

%%%%%%%%%%%%%
\section{Formats de text} \label{ss:formats}
Un \emph{text informatitzat} és, com qualsevol porció de dades
informatitzada, una \emph{seqüència de bits}, és a dir, d'\emph{uns} i
\emph{zeros} de l'estil de la següent:
\begin{center}
 \texttt{010000010100110101001001\ldots}
\end{center}

Com ja hem vist en l'apartat~\ref{ss:memoria}, els \emph{bits}
s'agrupen en grups de vuit (\emph{bytes} o \emph{octets}):
\begin{center}
 \texttt{01000001 01001101 01001001\ldots}
\end{center}

Hi ha moltes maneres d'organitzar aquests octets per a emmagatzemar
els textos; molts dels problemes que apareixen quan es tracten textos
amb l'ordinador provenen de discrepàncies quant a la manera de fer-ho.
En les seccions següents estudiarem dos aspectes importants que
anomenarem \emph{codificació} i \emph{format} pròpiament dit.  La
\emph{codificació} és l'assignació d'una seqüència concreta, d'un o
més \emph{octets}, a cada caràcter possible d'un text. El
\emph{format} d'un document és la part no textual d'aquest i que
serveix per a codificar informació estructural (sobre l'organització
del contingut del document) o presentacional (sobre l'aparença que
tindrà el document quan es presente).

\section{Codificació de caràcters}
La codificació dels caràcters d'un text consta de dues fases: 
\begin{enumerate}
\item S'assigna a cada caràcter un nombre enter positiu anomenat
  \emph{punt de codi} o simplement \emph{codi}; per exemple:
  ``\texttt{a}'' $\to$ 97; ``\texttt{?}'' $\to$ 63.
\item els codis numèrics es converteixen en octets assignant-los una
  determinada seqüència de bits; per exemple: 97 $\to$
  \texttt{01100001}; 63 $\to$ \texttt{00111111}).
  % Esta fase de vegades s'anomena \emph{serialització}.
\end{enumerate}

\subsection{ASCII}
Com ja s'ha comentat en la pàgina~\pageref{pg:ASCII}, per a
emmagatzemar textos s'ha usat històricament l'estàndard ASCII
(\emph{American Standard Code for Information Interchange}). Aquest
estàndard assigna un número del 0 al 127 a cada caràcter del alfabet
llatí usat en anglés i fa servir 7 bits per a
codificar-lo,\footnote{Amb 7 bits es poden fer $2^7=128$
  combinacions. Per això els codis assignats per ASCII van del 0 al
  127.} de manera que permet emmagatzemar un caràcter per octet i
encara en sobra un bit.\footnote{Inicialment aquest bit es feia servir
  com a bit de control per a detectar errades en la transmissió dels
  textos.} La taula~\ref{tb:ASCII} mostra alguns exemples de codis
ASCII. Els codis ASCII del 0 al 31 no corresponen a caràcters
imprimibles sinó a \emph{caràcters de control} que tenen noms
especials i s'usen per a un control rudimentari del format i de la
transmissió dels textos.

\begin{table}
\begin{center}
\begin{tabular}{c|l|l}
\hline\hline \textsc{Codi binari} & \textsc{codi decimal} &
\textsc{caràcter} \\
\hline
\texttt{0000000} & 0 & NUL (caràcter nul)  \\
$\cdots$ & $\cdots$ & $\cdots$ \\
\texttt{0001001} & 9 & TAB (tabulador) \\
\texttt{0001010} & 10 & NL (nova línia) \\
$\cdots$ & $\cdots$ & $\cdots$ \\
\texttt{0001101} & 13 & CR (retorn del carro) \\
$\cdots$ & $\cdots$ & $\cdots$ \\
\texttt{0100000} & 32 & (un espai en blanc) \\
\texttt{0100001} & 33 & \texttt{!} \\
\texttt{0100010} & 34 & \verb+"+ \\
$\cdots$ & $\cdots$ & $\cdots$ \\
\texttt{0110000} & 48 & \texttt{0} \\
\texttt{0110001} & 49 & \texttt{1} \\
$\cdots$ & $\cdots$ & $\cdots$ \\
\texttt{0111000} & 56 & \texttt{8} \\
\texttt{0111001} & 57 & \texttt{9} \\
$\cdots$ & $\cdots$ & $\cdots$ \\
\texttt{1000000} & 64 & \texttt{@} \\
\texttt{1000001} & 65 & \texttt{A} \\
\texttt{1000010} & 66 & \texttt{B} \\
$\cdots$ & $\cdots$ & $\cdots$ \\
\texttt{1011010} & 90 & \texttt{Z} \\
\texttt{1011011} & 91 & \texttt{[} \\
$\cdots$ & $\cdots$ & $\cdots$ \\
\texttt{1100000} & 96 & \verb+`+ \\
\texttt{1100001} & 97 & \texttt{a} \\
\texttt{1100010} & 98 & \texttt{b} \\
$\cdots$ & $\cdots$ & $\cdots$ \\
\texttt{1111010} & 122 & \texttt{z} \\
\texttt{1111011} & 123 & \verb+{+ \\
$\cdots$ & $\cdots$ & $\cdots$ \\
\texttt{1111110} & 126 & \verb+~+ \\
\end{tabular}
\end{center}
\caption{Alguns exemples del codi ASCII. Els codis del 0 al 31 no
  corresponen a caràcters imprimibles sinó a caràcters de control.}
\label{tb:ASCII}
\end{table}

L'estàndard ASCII té la limitació que no permet escriure caràcters
propis de moltes llengües europees, com ara lletres amb signes
diacrítics (\emph{á}, \emph{ò}, \emph{ç}, \emph{ñ}, \emph{ü}, etc.) o
lletres especials com \emph{ß}.


\subsection{Extensions d'ASCII} \label{s3:ISO}
Amb l'arribada dels microordinadors\footnote{Els primers ordinadors
  que tenien una grandària que permetia tindre un a casa.} en els anys
vuitanta es va decidir ampliar l'estàndard ASCII, de 7 bits i per tant
amb $2^7=128$ caràcters diferents, a un codi de 8 bits amb $2^8=256$
caràcters diferents. El vuité bit o ``bit 7''\footnote{Recordeu que en
  informàtica és comú comptar començant pel zero.}---el primer per
l'esquerra--- és 1 per als nous caràcters (numerats del 128 al 255) i
zero per als caràcters estàndards d'ASCII.

Hi ha diverses extensions d'ASCII, cadascuna adreçada a un conjunt de
llengües concret que fan servir (quasi) el mateix alfabet. En la
nostra àrea geogràfica s'usa normalment la codificació ISO-8859-1 o
\emph{Latin-1} (vegeu la taula~\ref{tb:ISO88591}); aquesta codificació
serveix per a les llengües següents: \emph{afrikaans} (llengua
germànica parlada en la República de Sudàfrica), alemany, anglés,
basc, català, danés, escocés, espanyol, feroés, finés, francés,
gallec, irlandés, islandés, italià, neerlandés, noruec, portugués i
suec.\footnote{Hi ha una modificació anomenada ISO-8859-15, que
  inclou, entre altres, el símbol de l'euro i resol alguns problemes
  referents al francés i al finés.}  El sistema operatiu Windows de
Microsoft usa la codificació de 8 bits anomenada CP-1252, també
anomenada \emph{WinLatin-1}, que és més àmplia que ISO-8859-1 ja que
usa alguns dels codis 128--159 per a caràcters (per exemple, usa el
codi 128 per al símbol de l'euro).

Hi ha altres codificacions en la família ISO-8859. Per exemple,
l'albanés, el bosni, el croat, el txec, l'hongarés i el romanés usen
una codificació anomenada ISO-8859-2 o \emph{Latin-2}; el letó, el
lituà i l'estonià usen l'ISO-8859-4 o \emph{Latin 4}; el rus usa
l'ISO-8859-5 que conté l'alfabet ciríl·lic a més de l'alfabet llatí
bàsic. Per això, és molt important conéixer quin esquema de
codificació de caràcters s'ha usat en un document de text determinat
per a poder-lo llegir correctament; alguns formats de text inclouen
aquesta informació dins del mateix document.

El fet que hi haja diverses maneres d'usar els nous codis fa que de
vegades els textos amb caràcters especials no queden bé quan passem
d'un processador de textos (o un editor de textos) a un altre (els
caràcters d'ASCII es veuen normalment bé: els que fallen són els
nous). Fixeu-vos que si en un document ISO-8859-1 s'escriu la frase
\emph{Què és això?}, on els caràcters ``\texttt{è}'', ``\texttt{é}'' i
``\texttt{ò}'' tenen codis per damunt de 127 (233, 232 i 242
respectivament), i provem a llegir-lo com si fóra un document
ISO-8859-2, llegirem \emph{Quč és aixň?}, perquè dos d'aquests codis
(233 i 242) tenen una altra interpretació en aquesta codificació
(``\texttt{č}'' i ``\texttt{ň}'' respectivament). Es per això que no
podem mesclar en un mateix document textos en llengües que fan servir
extensions d'ASCII diferents.

\begin{table}
\begin{center}
\begin{tabular}{c|l|l}
\hline\hline \textsc{Codi binari} & \textsc{codi decimal} &
\textsc{caràcter} \\
\hline
\texttt{10100000} & 160 & (espai no trencable)  \\
\texttt{10100001} & 161 & \texttt{¡} \\
$\cdots$          & $\cdots$ & $\cdots$ \\
\texttt{10110101} & 181 & $\mathtt{\mu}$ \\
\texttt{10110110} & 182 & \texttt{¶} \\
\texttt{10110111} & 183 & \texttt{·} \\
\texttt{10111000} & 184 & \texttt{¸} \\
$\cdots$          & $\cdots$ & $\cdots$ \\
\texttt{11000000} & 192 & \texttt{À} \\
\texttt{11000001} & 193 & \texttt{Á} \\
\texttt{11000010} & 194 & \texttt{Â} \\
\texttt{11000011} & 195 & \texttt{Ã} \\
\texttt{11000100} & 196 & \texttt{Ä} \\
\texttt{11000101} & 197 & \texttt{Å} \\
\texttt{11000110} & 198 & \texttt{Æ} \\
\texttt{11000111} & 199 & \texttt{Ç} \\
\texttt{11001000} & 200 & \texttt{È} \\
\texttt{11001001} & 201 & \texttt{É} \\
\texttt{11001010} & 202 & \texttt{Ê} \\
\texttt{11001011} & 203 & \texttt{Ë} \\
\texttt{11001100} & 204 & \texttt{Ì} \\
\texttt{11001101} & 205 & \texttt{Í} \\
\texttt{11001110} & 206 & \texttt{Î} \\
\texttt{11001111} & 207 & \texttt{Ï} \\
$\cdots$          & $\cdots$ & $\cdots$ \\
\texttt{11100000} & 224 & \texttt{à} \\
\texttt{11100001} & 225 & \texttt{á} \\
\texttt{11100010} & 226 & \texttt{â} \\
\texttt{11100011} & 227 & \texttt{ã} \\
\texttt{11100100} & 228 & \texttt{ä} \\
\texttt{11100101} & 229 & \texttt{å} \\
\texttt{11100110} & 230 & \texttt{æ} \\
$\cdots$          & $\cdots$ & $\cdots$ \\
\texttt{11111111} & 255 & \texttt{ÿ} \\
\end{tabular}
\end{center}
\caption{Alguns exemples d'ISO-8859-1 (\emph{Latin-1}). Els codis del
  0 al 127 són com els d'ASCII. Els codis del 128 al 159 no estan
  assignats.}
\label{tb:ISO88591}
\end{table}

\subsection{Unicode}
Els codis de 8 bits com ISO-8859-1 (\emph{Latin-1}) són adequats per a
la major part de les llengües europees, les quals es basen en
l'alfabet llatí amb algunes modificacions, però hi ha llengües al món
que tenen sistemes d'escriptura molt complexos amb milers de símbols
diferents, com ara el xinés o el japonés. Per a aquestes llengües 256
combinacions no són suficients i s'hi han proposat diverses solucions.
\emph{Unicode}\footnote{\url{http://www.unicode.org}} (ISO 10646) és
un nou estàndard per a codificar pràcticament els caràcters de totes
les llengües del món i fins i tot mesclar diversos alfabets en un
mateix fitxer.

Unicode fa servir 31 bits; és a dir, permet $2^{31}=2.147.483.648$
caràcters diferents. La versió més comunament usada d'Unicode (BMP,
\emph{Basic Multilingual Plane}) té 65.534 caràcters; això comportaria
l'ús de 2 octets (16 bits) en comptes d'un ($2^{16}=65.536$); això
faria que un text Unicode senzill fóra el doble de gran que el text
ASCII corresponent. Per tal d'estalviar espai, hi ha mètodes de
serialització d'Unicode, com l'UTF-8, que en el cas de les llengües
europees amb alfabet llatí estalvia espai perquè usa un únic octet per
als codis ASCII (del 0 al 127, els més freqüents), i més d'un octet
per als codis següents (així, a més, és compatible amb l'ASCII). En
concret, UTF-8 usa:
\begin{itemize}
\item per als codis del 0 al 127, 1 octet (compatible amb ASCII);
\item per als codis del 128 al 2047, 2 octets;
\item per als codis del 2048 al 65535, 3 octets, i així successivament.
\end{itemize}

\subsection{Limitacions}

Tot i que ampliem l'ASCII a ISO-8859 o Unicode, encara és molt
limitat.  Per exemple, si volem que un text tinga un cert format,
només podrem usar caràcters de control com ara l'espai en blanc, el
tabulador, el salt de línia, etc.  Per exemple, no podrem canviar
fàcilment de tipus o de grandària de lletra, o indicar que una
determinada part del text és el títol d'una secció o el text d'una
nota a peu de pàgina. De qualsevol manera, les extensions d'ASCII
(ISO-8859-$X$) i Unicode encara s'usen en aplicacions com ara el
correu electrònic, o quan volem que un text ---el contingut del qual
és molt més important que l'aparença--- puga ser llegit per qualsevol
usuari sense importar el processador de textos que use; els textos
d'aquesta mena s'anomenen de vegades \emph{textos plans} i
s'emmagatzemen normalment en fitxers amb noms que tenen l'extensió
\texttt{.txt}.  Aquests textos es poden produir i llegir amb qualsevol
\emph{editor de textos} (vegeu l'apartat~\ref{ss:proctext}).

\section{Format pròpiament dit}\label{ss:format}
Els documents de text són en general més rics que simples seqüències
de caràcters; els textos, a més de caràcters, contenen informació de
\emph{format}. Per això, és necessària l'assignació de \emph{codis}
(que també es convertiran en octets) per a regular altres
característiques del text com:
\begin{itemize}
\item l'aparença \emph{visual} que tindrà el document quan es presente
  (per exemple, ``inici cursives'', ``final negretes'', ``lletra de 16
  punts''), o
  \item l'\emph{estructura}, és a dir, l'organització del contingut
    del document (per exemple, ``títol de secció'', ``llista
    numerada'', ``nota a peu de pàgina'', ``fila d'una taula'', etc.)
\end{itemize}

Per a guardar aquesta informació, s'usen:
\begin{itemize}
\item D'una banda, codificacions o formats basats en text
  (ISO-8859-$X$, Unicode, etc.). Tal és el cas del format SGML
  (\emph{standardized generalized markup language}), la seua versió
  simplificada (i molt més estesa) XML (\emph{extensible markup
    language}), el format HTML (\emph{hypertext markup language};
  basat en SGML), el format RTF (\emph{rich text format}; proposat per
  Microsoft i sense relació amb SGML o XML), o el llenguatge per a
  impressores anomenat Postscript. Tots aquests formats usen
  combinacions especials\footnote{Combinacions de caràcters poc
    freqüents en textos usuals.} de caràcters de text per a indicar
  aquestes característiques d'estructuració o de
  presentació.\footnote{Aquests caràcters que indiquen el format no
    són normalment visibles per a la persona usuària mentre redacta o
    veu el document, excepte si demana explícitament que els vol
    veure.}
\item D'altra banda, codificacions o formats basats en codis binaris
  no interpretables com a caràcters. Tal és el cas dels formats
  particulars dels processadors de textos comercials com ara Corel
  WordPerfect o Microsoft Word.\footnote{Hi ha una tendència a
    considerar el format de document de Microsoft Word, amb extensió
    \texttt{.doc}, com la manera estàndard d'enviar documents de text
    annexos a un missatge electrònic, sense considerar el fet que
    aquest format és privat i està associat a l'ús d'un determinat
    producte no lliure i de codi font tancat. El format \texttt{.docx}
    també anomenat Office Open XML o OOXML, està millor documentat i
    estandarditzat i pot ser processat més satisfactòriament amb
    processadors lliures i de codi font obert.}
\end{itemize}
Com ja s'ha dit, l'ús de formats de text més avançats no només serveix
per a determinar-ne la presentació en la pantalla o quan són impresos;
com veurem més avall, en el cas de SGML i XML, el format serveix per a
\emph{estructurar} el document de text en unitats directament
relacionades amb el contingut del document, com ara seccions, títols
de secció, llistes, paràgrafs, etc.; aquesta estructuració interna del
document pot ser usada després per a fer recerques d'informació amb
l'ajuda de l'estructura definida, com ara buscar un mot concret només
en títols de secció, o també per a produir-ne una presentació concreta
del document, com veurem més endavant. De fet, recentment, amb
l'aparició de XML (vegeu l'apartat~\ref{s3:XML}), s'observa una
tendència cap a l'adopció de formats de document estructurats, és a
dir, no relacionats únicament amb la presentació, sinó també amb
l'estructura pròpia del document, formats normalment concebuts de
manera que la presentació desitjada es puga produir a partir de
l'estructura usant fitxers (anomenats \emph{fulls d'estil}) amb regles
d'estil ben definides (vegeu la secció~\ref{ss:separac}).

\section{SGML i XML} \label{s3:SGML} \label{ss:SGML}

\subsection{SGML}
SGML (\emph{standardized generalized markup language}), el llenguatge
estàndard generalitzat de marques, havia tingut un èxit relatiu fins a
mitjans dels noranta; però l'aparició cap a finals dels noranta d'una
versió restringida i simplificada de SGML anomenada XML
(\emph{extensible markup language}) ha impulsat enormement l'adopció
dels formats d'estructuració de documents, de tal manera que en
l'actualitat s'usa XML moltíssim més que el SGML original; per això,
ens centrarem en aquest últim format. De tota manera, encara hi ha
formats molt importants que es basen en SGML, com el llenguatge de
marques per a hipertextos HTML (vegeu l'apartat~\ref{s3:HTML})
(excepte pel més recent, HTML5, estandarditzat el 2014, que ja no és
una aplicació SGML). Hi ha també versions estàndards d'HTML conegudes
com XHTML, que es basen directament en XML (vegeu
l'apartat~\ref{s3:XML}).\footnote{Noteu que HTML5 també té una versió
  \emph{serialitzada en XML}, XHTML5.}

\subsection{XML} \label{s3:XML}

\subsubsection{Marques}
Un document XML és un document de text on, a més del text pròpiament
dit, podem trobar \emph{etiquetes} o \emph{marques} (en anglés
\emph{tags}) que donen informació sobre la naturalesa i l'organització
de cada un dels continguts del document; com ja s'ha dit, un document
XML és un document \emph{estructurat}. Per exemple, un document XML
corresponent a un missatge de correu electrònic podria tenir
l'aparença que es mostra en la figura~\ref{fg:faxXML}.  La primera
línia declara que el document és un document XML de la versió 1.0 i
que el joc de caràcters que usa és l'ISO-8859-1 (\emph{Latin-1}; vegeu
l'apartat~\ref{s3:ISO}). Com s'hi pot veure, les etiquetes que
apareixen entre parèntesis angulars indiquen les diverses parts del
document, anomenades \emph{elements}.  Típicament, s'obrin amb
\texttt{<}\emph{nom}\texttt{>} i es tanquen amb
\texttt{</}\emph{nom}\texttt{>}.  En l'exemple, es pot veure que un
missatge de correu (\texttt{<EMAIL>}\ldots\texttt{</EMAIL>}) té un
destinatari, un remitent, una data, un títol i un text; és a dir, els
elements poden contenir altres elements; les marques funcionen com a
parèntesis. Seguint amb la jerarquia d'inclusió d'elements en altres,
tant el destinatari com el remitent tenen nom i adreça, i el text es
compon de paràgrafs (\texttt{<P>}\ldots\texttt{</P>}).

\begin{figure}
\begin{center}
\begin{alltt}
<?xml version="1.0" encoding="ISO-8859-1"?>
<!DOCTYPE EMAIL SYSTEM "http://www.dlsi.ua.es/%7Efsanchez/tt/email.dtd">
<\textbf{EMAIL}>
  <\textbf{DESTINATAR}>
    <\textbf{NOM}>Mikel L. Forcada</\textbf{NOM}>
    <\textbf{ADREÇA}>mlf@dlsi.ua.es</\textbf{ADREÇA}>
  </\textbf{DESTINATARI}>
  <\textbf{REMITENT}>
    <\textbf{NOM}>Felipe Sánchez Martínez</\textbf{NOM}>
    <\textbf{ADREÇA}>fsanchez@dlsi.ua.es</\textbf{ADREÇA}>
  </\textbf{REMITENT}>
  <\textbf{DATA}>9 de novembre de 2015</\textbf{DATA}>
  <\textbf{ASSUMPTE}>Capítol 4 del llibre de TT</\textbf{ASSUMPTE}>
  <\textbf{TEXT}>
    <\textbf{P}>Mikel, estic acabant de fer modificacions al capítol 
    dedicat a textos i formats. Quan acabe t'avise.</\textbf{P}>
    <\textbf{P}>Per favor, envia'm els apunts que vam preparar 
    sobre traducció automàtica que no els trobe. Gràcies!</\textbf{P}>
  </\textbf{TEXT}>
</\textbf{EMAIL}>
\end{alltt}
\end{center}
\caption{El text d'un missatge de correu electrònic en XML.}
\label{fg:faxXML}
\end{figure}

\subsubsection{Documents XML ben formats}
Aquestes són algunes de les característiques que fan que un document
XML estiga \emph{ben format}, és a dir, siga un document XML i no una
altra cosa:
\begin{itemize}
\item Cada etiqueta d'inici d'element de la forma
  \texttt{<}\emph{nom}\texttt{>}, \texttt{<}\emph{nom}
  \emph{atribut}\texttt{=}\texttt{"}\emph{valor}\texttt{"}\texttt{>},
  \texttt{<}\emph{nom}
  \emph{atribut1}\texttt{=}\texttt{"}\emph{valor}\texttt{"}
  \emph{atribut2}\texttt{=}\texttt{"}\emph{valor}\texttt{"}\texttt{>},
  etc. (amb zero o més assignacions de valors a atributs) ha d'estar
  emparellat amb una etiqueta de final d'element de la forma
  \texttt{</}\emph{nom}\texttt{>}, sense atributs però amb el mateix
  nom.\footnote{En SGML es permet que alguns elements es tanquen
    \emph{implícitament}, sense necessitat d'una etiqueta de final
    d'element.} Si l'element és buit,
  \texttt{<}\emph{nom}\ldots\texttt{></}\emph{nom}\texttt{>}, també es
  pot escriure \texttt{<}\emph{nom}\ldots\texttt{/>}
\item Un element pot contenir qualsevol nombre d'elements.
\item Els elements no es poden solapar o creuar: no és possible
  escriure, per exemple, \texttt{<a>text<b>més text</a>més text encara
    </b>}.
\item El document conté un únic element \emph{arrel} que conté tots
  els elements del text.
\item El document pot contenir comentaris entre \texttt{<!--} i
  \texttt{-->} o instruccions de processament del tipus
  \texttt{<?}\emph{nom}\ldots\texttt{?>} en qualsevol lloc excepte
  dins de les etiquetes.
\item Els valors dels atributs han d'anar entre cometes dobles
  (\texttt{"}\emph{valor}\texttt{"}) o simples
  (\texttt{'}\emph{valor}\texttt{'}).
\item Un element no pot tenir dos atributs amb el mateix nom.
\item Els caràcters \texttt{<} i \texttt{\&} no poden aparéixer en el
  text dels elements ni dels atributs. Això és perquè \texttt{<}
  indica el començament d'une etiqueta i \texttt{\&} el començament
  d'una \emph{entitat} com ara \texttt{\&copy;} que es pot usar per a
  representar el càracter \emph{©}: si es necessiten aquests caràcters
  s'han d'escriure les entitats \texttt{\&lt;} i \texttt{\&amp;},
  respectivament.
\end{itemize}
Com es pot veure, aquestes regles que defineixen un document XML ben
format no diuen quines etiquetes són vàlides i quines no, o quins
atributs pot tenir un determinat element, o quins elements poden anar
dins de un determinat element, en quin ordre o en quina quantitat.

\subsubsection{Tipus de documents}
Per a especificar, per a un tipus determinat de document XML, quines
etiquetes són vàlides, quins atributs pot tenir cada element, o quins
elements poden anar dins d'un determinat element, en quin ordre o en
quina quantitat, es pot usar una DTD (\emph{document type definition}
o definició del tipus de document).\footnote{Les DTD no són l'única
  manera d'especificar famílies de documents XML; una altra manera més
  potent són els anomenats \emph{esquemes XML} (en anglés \emph{XML
    schema}).}

\begin{figure}
\begin{center}
\begin{alltt}
<?xml version="1.0" encoding="ISO-8859-1"?>
<!-- Aquest és l'exemple de DTD de EMAIL -->
<!ELEMENT EMAIL (DESTINATARI+, REMITENT?, DATA, ASSUMPTE, TEXT)>
<!ELEMENT DESTINATARI (NOM?, ADREÇA)>
<!ELEMENT REMITENT (NOM?, ADREÇA)>
<!ELEMENT NOM (#PCDATA)>
<!ELEMENT ADREÇA (#PCDATA)>
<!ELEMENT DATA (#PCDATA)>
<!ELEMENT ASSUMPTE (#PCDATA)>
<!ELEMENT TEXT (P)+>
<!ELEMENT P (#PCDATA)>
\end{alltt}
\end{center}
\caption{La DTD que defineix missatges de correu electrònic com el de
  la figura~\ref{fg:faxXML}.}
\label{fg:faxDTD}
\end{figure}

La segona línia del missatge de correu de la figura~\ref{fg:faxXML}
especifica el tipus del document tot indicant d'una banda l'etiqueta
arrel o principal del document (\texttt{EMAIL}) i l'URI
(\texttt{SYSTEM}) on es troba la DTD. Aquesta DTD es veu en la
figura~\ref{fg:faxDTD}; examinem ara la DTD línia a línia per a
comprendre com s'usen les DTD per a definir famílies (tipus) de
documents en XML:
\begin{enumerate}
\item La primera línia declara que la DTD és una DTD de la versió 1.0
  i que el joc de caràcters que s'usa és l'ISO-8859-1
  (\emph{Latin-1}):
\begin{small}\begin{alltt}
<?xml version="1.0" encoding="ISO-8859-1"?>
\end{alltt}\end{small}

\item La segona línia és un comentari. Els comentaris comencen amb
  \texttt{<!--} i acaben amb \texttt{-->} i es poden situar en
  qualsevol part d'una DTD.
\begin{small}\begin{alltt}
<!-- Aquest és l'exemple de DTD de EMAIL -->
\end{alltt}\end{small}

\item Les línies següents defineixen l'estructura del document
  definint els seus \emph{elements}. La línia
\begin{small}\begin{alltt}
<!ELEMENT EMAIL (DESTINATARI+, REMITENT?, DATA, ASSUMPTE, TEXT)>
\end{alltt}\end{small}
defineix l'element arrel o principal, \texttt{EMAIL}, i especifica que
es compon (en l'ordre especificat) d'un o més \texttt{DESTINATARI}s
(el símbol \texttt{+} indica que pot haver-hi un o més), d'un
\texttt{REMITENT} opcional (indicat amb \texttt{?}), d'una
\texttt{DATA}, d'un \texttt{ASSUMPTE} i d'un \texttt{TEXT}.

\item Un \texttt{DESTINATARI} del missatge de correu té dues parts: el
  \texttt{NOM} (opcional) i l'adreça de correu (\texttt{ADREÇA}):
\begin{small}\begin{alltt}
<!ELEMENT DESTINATARI (NOM?, ADREÇA)>
\end{alltt}\end{small}
\item El remitent es defineix igual:
\begin{small}\begin{alltt}
<!ELEMENT REMITENT (NOM?, ADREÇA)>
\end{alltt}\end{small}
\item El \texttt{NOM}, l'\texttt{ADREÇA} la \texttt{DATA} i el
  \texttt{ASSUMPTE} contenen text sense marques (indicat amb
  \texttt{\#PCDATA}):
\begin{small}\begin{alltt}
<!ELEMENT NOM (#PCDATA)>
<!ELEMENT ADREÇA (#PCDATA)>
<!ELEMENT DATA (#PCDATA)>
<!ELEMENT ASSUMPTE (#PCDATA)>
\end{alltt}\end{small}

\item El \texttt{TEXT} es compon d'un o més (\texttt{+}) paràgrafs
  (\texttt{P}). Si volguérem que el text estiguera compost per
  \emph{zero} o més paràgrafs usaríem ``\texttt{*}'' en comptes de
  ``\texttt{+}'':
\begin{small}\begin{alltt}
<!ELEMENT TEXT (P)+>
\end{alltt}\end{small}

\item Finalment, els paràgrafs contenen text:
\begin{small}\begin{alltt}
<!ELEMENT P (#PCDATA)>
\end{alltt}\end{small}
\end{enumerate}

Una de les aplicacions més importants de les DTD és que serveixen per
a la validació automàtica dels documents: un programa \emph{validador}
llegeix la DTD i el document XML i decideix si aquest últim és vàlid,
és a dir, si segueix l'especificació donada en la DTD. Per a què un
document siga vàlid con respecte a una DTD qualsevol, primer ha
d'estar ben format, és a dir, ha de complir amb les regles bàsiques
d'escriptura de documents XML esmentades més amunt.

Tot i que una DTD serveix per a la validació automàtica de documents
XML del tipus que la DTD defineix, el \emph{significat} de les
etiquetes (és a dir, quines conseqüències tindran quan es processe el
document XML) l'ha d'establir el programa o els programes que
processaran els documents.  Com ja s'ha dit, aquest significat pot
estar associat, per exemple, a la manera (\emph{estil}, vegeu la
p.~\pageref{pg:estil}) de presentar el document quan s'imprimeix (per
exemple, els destinataris del missatge de correu poden anar en
negretes), però també podria servir per a facilitar el processament de
la informació (per exemple, buscar tots els missatges que tenen un
determinat destinatari, o, en llibres codificats en XML, decidir
quines parts han de ser traduïdes automàticament de l'espanyol a
l'anglés i quines no perquè són cites literàries.\footnote{Hi ha un
  estàndard anomenat TEI, de l'anglés \emph{text encoding initiative},
  ``iniciativa de codificació de textos'' (\url{http://www.tei-c.org})
  que usa famílies de DTD per a definir diferents tipus d'obres
  (literàries i no literàries). De fet, existeixen, d'una banda, les
  antigues DTD per a SGML, i, d'altra, les DTD TEI per a XML.}  Fins i
tot fitxers que normalment no consideraríem documents, com ara les
memòries de traducció (vegeu el capítol~\ref{se:memtrad})
s'estructuren de manera estàndard usant un format basat en XML
anomenat TMX.

Una altra aplicació de les DTD és fer-les servir per a facilitar
l'edició de documents XML vàlids: un editor de documents XML pot
consultar la DTD per suggerir a la persona usuària l'element o
elements correctes en el context actual, o per emetre un missatge
d'error tan aviat com el document perda la seua validesa.

\subsection{(X)HTML}
\label{s3:HTML}

El format XHTML (\emph{extensible hypertext markup language} o
\emph{llenguatge extensible de marques per a hipertextos} és un dels
tipus de document que es poden definir amb XML i es correspon amb la
versió XML del llenguatge HTML (\emph{hypertext markup language}),
aquest últim basat en SGML (format precursor d'XML). Ambdós
llenguatges s'usen per a escriure els hipertextos d'Internet (vegeu el
capítol~\ref{se:Internet}) i són el que interpreten els navegadors
d'Internet (vegeu l'apartat~\ref{ss:navegadors}).


Tant en XHTML com en HTML, les marques tenen un significat determinat.
% Originalment, les marques estaven pensades per a expressar
% l'estructura del document, però amb el pas del temps el significat
% de les marques ha canviat i actualment algunes estan més aïna
% associades a la presentació del document durant la navegació (tot i
% que, darrerament, sembla que a poc a poc es va tornant al
% plantejament original)
Per exemple, (X)HTML indica el començament d'un segment de text
destacat (emfatitzat) amb la marca ``{\tt <em>}'' (4 caràcters ASCII)
i el final amb la marca ``{\tt </em>}'' (5 caràcters). (X)HTML serveix
per a codificar hipertextos: els enllaços (hiperreferències) a altres
documents (que al seu torn poden també ser hipertextos) comencen amb
``\texttt{<a href="} $\mathit{URI}$ \texttt{"}\texttt{>}'' ---on
$\mathit{URI}$ és el identificador del document enllaçat--- i acaben
amb ``\texttt{</a>}'', etc. Els documents (X)HTML comencen idealment
amb la marca ``\texttt{<html>}'' i acaben amb la marca
``\texttt{</html>}'', i tenen, entre altres elements, un títol
(``\texttt{<title>}\ldots\texttt{</title>}'') i un cos
(``\texttt{<body>}\ldots\texttt{</body>}'').  

La principal diferència entre HTML i XHTML és que com aquest últim
està basat en XML, el document ha de ser XML ben format i per tant no
pot haver-hi elements que s'obrin però no es tanquen; això sí és valid
en HTML quan es tracta d'elements buits com \texttt{img} o
\texttt{meta}. Altra diferència notable és que en XHTML els noms dels
elements van sempre en minúscula, mentres que en HTML poden anar en
majúscules.

\begin{figure}
\begin{center}
\begin{alltt}
<!DOCTYPE html PUBLIC "-//W3C//DTD XHTML 1.0 Strict//EN"
      "http://www.w3.org/TR/xhtml1/DTD/xhtml1-strict.dtd">
<\textbf{html}>
  <\textbf{head}>
    <\textbf{meta} http-equiv="Content-Type"
          content="text/html; charset=iso-8859-1"/>
    <\textbf{title}>Títol del document</\textbf{title}>
  </\textbf{head}>

  <\textbf{body}>
    <\textbf{h1}>Encapçalament de nivell 1</\textbf{h1}>

    <\textbf{h2}>Encapçalament de nivell 2</\textbf{h2}>

    <\textbf{p}>Aquest és el <\textbf{em}>primer</\textbf{em}> paràgraf 
    d'aquest document. El navegador decideix com dividir-lo 
    en línies per a presentar-lo. Idealment, hauria 
    d'acabar amb una marca de final de paràgraf. </\textbf{p}>

    <\textbf{h2}>Un altre encapçalament de nivell 2</\textbf{h2}>

    <\textbf{p}>Aquest és l'<\textbf{em}>últim</\textbf{em}> paràgraf 
    d'aquest document XHTML. Els documents XHTML poden contenir 
    <\textbf{a} href="http://www.apertium.org">enllaços</\textbf{a}> 
    a altres documents (X)HTML, locals o remots. </\textbf{p}>
  </\textbf{body}>
</\textbf{html}>
\end{alltt}
\end{center}
\caption{Un document XHTML, tal com el presentaria un editor de textos
  normal o usant l'opció ``view HTML source'' (veure font HTML) del
  navegador.}
\label{fg:HTML}
\end{figure}

\begin{figure}
\begin{center}
\includegraphics[scale=0.5]{vista-chromium.jpg}
  \end{center}
  \caption{El document XHTML de la figura~\protect\ref{fg:HTML}, vist
    a través d'un navegador d'Internet.}
\label{fg:HTMLnav}
\end{figure}

El document XHTML que es mostra en la figura~\ref{fg:HTML} es
mostraria en un navegador aproximadament com en la
figura~\ref{fg:HTMLnav}. Com es pot veure, la primera línia, que
comença amb ``\texttt{<!DOCTYPE}'' declara que el document és un tipus
de document XHTML estàndard segons la versió 1.0 \emph{estricta} de
XHTML (hi ha diverses versions).  En la tercera línia, l'etiqueta
``\texttt{<html>}'' indica el començament del document XHTML, i
l'etiqueta ``\texttt{</html>}'' del final indica el final del
document. Dins de l'element \texttt{html} trobem dos elements:
\texttt{head} (l'\emph{encapçalament}) i \texttt{body} (el \emph{cos}
del document).  Dins de l'encapçalament, un element \texttt{meta} que
no té contingut (fixeu-vos com s'obri i es tanca al mateix temps)
indica, a través de dues assignacions del tipus
\emph{atribut}\texttt{="}\emph{valor}\texttt{"}, que el \emph{joc de
  caràcters} que usa el document és l'ISO-8859-1, el més comú a Europa
occidental.\footnote{Per a escriure documents en \emph{txec} o en
  \emph{coreà}, caldria canviar part del valor de l'atribut
  \texttt{content} perquè la codificació ISO-8859-1 no permet escriure
  en aquests idiomes.} Dins de \texttt{head} també trobem l'element
\texttt{title}, el qual conté un títol que es presentarà, quan obriu
el document amb un navegador, en la barra del navegador, \emph{però no
  com a part del text del document}. Dins de \texttt{body} veiem
encapçalaments de nivell 1 (\texttt{h1}), encapçalaments de nivell 2
(\texttt{h2}), paràgrafs (\texttt{p}), parts del text destacades
(\texttt{em}), i enllaços (\texttt{a}). La taula~\ref{tb:15etiq}
descriu algunes de les etiquetes més importants que s'usen en XHTML.

Quan estem mirant un document HTML amb un navegador, podem veure les
etiquetes HTML que el formaten si seleccionem l'opció ``veure font
HTML'' (``view HTML source'') o similar que hi ha normalment en el
menú ``veure'' (``view'').

\begin{table}
\begin{center}
\begin{tabular}{l|p{3.2cm}|p{4.3cm}}
  \hline\hline
  \textsc{Element} & \textsc{Descripció} & \textsc{Més informació} \\
  \hline
  \texttt{<html>}\ldots\texttt{</html>} & Conté tot el document & \\
  \hline
  \texttt{<head>}\ldots\texttt{</head>} & Encapçalament & \\\hline
  \texttt{<body>}\ldots\texttt{</body>} & Cos & \\\hline
  \texttt{<meta}\ldots\texttt{/>} & Informació sobre el document & L'element és
  buit \\\hline
  \texttt{<title>}\ldots\texttt{</title>} & Conté el títol del document & \\\hline
  \texttt{<br/>} & Salt de línia forçat & L'element és buit. \\\hline
  \texttt{<h1>}\ldots\texttt{</h1>} & Encapçalament de nivell 1 & \\\hline
  \texttt{<h2>}\ldots\texttt{</h2>} & Encapçalament de nivell 2 & \\\hline
  \(\cdots\) & \(\cdots\) & \(\cdots\) \\\hline
  \texttt{<h6>}\ldots\texttt{</h6>} & Encapçalament de nivell 6 & \\\hline
  \texttt{<p>}\ldots\texttt{</p>} & Paràgraf & 
  \\\hline
  \texttt{<ul>}\ldots\texttt{</ul>} & Llista sense numerar & Conté elements \texttt{li} \\\hline
  \texttt{<ol>}\ldots\texttt{</ol>} & Llista numerada & Conté elements \texttt{li} \\\hline
  \texttt{<li>}\ldots\texttt{</li>} & Element de llista & Pot contenir una altra
  llista en el seu interior. 
  \\\hline
  \texttt{<em>}\ldots\texttt{</em>} & Èmfasi & \\\hline
  \texttt{<strong>}\ldots\texttt{</strong>} & Èmfasi fort & \\\hline
  \texttt{<code>}\ldots\texttt{</code>} & Exemple de codi & \\\hline
  \texttt{<a}\ldots\texttt{>}\ldots\texttt{</a>} & ``Àncora'' & Si porta un atribut
  \texttt{href="}URI\texttt{"}, el text entre ``\texttt{<a}\ldots\texttt{>}'' i
  ``\texttt{</a>}'' funciona com un enllaç al document que hi ha en l'URI.\\\hline
  \texttt{<img}\ldots\texttt{/>} & Imatge & L'atribut  \texttt{src="}URI\texttt{"} indica l'adreça on és la imatge. 
  L'atribut \texttt{alt="}\emph{text}\texttt{"} descriu la imatge amb paraules. L'element és buit.\\\hline
\end{tabular}
\end{center}
\caption{Alguns elements bàsics d'XHTML, la versió XML de HTML.}
\label{tb:15etiq}
\end{table}

\subsection{Altres formats basats en XML}
Hui dia s'han popularitzat els formats de documents basats en
XML. Alguns dels formats basats en XML que interessen als traductors
són: el format TMX (\emph{translation memory exchange}) que s'usa per
al intercanvi de memòries de traducció (fitxers amb extensió
\texttt{.tmx}, vegeu el capítol~\ref{se:memtrad}), el format TBX
(\emph{termbase exchange}) per al intercanvi de bases de dades
terminològiques (fitxers amb extensió \texttt{.tbx}, vegeu el
capítol~\ref{se:basesdades}) i el format XLIFF (\emph{XML localization
  interchange file format}). Aquest últim és un format creat per a la
estandardització del format empleat per les diverses ferramentes que
s'utilitzen durant el procés de \emph{localització} d'un producte
(vegeu el capítol~\ref{se:memtrad}).\footnote{La \emph{localització}
  es pot definit com el procés d'adaptació d'un producte als usos de
  una regió específica del món.}

A més dels formats descrits en el paràgraf anterior hi ha dos formats
usats pels processadors de textos més moderns que estan basats en
XML. Aquest formats són OpenDocument i Office Open XML. 

\subsubsection{OpenDocument}
OpenDocument és un format d'arxius obert i estàndard per a
emmagatzemar, entre altes, textos (fitxers amb extensió \texttt{.odt})
fulls de càlcul (fitxers amb extensió \texttt{.ods}) i presentacions
(fitxers amb extensió \texttt{.odp}). Aquest format es el emprat per
defecte per les aplicacions ofimàtiques LibreOffice i OpenOffice.org i
consisteix en un diversos documents XML ---per al contingut, els
estíls usats en el document, etc.--- comprimits amb ZIP.\footnote{ZIP
  és un format per a l'emmagatzemament de fitxers comprimits; els
  fitxers d'aquest tipus solen tenir l'extensió \texttt{.zip}.}

\subsubsection{Office Open XML}
Office Open XML, també conegut com OOXML o OpenXML, és un altre format
estàndard ---impulsat per Microsoft--- bastat en XML que s'utilitza
per a emmagatzemar textos (fitxers amb extensió \texttt{.docx}), fulls
de càlcul (fitxers amb extensió \texttt{.xlsx}) i presentacions
(fitxers amb extensió \texttt{.pptx}). Al igual que OpenDocument, un
arxiu OpenXML consisteix en diversos documents XML comprimits amb ZIP.

\section{Altres formats}
\subsection{RTF}
\label{s3:RTF}

RTF (\emph{rich text format}, és a dir, \emph{format de text ric}) va
ser un format impulsat per l'empresa Microsoft per a facilitar
l'intercanvi de documents entre processadors de textos mantenint-ne el
format, i que encara s'usa de vegades. RTF també té etiquetes, les
quals comencen normalment per una barra invertida
(\texttt{\textbackslash}); però els àmbits d'acció de les etiquetes
estan delimitats per claus (``\texttt{\{\ldots\}}'') en comptes de per
parelles d'etiquetes; per exemple, un segment en negretes s'indica amb
``\verb+{\b+\ldots\verb+}+'', mentre que en HTML s'usa
``\verb+<B>+\ldots\verb+</B>+''. La figura~\ref{fg:RTF} mostra part
d'un document RTF, en la qual es veuen algunes comandes de
l'encapçalament (començant amb ``\verb+{\rtf1+...'') i on també
  s'observa la manera especial com es codifiquen alguns caràcters.

\begin{figure}
\begin{center}
\begin{verbatim}
{\rtf1\ansi\ansicpg1252
\end{verbatim}
[\ldots]
\begin{verbatim}
\par
{\b T\`edtol en negretes}\par
Text del par\`a0graf en lletra normal amb alguns incisos 
{\i en cursives} i una marca de final de par\`a0graf al 
final.\par  
Els car\`a0cters que no pertanyen a l'ASCII est\`a0ndard 
s'indiquen amb codis especials (en aquest cas s'ha usat 
ANSI, amb {\i codepage} 1252, com es veu al principi del 
document), com per exemple en el mot 
{\i ling\'fc\'edstica}.\par
\end{verbatim}
[\ldots]
\end{center}
\caption{Part d'un document de text en format RTF.}
\label{fg:RTF}
\end{figure}


\subsection{PDF}
PDF (de l'anglés \emph{portable document format}, format portable de
document) és un altre format desenvolupat per a capturar completament
les característiques presentacionals dels documents. En PDF, el
document es mostra exactament amb la mateixa aparença independentment
de l'ordinador, sistema operatiu o aplicació que fem servir per
veure'l. Els documents PDF poden emmagatzemar, a banda del text, tipus
de lletra, gràfics, sons, etc. Aquest format va ser impulsat en els
anys noranta per l'empresa Adobe, que ofereix en l'actualitat un
programa gratuït\footnote{però no lliure ni de codi font obert}
anomenat Adobe Acrobat Reader DC--- per visualitzar els
documents;\footnote{Hi ha alternatives lliures i de codi font obert
  com ara Sumatra PDF, Evince, Okular, etc. Fins i tot, els mateixos
  navegadors vénen ja amb visors de PDF.}  per crear-los podem usar
programes especialitzats o qualsevol processador de textos que permeta
\emph{exportar} (realment \emph{imprimir}) el nostre document a PDF.

\section{Processadors de textos}\label{ss:proctext}
Un \emph{processador de textos} és un programa que permet crear i
modificar documents de text informatitzats. També s'hi poden usar
\emph{editors}: la diferència entre un processador de textos i un
\emph{editor} és que aquest últim programa és un processador de textos
plans (sense informació de format, etc.) que normalment s'usa per a
preparar textos en algun llenguatge artificial (per exemple, programes
escrits en algun llenguatge de programació) que serviran de entrada
per a un altre programa, o textos molt senzills on el format no és
crucial, com un missatge electrònic senzill.

El processament de textos també s'anomena \emph{tractament de textos}
(paral·lelament al francés, \emph{traitement textes}). En anglés,
l'èmfasi és sobre les paraules: \emph{word processing}.

Per descomptat, aquesta secció no pretén instruir en l'ús de cap
processador de textos concret, sinó que vol descriure breument algunes
característiques comunes als processadors de textos que s'usen en
l'actualitat. De fet, l'ús dels processadors de text s'aprén molt
millor en el laboratori; a més, en vista del fet que els processadors
de textos canvien constantment, potser és millor no aprendre a usar un
processador concret sinó a buscar en cada processador les eines que
necessitem. Això és possible perquè la major part dels processadors
van fornits de manuals o de sistemes d'ajuda en línia; alguns tenen
fins i tot ``assistents'' que observen el que fa la persona usuària i
li suggereixen ---amb més o menys fortuna--- possibles accions en cada
moment.

Quant a l'\emph{aparença} del programa, la major part dels
processadors de text es manifesten bàsicament com una o diverses
finestres, cada una de les quals mostra una secció d'algun dels
documents de text informatitzats que estem creant i modificant (els
documents que tenim \emph{oberts}).  La tendència actual afavoreix que
el text es mostre tan paregut com siga possible a la versió impresa
que se'n produirà, quant a format, tipus de lletra, etc.\ (en anglés,
aquest concepte de fidelitat visual es resumeix amb el mot {\em
  wysiwyg}, fet amb les sigles de ``what you see is what you get'', és
a dir, ``el que veieu és el que obtindreu''); la
secció~\ref{s3:problema_wysiwyg} descriu alguns problemes derivats
d'aquesta tendència.

Quant a l'\emph{operació}, els processadors de text assumeixen que la
major part dels caràcters que teclegem s'han d'inserir darrere del
caràcter que actualment es troba destacat amb una marca anomenada {\em
  cursor} de text (pot ser diferent del cursor o apuntador que indica
la posició virtual del ratolí en la pantalla), o bé l'han de
sobrescriure. No obstant això, es reserven determinades tecles
(algunes senzilles, i altres en combinació amb les tecles especials
``Alt'' o ``Control'') per a fer operacions, algunes molt bàsiques com
ara moure el cursor de text o esborrar caràcters i altres més
complexes, com ara apegar-hi un bloc de text que havíem esborrat
prèviament o enregistrar el text complet en el disc.\footnote{Aquestes
  tecles i combinacions de tecles que permeten un accés ràpid a
  operacions rutinàries se solen anomenar en anglés \emph{hotkeys};
  per exemple, en Windows, la combinació control--X retalla el text
  seleccionat, la combinació control--V insereix un text prèviament
  retallat, etc.} Però moltes d'aquestes operacions, conjuntament amb
d'altres que no s'usen tan sovint, també estan accessibles mitjançant
\emph{menús}; els noms d'aquests menús solen estar situats típicament
en la part de dalt de la finestra: si s'hi fa un clic del ratolí, es
despleguen i ens mostren les opcions que contenen, que podem elegir
amb el ratolí.

\paragraph{Sobre la cerca de paraules.}
Alguns processadors de textos permeten buscar usant les anomenades
\emph{expressions regulars}, les quals permeten, mitjançant caràcters
especials anomenats \emph{jòquers} (anglés \emph{wildcards}), buscar
tots els mots i totes les porcions de text que segueixen un patró
determinat. Per exemple, una recerca amb l'expressió regular
\texttt{pres*a} trobaria els mots \emph{prea}, \emph{presa},
\emph{pressa}, \emph{presssa}, etc., o l'expressió regular
\texttt{<[\^{}>]+>} que trobaria totes les etiquetes de l'estil de
XML, ja que comencen per \texttt{<}, tenen un o més (\emph{+})
caràcters que \emph{no} (\texttt{\^}) són \texttt{>}, i acaben amb
\texttt{>}. Per saber més sobre expressions regular podeu consultar la
pàgina de la Viquipèdia
\url{https://ca.wikipedia.org/wiki/Expressi%C3%B3_regular}.

% A més de l'accés als menús, hi ha operacions que normalment es fan amb
% el ratolí: una de les més importants és \emph{marcar} o
% \emph{seleccionar} una porció de text per a alguna operació posterior
% (per exemple, copiar-la o modificar-ne el tipus de lletra);
% típicament, es fa prement el botó principal del ratolí en un extrem
% del text que volem marcar i, anant, sense soltar-lo, a l'altre extrem.

% Heus ací algunes de les operacions bàsiques que es poden fer amb un
% processador de textos, organitzades de manera similar als menús que
% trobarem en un processador de text:
% \begin{itemize}
% \item Operacions amb fitxers:
%      \begin{itemize}
%      \item \emph{crear} un nou document de text;
%      \item \emph{obrir} un fitxer de document existent en el disc per
%        a treballar-hi;
%      \item \emph{guardar} el document en curs en un fitxer amb el
%        mateix nom i en el mateix format que tenia quan el vam obrir;
%      \item guardar el document en curs amb un altre nom o en un altre
%        format, per exemple el d'un altre processador de textos ({\em
%          guardar com});
%      \item \emph{imprimir} el document actual;
%      \item fer una \emph{presentació preliminar} en pantalla del
%        document tal com quedarà imprés (quan no és possible una
%        presentació completament \emph{wysiwyg})

%      \item \emph{eixir} del processador de textos.
%      \end{itemize}
%    \item Operacions d'edició. A més de les operacions que només es fan
%      des del teclat o amb el ratolí, com ara inserir un caràcter,
%      esborrar-lo, moure'ns pel text, o marcar-hi un passatge, hi ha
%      operacions de modificació del text molt importants que estan
%      accessibles en el menú d'\emph{edició}:
%      \begin{itemize}
%      \item esborrar (\emph{retallar}) la part marcada;
%      \item \emph{copiar} la part marcada a un \emph{portapapers} (una
%        memòria intermèdia) per a usar-la posteriorment;
%      \item inserir (\emph{enganxar} o \emph{apegar}) el contingut del
%        portapapers en un punt del document actual;
%      \item \emph{desfer} o invertir l'última operació (hi ha
%        processadors de text que recorden un nombre considerable
%        d'operacions bàsiques i permeten desfer-ne més d'una, en ordre
%        invers, per descomptat; altres només en recorden l'última
%        operació).
%      \end{itemize}
%    \item Operacions de cerca i substitució:
%      \begin{itemize}
%      \item \emph{buscar} una determinada paraula o seqüència de
%        caràcters en el text;\footnote{Alguns programes permeten buscar
%          usant les anomenades \emph{expressions regulars}, les quals
%          permeten, mitjançant caràcters especials anomenats
%          \emph{jòquers} (anglés \emph{wildcards}), buscar tots els
%          mots i totes les porcions de text que segueixen un patró
%          determinat. Per exemple, una recerca amb l'expressió regular
%          \texttt{pres*a} trobaria els mots \emph{prea}, \emph{presa},
%          \emph{pressa}, \emph{presssa}, etc., o l'expressió regular
%          \texttt{<[\^{}>]+>} que trobaria totes les etiquetes de
%          l'estil de XML, ja que comencen per \texttt{<}, tenen un o
%          més (\emph{+}) caràcters que \emph{no} (\texttt{\^}) són
%          \texttt{>}, i acaben amb \texttt{>}.}
%      \item repetir l'última recerca;
%      \item buscar una determinada paraula o seqüència de caràcters en
%        el text i \emph{substituir-la} per una altra (interactivament o
%        automàticament).
%      \end{itemize}
%    \item Operacions amb tipus de lletra:
%      \begin{itemize}
%      \item selecccionar la grandària de la lletra (per exemple, en
%        \emph{punts}, 1 polzada = 2,54~cm = 72 punts)
%      \item seleccionar la família tipogràfica (Times, Courier,
%        Helvetica...)
%      \item seleccionar l'estil de la lletra: redona, cursiva, negreta,
%        subíndex, superíndex, etc.
%      \end{itemize}
%    \item Operacions de formatatge (normalment els paràgrafs es
%      formaten sols, sense intervenció de la persona usuària en el
%      procés, segons que el va teclejant):
%      \begin{itemize}
%      \item Seleccionar l'alineació o la justificació de les línies del
%        text (alineades a la dreta, a l'esquerra, centrades, o
%        justificades\footnote{No s'ha de dir \emph{justificat a la
%            dreta}: la justificació és sempre als dos marges al mateix
%          temps; el que s'ha de dir en aquest cas és \emph{alineat a la
%            dreta}.});
%      \item seleccionar el format de la pàgina (grandària del paper,
%        marges inferior, superior, dret i esquerre), etc.  S'ha
%        d'esmentar que moltes operacions de formatatge es poden
%        automatitzar i regularitzar usant els estils
%        (p.~\pageref{pg:estil})
%      \end{itemize}
%    \item Altres eines
%      \begin{itemize}
%      \item demanar una \emph{correcció ortogràfica} del text;
%      \item accedir a un \emph{diccionari de sinònims}, etc.
%      \end{itemize}
% \end{itemize}

\section[Contingut, estructura i presentació]{Contingut, estructura i
  presentació dels documents} \label{ss:separac}

\subsection{El problema \emph{wysiwyg}}\label{s3:problema_wysiwyg}
La majoria dels processadors de textos actuals són \emph{wysiwyg} en
el sentit explicat més amunt: el text que s'edita es presenta
gràficament en la finestra pràcticament igual com es veurà en el paper
quan l'enviem a la impressora; això ha facilitat enormement l'accés de
tothom als processadors de textos. Però l'esquema \emph{wysiwyg},
completament generalitzat des de meitat dels vuitanta, té també, com
veurem, els seus inconvenients.  La persona escriptora tendeix a
centrar-se en els atributs \emph{visuals} del text (tipus i grandàries
de lletra, marges, etc.), ja que confia que una bona
\emph{presentació} transmetrà a les persones lectores l'estructura
\emph{lògica} que la persona escriptora té al cap.  Amb el document,
per tant, només es guardarà aquesta informació de presentació,
pràcticament sense cap indicació de l'estructura lògica dels
continguts. Imaginem les següents situacions problemàtiques:
  \begin{quote}
    Vladimir ha decidit que els títols de secció de l'informe anual
    que li han encarregat estaran en Helvetica de 14 punts, negreta i
    els de subsecció en Arial de 12 punts, negreta cursiva. A la seua
    directora no li agraden així i li'ls ha fet canviar a Lucida Sans
    de 14, negreta i Lucida de 12, negreta sense cursives. Com que
    l'informe ha d'estar acabat per a demà de matí, Vladimir es queda
    a l'oficina fins a les 11 de la nit, canviant un a un els tipus de
    lletra del títols de seccions i subseccions. A l'endemà, de matí,
    Marina, la directora, li passa un document amb una secció més que
    s'ha d'inserir entre la 4 i la 5. Vladimir no pot anar a esmorzar:
    ha de canviar els números de seccions i subseccions a partir de la
    5 i repassar si s'ha de canviar alguna referència que es faça des
    d'una part del text a una secció pel seu número.
  \end{quote}

  \begin{quote}
    Ens han encarregat traduir un text informatitzat. En la llengua
    d'origen és costum posar en \emph{cursives} tant els mots
    estrangers (``\emph{Sprachgefühl}'') com els termes quan es
    defineixen per primera volta (``Un \emph{octet} és...''), se sagna
    la primera línia de tots els paràgrafs, i els números de secció
    porten un punt al final (``1.1. Introducció''), però en la llengua
    d'arribada els termes nous van en negretes (``Un \textbf{octet} és
    ...''), se sagna la primera línia de tots els paràgrafs excepte la
    del primer paràgraf d'una secció, i els números de secció no
    porten punt al final (``1.1 Introducció''). El text ha estat
    traduït mantenint les convencions de la llengua d'origen: per a
    fer-lo adequat a la llengua d'arribada, ens toca, d'una banda,
    anar mirant un a un els segments de text en cursives, decidir si
    són definicions, i canviar-los a negretes si cal; d'altra banda,
    ens toca anar llevant el puntet final de tots els números de
    secció.
  \end{quote}
  
  En aquests dos casos, si la persona que va escriure els textos només
  va codificar informació relativa a la presentació visual no podrem
  evitar fer els treballs tediosos descrits. Se podria dir que si qui
  escriu es deixa portar per la filosofia ``what you see is what you
  get'' acaba amb ``what you see is \emph{all} you get'', és a dir,
  només té el que veu. Però la majoria dels processadors de textos
  \emph{wysiwyg} actuals permeten un cert nivell de codificació de
  l'\emph{estructura}, a través dels anomenats
  \emph{estils}:\footnote{Aquesta és la denominació usada per
    \emph{Word} i pels processadors lliures i de codi obert
    \emph{OpenOffice.org} i \emph{LibreOffice}.} hi ha \emph{estils de
    paràgraf} (paràgraf del cos de text, encapçalaments de diversos
  nivells, etc.) i \emph{estils de caràcter} (definicions, èmfasi,
  èmfasi forta, text d'ordinador, etc.).\label{pg:estil} A cada estil
  se li assignen unes determinades característiques de presentació:
  per exemple, els encapçalaments de nivell 2 van en Helvetica de 14
  punts negreta i numerats automàticament amb el número de la secció
  de nivell 1, un punt i el número de la secció de nivell 2; les
  definicions van en negreta i l'èmfasi en cursiva, etc. El
  processador de textos aplica automàticament les mateixes
  característiques de presentació \emph{a tots els segments del
    document} que tenen aquell mateix estil.  Això resoldria les
  situacions problemàtiques explicades més amunt.
  
  En el primer problema, si s'hagueren usat els estils com s'indica,
  la numeració i l'estil de les seccions es determinaria
  automàticament i només caldria indicar (només una vegada) quin tipus
  de lletra correspon als títols de secció; a més és renumerarien
  automàticament totes les seccions. Si les referències d'unes
  seccions a altres s'hagueren fet usant referències creuades
  simbòliques (molts processadors de textos les permeten), també
  s'actualitzarien automàticament.
  
  En el segon problema, si el document haguera contingut informació
  sobre quins termes són definicions i quins són mots estrangers,
  només caldria canviar l'estil de les definicions i totes quedarien
  en negretes. D'altra banda només caldria indicar que no és necessari
  l'últim punt en els números de secció i tots passarien
  automàticament al format desitjat.
  
  Aquests exemples il·lustren la conveniència que els autors dels
  documents se centren més en l'estructuració lògica del contingut del
  document que escriuen. Després, només cal indicar al processador
  quina ha de ser la presentació de cada element d'aquesta estructura
  lògica i obtindrem la presentació desitjada.

\subsection{Fulls d'estil}
En XML i HTML, aquesta separació entre l'estructura del contingut i la
presentació d'un document s'executa a través d'especificacions
anomenades \emph{fulls d'estil}. Un dels tipus més senzills de fulls
d'estil són els anomenats fulls d'estil en cascada\footnote{Més
  informació en \url{http://www.w3c.org/Style/CSS/}.} (CSS,
\emph{cascaded style sheets}) que s'usen sobre tot amb navegadors i
HTML, encara que també es poden usar per a presentar XML directament
en els navegadors.

Els fulls d'estil CSS assignen característiques de presentació a cada
element del document. Per exemple, l'ordre CSS
\begin{verbatim}
h2 {   display : block ;
       font-size : large ;
       font-family : sans-serif ;
       text-align : left ;
       margin-top: 0.2cm ;
       margin-bottom : 0.2cm ; }
\end{verbatim}
indica que tots els encapçalaments de segon nivell (\texttt{h2}) de
(X)HTML es visualitzen (\texttt{display}) com a blocs de text separat
(\texttt{block}), amb una grandària de lletra (\texttt{font-size})
gran (\texttt{large}) de la família \emph{sans serif}, alineat
(\texttt{text-align}) a l'esquerra, i amb màrgens superior i inferior
de 0,2 cm.

Els fulls d'estil es poden usar també per a visualitzar documents XML
directament en els navegadors més recents. Per exemple, podem fer que
la presentació visual del missatge de correu electrònic de la
figura~\ref{fg:faxXML} tinga un encapçalament amb el text
``\emph{Missatge de correu}'' centrat, gran i en negretes amb aquesta
ordre CSS (amb comentaris entre \texttt{/*} i \texttt{*/}):
\begin{verbatim}
EMAIL:before {                      /*Abans del EMAIL*/
   content : "Missatge de correu" ; /*El text desitjat*/
   display : block ;                /*com un bloc*/
   font-weight : bold ;             /*en negreta*/
   text-align : center ;            /*centrat */
   font-size : x-large ;            /*i amb lletra extra-gran*/
}
\end{verbatim}

En el cas dels fulls d'estil CSS, l'esquema d'ús és el que s'indica en
la figura~\ref{fg:CSS}: el navegador llig el document HTML o XML, hi
aplica els estils del full CSS, i genera una presentació. El full
d'estil CSS pot estar en el mateix fitxer que el document HTML o XML,
o en un fitxer extern.

\begin{figure}
$$
\left.
\begin{array}{rcl}
\mbox{\textsf{Document XML o HTML}} & 
\to \\
\mbox{\textsf{Full d'estil CSS}} &\to 
 \\
\end{array}
\right.
\mbox{\framebox{\parbox{1.8cm}{\textsf{Navegador}}}}
\to \mbox{\parbox{1.8cm}{\textsf{Presentació}}}
$$
\caption{Presentació de documents XML i HTML amb fulls d'estil CSS.}
\label{fg:CSS}
\end{figure}


Per a la presentació de documents XML, existeix un llenguatge de
programació de fulls d'estil molt més potent que CSS anomenat
XSL\footnote{Més informació en \url{http://www.w3c.org/Style/XSL/}.}
(\emph{extended stylesheet language}) que permet \emph{transformar} un
document XML (amb etiquetes que n'indiquen l'estructura del contingut)
en un altre document XML, HTML o de qualsevol altre format
(Postscript, PDF, RTF, etc.), per exemple, per a presentar-lo
visualment (veure figura~\ref{fg:XSL}).  Els navegadors més recents ja
són capaços d'aplicar fulls d'estil XSL a pàgines \emph{web} escrites
en XML i presentar-les com si estigueren escrites originalment en
HTML.


Aquest tipus de presentació visual no és l'única transformació
possible que podem obtindre amb fulls d'estil: com es mostra en la
figura~\ref{fg:braille}, amb el mateix document XML podem generar un
conjunt de \emph{vistes} del seu contingut en diferents mitjans
(\emph{media}) només usant el full d'estil adequat.

\begin{figure}
$$
\left.
\begin{array}{rcl}
\mbox{\textsf{Document XML}} & 
\to \\
\mbox{\textsf{Full d'estil XSL}} &\to 
 \\
\end{array}
\right.
\mbox{\framebox{\parbox{2.1cm}{\textsf{Processador de XSL}}}}
\to \mbox{\parbox{2.1cm}{\textsf{Document (XML, HTML, etc.) }}}
$$
\caption{Transformació de documents XML amb fulls d'estil XSL.}
\label{fg:XSL}
\end{figure}

\begin{figure}
  \centering \setlength{\unitlength}{1cm}
  \begin{picture}(11,6)(0,1.5)
\put(0,1){\makebox(3,2){\sf Fitxer de so}}
\put(4,1){\makebox(3,2){\sf Document Braille}}
\put(8,1){\makebox(3,2){\sf Document per a mòbils}}
\put(1.5,2.6){\makebox(0,0){\LARGE \twonotes}}
\put(5.5,2.6){\makebox(0,0){\LARGE \Printer}}
\put(9.5,2.6){\makebox(0,0){\LARGE \Mobilefone}}
\put(0,4){\framebox(3,1){\sf Full d'estil 1}}
\put(4,4){\framebox(3,1){\sf Full d'estil 2}}
\put(8,4){\framebox(3,1){\sf Full d'estil 3}}
\put(4,6){\makebox(3,2){\sf Document XML}}
\put(5.5,6.5){\line(0,-1){0.75}}
\put(1.5,5.75){\line(1,0){8}}
\put(1.5,5.75){\vector(0,-1){0.5}}
\put(5.5,5.75){\vector(0,-1){0.5}}
\put(9.5,5.75){\vector(0,-1){0.5}}
\put(1.5,4){\vector(0,-1){1}}
\put(5.5,4){\vector(0,-1){1}}
\put(9.5,4){\vector(0,-1){1}}
  \end{picture}
  \caption{Obtenció de tres presentacions diferents d'un únic document
    XML mitjançant fulls d'estil.}
  \label{fg:braille}
\end{figure}

\subsection{Accessibilitat}
En quasi tota la discussió anterior hem parlat de presentació
referint-nos sempre a un mitjà visual, de manera que hem exclòs, per
exemple, les persones que tenen discapacitats o limitacions
relacionades amb el sentit de la vista (poden no veure gens o veure
molt malament, o patir atacs epilèptics quan veuen una imatge que
canvia ràpidament de color).  Quan presentem un document visualment,
intentem que la representació visual de les diverses parts del
document comuniquen l'estructura lògica del contingut a la persona
lectora, però, com presentem l'estructura lògica d'un document a una
persona cega?  Mecanismes com els tipus o grandàries de lletra o la
forma o l'alineament visual dels paràgrafs, llistes o taules no li
serveixen; aquesta persona potser vol accedir als documents mitjançant
un tauler Braille (una espècie de pantalla tàctil on es formen els
signes de l'alfabet dels invidents, vegeu la figura~\ref{fg:braille})
o mitjançant un sistema de síntesi de veu que llig la pàgina en veu
alta.  

Si qui ha escrit el document només ha codificat l'estructura lògica
que tenia en el seu cap mitjançant indicadors visuals de format
(negretes o cursives per a l'èmfasi, paràgrafs d'una línia en lletra
més grossa per a títols, etc.) serà difícil transformar aquest format
per a una altra presentació.

\begin{figure}
  \centering
  \includegraphics[scale=0.5]{Refreshable_Braille_display.jpg}
  \caption{Tauler braille (imatge presa de l'entrada \emph{Refreshable Braille Display} de la Wikipedia en anglés)}
\end{figure}

Però no només les persones amb discapacitats visuals poden tenir
problemes; qui llig un document ho pot estar fent a través d'una
pantalla de text (no gràfica) o menuda (com la d'un telèfon mòbil), o
a través d'una connexió molt lenta a la xarxa, o pot estar en una
situació en la qual els seus ulls estiguen ocupats (per exemple, quan
condueix un vehicle). La presentació de documents en aquests mitjans
té problemes similars.

Si l'èmfasi en el document ha estat emmagatzemat com a èmfasi i no amb
lletra negreta o cursiva, si els títols de secció estan indicats com a
tals i no perquè són paràgrafs d'una única línia en negretes grosses,
serà molt més fàcil transformar-lo per a la presentar-lo en un mitjà
no visual o en una pantalla reduïda o limitada, per exemple, usant
\emph{fulls d'estil} especialment concebuts per a la presentació
\emph{aural} (sonora), \emph{tàctil} (Braille), etc.

La separació de l'estructuració del contingut, d'una banda, i dels
mecanismes de presentació del document, d'una altra banda, facilita
l'\emph{accessibilitat} al document a través de diversos mitjans a
persones discapacitades o en situacions especials.



\begin{persabermes}{tecnologies auxiliars per a la generació de textos}
  \paragraph{Reconeixement automàtic de la parla.}
  El \emph{reconeixement automàtic de la parla} (RAP) es pot definir
  com la producció de textos informatitzats ---en \emph{temps real},
  és a dir, tan instantàniament com siga possible--- a partir de la
  veu humana (vegeu \citealt{samuelson-brown96b}). El RAP de propòsit
  general està encara molt lluny de ser perfecte, i, de fet, és encara
  un camp de recerca actiu, encara que recentment s'ha incorporat
  bastant satisfactòriament en dispositius com ara telèfons
  mòbils (per exemple, els dispositius amb sistema operatiu Android
    permeten fer recerques per veu). En canvi, el RAP per a un
  propòsit específic (per exemple, la consulta telefònica d'horaris de
  trens o de les condicions del trànsit) està molt més avançat. La
  major part de la inversió de la comunitat internacional en RAP és,
  per raons òbvies, sobre l'anglés.

  El RAP genera text a partir de la veu recollida a través d'un
  micròfon utilitzant una dispositiu de captura per a digitalitzar-la
  i després un sistema de reconeixement automàtic de la veu
  (\emph{automatic speech recognition}) per a detectar fonemes,
  síl·labes o paraules completes (depén del sistema concret) i
  traduir-les posteriorment a un text informatitzat.  Hi ha sistemes
  de reconeixement {\em independents del parlant} i sistemes
  \emph{dependents del parlant} (els últims normalment han de ser
  \emph{entrenats} per la persona abans del seu ús).  El RAP és
  especialment difícil per la gran variabilitat acústica que presenten
  els fonemes:
  \begin{itemize}
  \item segons el context articulatori (per exemple, no és igual el so
    del fonema palatal representat pel dígraf \emph{ig} en ``passeig
    curt'' ---sord--- que en ``passeig allargat'' ---sonor---);
  \item segons el parlant (cada persona té uns òrgans fonadors de
    forma diferent ---acústicament diferents--- i processos de
    producció de la parla diferents ---per exemple, n'hi ha qui parla
    més a poc a poc i qui parla molt de pressa---);
  \item segons el dialecte del parlant (per exemple, els valencians
    fem africades les \emph{j} que en català central són palatals
    fricatives sonores).
  \item segons l'estat emocional del parlant, etc.
  \end{itemize}
  És un fet ben establert que, per a superar aquestes dificultats, els
  humans fem un ús molt intensiu dels coneixements lingüístics que
  tenim sobre l'idioma que estem escoltant i del context comunicatiu:
  així, si sentim dir ``\emph{percà nom} passes l'\emph{antre xoc} de
  \emph{craus}?'' a un amic quan veiem que no pot obrir el cotxe,
  entenem perfectament que ens vol dir \emph{Per què no em
      passes l'altre joc de claus?}, o si sentim dir en veu alta ``mu
  han dim moltis baltes'' és molt probable que entenguem clarament
  ``m'ho han dit moltes voltes'' a pesar dels canvis fonètics, ja que
  inconscientment busquem la interpretació correcta més propera al que
  hem sentit (en el context concret en què es diu la
  frase). Considereu aquest doblet anglés clàssic sobre el
    tema: \emph{people can easily recognize speech} no és molt
    diferent de {\em people can easily wreck a nice beach}; un altre
    doblet el formen les expressions \emph{sax and violins on TV} i la
    més versemblant \emph{sex and violence on TV}. Els resultats de
  la RAP són especialment dependents de les particularitats
  lingüístiques de la llengua involucrada i l'èxit depén de
  l'existència d'un bon \emph{model de llengua} ---ràpid i concís, és
  a dir, computacionalment eficient--- que simule la part no
  contextual de la comprensió humana i permeta obtenir el text més
  probable en un idioma determinat a partir del text en brut produït
  pel sistema de RAP. La major part dels sistemes usen vocabularis
  grans i models estadístics.

  \paragraph{Reconeixement automàtic de textos escrits.}
  El \emph{reconeixement automàtic de textos escrits} (RATE) es pot
  definir com la producció de textos informatitzats a partir de textos
  manuscrits o tipografiats. En el cas de textos tipografiats la tasca
  és molt més senzilla; en el cas de manuscrits, la complexitat és
  comparable a la del reconeixement de la parla.
  
  El RATE genera un text informatitzat a partir d'un document imprés,
  usant un escàner (o \emph{scanner}) i un programa de reconeixement
  òptic (també se'n diu {\em automàtic}) de caràcters (OCR,
  \emph{optical character recognition}).  Primerament, el document
  imprés és llegit (escanejat o escandit) usant l'escàner, i se'n
  genera un fitxer que en conté la imatge digital (per exemple, una
  graella molt fina de quadrats blancs i negres).  Després, el
  programa d'OCR llig la pàgina, descobreix on són els paràgrafs, les
  línies i, finalment, els caràcters concrets, i els transforma en un
  text informatitzat (normalment bastant imperfecte, especialment si
  és manuscrit).  Com en el cas del reconeixement de la parla, és
  crucial l'ús d'informació sobre l'idioma concret (diccionaris,
  estadística sobre les seqüencies de lletres) per a corregir els
  errors de l'OCR.  Per exemple, si un programa de lectura automàtica
  de textos produeix per error el text ``4ixò 6s uua mcrda'' no cal
  dir què hi llegim sense massa problemes, malgrat els errors en tots
  els mots; això és gràcies als nostres coneixements sobre les
  seqüències de lletres comunes en català.

\mbox{}
\end{persabermes}




\section{Qüestions i exercicis}
\begin{enumerate}
\item Per a validar un document XML necessitem {\ldots}
  \begin{enumerate}
  \item {\ldots} un altre document XML, aquest últim amb les marques
    sense contingut.
  \item {\ldots} un full d'estil CSS.
  \item {\ldots} una definició de tipus de document (DTD).
  \end{enumerate}

\item Com s'indica en una DTD que l'element \texttt{teixit} conté
  opcionalment els elements \texttt{grandaria} i \texttt{color} en
  aquest ordre?
  \begin{enumerate}
  \item \verb|<!MARK teixit grandaria, color #OPTIONAL>|
  \item \verb|<!MARK teixit (grandaria?,color?)>|
  \item \verb|<!ELEMENT teixit (grandaria?,color?)>|
  \end{enumerate}

\item Un document XML és \emph{vàlid} {\ldots}
  \begin{enumerate}
  \item {\ldots} si només usa els noms d'elements definits a la DTD;
    la resta de les directrius de la DTD només serveixen per fer
    documents \emph{ben formats}.
  \item {\ldots} si només usa les marques vàlides dels documents HTML.
  \item {\ldots} si segueix les regles de la DTD quan inclou un
    element dins d'un altre i, a més, no inclou cap element no definit
    a la DTD.
\end{enumerate}

\item Un text informatitzat es caracteritza principalment {\ldots}
  \begin{enumerate}
  \item {\ldots} pel seu format, d'una banda, i pel joc de caràcters
    amb què està codificat, d'altra.
  \item {\ldots} per la versió del sistema operatiu i el processador
    de textos amb què ha estat escrit.
  \item {\ldots} pel full d'estil que indica els aspectes estètics de
    la seua presentació.
\end{enumerate}

\item Què fa que el següent fragment de XML estiga \emph{mal format}?
  \begin{center}\verb|<tit int=hi>Zjuknim agarnow</tit>|\end{center}
  \begin{enumerate}
  \item Entre \verb|tit| i \verb|>| no pot haver-hi res.
  \item L'etiqueta \verb|tit| no és vàlida en XML; hauria de ser
    \verb|title|.
  \item Si hi ha algun atribut, el valor ha d'anar entre cometes.
  \end{enumerate}

\item Si en una DTD trobem les regles
\begin{verbatim}
<!ELEMENT taula (capçalera?,fila+)>
<!ELEMENT fila (casella*)>
<!ELEMENT casella (#PCDATA|taula)*>
\end{verbatim}
  quina de les tres situacions següents és vàlida d'acord amb aquesta
  DTD?
  \begin{enumerate}
  \item \verb|<taula></taula>|
  \item \verb|<taula><fila><casella>zz<taula><fila></fila></taula>zz|
    \verb|</casella><fila></taula>|
  \item \verb|<taula><fila><casella>zz</casella><casella>ww</casella>|
    \verb|</fila></taula>|    
  \end{enumerate}

\item Què indica el fragment \texttt{encoding="\ldots"} en la primera
  línia (\texttt{<?xml\ldots?>}) d'un document XML?
  \begin{enumerate}
  \item La versió de XML.
  \item On és la DTD necessària per a validar-lo.
  \item Quin és el joc de caràcters que usa el document XML.
  \end{enumerate}

\item Quants octets (\emph{bytes}) ocupa el segment de XML
  següent: \begin{center}\verb|<qq>ww</qq>|\end{center}
  \begin{enumerate}
  \item 11 com a mínim, depenent de la codificació.
  \item 11, independentment de la codificació.
  \item 4 exactament.
  \end{enumerate}

\item Quan les marques de format només especifiquen el
  \emph{contingut} d'un document (identificant les parts i
  l'estructura de cada una), com s'assigna una \emph{presentació}
  determinada al document?
  \begin{enumerate}
  \item Amb un o més fulls d'estil.
  \item Amb una codificació de caràcters (p.e., Unicode o ISO-8859-1).
  \item No s'hi pot assignar presentació.
  \end{enumerate}

\item Què es conserva d'ASCII en els sistemes de codificació de
  caràcters més avançats com Unicode UTF-8, ISO-8859-1
  (\emph{Latin-1}), etc.?
  \begin{enumerate}
  \item Els caràcters i els seus números de codi.
  \item Els caràcters, però amb números de codi diferents.
  \item No en queda res. S'ha reorganitzat tota la codificació.
  \end{enumerate}

\item Som a Eslovàquia, on s'usa la codificació de caràcters
  ISO-8859-2 (\emph{Latin-2}). Des d'Alacant, ens envien un document
  de text pla, escrit en codificació ISO-8859-1 (\emph{Latin-1}) i
  l'obrim com si fóra ISO-8859-2 (\emph{Latin-2}). Què passa?
  \begin{enumerate}
  \item No veiem bé cap lletra: tot són símbols estranys i
    inintel·ligibles.
  \item Veiem bé totes les lletres excepte les accentuades, les que
    porten dièresi, la \emph{ñ} o la \emph{ç}: en el seu lloc
    apareixen altres símbols o lletres típiques de les llengües
    d'Europa de l'Est.
  \item Veiem bé totes les lletres excepte les accentuades, les que
    porten dièresi, la \emph{ñ} o la \emph{ç}: en el seu lloc
    apareixen les versions sense accent, la \emph{n} o la \emph{c}.
\end{enumerate}

\item Què és RTF?
  \begin{enumerate}
  \item Un esquema avançat de codificació de caràcters.
  \item Un format obert d'intercanvi de memòries de traducció.
  \item Un format obert per a intercanviar documents de text entre
    processadors de textos.
  \end{enumerate}

\item Un document HTML té un enllaç amb el text ``Més informació'' i
  amb URI de destinació
  \verb|http://www.detalls-e.com/mes.html|. Com és aquest
  enllaç en HTML?
  \begin{enumerate}
  \item \verb|<a href="http://www.detalls-e.com/mes.html">Més informació</a>|
  \item \verb|<a href="Més informació">http://www.detalls-e.com/mes.html</a>|
  \item \verb|<a htxt="Més informació" href="http://www.detalls-e.|
    \verb|com/mes.html">|
  \end{enumerate}

\item On va el títol d'un document HTML (el que es mostra en la
  \emph{barra} del navegador)?
  \begin{enumerate}
  \item En un element \verb|title| dins de \verb|head|.
  \item En un element \verb|title| dins de \verb|body|.
  \item En un element \verb|h1| dins de \verb|head|.
  \end{enumerate}

\item Si els caràcters d'un text estan codificats usant el joc de
  caràcters ISO-8859-1 (\emph{Latin-1}), quins codis tenen les lletres
  de la \verb|A| a la \verb|Z|?
  \begin{enumerate}
  \item Depén del format del text (HTML, etc.).
  \item Els mateixos que en la codificació ASCII.
  \item Els que tenien en la codificació ASCII més 128.
  \end{enumerate}

\item Quants octets (\emph{bytes}) ocupa com a mínim el següent fitxer
  HTML?
  \begin{center}
    \verb|<html><body><p>Text.</p></body></html>|
  \end{center}
  \begin{enumerate}
  \item 11
  \item 38
  \item 76
  \end{enumerate}

\item En la codificació de caràcters ISO-8859-1 (\emph{Latin-1}), tots
  els caràcters accentuats de l'espanyol o del català tenen
  codis {\ldots}
  \begin{enumerate}
  \item {\ldots} entre 0 i 127.
  \item {\ldots} entre 128 i 255.
  \item {\ldots} més grans que 256.
  \end{enumerate}

\item Un text codificat en ISO-8859-1 (\emph{Latin-1}) té 1000
  caràcters justos (comptant els espais en blanc i els salts de
  línia). Quants octets (\emph{bytes}) ocupa?
  \begin{enumerate}
  \item 1000 exactament, 1 per caràcter.
  \item 2000 exactament, 2 per caràcter.
  \item entre 1000 i 2000, entre 1 octet i 2 octets per caràcter.
  \end{enumerate}

\item En XML, si s'obri un element amb la marca \verb|<frase>|, amb
  quina marca es tanca?
  \begin{enumerate}
  \item Amb \verb|</frase>|.
  \item Amb \verb|<frase>|.
  \item Automàticament quan s'obri qualsevol altre element.
  \end{enumerate}

\item Si en un document XML trobem la situació
  \begin{center}\verb|<rec><id>Zork</id><addr>Zmeggs</addr></rec>|\end{center}
  i una DTD defineix l'element \verb|rec| amb la regla
   \begin{center}\verb|<!ELEMENT rec (id,up?,addr*)>|\end{center} Pot
   ser que el document siga vàlid segons la DTD?
   \begin{enumerate}
   \item Depén de com siga d'estricte el programa validador.
   \item No, perquè aquesta situació no és vàlida.
   \item Sí, si la resta del document és vàlida.
   \end{enumerate}

\item Què veiem si obrim un text HTML amb un editor de textos senzill
  com el \emph{Bloc de notes}, \emph{Llibreta} o \emph{Notepad} de
  Windows?
  \begin{enumerate}
  \item El text HTML però sense les marques entre ``\verb|<|'' i
    ``\verb|/>|''.
  \item El text HTML tal com està fet per dins, amb les marques entre
    ``\verb|<|'' i ``\verb|>|'' i tot.
  \item Una pantalla en blanc.
  \end{enumerate}

\item En què es diferencien dues extensions d'ASCII diferents?
  \begin{enumerate}
  \item En els caràcters assignats als 256 codis.
  \item En els caràcters assignats als codis del 0 al 127.
  \item En els caràcters assignats als codis del 128 al 255.
\end{enumerate}

\item Si en una DTD trobem la regla
   \begin{center}\verb|<!ELEMENT cv (nom,any?,ob+)>|\end{center}
   quina de les tres situacions següents no és vàlida d'acord amb
   aquesta DTD?
   \begin{enumerate}
   \item \verb|<cv><nom>Pere</nom><any>1992</any></cv>|
   \item \verb|<cv><nom>Pere</nom><ob>Escrits</ob></cv>|
   \item \verb|<cv><nom>Pere</nom><any>1992</any><ob>Crits</ob><ob>Plors</ob></cv>|
   \end{enumerate}

 \item En un document HTML volem que la frase \emph{aquest document}
   siga un enllaç al document que té l'URI
   \verb|http://www.uc.za/t.html|: quina de les següents porcions de
   HTML és la correcta?
   \begin{enumerate}
   \item \verb|<a url="http://www.uc.za/t.html">aquest document</a>|
   \item \verb|<a href="http://www.uc.za/t.html">aquest document</a>|
   \item \verb|<link url="http://www.uc.za/t.html">aquest document</link>|
   \end{enumerate}

 \item El fragment de document HTML
   ``\verb|<strong><em>link</strong></em>|'' té una errada. Quina n'és
   la causa?
   \begin{enumerate}
   \item El nom de les marques no és vàlid, perquè no n'indica cap
     informació sobre el contingut.
   \item L'ordre de les marques d'obertura i clausura no és correcte.
   \item No s'ha indicat el valor de l'atribut \verb|href| de
     l'element \verb|em|.
   \end{enumerate}

 \item La longitud mitjana d'un mot en gondavés és de 5,5 caràcters i
   l'edició electrònica de \emph{Gundhawól Vlâj} (``La Veu de
   Gondàvia''), té uns 100.000 mots diaris com a mitjana. Si el
   gondavés s'escriu usant la codificació ISO-8859-1 (\emph{Latin-1}),
   quants exemplars del diari es poden guardar en un CD-ROM?
   \begin{enumerate}
   \item Més de dos anys.
   \item Un exemplar només.
   \item Un mes aproximadament.
   \end{enumerate}

 \item Com s'indica en una DTD que l'element \verb|<fitxa>| conté
   obligatòriament els camps \verb|<nom>| i \verb|<tel>| i,
   opcionalment, el camp \verb|<email>|?
   \begin{enumerate}
   \item \verb|<!ELEMENT fitxa (nom,tel,email?)>|
   \item \verb|<!ELEMENT fitxa (nom,tel,email+)>|
   \item \verb|<!ATTLIST fitxa nom CDATA #required| \newline
     \verb| tel CDATA #required email CDATA>|
   \end{enumerate}

% \item En XHTML (o HTML) hi ha elements buits (sense contingut).
%   Digues en quina d'aquestes tres opcions els dos elements són buits.
%   \begin{enumerate}
%   \item \texttt{h1} i \texttt{meta}.
%   \item \texttt{meta} i \texttt{img}.
%   \item \texttt{meta} i \texttt{head}.
%   \end{enumerate}

 \item En XML, què vol dir \texttt{<mang/>}?
   \begin{enumerate}
   \item No vol dir res, perquè no hi ha cap element que es diga
     \texttt{mang}.
   \item El mateix que \texttt{<mang></mang>}.
   \item No vol dir res, perquè hauria de ser \texttt{</mang>}.
   \end{enumerate}

 \item Quin d'aquests tres elements XHTML (o HTML) no pot anar dins de
   l'element \texttt{body}?
   \begin{enumerate}
   \item \texttt{meta}.
   \item \texttt{img}.
   \item \texttt{h1}.
   \end{enumerate}

 \item Tenim un document XML que és vàlid d'acord amb una determinada
   DTD. Esborrem un element complet (per exemple
   \texttt{<element>text</element>}), el qual no és l'element arrel
   del document. Quina d'aquestes tres situacions no és possible?
   \begin{enumerate}
   \item Que el document XML resultant no siga vàlid respecte de la
     DTD.
   \item Que el document XML siga vàlid respecte de la DTD.
   \item Que el document XML no siga un document XML ben format.
   \end{enumerate}

 \item En HTML, podem posar un enllaç \texttt{<a
     href="..."}\texttt{>...</a>} dins d'un element de llista
   \texttt{<li>...</li>?}
   \begin{enumerate}
   \item Sí.
   \item No, perquè el document estaria mal format.
   \item No, perquè el document no seria vàlid.
   \end{enumerate}

 \item Un fitxer de text escrit en anglés conté només caràcters
   ASCII. L'obrim amb un editor i el guardem en format Unicode
   UTF-8. Ara ocupa {\ldots}
  \begin{enumerate}
  \item {\ldots} el doble d'espai.
  \item {\ldots} exactament el mateix espai.
  \item {\ldots} la meitat d'espai.
  \end{enumerate}

\item Assenyala el fragment d'HTML que generarà el text més gran:
  \begin{enumerate}
  \item \texttt{<h1>Títol</h1>}
  \item \texttt{<h2>Títol</h2>}
  \item \texttt{<h3>Títol</h3>}
  \end{enumerate}

\item Com es diu el joc de caràcters estàndard universal, el que
  assigna un número de codi diferent i únic a cada un dels caràcters
  de cada una de les llengües del món?
  \begin{enumerate}
  \item Unicode.
  \item ISO-8859-1 (\emph{Latin-1}).
  \item XML.
  \end{enumerate}

\item Quan és preferible utilitzar el joc de caràcters Unicode en lloc
  de l'ISO-8859-1 (\emph{Latin-1})? 
  \begin{enumerate}
  \item Quan anem a mesclar text en diferents idiomes.
  \item Quan un text en espanyol té molts accents.
  \item Quan el text només s'usarà en una situació
    d'assimilació.
  \end{enumerate}

\item Podem emmagatzemar correctament el text és espanyol ``\emph{La
    España de charanga y pandereta, cerrado y sacristía, devota de
    Frascuelo y de María, de espíritu burlón y de alma quieta}'' usant
  el joc de caràcters ASCII? 
  \begin{enumerate}
  \item Sí, sense problemes.
  \item Sí, si abans l'hem convertit a UTF-8.
  \item No. 
  \end{enumerate}

\item Quant ocupa un fitxer de text que conté 2000 caràcters
  pertanyents a l'alfabet espanyol? 
  \begin{enumerate}
  \item Si s'ha codificat en ISO-8859-1 (\emph{Latin-1}), entre 2000 i
    4000 octets.
  \item Si s'ha codificat en ASCII, 2000 octets.  
  \item Si s'ha codificat en UTF-8, entre 2000 i 4000
    octets. 
  \end{enumerate}

\item Si en visualitzar el document de text \texttt{noticia.txt} en el
  navegador hi apareixen paraules com \texttt{camiÃ\(^3\)n} e
  \texttt{Informática}, quina en pot ser la causa? 
  \begin{enumerate}
  \item Un error per part de la persona que ha escrit el document de
    text.
  \item Que el navegador està usant per llegir el document un joc de
    caràcters diferent del que es va usar en escriure'l.
  \item Que el document no és un document HTML.
  \end{enumerate}

\end{enumerate}

\section{Solucions}
\begin{enumerate}
\item (c)
\item (c)
\item (c)
\item (a)
\item (c)
%
\item (c)
\item (c)
\item (a)
\item (a)
\item (a)
%
\item (b)
\item (c)
\item (a)
\item (a)
\item (b)
%
\item (b)
\item (b)
\item (a)
\item (a)
\item (c)
%
\item (b)
\item (c)
\item (a)
\item (b)
\item (b)
%
\item (a)
\item (a)
\item (b)
\item (a)
\item (c)
%
\item (a)
\item (b)
\item (a)
\item (a)
\item (a)
%
\item (c)
\item (c)
\item (b)
\end{enumerate}

