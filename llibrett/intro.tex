\chapter{Introducció}

Aquestes pàgines contenen la major part dels continguts\footnote{Hi ha
  continguts ---diguem-ne millor habilitats--- que s'aprenen com a
  part de les sessions de laboratori i que no figuren en aquest
  document.} de l'assignatura \emph{Tecnologies de la Traducció} que
cursarà l'alumnat de segon curs del grau en Traducció i Interpretació
de la Universitat d'Alacant; també poden ser útils per a assignatures
similars en altres universitats (per això s'hi ha inclòs material més
avançat que no s'estudia en Tecnologies de la Traducció).  La lectura
d'aquest manual ---que pot fins i tot contenir algun error no
detectat--- no pot mai substituir l'estudi de bons llibres sobre la
matèria, molts dels quals se citen en aquest text i es llisten en la
bibliografia.
  
Veureu que els continguts d'aquest manual es poden dividir en dues
parts: la primera presenta alguns conceptes bàsics de la informàtica
(capítol~\ref{se:OiP}), i, més concretament, d'Internet
(capítol~\ref{se:Internet}), sobre l'entrada i el processament de
textos (capítol~\ref{se:EPT}), i sobre les bases de dades
(capítol~\ref{se:basesdades}), i la segona és una introducció a alguns
aspectes generals de la traducció automàtica (capítols~\ref{se:TiTA} a
\ref{se:ASTA}) i a la traducció assistida per ordinador amb memòries
de traducció (capítol~\ref{se:memtrad}). Finalment, un apèndix
discuteix la problemàtica de la traducció espanyol--català i alguns
dels sistemes existents; aquesta informació pot servir com a
il·lustració en un cas concret del que s'ha estudiat sobre traducció
automàtica. Els continguts d'aquesta tercera part són, per tant,
complementaris.
  
De fet, aquestes notes es poden millorar molt, i n'anirem fent
versions noves.
%   Heus ací una llista d'algunes de les coses que sabem que ens queden
%   a fer:\todo{Repassar aquesta llista}
% \begin{itemize}
% \item Potser s'ha de millorar una miqueta el capítol que tracta sobre
%   conceptes bàsics de la informàtica.\footnote{En aquest sentit, volem
%     agrair les contribucions de Raül Canals i Marote, qui a més ha
%     trobat i corregit algunes errades del text.}  També hi falten
%   referències bibliogràfiques.
% \item La descripció dels conceptes de fitxer i directori és molt curta
%   i cal ampliar-la, amb gràfics si és possible.
% \item Cal explicar com funcionen els diversos tipus d'impressora.
% \item Faltaria una secció que descriga recursos per a traductors que
%   es poden trobar en Internet: diccionaris, traductors, lliçons
%   introductòries (\emph{tutorials}), etc.
% \com{Juan Antonio?}
% \item Cal explicar altres modalitats d'accés domèstic a Internet
%   (ADSL, cable). \com{En marxa}
% \item Hi ha capítols sobre temes de molt interés que encara són massa
%   curts (per exemple, els capítols~\ref{se:basesdades} i
%   \ref{se:memtrad}); en tots dos falten esquemes i gràfics que
%   expliquen millor els conceptes i els processos.
%   \jacom{Fet!}
% \item S'ha de millorar la descripció de les tècniques de resolució de
%   l'ambigüitat lèxica de transferència (polisèmia), tot donant un
%   exemple de xarxa semàntica.
% \item Potser cal millorar l'apartat sobre resolució de l'ambigüitat
%   estructural pura.
% \item Cal ampliar l'apartat sobre llenguatges controlats.
%   \jacom{L'he augmentat una mica.}
% \item Potser s'ha d'ampliar la descripció dels sistemes de traducció
%   automàtica basats en transferència semàntica i en interlingua.
% \item Cal fer més èmfasi sobre les aplicacions de codi obert i en
%   concret a la possibilitat d'una estació de treball de traducció
%   basada en elles: GNU/Linux (p.ex., Ubuntu Linux), OpenOffice.org,
%   Mozilla, el programa de traducció assistida OmegaT i l'alineador
%   bitext2tmx, etc.\
%\item Cal descriure la plataforma de traducció automàtica de codi
%  obert Apertium, i en tot cas explicar interNOSTRUM com a precursor.
% \item Cal actualitzar les descripcions de sistemes de traducció
%   automàtica espanyol--català que hi ha al final del llibre.
% \item Cal ampliar el nombre de qüestions i exercicis d'alguns
%   capítols (alguns, de fet, no en tenen), recuperant els d'exàmens,
%   etc.  \com{Aquesta faena està en marxa} \jacom{Feta!}
% \end{itemize}
A més, és segur que hi deu haver errades que s'han de corregir. El
text està obert, per descomptat, a suggeriments i a correccions que el
facen més útil, tant a l'alumnat de l'assignatura com a altres
persones que vulguen saber sobre el tema.  De fet, aprofitem per donar
les gràcies a totes les persones (alumnat, professorat, etc.)  que,
amb els seus comentaris crítics, han anat millorant aquest
text.\footnote{Aquest llibre està basat en una obra anterior,
  \protect\citep{forcada09b}, usada per a la llicenciatura en
  Traducció i Interpretació.}  Són massa gent per a esmentar-los tots,
però no volem acabar sense agrair les aportacions de Raül Canals i
Marote, que va corregir errades de versions anteriors i va fer
aportacions en la part de conceptes bàsics de la informàtica, de Gema
Ramírez Sánchez, particularment en el capítol de memòries de
traducció, i de Sandra Montserrat, en la discussió sobre divergències
lingüístiques espanyol--català de l'apèndix.

En el servidor \url{https://github.com/mlforcada/llibre-tecnol-trad}
trobareu els fitxers font (\LaTeX, .eps, etc.) necessaris per a tornar
a generar el llibre, de manera que, si ho desitgeu, els podeu
modificar per a generar un text nou i publicar-lo vosaltres, però
sempre d'acord amb les condicions de la Llicència General Pública de
GNU versió 3 (descrita en
\url{http://www.gnu.org/licenses/gpl-3.0.txt}) o de la llicencia
Creative Commons Reconeixement-CompatirIgual 4.0 Internacional
(descrita en
\url{http://creativecommons.org/licenses/by-sa/4.0/}). Aquestes
llicències us obliguen, bàsicament, a publicar qualsevol treball
derivat d'aquest amb la mateixa llicència. Així garantim que el nostre
treball està sempre accessible per a qualsevol persona que el
considere útil per a l'ensenyament. Si aquest és el teu cas,
t'estaríem molt agraïts si ens enviares un correu electrònic dient-nos
per a quina assignatura l'estas usant.
