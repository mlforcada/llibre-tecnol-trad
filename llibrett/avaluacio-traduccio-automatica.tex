\chapter[Avaluació de la traducció automàtica]{L'avaluació dels sistemes de traducció
  automàtica}
\label{se:ASTA}



\todo{A integrar en aquest capítol}
\begin{persabermes}{anàlisi de costos en postedició}
  L'anàlisi de costos sol ser en general més complexa, i ha de tenir
  en compte totes les despeses en què s'incorre quan s'adopta la
  traducció automàtica seguida de postedició:
  \begin{itemize}
  \item \textbf{Costos de \emph{funcionament}} (cost efectiu per mot),
    que ha de tenir en compte:
    \begin{itemize}
    \item l'amortització del traductor automàtic (en cas d'adquisició)
    \item el servei tècnic i el manteniment del sistema
    \item la migració (adaptació dels programes que s'usen,
      l'adquisició de sistemes informàtics)
    \end{itemize}

  \item \textbf{Costos de \emph{preedició} i de preparació}: cal
    preparar i potser preeditar (vegeu l'apartat~\ref{ss:preedposted})
    els textos que s'han de traduir

  \item \textbf{Costos de postedició}: depén de la \emph{qualitat} del
    text en brut i de la formació dels posteditors, als quals se'ls
    pot pagar per hores de treball, per quantitat de text corregit,
    etc.

  \item \textbf{Costos de formació}, ja que els professionals han
    d'aprendre a usar una nova tecnologia:
    \begin{itemize}
    \item \textbf{Formació en l'ús del programa de traducció
        automàtica}: els professionals han d'aprendre a usar,
      configurar i potser mantenir el nou programari associat
    \item \textbf{Formació en postedició,} la qual ha de permetre que
      els professionals
      \begin{itemize}
      \item coneguen el comportament del programa de traducció
        automàtica (per exemple, quins són els errors típics que
        comet);
      \item aprenguen tècniques de correcció, com ara l'ús avançat del
        processador de textos (macroinstruccions, substitució de
        patrons, etc.)
      \end{itemize}
    \end{itemize}
  \end{itemize}
\end{persabermes}
















Aquest capítol pretén enunciar i descriure molt breument
alguns dels aspectes rellevants de l'avaluació dels sistemes de
traducció automàtica i donar algunes referències que
puguen ser d'interés per a qui vulga aprofundir en aquest tema.

\section{Qüestions bàsiques} 
Quan ens plantegem l'avaluació dels
sistemes de traducció automàtica (TA), hi ha algunes preguntes
bàsiques que cal respondre. \citet{arnold94b} plantegen
el problema així:
\begin{itemize}
\item Com es pot decidir si un sistema de TA és \emph{bo}?
\item Com es pot decidir si un sistema de TA és \emph{millor} que
  un altre?
\end{itemize}
i afegeixen la pregunta clau: 
``Què vol dir \emph{bo} o \emph{millor} en aquest
context?'' La resposta a totes aquestes preguntes és molt difícil,
com diu \citet{minnis94j}: ``el fet que no s'haja proposat cap métode
d'avaluació o de mesurament estàndard és un bon indicador
de la magnitud del problema''. 

\section{Tipus d'avaluació}
La naturalesa de l'avaluació d'un sistema de TA
depén de diversos factors:
\begin{enumerate}
\item \emph{Per a què} es fa l'avaluació? \citet{hutchins96u} distingeix
  entre tres tipus bàsics d'avaluació: 
\begin{enumerate}
\item l'\emph{avaluació
  d'adequació}, que serveix per a ``determinar la idoneïtat
dels sistemes de TA en un context operacional especificat'' ---per
exemple, per a decidir si el sistema de TA és bo per a traduir el correu
comercial d'una empresa alimentària---;
\item l'{\em
  avaluació diagnòstica}, que serveix per a ``identificar
limitacions, errors o deficiències, les quals poden ser corregides
o millorades'' ---per exemple, defectes en el tractament de la
concordança verbal de les oracions subordinades---, i
\item l'\emph{avaluació de funcionament},  ``per a valorar l'estat de
  desenvolupament del sistema o les diferents realitzacions
  tècniques'' ---per exemple, si el programa és robust,
  ràpid, fa un ús racional de la memòria del sistema, etc.
\end{enumerate}
Quant a l'\emph{avaluació diagnòstica}, vegeu en la
secció~\ref{ss:avalpred} la discussió sobre la noció
d'\emph{avaluació predictiva}.
\item \emph{Qui} fa l'avaluació? L'avaluació la poden fer:
  \begin{enumerate} 
  \item les
  persones que presumiblement usaran el sistema o l'adquiriran per a
  una empresa   (avaluació d'adequació) o professionals externs
  (\emph{consultors}) contractats a l'efecte;
  \item els investigadors, equips de desenvolupament, programadors
    (avaluació diagnòstica), molt especialment durant el
    desenvolupament d'un sistema de TA;
  \item qualsevol dels dos grups anteriors (avaluació del funcionament).
  \end{enumerate}
\item \emph{Com} es fa l'avaluació? Quan s'avalua un sistema de
  TA es tenen en compte:
\begin{enumerate}
\item \emph{La qualitat de les traduccions en brut}
  produïdes pel sistema. La qualitat és una combinació
  (en proporcions difícils de determinar\footnote{\citet{minnis94j}
    diu: ``La raó per la qual el mesurament de la qualitat és
    difícil és, per descomptat, el fet que la qualitat siga
    un concepte tan polifacètic i intangible''.})
  de diversos factors, com ara: 
  l'\emph{intel·ligibilitat} 
  dels documents traduïts per part dels usuaris; 
  la \emph{precisió} o \emph{fidelitat} amb què el text
  traduït comunica el significat del document original (les quals
  han de ser jutjades per part
  de persones bilingües coneixedores de la temàtica dels
  documents); l'adequació de l'estil o del registre dels documents
  traduïts, etc. 

  Aquesta avaluació es pot fer mitjançant
  l'ús de col·leccions de documents típics o representatius 
  (com se sol
  fer en les avaluacions d'adequació) o mitjançant sèries de
  proves objectives (angl.\ \emph{test suites}),   
  usades en les avaluacions diagnòstiques\footnote{Però no
    únicament, com indica  \citet{lewis97j}, ja que també poden
    servir perquè els usuaris jutgen l'adequació de l'eixida
    produïda pel sistema.} i dissenyades per a
  abraçar conjunts complets de fenòmens lingüístics que 
  es manifesten en la traducció.\footnote{Per exemple, el reordenament
  dels mots dels sintagmes nominals quan es tradueix de l'anglés
  a l'espanyol \citep{mira98j,forcada00p}.}

Una possible mesura quantitativa de la qualitat d'un text produït per
un sistema de TA a partir d'un text representatiu és el nombre mínim
d'insercions, esborraments i substitucions de mots necessaris (per
exemple, per cada 100 mots de text) per a transformar-lo en un text
que siga una traducció acceptable\footnote{De qualsevol manera, queda
  el problema que un text que per a un lector és acceptable pot no
  ser-ho per a un altre.} de l'original. A més, la determinació de la
qualitat d'una traducció en brut per còmput del nombre de correccions
necessàries no està exempta de problemes:
\begin{itemize}
\item Aquest mètode dóna la mateixa importància a totes les operacions
  de correcció, independentment del mot. Això pot no ser adequat.
\item Si suposem que existeix una única traducció acceptable del text
  origen i l'usem com a referència, hi ha més d'una manera de corregir
  la traducció en brut de manera que el resultat siga idèntic al de
  referència. Per a poder fer comparacions, estem interessats en la
  correcció produïda amb el nombre mínim d'operacions d'inserció,
  esborrament i substitució de mots; aquest nombre mínim es pot
  considerar una \emph{distància}.\footnote{De fet, matemàticament, ho
    és: s'anomena \emph{distància d'edició} (angl.~\emph{edit
      distance}).} La recerca d'aquesta manera òptima de corregir pot
  no ser trivial per a una persona, especialment si els errors
  apareixen junts i agrupats. 
\item Però és que, a més, la traducció de referència pot no estar
  disponible; a més, en la majoria dels casos no hi ha una única
  traducció acceptable. De nou, si volem comparar, voldríem trobar la
  traducció acceptable més pròxima a la traducció en brut, és a dir,
  la que s'obté amb el mínim de correccions possibles. És a dir,
  avaluar (corregir) la traducció en brut comporta per tant fer una
  doble recerca: la persona que corregeix ha de buscar mentalment la
  traducció acceptable més propera (tot tenint en compte els criteris
  que fan acceptable una traducció, els quals poden no ser fàcils
  d'aplicar), però la \emph{distància} entre els dos textos també es
  calcula fent una recerca mental del nombre mínim de correccions
  necessàries.
\end{itemize}
\com{Repassar aquesta discussió i millorar-la?}  El fet que és
possible que l'avaluació per recompte de correccions no siga òptima en
vista d'aquests problemes fa que, a més, siga especialment difícil
comparar les avaluacions fetes per persones diferents. A més, aquest
tipus d'avaluació és molt costós, ja que per a obtenir una medició
fiable de la qualitat és necessari corregir textos de milers de
paraules.

D'altra banda, sempre s'ha de tenir en compte que els mètodes
d'avaluació de la qualitat depenen de l'ús que es pensa donar al
sistema de TA \citep{arnold93j}; per exemple, l'avaluació d'un sistema
que s'usa per a la \emph{disseminació} de material és ben diferent de
l'avaluació d'un sistema usat per a l'\emph{assimilació}
d'informació.\footnote{De nou, com s'ha discutit en el capítol~\ref{se:TiTA}, la
  noció central és la de \emph{propòsit} de la traducció.}

\item \emph{La qualitat del sistema de TA mateix}: per exemple, ``la facilitat amb
  què es poden crear i actualitzar diccionaris, posteditar els
  textos, controlar el llenguatge d'entrada'' o
  ``l'extensibilitat [del sistema] a parells nous d'idiomes o a noves temàtiques''
  \citep{hutchins96u}.
\item \emph{La comparació dels costos i dels beneficis} d'usar un
  sistema de TA en comptes dels serveis de professionals de la
  traducció: per exemple, si costa més (en despeses de personal) la
  postedició (revisió) dels textos meta produïts pel sistema
  (afegint-hi el cost d'usar el sistema de TA) que la traducció
  completa dels textos origen per part de professionals, l'adopció del
  sistema de TA no convé a una empresa (p.~\pageref{pg:cost}).
\end{enumerate}
\end{enumerate}

\section[Traducció automàtica i traducció humana]{Sobre la comparació entre traducció auto\-mà\-tica i
  traducció humana} 
\label{ss:humaut}
Una visió predominant de l'avaluació dels sistemes de TA és
l'anomenada \emph{metàfora del traductor humà}, segons la qual
\citep{krauwer93j} la tasca consisteix a ``determinar fins a quin punt
els constructors del sistema han aconseguit imitar el comportament
d'un traductor humà''. \citet{sager93b} ho formula dient que ``S'ha
argumentat que la qualitat dels documents produïts mitjançant
traducció automàtica s'hauria d'avaluar en termes de la identitat amb
productes humans''.
  
Tant \citet{krauwer93j} com \citet{sager93b} qüestionen aquesta visió;
aquest últim argumenta que ``s'ha d'acceptar que no hi ha cap situació
que puga servir com a punt de comparació entre la traducció humana i
l'automàtica, i que potser no hi ha cap situació en la qual la
traducció humana i l'automàtica siguen igualment adequades'' i proposa
que, en canvi, les traduccions poden ser comparades per a veure ``si
satisfan, i fins a quin punt, les expectatives de l'usuari final [dels
documents traduïts]'', ja que la traducció és una ``activitat de
mediació, la forma particular de la qual està determinada tant pel
text com per les circumstàncies comunicatives que requereixen aquesta
mediació''.  En concret, la
traducció automàtica pot ser la més adequada en algunes
circumstàncies, en vista de l'enorme demanda general existent i, més
concretament, de la demanda de traduccions ràpides i barates que no
poden ser produïdes per professionals.

\section{Avaluació predictiva}
\label{ss:avalpred}

Hi ha un tipus d'avaluació que es pot considerar com a cas particular
de l'avaluació diagnòstica definida més amunt, encara que no s'use
estrictament per a millorar el funcionament d'un sistema, sinó només
per a predir el comportament del sistema en situacions noves.
L'anomenarem ací \emph{avaluació predictiva}.

Per a poder fer l'avaluació predictiva, és crucial que els avaluadors
tinguen, en primer lloc, un model que descriga aproximadament el
funcionament del sistema de traducció automàtica (relacionat amb la
tipologia del sistema, és a dir, de transferència morfològica,
sintàctica, etc., vegeu el cap.~\ref{se:TdTA}), i, en segon lloc, un
conjunt de textos o frases d'avaluació (\emph{test suite}) que els
permeta obtenir detalls concrets sobre les dades lingüístiques (p.ex.,
regles) que usa aquell model.  Les prediccions serien de l'estil de
``com que sembla usar regles patró--acció de l'estil de ``si troba un
patró $X$ farà l'acció $Y$'' i en una sèrie de casos troba el patró
$X_1$ i fa l'acció $Y_1$, podem predir que sempre que trobe aquest
mateix patró farà la mateixa acció''. Com que la majoria dels sistemes
comercials no ens donen suficient informació sobre la naturalesa del
model, els haurem de tractar com una \emph{caixa negra}; la
intuïció de la persona avaluadora, el seu coneixement d'altres
sistemes o de la història de les empreses involucrades (per exemple,
quant a l'adquisició de tecnologia d'altres empreses) i la seua
habilitat per a elegir exemples reveladors li permetran determinar
aspectes bàsics del model de traducció. En particular, qualsevol
avaluació predictiva necessita tenir una idea clara sobre el nivell
d'anàlisi que es fa en el sistema de TA, ja que el nivell d'anàlisi és
el que determina més la naturalesa d'un sistema (vegeu
l'apartat~\ref{ss:classtrans}).  L'avaluació predictiva pot tenir, a
més, un paper fonamental en l'educació dels futurs traductors
\citep{mira98j,forcada00p}, en vista de la disponibilitat creixent de
sistemes comercials (per exemple, a través d'Internet).\footnote{Una
  altra aplicació de les tècniques descrites és l'anomenada
  \emph{enginyeria inversa}, o determinació detallada de l'estratègia
  usada per un programa (en aquest cas, de traducció automàtica) per a
  programar-la en un altre.}

D'altra banda, perquè l'avaluació siga útil els conjunts de prova
haurien d'estar dissenyats de manera que abraçaren conjunts complets
de fenòmens lingüístics que es manifesten amb freqüència rellevant en
les situacions reals de traducció que es volen avaluar, ja que es vol
predir el comportament del sistema en aquestes situacions concretes.


\section{Qüestions i exercicis}

\begin{enumerate}
\item Quina característica d'un sistema de traducció automàtica
   s'ha de considerar com a especialment important quan s'avalua
   l'aplicació del sistema a l'\emph{assimilació}?
   
\begin{enumerate}
\item els seus útils de postedició assistida
\item els seus útils de preedició assistida
\item la velocitat de resposta
\end{enumerate}

\item Per què és difícil avaluar la qualitat d'una traducció automàtica
   comptant la quantitat mínima de postedició necessària per fer-lo adequat
   quan no hi ha una traducció de referència?
   
\begin{enumerate}
\item No és que siga difícil; sense
      traducció de referència és absolutament impossible.
\item Perquè aquesta tasca no es pot fer sense conéixer
      profundament l'estratègia usada pel sistema de traducció
      automàtica.
\item És relativament senzill corregir el text perquè
      siga adequat però és molt difícil fer-ho fent-hi el mínim nombre
      de canvis necessaris.
\end{enumerate}

\item Quan es vol usar un sistema de traducció automàtica per a
  l'\emph{assimilació} d'informació, a què donaríeu \emph{menys} pes
  en l'avaluació?
   
\begin{enumerate}
\item Facilitat de postedició de la traducció en brut.
\item Intel·ligibilitat de la traducció en brut.
\item Velocitat.
\end{enumerate}


% < QT11

\end{enumerate}

Quant al concepte d'\emph{avaluació predictiva}, mireu a més els
exercicis~\ref{ex:cascat}, \ref{ex:zkanagg} i \ref{ex:postres} del capítol~\ref{se:TdTA}.


\section{Solucions}


\begin{enumerate}
\item (c)
\item (c)
\item (a)

% < SL11

\end{enumerate}


