\chapter{Traducció i traducció automàtica}
\label{se:TiTA}

Una de les aplicacions més importants de la informàtica a la
traducció és la \emph{traducció automàtica} (TA).  Abans de considerar
l'automatització de la traducció fóra bo que ens paràrem un poc per a
discutir què vol dir exactament el mot \emph{traducció}. Sobre la
relació entre traducció humana i traducció automàtica, vegeu la
secció~\ref{ss:humaut}.

\section{Què és la traducció?}
\label{ss:trad}

Per començar, s'ha de tenir en compte que 
el mot \emph{traducció} és ambigu\footnote{Com molts altres
  substantius acabats en \emph{-ció}} perquè es pot
referir al \emph{procés} de traduir o al
\emph{producte} (resultat) d'aquest procés.

\citet{sager93b}\footnote{Els conceptes
d'aquesta secció estan presos quasi íntegrament d'aquesta
obra.} comença la seua definició dient que, com a procés,
es pot anomenar traducció ``un rang d'activitats humanes
deliberades, que es fan com a resultat d'instruccions rebudes d'un tercer,
i que consisteixen en la producció de textos en una llengua meta (LM),
basada, entre altres coses, en la modificació d'un text en una llengua
origen (LO) per a fer-lo adequat a un propòsit nou'', però encara no
descriu la naturalesa de la modificació.

Com a producte, una \emph{traducció} es pot identificar com a tal
perquè és un document (en LM)  derivat d'un altre document en un altre
idioma (LO), i que manté una certa similitud de contingut amb aquest.

Es poden dir encara més coses sobre la traducció:
\begin{itemize}
\item les traduccions solen estar escrites en un subllenguatge particular
(registre, especialitat...) de la comunitat lingüística de la LM, basat en un subllenguatge
paral$\cdot$lel de la LO;
\item els documents i les traduccions corresponents es poden classificar en
tipus\footnote{Per exemple, \emph{carta comercial}, \emph{edicte municipal},
  \emph{comentari editorial}, \emph{manual tècnic informàtic} o \emph{recull de poemes}. } i aquesta tipologia afecta la traducció;
\item la traducció es veu afectada per elements
extralingüístics perquè, normalment, els documents són
entitats que uneixen l'expressió lingüística amb
l'expressió no lingüística;
\item les traduccions tenen un \emph{receptor} o \emph{lector}; una 
traducció, com a acte comunicatiu ha de considerar, a  més de la
intenció de la traducció, les expectatives dels
lectors, que resulten del seu rerefons cultural i de les seues necessitats
comunicatives, i que influeixen en la recepció del text traduït;
\item la traducció sempre té una motivació: la superació de
barreres comunicatives; per això se n'ha creat una professió. 
\end{itemize}

Es pot aprofundir un poc més en la definició de
traducció que hem considerat més amunt, revisant definicions
existents (algunes preses de \citealt{sager93b}):
\begin{itemize}
\item \citet[p.~19]{nida59b}: ``La traducció consisteix a produir en la llengua meta
l'equivalent natural més proper del missatge en la llengua origen,
primerament quant al significat i segonament quant a l'estil.'' Sager diu
que, més que \emph{natural} (en sentit absolut) caldria dir \emph{adequat}
(a la tasca concreta). Aquesta definició introdueix dues de les tres
dimensions bàsiques d'un document escrit (original o traduït):
el \emph{contingut} (significat) i la \emph{forma} (estil), però oblida el 
\emph{propòsit}. 
\item \citet{flamand83b}: traduir és representar amb precisió
(fidelitat a l'autor) un missatge
en LO 
en una forma autèntica i correcta de la LM,
adaptada al contingut i al receptor (fidelitat al lector)''. El problema
d'aquesta definició
és la indefinició del concepte de \emph{fidelitat}.
\item \citet{jakobson66b}: ``Traducció és la interpretació de signes
verbals per mitjà d'una altra llengua''. Aquesta definició evita el
concepte d'\emph{equi\-va\-lèn\-cia} i introdueix el d'\emph{interpretació}
com a conjunt de processos cognitius que tenen lloc en la ment del
traductor.
\item En el \emph{Diccionari de la Llengua
    Catalana}\footnote{Editorial Enciclopedia Catalana, 7a ed.,
    1987-}  
es defineix {\em
    traducció} com la
``reproducció del contingut
d'un text o d'un enunciat oral, formulat en una llengua, en formes
pròpies d'una altra llengua'' (i \emph{traduir} com  ``escriure o dir en una
llengua allò que ha estat escrit o dit en una altra''). La 
definició
inclou, per tant, el tractament i la producció de missatges
no textuals (orals).
\item \citet{alcaraz97b} defineixen
  traducció com ``expresión de un
enunciado en la lengua de llegada [lengua meta] que sea equivalente al de la
lengua de partida [lengua origen]''; queda per definir la noció
d'\emph{equivalència}, que els mateixos autors defineixen així: ``la
posesión del mismo valor por parte de los enunciados de la lengua de
partida y de la de llegada''; l'equivalència pot ser {\em
semàntica}, \emph{estilística} i \emph{textual}. 
\end{itemize}

Per a acabar aquest apartat, convé esmentar alguns processos que no
s'anomenaran \emph{traducció} en aquest curs:
\begin{itemize}
\item l'adaptació de textos antics a la forma moderna d'un idioma;
\item la traducció de mots i frases quan s'ensenya un nou idioma;
\item la interpretació (de missatges parlats);
\item la codificació (en Morse, etc.).
\end{itemize}


\section{Traducció automàtica}

\subsection{Definició}

La traducció automàtica (TA) es pot definir com el procés (o el
producte) de traduir un text informatitzat\footnote{Anomenarem {\em
    text informatitzat} un fitxer  d'ordinador que conté un
  text codificat en un format conegut (vegeu el capítol~\ref{se:EPT}.)
  } en una llengua origen a un text informatitzat en una llengua meta
mitjançant l'ús d'un programa d'ordinador. Normalment es reserva
la denominació \emph{traducció automàtica} per a la completament
automàtica; quan s'hi produeix intervenció humana es parla de {\em
  traducció assistida per l'ordinador} o de \emph{traducció
  semi-automàtica}.  El capítol~\ref{se:UTA} està dedicat a analitzar
les diverses modalitats d'interacció entre persones i màquines en
traducció.

Un aclariment és necessari sobre el tractament dels textos
informatitzats. Quan els programes de traducció automàtica i
semiautomàtica han de tractar documents estructurats (com els
discutits en els epígrafs~\ref{s3:SGML} i \ref{s3:RTF}) han de ser
capaços d'identificar les parts dels documents que corresponen als
textos que s'han de traduir, destriant-les de les etiquetes.
Normalment, els programes tenen un mòdul inicial que podríem anomenar
\emph{desformatador} i un mòdul final que podríem anomenar
\emph{reformatador} i que restitueix les etiquetes de manera que el
format i l'estructura del document es conserven tant com siga
possible. En general, aquestes operacions es poden considerar
bàsicament independents del procés de traducció mateix ---com farem en
aquest llibre---, però hi ha programes més avançats que són fins i tot
capaços d'usar la informació de les etiquetes com a context per a
elegir una traducció on hi ha més d'una alternativa.

 Les referències que s'han fet en l'epígraf~\ref{ss:trad} al propòsit o
motivació de la traducció i a la tipologia dels documents que han de
ser traduïts són també molt importants a l'hora d'analitzar la
traducció automàtica.


\subsection{Sobre el nom en altres llengües}

Convé comentar de pas que en anglés, la traducció automàtica
s'anomena \emph{machine translation} i s'abreuja MT, paral$\cdot$lelament a
l'alemany, que usa la denominació \emph{maschinelle
  übersetzung}; en  aquestes dues llengües s'expressa la
noció d'automatisme mitjançant la referència a una {\em
  màquina}. En
canvi, en francés o en rus es parla, com en català o en espanyol, de {\em
traduction automatique} o \emph{avtomatitxeski perevod},
respectivament.  


\section{Història de la traducció automàtica}

La major part del discutit en aquest apartat està pres dels treballs
de John Hutchins, especialment de \cite{hutchins1995}
i \cite{hutchins2001}.

\subsection{Els pioners, --1954}

La traducció mitjançant màquines és una ambició humana des de fa
segles que no es va fer 
realitat fins al XX. No feia molt que s'havia creat el primer
ordinador, 
quan ja es va 
començar a pensar en la possibilitat d'usar-los
per a traduir llenguatges 
humans. 

Tot i que en els decennis dels 1930 i 1940 
hi va haver alguns treballs precursors,
és als primers cinquanta quan comença realment la recerca en TA en moltes 
universitats arreu al món, especialment als 
Estats Units. Els recursos de maquinari, programari i 
llenguatges de programació eren massa
reduïts i la primera aproximació va ser la traducció 
mot per mot 
basada en diccionari amb algunes regles 
senzilles de reordenament (actualment anomenada \emph{traducció
  directa} (vegeu l'apartat~\ref{ss:traddir}). Aquesta manca
de recursos va fer que els primers objectius foren
molt modests i, així, els primers investigadors van concentrar-se
en el desenvolupament de llenguatges 
controlats (vegeu~\ref{ss:llecon}) i 
en l'ajuda humana en tasques de preedició i postedició 
(vegeu~\ref{ss:preedposted}); 
era prou clar
que els sistemes reals no podrien produir 
més que traduccions de molt baixa qualitat. 
El 1952 es va celebrar als Estats Units 
el primer 
congrés sobre TA on es van definir les línies fonamentals a seguir.

\subsection{El decenni de l'optimisme, 1954--1966}

La primera demostració pública d'un sistema de TA
va ser desenvolupada per IBM i la Universitat Georgetown el 1954. 
Es va traduir a l'anglés un conjunt de 49 frases en rus\footnote{Per 
raons polítiques i militars 
aquestes llengues van ser les elegides per als primers sistemes de
TA.} usant un diccionari de 
només 250 mots i 6 regles gramaticals.
Tot i que els resultats no eren massa bons, 
el públic i la indústria van creure que en uns anys es 
podrien aconseguir traduccions automàtiques de 
qualitat de documents científics i tècnics. 
Aquesta idea es va reforçar pel fet que van començar a 
aparèixer millores significatives en el maquinari, 
els primers llenguatges de programació i moltes 
millores en la lingüística formal (especialment 
en l'àrea de la sintaxi). L'entusiasme va fer que es 
finançaren un munt de projectes entre la meitat dels 50 i la meitat
dels 60, 
projectes dins els 
quals van nàixer la major part de les tècniques actuals, 
com ara la traducció indirecta per 
transferència o la traducció per interlingua (vegeu el
capítol~\ref{se:TdTA}).


L'objectiu era el desenvolupament de sistemes perfectes. Calia reduir 
al mínim la intervenció 
humana en el procés de TA, fins assolir la independència total i una 
qualitat comparable a la dels 
humans. Pràcticament ningú va considerar com es podria traure 
profit d'un sistema imperfecte: 
per què pensar-hi si aviat es disposaria de sistemes perfectes? Els
traductors es van sentir 
amenaçats. No obstant això, algunes veus es pronunciaren en contra 
del perfeccionisme dominant i 
defensaren una aproximació 
més a llarg termini al problema i 
la construcció de sistemes que feren 
un ús efectiu de la interacció persona-màquina.

Un decenni després, i com que les expectatives eren tan 
altes, els avanços eren escassos
i el futur pròxim no semblava poder millorar la situació. 
Molts investigadors començaven a 
trobar barreres de tot tipus, especialment semàntiques,
que semblaven massa difícils de superar
i que exigien mètodes més complexos. La Acadèmia Nacional de 
les Ciències dels Estats Units va publicar el 1966
l'informe ALPAC (Automatic Language Processing 
Advisory Committee) en el qual es recomanava que els nombrosos 
recursos que es dedicaven a 
la recerca en TA s'utilitzaren per tasques menys pretensioses 
i més bàsiques relacionades amb 
el processament del llenguatge natural i amb el desenvolupament 
d'eines de suport per als 
traductors com ara diccionaris automàtics.
La conclusió era que només després de conéixer les arrels del 
problema, podria estudiar-se la realització 
d'un sistema de TA real. L'informe assegurava que la TA era més 
lenta i menys exacta que la feta pels humans, 
a més de ser el doble de cara, i que no hi havia 
cap indici de l'obtenció en el futur més o menys inmediat d'un 
sistema de TA útil.
L'informe va fer que es reduira significativament el nombre de 
persones que es dedicaven a la TA i que 
els laboratoris començaren a treballar en el que es va conéixer 
com a lingüística computacional.

\subsection{Des de l'informe ALPAC (1966) fins als vuitanta}

L'informe va acabar quasi virtualment amb la 
recerca en TA als Estats Units (també va tenir un 
impacte negatiu en els projectes desenvolupats 
a la resta del món) i durant molts anys la TA va 
ser percebuda com un autèntic fracàs. Tot i això, 
alguns grups van continuar treballant a Canadà i a Europa
i van aparéixer els primers sistemes que funcionaven; 
el 1970 el sistema Systran va començar a ser usat 
per la USAF (United States Air Force) i 
el 1976 per la Comissió de la Comunitat Europea.
També el 1976 apareix Metéo, 
desenvolupat per la Universitat de Montréal,
que tradueix al francés els informes meteorològics. 
Per aquesta època, a més, 
els sistemes de TA comencen a ser demanats per empreses 
i administracions i no sols per
traduir textos científics i tècnics.

Des de l'informe ALPAC
el camp va patir una redefinició progressiva vers una 
concepció de la TA 
com un procés en el qual els traductors humans juguen 
un paper bàsic, 
i comencen a desenvolupar-se eines de traducció 
pensant en aquesta intervenció.

Els principals corrents dins la TA des dels 70 són, per tant: 
eines de suport a la traducció per a traductors, 
sistemes de TA amb intervenció humana i recerca teòrica 
vers un sistema completament 
automàtic de traducció.

\subsection{Els primers vuitanta}

Als 1980 apareixen nous sistemes de TA arreu del món
amb expectatives més reals i l'interés en la TA
resorgeix.
Són especialment importants els resultats obtinguts a 
diverses empreses com Xerox on s'elimina 
quasi completament la postedició (vegeu la p.~\pageref{pg:homografia}) gràcies al control 
de la llengua origen; això permet la traducció 
senzilla dels manuals tècnics en anglés de la companyia 
a un gran nombre d'idiomes (francés, 
alemany, italià, espanyol, portugués i llengues escandinaves).

Durant aquest decenni els esforços es dirigeixen vers la traducció indirecta 
amb representacions 
intermèdies o sense (com la interlingua; 
vegeu l'apartat~\ref{ss:interlingua}) 
mitjançant
anàlisis morfològiques i sintàctiques i, de vegades, 
coneixements no lingüistics. 
Els projectes més notables són GETA-Ariane (Grenoble), SUSY 
(Saarbrücken), Mu (Kyoto), DLT (Utrecht), Rosetta (Eindhoven), 
el projecte de la Universitat 
Carnegie-Mellon (Pittsburgh) i dos projectes internacionals: 
Eurotra, finançat per la Comunitat 
Europea i el projecte japonés CICC amb participants 
a la Xina, Indonèsia i Tailàndia.

Eurotra és un dels projectes de traducció més coneguts del decenni 
dels 80. El seu objectiu era la 
construcció d'un sistema de transferència multilingüe 
que permetera la traducció entre totes les llengües 
de la Comunitat Europea. 
Tot i que la traducció resultant 
tenia bastanta qualitat, necessitava una gran quantitat de 
postedició. El projecte va estimular la investigació a
tot Europa, però va ser abandonat finalment el 1992.

En aquests anys es consolida la idea que els
sistemes de TA no són per a traductors; un traductor necessita eines 
que li faciliten el treball: diccionaris, 
bases de dades terminològiques (vegeu el capítol~\ref{se:basesdades}), 
sistemes de 
comunicació, memòries de traducció (vegeu el
capítol~\ref{se:memtrad}), etc.
De fet, actualment la postedició no s'encarrega a 
traductors (que no consideren açò 
com a part del seu treball), 
sinó a persones preparades específicament.

Tothom accepta ja en aquest decenni la importància dels llenguatges
controlats i els subllenguatges en la TA, com ja havien defensat els
precursors de la TA durant el decenni dels cinquanta.

El sistema comercial més sofisticat dels 1980 és Metal (1988),
finançat per Siemens i 
que tradueix de l'alemany a l'anglés. 
Es tracta bàsicament d'un sistema per transferència, indicat per 
a la traducció de documents 
relacionats amb el processament de dades i les telecomunicacions. 

Al final dels 1980 comença l'aplicació de tècniques d'inteligència
artificial
al processament del llenguatge 
humà (sistemes experts i sistemes basats 
en el coneixement dissenyats per entendre els textos).

\subsection{Els primers noranta}

Tots els sistemes de TA dels vuitanta, tant els de transferència 
com els d'interlingua, funcionen bàsicament a 
partir de regles lingüístiques. Als 1990, però,
apareixen noves estratègies conegudes com a mètodes basats en 
corpus. Els mètodes basats en corpus es poden dividir 
en dos grups: estadístics i basats en exemples.

Els mètodes estadístics ja van ser considerats als anys seixanta, però
aviat van ser descartats perquè els resultats obtinguts no eren
acceptables. Ara, però, el descobriment de noves tècniques va fer
possible projectes com Candide a IBM. Candide usa mètodes estadístics
per a l'anàlisi i la generació, però cap regla lingüística. Els
treballs a IBM van utilitzar el gran corpus de textos en anglés i
francés resultants de les sessions del Parlament de Canadà.  El mètode
consisteix a alinear en primer lloc les frases, els grups de mots i
els mots en els dos textos i calcular després la probabilitat que un
mot del text origen corresponga a un o més mots del text meta amb el
qual ha estat alineat.

Els metodes basats en exemples (vegeu el capítol~\ref{se:memtrad})
s'aprofiten també de l'existència de grans corpora de textos
traduïts (per això també s'en diu basats en memòria). La idea
fonamental es que el procés de traducció es pot fer sovint consultant
traduccions anteriors i identificant frases o grups de mots en el
corpus ja traduït. Per poder dur a terme la traducció és necessari que
els textos del corpus hagen estat alineats prèviament (mitjançant
mètodes estadístics o mètodes basats en regles).

Tot i que la gran innovació dels noranta van ser els mètodes descrits
abans, 
la recerca i el desenvolupament 
dels sistemes clàssics també va continuar: 
per exemple, el projecte Eurolang basat en el sistema de 
transferència Metal pot traduir de l'anglès al 
francés, alemany, italià i espanyol, i 
viceversa.
Ens els darrers 10 anys, un dels camps amb més 
investigacions ha estat el de traducció de la 
parla, una idea que evidentment ha estat 
present des de fa dècades, però que només ara es pot 
materialitzar parcialment. 
L'objectiu no és obtenir un sistema de traducció perfecta, 
sinó un sistema adequat per a aplicacions amb 
llenguatges, dominis i usuaris restringits. El 
principals són els desenvolupats a ATR, 
CMU i el projecte Verbmobil.

Una característica important dels primers 1990 és l'aparició 
de les primeres aplicacions 
pràctiques per a traductors: eines de suport a la traducció, 
diccionaris i bases de dades 
terminològiques, processadors de text multilingües, 
accés a glossaris i terminologies 
electròniques, eines de comunicació (escàners, OCRs, Internet;
vegeu els capítols~\ref{se:Internet} i~\ref{se:EPT}) o 
eines per a entorns restringits. 
La combinació d'algunes d'aquestes eines en un 
programari concret és el que es coneix com 
\emph{estacions de treball per a traductors} (per exemple, el 
Translation Manager d'IBM o el Translator 
Workbench de Trados). La major part 
d'aquestes estacions de treball estan disponibles per a 
ordinadors personals.

\subsection{Dels darrers noranta a l'actualitat}

La TA i les eines de suport a la traducció son cada 
vegada més usades per les grans empreses i 
per les administracions, principalment per a la 
traducció de documentació tècnica. 

Al llarg dels darrers anys, amb la generalització de l'ús
d'Internet, s'han desenvolupat serveis de traducció 
disponibles en línia i nombroses eines per a l'assimilació 
de continguts electrònics com ara documents HTML i missatges de
correu electrònic.

Des dels seus inicis, quasi tota la recerca i quasi tots els 
sistemes comercials de TA s'han centrat en els principals 
idiomes internacionals: anglés, francés, espanyol, japonés, rus, etc.
Encara resta molt a fer amb les 
altres llengues del món.

\com{2003}
En el moment d'escriure aquestes línies s'observa un resorgiment de
les tècniques de \emph{traducció automàtica estadística} com les que
es van iniciar en IBM durant la dècada dels 1990; fins i tot han
aparegut empreses que es proposen comercialitzar productes que
\emph{aprenen} el vocabulari i les regles de traducció directament de
grans textos bilingües.
