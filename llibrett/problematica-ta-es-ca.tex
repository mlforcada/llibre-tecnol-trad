\chapter{Problemàtica de la TA espanyol--català}
\label{se:PdTACC}

La inclusió d'aquest capítol en el llibre té dues raons:
\begin{itemize}
\item D'una banda, estudiar amb una miqueta més de detall els
  problemes que planteja la traducció automàtica entre dues llengües
  emparentades. Podríeu pensar que, sent l'espanyol i el català tan
  similars sintàcticament, els problemes serien poc importants: el
  capítol intenta convéncer-vos que les coses no són tan senzilles com
  podrien semblar a primera vista.
\item D'altra banda, serveix per a il·lustrar, amb un parell de
  llengües concret, alguns dels conceptes tractats en els capítols anteriors.
\end{itemize}




\section{Introducció}

Les aplicacions potencialment més interessants de la TA
espanyol--català s'em\-mar\-quen dins de l'anomenada {\em
  normalització lingüística}, és a dir, l'esforç de les societats
de parla catalana per promoure'n l'ús normal en tots els àmbits; un
exemple actual el constitueixen els servidors d'Internet
d'institucions públiques i d'empreses privades dels Països Catalans,
on la presència del català és encara minoritària. Quan la llengua
original dels documents és l'espanyol, es podria usar un sistema de
TA per a generar esborranys de documents catalans (o, fins i tot,
documents correctes si els documents espanyols estan escrits en un
llenguatge controlat).

En el cas concret de l'espanyol i el català, la proximitat lingüística
entre les dues llengües fa que siga abordable el disseny de sistemes
de traducció automàtica que generen textos d'un nivell de correcció
tal que resulte més rendible revisar el resultat en brut produït pel
programa que fer la traducció completa (mireu la p.~\ref{pg:cost}).


En aquest capítol es presenten alguns dels problemes més importants
amb què es pot trobar qui vulga dissenyar un sistema de traducció
automàtica per a traduir textos de l'espanyol al català. A la vista de
la notable similitud lingüística existent entre les dues llengües, es
podria pensar que la tasca de traducció automàtica podria, en la
majoria dels casos, ser tan senzilla com substituir un a un els mots
espanyols pels seus equivalents catalans. De fet, el model de
traducció automàtica \emph{mot per mot} (definit en la
pàg.~\pageref{pg:mpm}, i que no s'ha de confondre amb el que els
traductors humans anomenen \emph{traducció literal}) és el model de
referència que usarem en aquest capítol: els tres grups de {\em
  problemes} que es presenten en aquest capítol són alguns ---no
tots--- dels que no resol el model mot per mot: la segmentació del
text origen, l'homografia i les divergències sintàctiques.

\section{Segmentació del text origen}

La segmentació d'un text en mots sol ser normalment molt senzilla: el
programa pot usar els blancs, els tabuladors, els finals de línia o
els signes de puntuació com a fronteres entre mots. Però de vegades no
és tan fàcil: per exemple, l'espanyol uneix moltes vegades diversos
mots en un sol mot, sense que s'hi puguen distingir els mots
components; en català, llevat de contraccions com \emph{al},
\emph{pels}, i \emph{del}, sempre queda alguna indicació d'aquesta
unió, com ara un apòstrof o un guionet. Per exemple, en espanyol, els
pronoms enclítics s'uneixen a l'imperatiu, a l'infinitiu i al gerundi,
i moltes voltes fan que en canvie la forma (vegeu
l'epígraf~\ref{s3:STMorf}). Per sort, aquests problemes es poden
resoldre de manera senzilla usant analitzadors morfològics com els que
es descriuen en l'epígraf~\ref{s3:anmor}.

\section{Homografia}

L'homografia pot produir ambigüitat lèxica categorial o fins i tot aparéixer
entre mots de la mateixa categoria lèxica. L'homografia apareix quan un mot
(anomenat usualment \emph{homògraf}) té més d'una anàlisi morfològica
possible (vegeu l'apartat~\ref{ss:amblex}). L'espanyol ---com les
altres llengües romà\-ni\-ques--- té molts homògrafs. Una de les fonts més
importants d'homografia és la coincidència entre algunes terminacions
de la flexió verbal i algunes terminacions de la flexió nominal i
adjectival (\emph{-a}, \emph{-as}, \emph{-o}, \emph{-e}, \emph{-es}), ja
que involucra categories lèxiques obertes amb molts
membres.\footnote{De vegades, els mots homògrafs comparteixen una
  semàntica relacionada, com \emph{ahorro}, i altres vegades no, com
  \emph{oso}.} Però hi ha, a més, altres fonts menys productives
d'ambigüitat, com ara la coincidència d'algunes de les terminacions
del present d'indicatiu dels verbs en \emph{-ar} amb les del present de
subjuntiu dels verbs en \emph{-er} i \emph{-ir} i al revés. Finalment, hi ha
algunes homografies fortuïtes (com ara \emph{para}: preposició i verb).

Per a il·lustrar aquest fet, es presenta un assaig de classificació 
---no exhaustiva---  dels homògrafs espanyols:

\begin{enumerate}
\item Homografia verb conjugat--substantiu:
 \begin{enumerate}

   \item En \emph{-a}: 
   \begin{itemize}
   \item Pres.\ ind., 3a pers.\  sing.\ (1a conj.) / subst.\ fem.\ sing.: \emph{casa}, \emph{pinta},
   \emph{ sala}, \emph{toma}, \emph{entrega}, \emph{osa}.
   \item Pres. subj., 1a i 3a pers.\ sing.\ (2a i 3a conj.) / subst.\ fem.\ sing.:
       \emph{bata}, \emph{tema}, \emph{meta}
   \item Altres: \emph{era} (verb \emph{ser}, 1a i 3a pers.\ sing.\
     pretèrit imperf. i subst\. fem\. sing.).
   \end{itemize} 

   \item En \emph{-as}:
   \begin{itemize}
   \item Pres.\ ind. 2a pers.\ sing. (1a conj.) / subst.\ fem.\ pl.:
       \emph{casas}, \emph{salas}, \emph{tomas}, \emph{entregas}, \emph{osas};
   \item Pres.\ ind. 2a pers.\ sing (2a i 3a  conj.) / subst.\ fem.\ pl.:
       \emph{batas}, \emph{temas}, \emph{metas}.
   \item Altres: \emph{eras}.
   \end{itemize}

   \item En \emph{-e}:
   \begin{itemize}
   \item Pres.\ subj., 1a/3a pers.\ sing.\ (1a conj.) / subst.\ masc.\  i fem.\
sing.:
       \emph{cante}, \emph{deje}, \emph{sobre}, \emph{pose}, \emph{apunte}
   \item Pres. ind., 3a pers.\ sing.\ (1a conj.) / subst.\ masc.\  i fem.\
sing.: \emph{vale}.
   \item Altres: \emph{traje} (verb \emph{traer}, 1a pers.\ sing.\
     pretèrit indefinit i subst.\ masc.\ sing.)
   \end{itemize}

   \item En \emph{-es}:
   \begin{itemize}
   \item Pres.\ subj., 2a pers.\ sing.\ (1a conj.) / subst.\ masc.\  i fem.\
pl.: \emph{sales} (verb \emph{salar}), \emph{ases} (verb \emph{asar}), 
     \emph{cantes}, \emph{dejes}, \emph{sobres}, \emph{poses}, \emph{apuntes}
   \item Pres. ind., 2a pers.\ sing.\ (1a conj.) / subst.\ masc.\  i fem.\
pl.: \emph{vales}, \emph{sales} (verb \emph{salir}), \emph{ases} (verb
\emph{asir}).
   \end{itemize}
 

   \item En \emph{-o}: 
   \begin{itemize}
     \item 1a pers.\ del
      present d'indicatiu / subst.\ masc.\ sing.: \emph{oso}, \emph{remiendo}, \emph{riego}, \emph{mando}, 
      \emph{canto}, \emph{cardo}, \emph{recibo}, \emph{abono}, \emph{saldo};
     \item altres: \emph{vino}.
   \end{itemize}  

   \item En \emph{-os}:
       marchamos (1a pers.\ pl. present i pretèrit perfet simple
       d'indicatiu i subst.\ masc.\ pl.).

   \item Altres terminacions: \emph{sal} (verb \emph{salir})
       \emph{mentís}, \emph{pagaré}.
  
   
\end{enumerate}
\item Homografia verb conjugat--adjectiu:
 \begin{enumerate}

   \item En \emph{-a}: 
   \begin{itemize}
   \item Pres.\ ind., 3a pers.\  sing.\ (1a conj.) / adj.\ fem.\ sing.: {\em
       pinta}, \emph{monda},
   \emph{ baja}, \emph{linda}.
   \item Pres. subj., 1a i 3a pers.\ sing.\ (2a i 3a conj.) / adj.\ fem.\ sing.:
       \emph{viva}.
   \end{itemize} 

   \item En \emph{-as}:
   \begin{itemize}
   \item Pres.\ ind. 2a pers.\ sing. (1a conj.) / adj.\ fem.\ pl.:
       \emph{pintas}, \emph{bajas}, \emph{mondas}, \emph{lindas};
   \item Pres.\ ind. 2a pers.\ sing (2a i 3a  conj.) / adj.\ fem.\ pl.:
       \emph{vivas}.
   \end{itemize}

   \item En \emph{-e}: 
   \begin{itemize}
   \item Pres.\ subj.\ 1a/3a. pers.\ sing. (1a conj.) / adj.\ masc.\ i fem.\
     sing.: \emph{leve},  \emph{ausente}, \emph{presente}.
   \end{itemize}

   \item En \emph{-es}: 
   \begin{itemize}
   \item Pres.\ subj.\ 2a pers.\ sing. (1a conj.) / adj.\ masc.\ i fem.\
     sing.: \emph{leves}, \emph{ausentes}, \emph{presentes}.
   \end{itemize}

   \item En \emph{-o}: 
   \begin{itemize}
     \item 1a pers.\ del
      present d'indicatiu / adj.\ masc.\ sing.: \emph{pinto}, \emph{mondo}, \emph{bajo}, \emph{lindo},  \emph{vivo}.
   \end{itemize}  

\end{enumerate}

\item Homografia verb conjugat--verb conjugat (molt difícil de resoldre):
\begin{enumerate}
\item Entre verbs de la 1a conj.\ i verbs de la 2a o 3a conj.:
  \begin{itemize}
  \item \emph{sentir}/\emph{sentar}: \emph{siento}, \emph{sientes}, {\em
      siente}, \emph{sienten}, \emph{sienta}, \emph{sientas}, \emph{ sientan}.
  \item \emph{mentir}/\emph{mentar}: com \emph{sentir}/\emph{sentar}
  \item \emph{vendar}/\emph{vender}: \emph{vendo}, \emph{venda}, {\em
      vendas}, \emph{vendamos}, \emph{vendáis}, \emph{vendan},
    \emph{vende}, \emph{vendes}, \emph{vendemos}, \emph{vendéis}, {\em
      venden}.
  \item \emph{salir}/\emph{salar}: \emph{sales}, \emph{sale}, \emph{salen}
  \item \emph{asir}/\emph{asar}: como \emph{salir}/\emph{salar}
  \item \emph{poder}/\emph{podar}: \emph{podamos}, \emph{podáis}, {\em
      podemos}, \emph{podéis}.
  \item \emph{vengar}/\emph{venir}: \emph{vengo}, \emph{vengas}, {\em
      venga}, \emph{vengamos}, \emph{vengáis}, \emph{vengan}.
  \end{itemize}

\item  Entre la 1a pers.\ pl. del present d'indicatiu i del pretèrit perfet
     simple (``pretèrito indefinido'') dels verbs regulars de la 1a i
     3a conjs.: {\em
       amamos}, \emph{cantamos}, \emph{conseguimos}, etc.
\item Altres casos: \emph{amase},  \emph{amasen}, \emph{amases} ({\em
    amar}, \emph{amasar}); fui, fuiste, ... (\emph{ir} i \emph{ser}),
  \emph{ven} (\emph{ver} i \emph{venir}), etc.
\end{enumerate}


\item Homògrafs verb conjugat--preposició: \emph{bajo}, {\em
    cabe}, \emph{entre}, \emph{para}, \emph{sobre}.

\item Homògrafs adjectiu--preposició: \emph{bajo}.

\item Homògrafs substantiu--preposició: \emph{ante}, \emph{sobre}.

\item Homògrafs verb conjugat--determinant:  \emph{uno}, \emph{una},
  \emph{unas} (\emph{unir})
    
\item Homògrafs verb conjugat--adverbi: \emph{así} (\emph{asir}),
    \emph{fuera} (\emph{ser}, \emph{ir}), \emph{ arriba} (\emph{arribar}), {\em
      adelante} (\emph{adelantar}), \emph{cerca} (\emph{cercar}).

\item Homògrafs adjectiu--adverbi: \emph{mucho}, \emph{poco}, {\em
    fuerte}...

\item Homògrafs substantiu--adverbi: \emph{antes}, \emph{tanto},
  \emph{mal}, \emph{bien}...

\item Homògrafs adjectiu--substantiu: \emph{complejo}, {\em
    impreso}, \emph{derecho}...

\item Homògrafs determinant--pronom
    \emph{la}, \emph{los}, \emph{las}, \emph{lo} (en ``lo que'', ``lo grande'')

\item Altres homògrafs: \emph{como} (conjunció i forma de \emph{comer}), \emph{ora} 
  (conjunció i forma de \emph{orar}), \emph{bien} (conjunció, substantiu i
adverbi)

\end{enumerate}


\section{Divergències de traducció}

Imaginem que hem pogut segmentar el text espanyol i que hem resolt 
correctament les ambigüitats
lèxiques; si encara decidim fer la traducció mot per mot, ens
trobarem que hi ha certes construccions per a les quals la
traducció no és correcta, ja que els mots catalans no es
corresponen mot per mot amb els espanyols. Vegem quins són 
alguns dels problemes:

\begin{description}
\item[Concordança de gènere i nombre:] De vegades el gènere i el
  nombre d'un mot varien de l'espanyol al català. La dificultat per a
  un sistema de traducció automàtica apareix a l'hora de propagar el
  gènere i el nombre del nucli d'un sintagma als modificadors que hi
  hagen de concordar: \emph{su único amparo} $\rightarrow$ \emph{la
    seua única empara}; \emph{un buen postre} $\rightarrow$ \emph{unes
    bones postres}. Els problemes augmenten si la concordança s'ha
  de produir entre sintagmes distants: \emph{el calor producido por el
    motor ha resultado ser nefasto} $\rightarrow$ \emph{la calor
    produïda pel motor ha resultat ser nefasta}. Un problema similar
  el presenta l'establiment (opcional) de la concordança del
  participi, inexistent en espanyol en situacions com ara
  \emph{todavía no la hemos estudiado con profundidad} $\rightarrow$
  \emph{encara no l'hem estudiada amb profunditat}.
\item[L'article neutre:] L'espanyol posseeix l'anomenat {\em
    article neutre}, que no té correspondència en  català
  (\emph{lo que me dijiste} $\rightarrow$ \emph{el que em vas dir});
  presenten dificultat especial les construccions usades per a
  expressar l'abstracció o la intensitat: 
  \emph{recibirá el informe lo más pronto posible}
  $\rightarrow$ \emph{rebrà l'informe el més aviat possible};
  \emph{me asusta lo grande que es} $\rightarrow$ \emph{m'espanta com
    és de gran}.
\item[Els possessius:] De vegades, el català usa articles
  determinats i construccions amb el pronom feble \emph{en} 
  on l'espanyol usa possessius: \emph{cuando hagas cosas así
    debes valorar sus consecuencias} $\rightarrow$ \emph{quan
    faces coses així n'has de valorar les
    conseqüències}. 
\item[Els relatius:]
El principal problema apareix quan es volen traduir oracions que
contenen 
el relatiu possessiu \emph{cuyo},
inexistent en català, on el més senzill és 
usar una construcció amb {\em
  qual}, que, a més, presenta un esquema de concordança
diferent (\emph{qual} ha de concordar amb l'antecedent, mentre {\em
  cuyo} concorda amb el nom que el segueix):
\emph{el contribuyente cuyos informes hemos solicitado llegará tarde} 
$\rightarrow$ \emph{el contribuent els informes del qual hem
  sol·licitat 
arribarà tard} (vegeu el final de l'apartat~\ref{s3:transyn}).
\item[Els pronoms febles:]
Els principals problemes es troben en la traducció de \emph{lo}, ja que
pot correspondre 
en català a alguna forma del pronom masculí singular {\em
  lo} o a alguna forma del
pronom neutre \emph{ho}; en la traducció de 
\emph{se}, el qual correspon normalment al reflexiu 
català \emph{se} però
en les combinacions espanyoles \emph{se la}, \emph{se lo}, etc. pot
correspondre de vegades a alguna forma de \emph{li} o \emph{els}, i en
el fet que l'espanyol no té equivalents dels pronoms catalans
adverbials \emph{en} i \emph{hi}  (\emph{me $\emptyset$ dio uno}
$\rightarrow$
\emph{me'n va donar un}); \emph{$\emptyset$ había dos salidas}
$\rightarrow$ \emph{hi havia dues eixides}; \emph{no $\emptyset$
  $\emptyset$ dejó una} $\rightarrow$ \emph{no n'hi va deixar cap}).
\item[Règim preposicional:] Hi ha diferències notables entre
  els règims preposicionals espanyol i català: les
  preposicions espanyoles davant de \emph{que} completiu no apareixen
  en català (\emph{el hecho de que me hable} $\rightarrow$ \emph{el
    fet que em parle}); algunes preposicions no són possibles en català
  davant d'infinitiu (\emph{el juego consiste en ganar...}
  $\rightarrow$ \emph{el joc consisteix a guanyar...}), etc.
\end{description}

\section{Qüestions i exercicis}
\begin{enumerate}
\item Quina d'aquestes tres tasques és més difícil en un sistema de
  traducció automàtica espanyol--català?
\begin{enumerate}
\item Decidir la traducció del pronom espanyol \emph{se} (pot ser \emph{se},
\emph{li} o \emph{els}).
\item Detectar les formes de \emph{tener que} i traduir-les per \emph{haver de}.
\item Fer l'anàlisi morfològica de verbs seguits d'enclítics 
com ara \emph{estudiémonoslos} o \emph{dándoselo}.
\end{enumerate}

\item Indica quina d'aquestes tres 
és la font més important d'homografia (ambigüitat lèxica
categorial) de l'espanyol:
\begin{enumerate}
\item Les coincidències d'algunes formes d'alguns  
noms i d'alguns adjectius amb certes formes conjugades d'alguns verbs.
\item Les coincidències d'algunes formes de noms amb preposicions.
\item Les coincidències d'algunes formes de noms amb adverbis.
\end{enumerate}

\item El català no té cap construcció equivalent al \emph{cuyo}
  espanyol. En traducció automàtica de l'espanyol al català, una
  alternativa interessant és posar primer el sintagma nominal que
  segueix al \emph{cuyo} i després, una forma de \emph{del
    qual} que concorde amb l'antecedent. Es pot fer sempre
  correctament aquesta operació 
  en un sistema de traducció automàtica que no faça anàlisi
  sintàctica?
\begin{enumerate}
\item Sí, hi ha prou amb fer l'anàlisi morfològica.
\item No, perquè cal determinar bé la longitud del sintagma nominal
  que segueix a \emph{cuyo} per a poder posar \emph{del qual} en la
  posició correcta.
\item No, perquè \emph{cuyo} no té un equivalent morfològic en
  espanyol.
\end{enumerate}


% < QT12

\end{enumerate}

\section{Solucions}
\begin{enumerate}
\item (a)
\item (a)
\item (b)

% < SL12

\end{enumerate}
