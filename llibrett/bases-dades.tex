\chapter{Bases de dades}
\label{se:basesdades}

  Els gestors de terminologia són uns dels programes de MAHT (traducció
 humana assistida per una màquina; vegeu
 l'apartat~\ref{se:cat})
  més usats pels
  traductors. Com que es tracta d'un cas especial d'allò que s'anomena
 en informàtica
 \emph{bases de dades}, introduirem primer aquest concepte i llavors
 n'estudiarem l'aplicació a la gestió terminològica
 a través de les bases de dades terminològiques.


\section{Què és una base de dades?} 

Com s'explica en la pàg.~\pageref{pg:fitxer}, anomenem
\emph{fitxers} els conjunts de dades que es guarden en un mitjà
d'emmagatzematge secundari, que es manipulen com un tot i que
s'identifiquen per un nom. Molts dels fitxers que usen les persones
que es dediquen a la traducció són fitxers (o \emph{documents}) de
text de diversos formats, com els descrits en
l'epígraf~\ref{ss:formats}, però també n'hi ha que es corresponen amb
el significat del mot \emph{fitxer} fora de la informàtica: contenen
\emph{fitxes}, totes amb un format més o menys constant; per exemple,
totes les fitxes d'un fitxer bibliogràfic contenen informació sobre
els autors, el títol, l'any de publicació, etc. 

En informàtica, els fitxers d'aquesta mena se solen anomenar
normalment \emph{bases de dades} (BD); les fitxes s'anomenen
\emph{registres} i cada element d'informació de la fitxa s'anomena
\emph{camp}. 

Més generalment, Una BD es pot veure (en l'anomenat model \emph{pla} o
\emph{de taules}) com un conjunt de taules en les quals les columnes o
\emph{camps} guarden valors del mateix tipus i on els elements d'una
fila o \emph{registre} estan relacionats entre ells. Així, una taula
és un conjunt de registres (fitxes) cada un dels quals té la mateixa
estructura de camps (informacions), emmagatzemades en un fitxer
informàtic.

Així, els registres d'una base de dades bibliogràfica contenen
informació en camps: un per als autors, altre per al títol, etc.;
d'altra banda, els registres de les bases de dades usades per a la
gestió d'un vídeoclub contenen camps per a emmagatzemar la informació
referent als socis (com ara, el nom, el telèfon o el domicili), a les
pel·lícules (el títol, la persona que l'ha dirigida o el nombre de
còpies disponibles) i als prèstecs (les dades d'inici i termini del
prèstec o el preu de lloguer). Els camps poden ser de diversos
\emph{tipus}, segons la naturalesa de les dades que s'hi guarden
(cadenes de caràcters, valors numèrics enters o amb decimals, dates de
calendari, etc.)

\section{Operacions amb bases de dades}

Les operacions més comunes que es realitzen sobre una base de dades
també són similars a aquelles que podem fer amb un fitxer de fitxes de
cartolina, però la gestió és més senzilla i són possibles molts més usos:
\begin{itemize}
\item \emph{Creació de l'estructura de la base de dades}: definir cada taula (definint
  l'estructura de les fitxes)
\item \emph{Altes o addicions:} afegir un nou registre a la base de dades,
  tot fent que els seus camps prenguen els valors corresponents.
\item \emph{Baixes o esborrats:} eliminar un o més registres.
\item \emph{Modificacions:} canviar el valor d'un o més camps d'un o
  més registres de la base de dades.
\item \emph{Recerques o consultes:} buscar un o més registres que
  compleixen un determinat criteri de recerca. L'organització de la
  informació en forma de base de dades simplifica enormement les
  consultes, ja que els ordinadors són molt més ràpids i segurs a
  l'hora de, per exemple, comparar el contingut d'un determinat camp
  de totes les fitxes amb un cert valor o patró (per exemple, els
  autors que comencen per \emph{Per}) i llistar el contingut d'un
  altre camp (per exemple, el títol) per a cada fitxa coincident.
\item \emph{Explotació:} a més de fer recerques o consultes, es pot
  explotar la informació obtinguda de la base de dades, combinant
  diverses estratègies de recerca i presentant només els camps
  necessaris en algun format convenient per a la reutilització
  informatitzada o per a la publicació: per exemple, generar, a partir
  d'un fitxer bibliogràfic, les referències bibliogràfiques citades en
  un text ordenades alfabèticament i en el format requerit per una
  determinada revista, o generar, a partir d'una base de dades de
  clients, una carta de recordatori per als morosos, amb detalls sobre
  els seus deutes.
\end{itemize}
El programa que permet fer, entre altres, aquestes operacions de
manera senzilla o fins i tot automàtica (aspecte molt important quan
la base de dades conté milers de registres) és un programa
\emph{gestor de bases de dades}, o bé el programa inclou un gestor de
bases de dades o l'invoca.\footnote{De vegades, quan es parla
  descuradament, s'anomena per metonímia \emph{base de dades} al
  programa gestor.}

Normalment, els usuaris reals no executen un programa gestor de bases
de dades universal o genèric, sinó que usen programes o \emph{aplicacions} que
simplifiquen la creació, el manteniment i l'ús de la base de dades per
a un perfil d'usuari concret.

En particular, és possible organitzar les bases de dades de manera que
estiguen instal·lades en un o més servidor es puguen consultar i
explotar des d'un altre ordinador a través d'Internet.


\subsection{Recerques}

Quan volem buscar una determinada informació en un fitxer de fitxes de
cartolina (per exemple, quins autors han usat el
mot \emph{arbre} en el
títol de les seues obres), i aquest fitxer no està ordenat d'acord amb cap
criteri convenient, ens veurem obligats a mirar totes les fitxes una
per una.  Però és comú que els fitxers estiguen ordenats segons un
dels seus camps: per exemple, un fitxer bibliogràfic pot estar ordenat
pel cognom del primer autor, o per la matèria. Si la consulta o la
recerca que volem fer es refereix al camp pel qual s'ha establit
l'ordenació, és molt més senzilla que si es refereix a un altre camp,
i es pot completar sense mirar totes les fitxes; per exemple, fent-hi
una \emph{recerca dicotòmica}.

En una recerca dicotòmica mirem la fitxa que hi ha enmig del fitxer;
si ens hem passat, repetim l'operació amb la primera meitat del
fitxer, i si ens hem quedat curts, ho fem amb la segona meitat. Es pot
demostrar que la recerca dicotòmica mira com a molt $n$ fitxes si el
fitxer té entre $2^{n-1}$ i $2^n$ fitxes, perquè després de cada
consulta es redueix a la meitat la grandària del fitxer que cal
explorar; per exemple, si el fitxer té 1234 fitxes, hi ha prou amb
$n=11$ recerques perquè $2^{10}=1024$ i $2^{11}=2048$.

Per descomptat, és possible calcular el nombre màxim de consultes de
manera més pedestre, dividint el nombre de fitxes per 2 i posant-nos
en el cas pitjor. En l'exemple de les 1234 fitxes:
\begin{itemize}
\item Després de la 1a consulta, si la fitxa central no és la que busquem, ens queden dues meitats: una de 616 fitxes, i una altra de 617. Imaginem que anem al pitjor cas: 617 fitxes.
\item Després de la 2a consulta, si la fitxa central no és la que busquem, ens queden dues meitats de 308 fitxes.
\item 3a consulta: o la trobem, o hem de buscar en 154 fitxes.
\item 4a consulta: o la trobem, o hem de buscar en 77 fitxes.
\item 5a consulta: o la trobem, o hem de buscar en 38 fitxes.
\item 6a consulta: o la trobem, o hem de buscar en 19 fitxes.
\item 7a consulta: o la trobem, o hem de buscar en 9 fitxes.
\item 8a consulta: o la trobem, o hem de buscar en 4 fitxes.
\item 9a consulta: o la trobem, o hem de buscar en 2 fitxes.
\item 10a consulta: o la trobem, o hem de buscar en 1 fitxa.
\item 11a consulta: o és la que busquem, o no ho és.
\end{itemize}
Total, 11 consultes com a molt.

Un altre exemple: la taula~\ref{tb:dicotomica} conté una llista de cognoms ordenats
alfabèticament. Imaginem, en primer lloc, que volem buscar l'element
``Garrido''. Inicialment, mirem l'element d'enmig (el 13) de la llista
sencera (la llista que inclou els elements de l'1 al 25) i hi trobem
``Larrañaga''; com que ens hem passat, ens quedem amb la meitat baixa
de la llista (que inclou els elements de l'1 al 12) oblidant-nos de
l'altra meitat. Ara repetim el procés i mirem l'element d'enmig (el 6)
de la nova subllista i hi trobem l'element ``Esteve''; com que és
menor alfabèticament que l'element que estem buscant, ens quedem amb
la meitat alta (que inclou els elements del 7 al 12) de la subllista
actual. Tornem a repetir el procés, aquesta vegada considerant només
la subllista d'elements entre el 7 i el 12; tenim sort i en mirar
l'element central (el 9) ens adonem que coincideix amb l'element
buscat: la recerca, doncs, acaba amb èxit.  Podeu veure una
representació gràfica de tot el procés en les taules~\ref{tb:dico1}
a~\ref{tb:dicoult}: s'hi mostra en negretes l'element consultat en
cada moment i ombrejada la part de la llista descartada.

  Si la recerca haguera estat de l'element ``González'', els passos
  anteriors haurien estat els mateixos, però a més hauríem mirat
  l'element 11, l'element 10 i, finalment, hauríem conclòs que
  l'element no es troba a la llista. En aquest cas, el nombre de
  consultes hauria estat de 5 i, doncs, es compleix l'afirmació
  anterior: la recerca dicotòmica mira com a molt $n$ fitxes si
  el fitxer té entre $2^{n-1}$ i $2^n$ fitxes; ací 
 el nombre d'elements total
  (25) està entre $2^5=32$ i $2^4=16$,
  i el nombre
  d'elements consultats ha estat $n=5$.


\begin{table}
\begin{center}
  \begin{small}
\begin{tabular}{|l||l||l||l||l|}
\hline
  \multicolumn{1}{|>{\columncolor[gray]{1}}l||}{1 Beléndez} &
  \multicolumn{1}{>{\columncolor[gray]{1}}l||}{2 Canals} &
  \multicolumn{1}{>{\columncolor[gray]{1}}l||}{3 Carrasco} &     
  \multicolumn{1}{>{\columncolor[gray]{1}}l||}{4 Escolano} &
  \multicolumn{1}{>{\columncolor[gray]{1}}l|}{5 Espí} \\
\hline
  \multicolumn{1}{|>{\columncolor[gray]{1}}l||}{6 Esteve} &
  \multicolumn{1}{>{\columncolor[gray]{1}}l||}{7 Forcada} &
  \multicolumn{1}{>{\columncolor[gray]{1}}l||}{8 Garcia} &
  \multicolumn{1}{>{\columncolor[gray]{1}}l||}{9 Garrido} &       
  \multicolumn{1}{>{\columncolor[gray]{1}}l|}{10 Gómez} \\
\hline
  \multicolumn{1}{|>{\columncolor[gray]{1}}l||}{11 Guardiola} &   
  \multicolumn{1}{>{\columncolor[gray]{1}}l||}{12 Iturraspe} &
  \multicolumn{1}{>{\columncolor[gray]{1}}l||}{13 Larrañaga} &   
  \multicolumn{1}{>{\columncolor[gray]{1}}l||}{14 Marco} &
  \multicolumn{1}{>{\columncolor[gray]{1}}l|}{15 Micó} \\ 
\hline 
  \multicolumn{1}{|>{\columncolor[gray]{1}}l||}{16 Montserrat} &
  \multicolumn{1}{>{\columncolor[gray]{1}}l||}{17 Muñoz} &       
  \multicolumn{1}{>{\columncolor[gray]{1}}l||}{18 Nogueroles} &
  \multicolumn{1}{>{\columncolor[gray]{1}}l||}{19 Odriozola} &   
  \multicolumn{1}{>{\columncolor[gray]{1}}l|}{20 Oncina} \\
\hline
  \multicolumn{1}{|>{\columncolor[gray]{1}}l||}{21 Ortiz} &       
  \multicolumn{1}{>{\columncolor[gray]{1}}l||}{22 Pastor} &
  \multicolumn{1}{>{\columncolor[gray]{1}}l||}{23 Pérez} &       
  \multicolumn{1}{>{\columncolor[gray]{1}}l||}{24 Sempere} &
  \multicolumn{1}{>{\columncolor[gray]{1}}l|}{25 Zubizarreta} \\
\hline
\end{tabular}
\end{small}
\end{center}  
\caption{Llista ordenada alfabèticament usada per mostrar un exemple
 de recerca dicotòmica.}
\label{tb:dicotomica}
\end{table}

\begin{table}
\begin{center}
  \begin{small}
\begin{tabular}{|l||l||l||l||l|}
\hline
  \multicolumn{1}{|>{\columncolor[gray]{1}}l||}{1 Beléndez} &
  \multicolumn{1}{>{\columncolor[gray]{1}}l||}{2 Canals} &
  \multicolumn{1}{>{\columncolor[gray]{1}}l||}{3 Carrasco} &     
  \multicolumn{1}{>{\columncolor[gray]{1}}l||}{4 Escolano} &
  \multicolumn{1}{>{\columncolor[gray]{1}}l|}{5 Espí} \\
\hline
  \multicolumn{1}{|>{\columncolor[gray]{1}}l||}{6 Esteve} &
  \multicolumn{1}{>{\columncolor[gray]{1}}l||}{7 Forcada} &
  \multicolumn{1}{>{\columncolor[gray]{1}}l||}{8 Garcia} &
  \multicolumn{1}{>{\columncolor[gray]{1}}l||}{9 Garrido} &       
  \multicolumn{1}{>{\columncolor[gray]{1}}l|}{10 Gómez} \\
\hline
  \multicolumn{1}{|>{\columncolor[gray]{1}}l||}{11 Guardiola} &   
  \multicolumn{1}{>{\columncolor[gray]{1}}l||}{12 Iturraspe} &
  \multicolumn{1}{>{\columncolor[gray]{0.75}}l||}{\bf 13 Larrañaga} &   
  \multicolumn{1}{>{\columncolor[gray]{0.75}}l||}{14 Marco} &
  \multicolumn{1}{>{\columncolor[gray]{0.75}}l|}{15 Micó} \\ 
\hline 
  \multicolumn{1}{|>{\columncolor[gray]{0.75}}l||}{16 Montserrat} &
  \multicolumn{1}{>{\columncolor[gray]{0.75}}l||}{17 Muñoz} &       
  \multicolumn{1}{>{\columncolor[gray]{0.75}}l||}{18 Nogueroles} &
  \multicolumn{1}{>{\columncolor[gray]{0.75}}l||}{19 Odriozola} &   
  \multicolumn{1}{>{\columncolor[gray]{0.75}}l|}{20 Oncina} \\
\hline
  \multicolumn{1}{|>{\columncolor[gray]{0.75}}l||}{21 Ortiz} &       
  \multicolumn{1}{>{\columncolor[gray]{0.75}}l||}{22 Pastor} &
  \multicolumn{1}{>{\columncolor[gray]{0.75}}l||}{23 Pérez} &       
  \multicolumn{1}{>{\columncolor[gray]{0.75}}l||}{24 Sempere} &
  \multicolumn{1}{>{\columncolor[gray]{0.75}}l|}{25 Zubizarreta} \\
\hline
\end{tabular}
\end{small}
\end{center}  
\caption{Primer pas de la recerca dicotòmica de l'element ``Garrido'':
  mirem l'element 13 i ens quedem amb la meitat baixa de la llista.}
\label{tb:dico1}
\end{table}

\begin{table}
\begin{center}
  \begin{small}
\begin{tabular}{|l||l||l||l||l|}
\hline
  \multicolumn{1}{|>{\columncolor[gray]{0.75}}l||}{1 Beléndez} &
  \multicolumn{1}{>{\columncolor[gray]{0.75}}l||}{2 Canals} &
  \multicolumn{1}{>{\columncolor[gray]{0.75}}l||}{3 Carrasco} &     
  \multicolumn{1}{>{\columncolor[gray]{0.75}}l||}{4 Escolano} &
  \multicolumn{1}{>{\columncolor[gray]{0.75}}l|}{5 Espí} \\
\hline
  \multicolumn{1}{|>{\columncolor[gray]{0.75}}l||}{\bf 6 Esteve} &
  \multicolumn{1}{>{\columncolor[gray]{1}}l||}{7 Forcada} &
  \multicolumn{1}{>{\columncolor[gray]{1}}l||}{8 Garcia} &
  \multicolumn{1}{>{\columncolor[gray]{1}}l||}{9 Garrido} &       
  \multicolumn{1}{>{\columncolor[gray]{1}}l|}{10 Gómez} \\
\hline
  \multicolumn{1}{|>{\columncolor[gray]{1}}l||}{11 Guardiola} &   
  \multicolumn{1}{>{\columncolor[gray]{1}}l||}{12 Iturraspe} &
  \multicolumn{1}{>{\columncolor[gray]{0.75}}l||}{13 Larrañaga} &   
  \multicolumn{1}{>{\columncolor[gray]{0.75}}l||}{14 Marco} &
  \multicolumn{1}{>{\columncolor[gray]{0.75}}l|}{15 Micó} \\ 
\hline 
  \multicolumn{1}{|>{\columncolor[gray]{0.75}}l||}{16 Montserrat} &
  \multicolumn{1}{>{\columncolor[gray]{0.75}}l||}{17 Muñoz} &       
  \multicolumn{1}{>{\columncolor[gray]{0.75}}l||}{18 Nogueroles} &
  \multicolumn{1}{>{\columncolor[gray]{0.75}}l||}{19 Odriozola} &   
  \multicolumn{1}{>{\columncolor[gray]{0.75}}l|}{20 Oncina} \\
\hline
  \multicolumn{1}{|>{\columncolor[gray]{0.75}}l||}{21 Ortiz} &       
  \multicolumn{1}{>{\columncolor[gray]{0.75}}l||}{22 Pastor} &
  \multicolumn{1}{>{\columncolor[gray]{0.75}}l||}{23 Pérez} &       
  \multicolumn{1}{>{\columncolor[gray]{0.75}}l||}{24 Sempere} &
  \multicolumn{1}{>{\columncolor[gray]{0.75}}l|}{25 Zubizarreta} \\
\hline
\end{tabular}
\end{small}
\end{center}  
\caption{Segon pas de la recerca dicotòmica de l'element ``Garrido'':
  mirem l'element 6 i ens quedem amb la meitat alta de la subllista
anterior.}
\label{tb:dico2}
\end{table}

\begin{table}
\begin{center}
  \begin{small}
\begin{tabular}{|l||l||l||l||l|}
\hline
  \multicolumn{1}{|>{\columncolor[gray]{0.75}}l||}{1 Beléndez} &
  \multicolumn{1}{>{\columncolor[gray]{0.75}}l||}{2 Canals} &
  \multicolumn{1}{>{\columncolor[gray]{0.75}}l||}{3 Carrasco} &     
  \multicolumn{1}{>{\columncolor[gray]{0.75}}l||}{4 Escolano} &
  \multicolumn{1}{>{\columncolor[gray]{0.75}}l|}{5 Espí} \\
\hline
  \multicolumn{1}{|>{\columncolor[gray]{0.75}}l||}{6 Esteve} &
  \multicolumn{1}{>{\columncolor[gray]{0.75}}l||}{7 Forcada} &
  \multicolumn{1}{>{\columncolor[gray]{0.75}}l||}{8 Garcia} &
  \multicolumn{1}{>{\columncolor[gray]{1}}l||}{\bf 9 Garrido} &       
  \multicolumn{1}{>{\columncolor[gray]{0.75}}l|}{10 Gómez} \\
\hline
  \multicolumn{1}{|>{\columncolor[gray]{0.75}}l||}{11 Guardiola} &   
  \multicolumn{1}{>{\columncolor[gray]{0.75}}l||}{12 Iturraspe} &
  \multicolumn{1}{>{\columncolor[gray]{0.75}}l||}{13 Larrañaga} &   
  \multicolumn{1}{>{\columncolor[gray]{0.75}}l||}{14 Marco} &
  \multicolumn{1}{>{\columncolor[gray]{0.75}}l|}{15 Micó} \\ 
\hline 
  \multicolumn{1}{|>{\columncolor[gray]{0.75}}l||}{16 Montserrat} &
  \multicolumn{1}{>{\columncolor[gray]{0.75}}l||}{17 Muñoz} &       
  \multicolumn{1}{>{\columncolor[gray]{0.75}}l||}{18 Nogueroles} &
  \multicolumn{1}{>{\columncolor[gray]{0.75}}l||}{19 Odriozola} &   
  \multicolumn{1}{>{\columncolor[gray]{0.75}}l|}{20 Oncina} \\
\hline
  \multicolumn{1}{|>{\columncolor[gray]{0.75}}l||}{21 Ortiz} &       
  \multicolumn{1}{>{\columncolor[gray]{0.75}}l||}{22 Pastor} &
  \multicolumn{1}{>{\columncolor[gray]{0.75}}l||}{23 Pérez} &       
  \multicolumn{1}{>{\columncolor[gray]{0.75}}l||}{24 Sempere} &
  \multicolumn{1}{>{\columncolor[gray]{0.75}}l|}{25 Zubizarreta} \\
\hline
\end{tabular}
\end{small}
\end{center}  
\caption{Tercer i últim pas de la recerca dicotòmica de l'element ``Garrido'':
  mirem l'element 9 i trobem l'element buscat.}
\label{tb:dicoult}
\end{table}

\com{Repassar les fórmules, que potser estan malament.}

La recerca dicotòmica és només una de les moltes tècniques que es
poden fer servir per a accelerar les consultes que es refereixen a un
camp ordenat; el programa gestor de bases de dades pot usar una altra
tècnica de recerca depenent del seu disseny o del tipus de camp.

Però un fitxer de fitxes de cartolina només es pot ordenar seguint un
únic criteri. Si volem facilitar les consultes associades a més d'un
camp (per exemple, autors i matèries) ens veurem obligats a mantenir
dues còpies del fitxer sencer, cada còpia ordenada per un criteri; amb
un sistema de bases de dades no cal fer aquesta duplicació: només cal
marcar els camps pels que buscarem més freqüentment (els quals de
vegades se solen anomenar \emph{índexs}), i el programa gestor de
bases de dades \emph{indexarà} la base de dades per a permetre
recerques ràpides per aquests camps.


\begin{persabermes}{la indexació de bases de dades}

  Si les consultes més freqüents d'una base de dades es refereixen a
  un camp ---o a una combinació de camps, com ara el dia, el mes i
  l'any que formen una data--- que pren valors que es poden ordenar,
  els registres es poden ordenar per aquest camp, igual que el fitxer
  de fitxes de cartolina. Però un dels avantatges més clars de les
  bases de dades és que permeten que els registres estiguen ordenats
  per més d'un camp, sense haver de duplicar la base de dades (Tot i
  que la duplicació d'una base de dades menuda pot no semblar en
  principi problemàtica, les coses canvien si considerem una base de
  dades amb milers de registres i amb un gran nombre de camps, tots
  plens d'informació.  És evident, doncs, que cal establir un sistema
  alternatiu per poder fer recerques ràpides per més d'un camp).  Això
  s'aconsegueix mitjançant un procediment anomenat \emph{indexació}:
  bàsicament, s'assigna un número a cada fitxa i es construeix una
  taula o \emph{índex} ordenat (una altra base de dades) que conté
  registres amb dos camps: un, el camp pel qual es vol ordenar, i
  l'altre, la posició en la base de dades del registre que conté
  aquest valor del camp (en cert sentit, aquest índex no és massa
  diferent de l'índex alfabètic que hi ha al final d'alguns llibres:
  busquem el mot alfabèticament i ens diu en quina o en quines pàgines
  se'n parla).

Es pot construir un índex per a cada un dels camps associats a les
consultes més freqüents i així s'evita recórrer tota la base de dades
cada vegada que es fa una consulta: es busca el valor del camp en el
registre corresponent i, quan es troba, s'usa la posició del registre
per a accedir-hi directament. Una base de dades amb aquestes
propietats està \emph{indexada}. 
%A més de fer-hi \emph{consultes}, hi
%ha altres operacions que es poden fer amb una base de dades; les més
%importants són les \emph{altes} o addicions de registres, les
%\emph{baixes} o eliminacions de registres, i les \emph{modificacions}
%de la informació continguda en un o més registres. 
Els índexs s'han de
refer parcialment quan es fan altes, baixes o modificacions de registres per tal que les
consultes continuen sent eficients. Quan creem una nova base de dades i definim
l'estructura de camps que tindrà cada un dels seus registres, podem
designar quins dels camps corresponen als índexs; el gestor de la base
de dades crearà automàticament els índexs corresponents.

  Considereu la base de dades amb 10 registres de la taula
%~\ref{tb:indextot}.
següent:
% \begin{table}
 \begin{center}
 \begin{tabular}{clll}
 \hline
 {\sc núm} & {\sc basc} & {\sc serbo-croat} & {\sc català} \\
 \hline 
 \hline
  1 & bat  &     jedin &       un \\
  2 & bi   &     dva   &       dos \\
  3 & hiru &     tri   &       tres \\
  4 & lau  &     \v{c}etiri &      quatre \\
  5 & bost &     pet    &      cinc \\
  6 & sei  &     \v{s}est   &      sis \\
  7 & zazpi &    sedam  &      set \\
  8 & zortzi &   osam   &      vuit \\
  9 & bederatzi & devet &       nou \\
 10 & hamar &    deset  &      deu \\
 \hline
 \end{tabular}
% \caption{Exemple de base de dades amb 10 fitxes.}
% \label{tb:indextot}
 \end{center}
% \end{table}

 Com hi
  podeu veure, la base de dades conté 10 registres amb 4 camps; els
  registres estan ordenats pel camp ``\texttt{NÚM}'', que indica la
  posició de cada registre. Si volem fer consultes ràpides pels camps
  ``\texttt{BASC}'' i ``\texttt{SERBO-CROAT}'' sense haver de visitar
  (en el cas pitjor) tots els registres, el gestor de bases de dades
  ha de definir un índex per a cada un d'aquests camps com els de les
  taules següents, que mostren els índexs corresponents 
    als camps ``\texttt{SERBO-CROAT}'' (esquerra) i 
    ``\texttt{BASC}'' (dreta) de la base de dades de la
    taula anterior:
 \begin{center}
 \parbox{0.25\textwidth}{
 \begin{tabular}{lc}
 \hline
 {\sc serbo-croat} & {\sc núm} \\
 \hline
 \hline
 \v{c}etiri &     4 \\
 deset &      10 \\
 devet &      9 \\
 dva &        2 \\
 jedin &      1 \\
 osam &       8 \\
 pet &        5 \\
 sedam &      7 \\
 \v{s}est &       6 \\
 tri &        3 \\
 \hline
 \end{tabular}}%
\hspace{0.2\textwidth}%
\parbox{0.25\textwidth}{
   \begin{tabular}{lc}
   \hline
   {\sc basc\phantom{o-croat}} & {\sc núm} \\
   \hline
   \hline
  bat &       1 \\
  bederatzi  & 9 \\  
  bi &        2 \\   
  bost &      5 \\   
  hamar &     10 \\   
  hiru &       3 \\   
  lau &       4 \\   
  sei &        6 \\    
  zazpi &      7 \\ 
  zortzi &    8 \\
   \hline
   \end{tabular}}%
  \end{center}
 Hi podeu comprovar com cada fitxa de
  l'índex conté només el camp indexat i una referència a la seua
  posició en la base de dades. Si penseu en el cas que la base de
  dades de la taula anterior tinguera, diguem-ne, 20 camps més
  (amb els equivalents en 20 llengües més), us podeu fer una idea de
  l'estalvi que s'aconsegueix respecte a la duplicació sencera.

 % \begin{table}
 % \begin{center}
 % \begin{tabular}{clll}
 % \hline
 % {\sc núm} & {\sc basc} & {\sc serbo-croat} & {\sc català} \\
 % \hline 
 % \hline
 %  1 & bat  &     jedin &       un \\
 %  2 & bi   &     dva   &       dos \\
 %  3 & hiru &     tri   &       tres \\
 %  4 & lau  &     \v{c}etiri &      quatre \\
 %  5 & bost &     pet    &      cinc \\
 %  6 & sei  &     \v{s}est   &      sis \\
 %  7 & zazpi &    sedam  &      set \\
 %  8 & zortzi &   osam   &      vuit \\
 %  9 & bederatzi & devet &       nou \\
 % 10 & hamar &    deset  &      deu \\
 % \hline
 % \end{tabular}
 % \caption{Exemple de base de dades amb 10 fitxes.}
 % \label{tb:indextot}
 % \end{center}
 % \end{table}

% \begin{table}
%  \begin{center}
%  \parbox{0.25\textwidth}{
%  \begin{tabular}{lc}
%  \hline
%  {\sc serbo-croat} & {\sc núm} \\
%  \hline
%  \hline
%  \v{c}etiri &     4 \\
%  deset &      10 \\
%  devet &      9 \\
%  dva &        2 \\
%  jedin &      1 \\
%  osam &       8 \\
%  pet &        5 \\
%  sedam &      7 \\
%  \v{s}est &       6 \\
%  tri &        3 \\
%  \hline
%  \end{tabular}}%
% \hspace{0.2\textwidth}%
% \parbox{0.25\textwidth}{
%    \begin{tabular}{lc}
%    \hline
%    {\sc basc\phantom{o-croat}} & {\sc núm} \\
%    \hline
%    \hline
%   bat &       1 \\
%   bederatzi  & 9 \\  
%   bi &        2 \\   
%   bost &      5 \\   
%   hamar &     10 \\   
%   hiru &       3 \\   
%   lau &       4 \\   
%   sei &        6 \\    
%   zazpi &      7 \\ 
%   zortzi &    8 \\
%    \hline
%    \end{tabular}}%
%   \end{center}
%   \caption{Índexs corresponents 
%     als camps ``\texttt{SERBO-CROAT}'' (esquerra) i 
%     ``\texttt{BASC}'' (dreta) de la base de dades de la
%     taula~\ref{tb:indextot}.}
%    \label{tb:indexbasc}
% \end{table}
\end{persabermes}


%\com{Alguna cosa sobre recerques combinades o parcials i la facilitat
%  de la seua automatització amb BBDD?}

\section{Bases de dades lèxiques o terminològiques}
\label{ss:bdterm}

Un dels programes més comunament usats pels professionals de la
traducció són els gestors de bases de dades lèxiques (normalment
anomenats gestors de bases de dades \emph{terminològiques}, encara que
es poden usar per a moltes altres aplicacions a més de les
estrictament terminològiques). Els professionals de la traducció
gestionen, amb aquests gestors, bases de dades lèxiques que els ajuden
a traduir consistentment els termes i, potser, les locucions i frases
d'una determinada àrea de coneixement (terminologia). 

Els registres o les fitxes d'una base de dades lèxica multilingüe
estableixen correspondències entre els termes usats en diverses
llengües en un camp determinat de coneixement. En aquestes bases de
dades, \emph{cada fitxa representa un concepte} i conté un camp índex
per a emmagatzemar el terme corresponent en cada llengua. Aquesta
organització (una fitxa per concepte) és coherent amb la definició de
terminologia com la disciplina l'objecte de la qual és l'estudi
sistemàtic de l'etiquetatge o designació de \emph{conceptes}
particulars d'una o més àrees temàtiques o d'un o àmbits de
l'activitat humana amb el propòsit de documentar i promoure l'ús
correcte;\footnote{\url{https://ca.wikipedia.org/wiki/Terminologia},
  24 de novembre de 2015 a més, és especialment adequada quan la base
  de dades terminològica és multilingüe.}

Les bases de dades lèxiques o terminològiques poden contenir molts
tipus de camps:
\begin{itemize}
\item El terme en cada una de les llengües (camps que normalment s'usen
  d'índex per a fer les recerques més eficients).
\item El sentit (entre els possibles sentits del terme) al qual es
  refereix la fitxa o registre actual.
\item L'autor de la fitxa (quan més d'una persona gestiona la base de
  dades).
\item La data de creació i de modificació de la fitxa.
\item La definició del terme en una o més llengües. La major part dels
  gestor permeten que els termes usats en la definició d'un determinat
  terme es marquen \emph{remissions}, és a dir, enllaços actius a les
  fitxes on es defineixen. 
\item El camp temàtic de la fitxa.
\item Altres termes relacionats.
\item Informació sobre la morfologia o la flexió del terme en cada una
  de les llengües.
\item 
    Variants ortogràfiques o geogràfiques; sinònims; antònims;
    etimologia, etc. 
  
\end{itemize}

Una base de dades d'aquesta mena la pot consultar una persona mentre
està fent una traducció manualment o pot estar inclosa dins d'un
programa de traducció automàtica o assistida per ordinador. Per
exemple, molts programes de traducció assistida per ordinador (vegeu
el capítol~\ref{se:memtrad}) inclouen bases de dades terminològiques
d'aquesta mena i permeten que la persona usuària les mantinga i les
consulte, bé usant un programa independent, o bé des del processador
de textos que preferisca. 

També hi ha bases de dades terminològiques que es poden consultar en
línia: 
\begin{itemize}
\item L'Institut d'Estudis Catalans manté el TERMCAT
(\url{http://www.termcat.cat}). La intenció de TERMCAT és principalment normativa: s'hi associa a cada concepte el terme preferit en català (i també els termes usuals en espanyol, anglés, francés, etc.).
\item IATE (\emph{Inter-Active Terminology for Europe},
  \url{http://iate.europa.eu/}) és la base de dades terminològica de
  referència de la Unió Europea i proporciona termes per a cada
  concepte en les 24 llengües oficials de la Unió Europea i en
  llatí. La base de dades, en format TBX (vegeu
  l'apartat~\ref{s3:tbx}) es pot descarregar total o parcialment per a
  usar-la fora de línia.
\end{itemize}

\subsection{L'intercanvi de bases de dades terminològiques}
\label{s3:tbx}

La creació i el manteniment d'una bona base de dades terminològica
requereix un gran esforç i moltes hores de treball. Això la converteix
en un recurs valuós i sovint els traductors en fan intercanvi. Però la
informació emmagatzemada sobre cada terme pot ser molt diferent d'una
base de dades a una altra i, per tant, fer servir una base de dades
terminològica en diferents sistemes alhora no és una tasca
fàcil. Afortunadament, en els darrers anys s'ha desenvolupat un
conjunt de formats estàndards per facilitar aquest intercanvi; un dels
més coneguts es TBX\footnote{L'URI per a més informació és
  \url{http://www.lisa.org/tbx/}.}, (per \emph{TermBase exchange},
``intercanvi de bases de dades terminològiques''), encara que hi ha
altres com OLIF (\emph{Open Lexicon Interchange Format}). Amb el
temps, els programes que inclouen una base de dades terminològica van
incorporant la capacitat de llegir i escriure documents en format TBX.

\begin{persabermes}{el format TBX}
  

El format TBX segueix les especificacions XML (vegeu l'apartat 
\ref{s3:SGML}); és a dir, els documents TBX són un tipus de document
XML definit per una DTD concreta. A més, també segueix les directrius
de l'estàndar ISO 12620, que defineix un conjunt de camps, i els
seus possibles valors, per a la informació terminològica.

Heus ací un exemple de document TBX:
\begin{verbatim}
<?xml version='1.0'?>
<!DOCTYPE martif SYSTEM  "./TBXcoreStructureDTD-v-1-0.DTD">
<martif type='TBX' xml:lang='en' >
 <martifHeader>
  <fileDesc>
   <sourceDesc>
    <p>from an Oracle corporation termBase</p>
   </sourceDesc>
  </fileDesc>
  <encodingDesc>
   <p type='DCSName'>TBXdefaultXCS-v-1-0.XML</p>
  </encodingDesc>
 </martifHeader>
 <text> 
  <body>
    <termEntry id='eid-Oracle-67'>
      <descrip type='subjectField'>
        manufacturing
      </descrip>
      <descrip type='definition'>
        A value between 0 and 1 used in ...
      </descrip>
      <langSet xml:lang='en'>
       <tig>
        <term tid='tid-Oracle-67-en1'>
          alpha smoothing factor
        </term>
        <termNote type='termType'>
          fullForm
        </termNote>
       </tig>
      </langSet>
      <langSet xml:lang='hu'>
        <tig>
         <term tid='tid-Oracle-67-hu1'>
           Alfa sim&#x00ED;t&#x00E1;si t&#x00E9;nyez&#x0151; 
         </term>
        </tig>
      </langSet>
    </termEntry>
  </body> 
 </text>
</martif>
\end{verbatim}

La informació de cada terme (en l'exemple només un) s'inclou dins de
l'element {\tt termEntry}, que conté descripcions (\texttt{descrip})
sobre el domini d'ús i la definició del terme, a més d'una secció 
{\tt langSet} per a cada idioma (ací, anglés i hongarés) on
s'especifica el terme (\texttt{term}) i informació sobre ell 
(\texttt{termNote}). Abans del cos (\texttt{body}), l'element 
arrel {\tt martif} conté una capçalera (\texttt{martifHeader})
amb informació sobre l'origen de la base de dades. El terme hongarés
és ``{\usefont{T1}{ec}{m}{n}Alfa simítási tényez\symbol{'256}}''; en
el document d'exemple, els caràcters especials s'indiquen amb el seu
codi Unicode (per exemple,\, \verb+&#x0151;+ \ és el caràcter
``{\usefont{T1}{ec}{m}{n}\symbol{'256}}'')
\end{persabermes}


\section{Qüestions i exercicis}
\begin{enumerate}

\item 
   Tenim una base de dades amb fitxes de 100 alumnes ordenada pel NIF.
   Si busquem una alumna pel NIF, quantes consultes fa, com a màxim,
   el programa gestor de la base de dades fins arribar a la fitxa
   desitjada?
   
\begin{enumerate}
\item 100
\item 50
\item 7
\end{enumerate}

\item 
   Volem tenir una base de dades ordenada simultàniament per dos
 camps per a optimitzar les recerques. És això possible? Com?
   
\begin{enumerate}
\item No és possible.
\item Sí, però és necessari duplicar totes les fitxes en
     memòria.
\item Sí. S'han de construir dos
     índexs, un per a cada camp.
\end{enumerate}

\item És necessari tenir dues còpies de totes les fitxes d'una base
   de dades per a poder-la tenir ordenada per més d'un camp?
\begin{enumerate}
\item Depén; si els camps són numèrics no és necessari.
\item No hi ha cap altre remei: cal duplicar la base de dades.
\item No; s'hi pot crear més d'un \emph{índex}.
\end{enumerate}
\item Quan usem un programa gestor de bases de dades
   terminològiques per a buscar un terme en una base de dades{\ldots}
   
\begin{enumerate}
\item {\ldots} alguns gestors ens permeten buscar-lo sense
      conéixer-ne la forma exacta.
\item {\ldots} hem de conéixer com s'escriu exactament el terme
      per a poder trobar-lo.
\item {\ldots}sempre és necessari conéixer-ne la categoria lèxica
      i indicar-la al gestor
\end{enumerate}

\item 
    Si busquem en una base de dades de 240 fitxes d'alumnes una fitxa
    (un registre) pel número de telèfon, un camp pel qual no està
    ordenada, quantes consultes haurà de fer, com a màxim, el programa
    gestor de la base de dades fins arribar a la fitxa desitjada?
   
\begin{enumerate}
\item 240
\item 120
\item 9
\end{enumerate}
\item 
   Quan volem tenir els registres (les fitxes) d'una base de dades
 ordenada simultàniament per dos camps per a optimitzar les recerques
 construïm dos índexs. Els índexs contenen{\ldots}
   
\begin{enumerate}
\item Entrades compostes pel valor del camp índex i el número
     de la fitxa (del registre).
\item Només els números de les fitxes ordenats segons el valor del
     camp.
\item Una còpia de les fitxes ordenades pel camp índex
     corresponent.
\end{enumerate}

\item Què tenen en comú tots els camps d'una fitxa terminològica?
   
\begin{enumerate}
\item Es refereixen al mateix concepte.
\item Es refereixen al mateix terme.
\item Es troben en el mateix índex.
\end{enumerate}
\item En què es diferencia un camp clau o camp índex de la resta
   dels camps d'una fitxa?{\ldots}
   
\begin{enumerate}
\item Es guarda en un tipus de memòria RAM més ràpida anomenada
      \emph{cache} o \emph{memòria cau}.
\item La manera d'emmagatzemar el camp és diferent (els camps índex o clau
      s'emmagatzemen de manera comprimida i els altres no).
\item Les recerques de fitxes per aquest camp són
      molt més ràpides que les que es facen per camps que no són clau
      o índex.
\end{enumerate}
\item Quan es duplica el nombre de fitxes d'una taula determinada
   d'una base de dades, què succeeix amb el nombre de consultes que
   realitza una recerca dicotòmica?
   
\begin{enumerate}
\item Es duplica.
\item Es queda exactament com està.
\item S'incrementa en 1 com a promedi.
\end{enumerate}

% < QT5
\item En una base de dades lèxica o terminològica amb 2.000 fitxes, quan demanem al programa gestor que busque un determinat terme en una determinada llengua, quantes consultes ha de fer en el pitjor cas per a entregar-nos la fitxa?
  \begin{enumerate}
  \item Ha de fer, necessàriament, 2.000 consultes perquè hi ha 2.000 fitxes.
  \item No ha de fer cap consulta, va directament a la fitxa.
  \item Depén. Si està indexada pel terme en aquella llengua, farà com a molt 11 consultes, i si no, com a molt 2.000 consultes.
  \end{enumerate}

\item En una base de dades terminològica, cada concepte \ldots 
  \begin{enumerate}
  \item \ldots es correspon amb un camp d'un registre
      general.
  \item \ldots es correspon amb múltiples
      registres, un per cadascun dels termes usats en cada llengua per
      a representar aquest concepte.
    \item \ldots es correspon amb un registre en el qual es troben els
      termes usats en cada llengua per a representar aquest concepte.
  \end{enumerate}

\item Hem cronometrat el temps que tarda un programa gestor de bases
  de dades en trobar la fitxa que té un determinat valor per un camp
  determinat, quan augmenta el nombre de fitxes. Els resultats són:
\begin{center}
\begin{tabular}{c|c}
\hline\hline
\textsc{Nombre de fitxes} & \textsc{Temps} \\
\hline
1.000.000 & 4,9~s \\
2.000.000 & 5,1~s \\
3.000.000 & 5,3~s \\
4.000.000 & 5,4~s \\
6.000.000 & 5,5~s \\
\hline
\end{tabular}
\end{center}
Què podem dir de la base de dades?
\begin{enumerate}
\item Que no està ordenada pel camp pel qual estem buscant.
\item Que usa XML per a obtenir una velocitat acceptable.
\item Que està ordenada pel camp pel qual estem buscant.
\end{enumerate}

\item 
El programa gestor d'una base de dades lèxica feia en el pitjor cas 480 consultes abans de trobar la fitxa corresponent a un terme anglés concret, abans d'ordenar-la pel terme en anglés. Després d'ordenar-la fa
\begin{enumerate}
\item 240, la meitat.
\item moltes menys consultes: 9.
\item 480 consultes igualment.
\end{enumerate}


\end{enumerate}

\section{Solucions}
\com{Repassar-les}
\begin{enumerate}
\item (c)
\item (c)
\item (c)
\item (a)
\item (a)
\item (a)
\item (a)
\item (c)
\item (c)
\item (c)
\item (c)
\item (c)
\item (b) 

% < SL5

\end{enumerate}
