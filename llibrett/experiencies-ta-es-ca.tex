\chapter{Experiències de TA espanyol--català}
\label{se:ETACC}
\com{Actualitzar totes les descripcions. Explicar la raó d'aquest
  capítol (com a exemple)}

En aquest capítol es descriuen breument cinc experiències de traducció
automàtica de l'espanyol al català: SALT, Ara, Es-Ca, els traductors
Automatictrans i el del \textit{El Periòdico de Catalunya} i una
altra, interNOSTRUM, amb una miqueta més de detall.

\section{SALT, de la Generalitat Valenciana}

\comprof{Estudiar i descriure SALT 2}

El programa SALT (versió actual, 2.0) porta el nom del \emph{Servei d'Assessorament
  Lingüístic i Traducció} de la Conselleria de Cultura, Educació i
Ciència de la Generalitat Valenciana; es tracta d'un programa per al
sistema operatiu Windows que ha desenvolupat un equip de programadors
dirigit per Rafael Pinter sota la direcció lingüística de Josep
Lacreu, que va ser responsable d'aquest servei fins fa poc.
Inicialment, la disponibilitat del programa va ser més aviat reduïda;
actualment es pot descarregar gratuïtament de diversos servidors
d'Internet (per exemple,
\url{http://www.edu.gva.es/polin/val/salt/apolin_salt.htm}
o \url{www.softcatala.org}) i també el distribueixen els serveis
de normalització lingüística d'algunes universitats.  SALT tradueix
textos (ANSI, RTF o documents de Microsoft Word) en espanyol a la
variant valenciana del català ---l'estàndard lingüístic dels textos
meta es pot regular usant un menú molt senzill--- o corregeix una bona
part de les errades típiques dels textos valencians. El programa pot
ser interactiu i preguntar (al final de la traducció o durant el
procés) a la persona usuària com ha de resoldre algunes ambigüitats, i
dialoga sempre en català; a més, l'usuari pot seguir visualment el
procés de traducció (mot a mot amb modificacions locals) en dues
passades. Els resultats són molt interessants. El programa està
bàsicament concebut com una ajuda a les persones que volen
començar a generar documents en valencià (entre altres eines,
inclou una completíssima guia interactiva de gramàtica i estil).

L'Acadèmia Valenciana de la Llengua ha declarat \emph{oficials} ``els
continguts'' del programa SALT 2 (acord de 20 de maig del 2002).

\section{Ara, d'Autotrad}

El programa Ara, llançat l'any 2000 per l'empresa Autotrad de
València\footnote{URI \url{http://www.ara-autotrad.com}} ---el
gerent de la qual és Rafael Pinter, responsable informàtic de SALT---
és bàsicament una versió bastant millorada del SALT, amb una aparença
molt similar però amb algunes diferències: p.e., produeix textos en
català oriental estàndard, pot dialogar amb la persona usuària en
espanyol i en català, i permet programar tasques de traducció perquè
s'executen sense necessitat que la persona usuària les atenga. El cost
(l'any 2004) és de 45 euros per llicència.

\section{Es-Ca, de Incyta}

El sistema de traducció automàtica Es-Ca, originalment desenvolupat
per l'empresa Incyta de Cornellà (que després d'una etapa en la qual
va formar part de l'extinta multinacional Sail Labs, ha renascut amb
el nom d'Incyta Multianguage, S.L.) en col·laboració amb la
Universitat Autònoma de Barcelona és un sistema de transferència
sintàctica estàndard, hereu del sistema METAL de l'empresa Siemens. El
sistema no es distribueix com a programa, sinó que es troba en
Internet (\url{http://www.incyta.com}): l'usuari inscrit envia el
text i el servidor li'l retorna traduït; els costos en 1999 eren de 3
pessetes per paraula.  Els resultats del traductor, quan tenia una
demostració disponible en Internet, eren molt acceptables en la major
part dels casos (en l'actualitat no és possible avaluar el programa
gratuïtament).

\section{El traductor d'\emph{El Periódico de Catalunya} i AutomaticTrans}
\label{ss:ePdC}

Una experiència interessant de traducció espanyol--català per a la
disseminació és l'edició bilingüe del diari \emph{El Periódico de
  Catalunya}\footnote{Disponible per Internet:
  \url{http://www.elperiodico.es}.}; el text original ---en
espanyol la major part de les vegades--- es tradueix usant una tècnica
similar a les \emph{memòries de traducció} (vegeu el
capítol~\ref{se:memtrad}) i després és revisat per un equip de
posteditors del mateix periòdic abans de ser publicat.

Un programa similar (i, segons les nostres notícies, d'origen comú) a
l'usat per \emph{El Periódico de Catalunya} s'anomena AutomaticTrans i
es pot provar (sobre textos de 50 paraules) en
Internet.\footnote{\url{http://www.automatictrans.es}}


\section{interNOSTRUM}


Un equip d'investigadors de la Universitat d'Alacant, finançat per la
Caja de Ahorros del Mediterràneo i la mateixa Universitat, està
desenvolupant actualment sota la direcció d'un dels autors d'aquest
llibre un sistema de traducció automàtica espanyol--català anomenat
\textsf{interNOSTRUM} \citep{canals01a,canals01b}. Més concretament
l'objectiu del projecte (vigent des de novembre de 1998) és
desenvolupar un sistema de traducció automàtica de l'espanyol a les
variants estàndards del català i el sistema invers corresponent.

La versió actual d'{\sf interNOSTRUM} (accessible de forma gratuïta a
través d'Internet, \url{http://www.internostrum.com}) no és
encara un producte acabat, però ja pot ser usat per a generar, quasi
instantàniament, esborranys de traduccions al català llestes per a ser
corregides (posteditades).

Actualment, {\sf interNOSTRUM} tradueix textos en formats ANSI, HTML i
RTF de l'espanyol al català oriental i al revés (sobre formats, vegeu
l'epígraf~\ref{ss:formats}) i permet la navegació traduïda per
Internet (és a dir, permet la traducció instantània dels documents que
es vagen visitant sense haver d'invocar explícitament el traductor).
Al final del projecte, es disposarà de traductors que produïsquen i
accepten altres variants estàndard del català\footnote{En l'actualitat
  el traductor accepta nombroses variants balears i valencianes
  quan tradueix del català a l'espanyol.}. El traductor
català--espanyol està menys desenvolupat que el traductor
espanyol--català.

\subsection{Característiques informàtiques}

El traductor s'executa actualment sobre el sistema operatiu GNU/Linux i és
accessible, com ja s'ha dit, a través d'un servidor
d'Internet;\footnote{També hi ha disponible una versió per a servidors
  basats en el sistema operatiu Windows.} está constituït per 8
subprogrames independents que s'executen simultàniament (en
paral·lel), elaboren la traducció per etapes i es comuniquen
mitjançant canals de text ANSI legible\footnote{El sistema operatiu
  GNU/Linux permet construir una canonada (\emph{pipeline}) o cadena de
  muntatge en la qual el text d'eixida d'un programa s'envia com a
  entrada a un altre (en comptes d'enviar-lo a la pantalla), i així
  successivament.} (cosa que facilita enormement el diagnòstic dels
errors de traducció). Sis dels 8 subprogrames es generen
automàticament a partir de les dades lingüístiques
corresponents,\footnote{Característica que permet estendre fàcilment
  el producte a altres idiomes} mitjançant ferramentes informàtiques
desevolupades en el projecte; els altres dos mòduls (el primer i
l'últim de la cadena de muntatge) s'encarreguen, el primer
(\emph{desformatador}), de separar els codis de format HTML o RTF del
text, i l'últim (\emph{reformatador}), de tornar a combinar-los per
tal que la traducció conserve el format del text origen.  La
figura~\ref{fg:modules} mostra un diagrama de blocs del traductor. La
velocitat actual del sistema és de l'ordre de milers de mots per segon
sobre un PC estàndard (molt més ràpida que SALT o Ara, que tradueixen
a velocitats de l'ordre de desenes de molts per segon).

\begin{figure}
{\small
\setlength{\tabcolsep}{0.5mm}
\begin{center}
\begin{tabular}{cccccccccc}
\\
\parbox{0.7cm}{text LO} $\rightarrow$ &
\framebox{\parbox{1.4cm}{desfor\-matador}} $\rightarrow$ &
\framebox{\parbox{0.8cm}{anal. morf.}}  $\rightarrow$ &
\framebox{\parbox{1.2cm}{desamb. categ.}} $\rightarrow$ & \framebox{\parbox{1.1cm}{transf.\ estruc.}} $\rightarrow$ &
\framebox{\parbox{0.8cm}{gen. morf.}} 
$\rightarrow$ & \framebox{\parbox{1.0cm}{post\-genera\-dor}}
$\rightarrow$ & \framebox{\parbox{1.2cm}{reforma\-tador}} $\rightarrow$ &
\parbox{0.7cm}{text LM}\\\\
& &  & & $\updownarrow$   & \\\\
& &
 &  & \framebox{\parbox{1.0cm}{transf.\ lèxica}}                &   \\
\end{tabular}
\end{center}
}
\caption{Els 8 mòduls que formen la cadena de muntatge d'interNOSTRUM.}
\label{fg:modules}
\end{figure}


\subsection{Característiques lingüístiques}

{\sf interNOSTRUM} és un sistema clàssic de traducció indirecta per
transferència morfològica avançada (vegeu l'epígraf~\ref{s3:STMorf}),
amb les fases lingüístiques següents:
\begin{enumerate}
\item ANÀLISI: \begin{itemize}
      \item Anàlisi morfològica
      \item Desambiguació lèxica categorial
      \end{itemize}
\item TRANSFERÈNCIA:\begin{itemize}
      \item Transferència lèxica: consulta del diccionari bilingüe
      \item Transferència estructural: tractament de patrons
        (concordança, reordenament, canvis lèxics)\end{itemize}
\item GENERACIÓ:\begin{itemize}
                \item Generació morfològica
                \item Postgeneració (apostrofació, etc.)
                \end{itemize}
\end{enumerate}

\subsubsection{Subprogrames basats en tècniques d'estats finits} 

Els subprogrames d'\emph{anàlisi morfològica}, \emph{consulta del
  diccionari bilingüe}, \emph{generació morfològica} i {\em
  postgeneració} estan basats en \emph{transductors d'estats finits},
similars als descrits en l'epígraf~\ref{s3:anmor}. Aquesta
tecnologia permet velocitats de processament de l'ordre de 10.000 mots
per segon, velocitats que pràcticament no depenen de la grandària dels
diccionaris. Els \emph{transductors d'estats finits} usats en
interNOSTRUM lligen l'entrada símbol a símbol; cada vegada que es
llegeix una lletra canvien d'estat i van produint, també lletra a
lletra, una o més sortides.
\begin{description}
\item[Anàlisi morfològica:] El subprograma d'anàlisi morfològica, que es
  genera automàticament a partir d'un \emph{diccionari morfològic} de
  la llengua origen (LO), el qual conté els lemes, els paradigmes de flexió
  i les connexions entre ells. L'entrada són les formes superficials
  del text i la sortida, formes lèxiques consistents en lema, categoria
  lèxica i informació de flexió. 
\item[Consulta del diccionari bilingüe] (transferència lèxica): El subprograma de consulta del
  diccionari bilingüe és invocat pel subprograma de tractament de patrons
  (vegeu més avall); es genera automàticament a partir d'un fitxer que
  conté les correspondències bilingües. L'entrada és la forma lèxica
  de la LO i la sortida, la forma lèxica corresponent en
  la llengua meta (LM).
\item[Generació morfològica:] El generador morfològic fa l'operació
  inversa a l'analitzador morfològic però amb formes de la LM i es
  genera automàticament a partir d'un diccionari morfològic de la LM.
\item[Postgeneració:] Les formes superficials que estan implicades en
  processos d'apostrofació i guionatge (pronoms febles, articles,
  algunes preposicions, etc.) activen aquest subprograma, que
  normalment es troba inactiu. El postgenerador es genera a partir de
  regles senzilles d'apostrofació, guionatge i combinació de pronoms febles.
\end{description}
Com ja s'ha discutit en el capítol~\ref{se:PdTACC}, la divisió d'un
text en mots presenta alguns aspectes no trivials; se n'esmenten dos:
les \emph{locucions} (o \emph{girs}) i els pronoms enclítics.

\paragraph{Locucions i girs:} Hi ha nombroses locucions i girs que es poden tractar com a \emph{unitats
  multimot} i s'estan incorporant gradualment als diccionaris
morfològics de les dues llengües i al diccionari bilingüe:
\begin{itemize}
\item \emph{con cargo a} $\rightarrow$ \emph{a càrrec de}
\item \emph{por adelantado} $\rightarrow$ \emph{per endavant}, \emph{a la bestreta}
\item \emph{el abajo firmante} $\rightarrow$ \emph{el sotasignat}
\item \emph{{\bf echar} de menos} $\rightarrow$ \emph{{\bf trobar} a faltar}
\end{itemize}
En l'últim exemple, el gir no és invariable sinó que té un element que
es flexiona (en negretes).

\paragraph{Pronoms enclítics:} El subprograma d'anàlisi morfològica també és capaç de resoldre les
combinacions de verbs i pronoms febles enclítics en espanyol, les
quals presenten variacions ortogràfiques com ara canvis d'accentuació
o pèrdua de consonants:
\begin{itemize}
\item \emph{d\'{a}melo} = \emph{da} $+$ \emph{me} $+$ \emph{lo} $\rightarrow$
\emph{dóna} $+$ \emph{me} $+$ \emph{lo} = \emph{dóna-me'l}
\item \emph{pong\'{a}monos} = \emph{pongamos} $+$ \emph{nos} $\rightarrow$ {\em
posem} $+$ \emph{nos} = \emph{posem-nos}.  
\end{itemize}

El sistema interNOSTRUM tracta aquests dos problemes amb l'analitzador
morfològic, el qual és capaç de decidir quan un grup de mots s'ha de
tractar conjuntament o per separat. 

\subsubsection{El subprograma de desambiguació lèxica categorial} 

Aquest subprograma usa un model de llenguatge basat en trigrames
(se\-qüèn\-cies de tres categories lèxiques), de l'estil del descrit
en la secció~\ref{s3:reshom}. Aquest model es basa en les freqüències
observades per a aquests trigrames en un corpus de referència, i
assigna una probabilitat a cada possible desambiguació de la frase que
conté mots amb ambigüitat categorial. La desambiguació més probable
(la més versemblant) és l'elegida. En l'actualitat, les prestacions
d'aquest subprograma són molt millorables perquè els corpus de
referència actualment en ús no són encara suficientment
representatius.

Els pocs errors en homògrafs \emph{difícils} i \emph{freqüents}, com
ara \emph{una} (article/verb, freqüència 0,77\%), \emph{para}
(verb/prep., freqüència 0,77\%) i \emph{como} (conj./verb, freqüència
0,43\%) degraden actualment molt la qualitat de la traducció. Altres
homògrafs freqüents no són tan difícils de desambiguar.

Les ambigüitats lèxiques no categorials s'aborden amb estratègies {\em
  ad hoc} provisionals. En el futur, molts d'aquests homògrafs
s'inclouran en la definició d'un \emph{llenguatge controlat},
alternativa a la preedició consistent en l'aplicació de restriccions
lèxiques, sintàctiques i d'estil als textos de la LO. La CAM ha
encarregat també el disseny d'un espanyol controlat per a textos
financers i dels \emph{assistents d'estil} corresponents per als
autors.


\subsubsection{El subprograma de tractament de patrons (transferència
  estructural)} 
Malgrat la gran semblança entre l'espanyol i el català, hi ha
divergències gramaticals considerables (vegeu el capítol~\ref{se:PdTACC}):
\begin{itemize}
\item perífrasis modals: \emph{tienen que firmar}
$\rightarrow$ \emph{han de firmar}; 
\item canvis de gènere i nombre: \emph{la deuda
  contraída} $\rightarrow$ \emph{el deute contret} (masc.);
\item caiguda de preposicions: \emph{la intención de que el cliente} $\rightarrow$ \emph{la
intenció $\emptyset$ que el client}; 
\item construccions relatives: \emph{la cuenta cuyo titular es} $\rightarrow$ \emph{el compte el
    titular del qual és}.
\end{itemize}
Aquestes divergències s'han de tractar amb les regles gramaticals
escaients.  

\com{Fer referència a les seccions corresponents de la resta del llibre!!!}
La solució elegida (estàndard en sistemes comercials, vegeu
\citet{mira98j}, \citet{forcada00p}) es basa en la detecció i el
tractament de seqüències predefinides de categories lèxiques
(anomenades \emph{patrons}), és a dir, una mena de sintagmes
rudimentaris, com ara {\bf art}--{\bf nom} o {\bf art}--{\bf
  nom}--{\bf adj}. Les seqüències considerades pel subprograma en
formen el \emph{catàleg} de patrons. El funcionament del subprograma es
basa en un esquema patró--acció:
\begin{itemize}
\item Llegeix el text (analitzat i desambiguat) d'esquerra a dreta,
  categoria lèxica a categoria lèxica.
\item Busca, en la posició actual de la frase, el patró més llarg que
  concorda amb un patró del seu catàleg (per exemple, si en la posició
  actual es llegeix ``un senyal inequívoc\ldots'', tria {\bf
    art}--{\bf nom}--{\bf adj} en comptes de {\bf art}--{\bf nom}).
\item Opera sobre aquest patró (propagació de gènere i nombre,
  reordenament, canvis lèxics) seguint les regles associades a ell.
\item Continua immediatament darrere del patró tractat (no torna a
  visitar els mots sobre els quals ha operat).
\end{itemize}
Quan no es detecta cap patró en la posició actual, es tradueix
literalment un mot i es torna a iniciar el procés.  Els fenòmens ``a
la llarga'' com la concordança subjecte--predicat són una mica més
difícils de tractar; s'usa un registre \emph{estat} o  \emph{memòria}
que recorda certes informacions al llarg del procés.

El subprograma de tractament de patrons es genera automàticament a
partir d'un fitxer de regles que especifica els patrons i les accions
associades \citep{garridoalenda01j}.  Aquest és molt probablement el
subprograma més lent (uns 5.000 mots/segon).

% \subsection{Eines de suport a {\sf interNOSTRUM}}
% \com{Fa temps que ``es projecta'' i no fem res: llevar?}

% Es projecta construir les eines següents:
% \begin{itemize}
% \item Un assistent d'estil que permetrà l'autor d'un text en espanyol
%   evitar moltes ambigüitats difícils de resoldre usant regles
%   lèxiques, sintàctiques i d'estil (un  \emph{llenguatge controlat},
%   vegeu la secció~\ref{ss:llecon}).
% \item Un assistent de preedició, que permetrà una desambiguació manual
%   de mots i estructures problemàtiques (simplement fent-hi clic per
%   accedir als menús corresponents) quan els mètodes estadístics
%   indicats més amunt siguen incapaços de fer les tries correctes.
% \item Un assistent de postedició, que permetrà fer clic sobre un mot
%   sospitós de ser una traducció incorrecta i substituir-lo per altres
%   alternatives tenint en compte el text original i farà possible en
%   general qualsevol canvi del text meta.
% \end{itemize}
