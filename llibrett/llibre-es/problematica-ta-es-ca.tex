\chapter{Traducción automática español--catalán} \label{se:PdTACC} 

La inclusión de este apéndice en el libro tiene tres objetivos: \begin{enumerate} \item Estudiar con algo más de detalle los problemas que plantea la traducción automática entre dos lenguas emparentadas. Podríais pensar que, siendo el español y el catalán tan similares sintácticamente, los problemas serían poco importantes: el capítulo intenta convenceros de que las cosas no son tan sencillas como podrían parecer a primera vista. \item Ilustrar, con un par de lenguas concreto, algunos de los conceptos tratados en los capítulos anteriores. \item Proporcionar una breve descripción de las experiencias de traducción automática español--catalán existentes. \end{enumerate} 

\section{Problemática de la traducción automática español--catalán} 

\subsection{Introducción} 

Las aplicaciones potencialmente más interesantes de la TA español--catalán se enmarcan dentro de la llamada {\em normalización lingüística}, es decir, el esfuerzo de las sociedades de habla catalana para promover su uso normal en todos los ámbitos; un ejemplo actual lo constituyen los servidores de Internet de instituciones públicas y de empresas privadas de los territorios de habla catalana, donde la presencia del catalán es todavía minoritaria. Cuando la lengua original de los documentos es el español, se podría usar un sistema de TA para generar borradores de documentos en catalán (o, incluso, documentos prácticamente correctos si los documentos en español están escritos en un lenguaje controlado). 

En el caso concreto del español y el catalán, la proximidad lingüística entre las dos lenguas hace que sea abordable el diseño de sistemas de traducción automática que generen textos de un nivel de corrección tal que resulte más rentable revisar el resultado en bruto producido por el programa que hacer la traducción completa (véase la p.~\ref{pg:cost}). 

En este capítulo se presentan algunos de los problemas más importantes con los que se puede encontrar quién quiera diseñar un sistema de traducción automática para traducir textos del español al catalán. A la vista de la notable similitud lingüística existente entre las dos lenguas, se podría pensar que la tarea de traducción automática podría, en la mayoría de los casos, ser tan sencilla como sustituir una a una las palabras en español por sus equivalentes en catalán. De hecho, el modelo de traducción automática \emph{palabra por palabra} (definido en la pág.~\pageref{pg:mpm}, y que no se debe confundir con lo que se suele denominar \emph{traducción literal}) es el modelo de referencia que usaremos en este capítulo: los tres grupos de {\em problemas} que se presentan en este capítulo son algunos ---no todos--- de los que no resuelve el modelo palabra por palabra: la segmentación del texto origen, la homografía y las divergencias sintácticas. 

\subsection{Segmentación del texto origen} 

La segmentación de un texto en palabras suele ser normalmente muy sencilla: el programa puede usar los blancos, los tabuladores, los finales de línea o los signos de puntuación como fronteras entre palabras. Pero a veces no es tan fácil: por ejemplo, el español une muchas veces varias palabras en una sola palabra, sin que se  puedan distinguir las palabras componentes; en catalán, salvo contracciones como \emph{al}, \emph{pels}, y \emph{del}, siempre queda alguna indicación de esta unión, como por ejemplo un apóstrofo o un guion. Por ejemplo, en español, los pronombres enclíticos se unen al imperativo, al infinitivo y al gerundio, y muchas veces hacen que  cambie la forma (véase el epígrafe~\ref{s3:STMorf}). Por suerte, estos problemas se pueden resolver de manera sencilla usando analizadores morfológicos como los que se describen en el epígrafe~\ref{s3:anmor}. 

\subsection{Homografía} 

La homografía puede producir ambigüedad léxica categorial o incluso aparecer entre palabras de la misma categoría léxica. La homografía aparece cuando una palabra (denominada usualmente \emph{homógrafa}) tiene más de un análisis morfológico posible (véase el apartado~\ref{ss:amblex}). El español ---como las otras lenguas románicas--- tiene muchos homógrafos. Una de las fuentes más importantes de homografía es la coincidencia entre algunas terminaciones de la flexión verbal y algunas terminaciones de la flexión nominal y adjetival (\emph{-a ,} \emph{-as}, \emph{-o ,} \emph{-e}, \emph{-es}), puesto que involucra categorías léxicas abiertas con muchos miembros.\footnote{A veces, las palabras homógrafas comparten una semántica relacionada, como \emph{ahorro}, y otras veces no, como \emph{oso}.} Pero hay, además, otras fuentes menos productivas de ambigüedad, como por ejemplo la coincidencia de algunas de las terminaciones del presente de indicativo de los verbos en \emph{-ar} con las del presente de subjuntivo de los verbos en \emph{-er} e \emph{ir} y a la inversa. Finalmente, hay algunas homografías fortuitas (algunas particularmente frecuentes, como por ejemplo \emph{para}, preposición y verbo; \emph{una}, determinante y verbo, y \emph{como}, adverbio relativo, preposición y verbo). 

Para ilustrar este hecho, se presenta un ensayo de clasificación ---no exhaustiva--- de los homógrafos españoles: 

\begin{enumerate} \item Homografía verbo conjugado--sustantivo: \begin{enumerate} 

\item En \emph{-a :} \begin{itemize} \item Pres.\ ind., 3.ª pers.\  sing.\ (1.ª conj.) / sust.\ fem.\ sing.: \emph{casa}, \emph{pinta}, \emph{ sala}, \emph{toma}, \emph{entrega}, \emph{osa}. \item Pres. subj., 1.ª y 3.ª pers.\ sing.\ (2.ª y 3.ª conj.) / sust.\ fem.\ sing.: \emph{bata}, \emph{tema}, \emph{meta} \item Otros: \emph{era} (verbo \emph{ser}, 1.ª y 3.ª pers.\ sing.\ pretérito imperf.  y sust.\ fem.\  sing.). \end{itemize} 

\item En \emph{-as}: \begin{itemize} \item Pres.\ ind. 2.ª pers.\ sing. (1.ª conj.) / sust.\ fem.\ pl.: \emph{casas}, \emph{salas}, \emph{tomas}, \emph{entregas}, \emph{osas}; \item Pres.\ ind. 2.ª pers.\ sing (2.ª y 3.ª conj.) / sust.\ fem.\ pl.: \emph{batas}, \emph{temas}, \emph{metas}. \item Otros: \emph{eras}. \end{itemize} 

\item En \emph{-e}: \begin{itemize} \item Pres.\ subj., 1.ª y 3.ª pers.\ sing.\ (1.ª conj.) / sust.\ masc.\  y fem.\ sing.: \emph{cante}, \emph{deje}, \emph{sobre}, \emph{pose}, \emph{apunte} \item Pres. ind., 3.ª pers.\ sing.\ (1.ª conj.) / sust.\ masc.\  y fem.\ sing.: \emph{vale}. \item Otros: \emph{traje} (verb \emph{traer}, 1.ª pers.\ sing.\ pretérito indefinido y sust.\ masc.\ sing.) \end{itemize} 

\item En \emph{-es}: \begin{itemize} \item Pres.\ subj., 2.ª pers.\ sing.\ (1.ª conj.) / sust.\ masc.\  y fem.\ pl.: \emph{sales} (verb \emph{salar}), \emph{ases} (verbo \emph{asar}), \emph{cantes}, \emph{dejes}, \emph{sobres}, \emph{poses}, \emph{apuntes} \item Pres. ind., 2.ª pers.\ sing.\ (1.ª conj.) / sust.\ masc.\  y fem.\ pl.: \emph{vales}, \emph{sales} (verb \emph{salir}), \emph{ases} (verb \emph{asir}). \end{itemize} 

\item En \emph{-o :} \begin{itemize} \item 1.ª pers.\ del presente de indicativo / sust.\ masc.\ sing.: \emph{oso}, \emph{remiendo}, \emph{riego}, \emph{mando}, \emph{canto}, \emph{cardo}, \emph{recibo}, \emph{abono}, \emph{saldo}; \item otros: \emph{vino}. \end{itemize} 

\item En \emph{-os}: \emph{marchamos} (1.ª pers.\ pl. presente y pretérito perfecto simple de indicativo y subst.\ masc.\ pl.). 

\item Otras terminaciones: \emph{sal} (verbo \emph{salir}) \emph{mentís}, \emph{pagaré}. 

\end{enumerate} \item Homografía verbo conjugado--adjetivo: \begin{enumerate} 

\item En \emph{-a}: \begin{itemize} \item Pres.\ ind., 3.ª pers.\  sing.\ (1.ª conj.) / adj.\ fem.\ sing.: {\em pinta}, \emph{monda}, \emph{ baja}, \emph{linda}. \item Pres. subj., 1.ª y 3.ª pers.\ sing.\ (2.ª y 3.ª conj.) / adj.\ fem.\ sing.: \emph{viva}. \end{itemize} 

\item En \emph{-as}: \begin{itemize} \item Pres.\ ind. 2.ª pers.\ sing. (1.ª conj.) / adj.\ fem.\ pl.: \emph{pintas}, \emph{bajas}, \emph{mondas}, \emph{lindas}; \item Pres.\ ind. 2.ª pers.\ sing (2.ª y 3.ª conj.) / adj.\ fem.\ pl.: \emph{vivas}. \end{itemize} 

\item En \emph{-e}: \begin{itemize} \item Pres.\ subj.\ 1.ª y 3.ª pers.\ sing. (1.ª conj.) / adj.\ masc.\ y fem.\ sing.: \emph{leve}, \emph{ausente}, \emph{presente}. \end{itemize} 

\item En \emph{-es}: \begin{itemize} \item Pres.\ subj.\ 2.ª pers.\ sing. (1.ª conj.) / adj.\ masc.\ y fem.\ sing.: \emph{leves}, \emph{ausentes}, \emph{presentes}. \end{itemize} 

\item En \emph{-o}: \begin{itemize} \item 1.ª pers.\ del presente de indicativo / adj.\ masc.\ sing.: \emph{pinto}, \emph{mondo}, \emph{bajo}, \emph{lindo}, \emph{vivo}. \end{itemize} 

\end{enumerate} 

\item Homografía verbo conjugado--verbo conjugado (muy difícil de resolver): \begin{enumerate} \item Entre verbos de la 1.ª conj.\ y verbos de la 2.ª o 3.ª conj.: \begin{itemize} \item \emph{sentir}/\emph{sentar}: \emph{siento}, \emph{sientes}, {\em siente}, \emph{sienten}, \emph{sienta}, \emph{sientas}, \emph{ sientan}. \item \emph{mentir}/\emph{mentar}: como \emph{sentir}/\emph{sentar} \item \emph{vendar}/\emph{vender}: \emph{vendo}, \emph{venda}, {\em vendas}, \emph{vendamos}, \emph{vendáis}, \emph{vendan}, \emph{vende}, \emph{vendes}, \emph{vendemos}, \emph{vendéis}, {\em venden}. \item \emph{salir}/\emph{salar}: \emph{sales}, \emph{sale}, \emph{salen} \item \emph{asir}/\emph{asar}: como \emph{salir}/\emph{salar} \item \emph{poder}/\emph{podar}: \emph{podamos}, \emph{podáis}, {\em podemos}, \emph{podéis}. \item \emph{vengar}/\emph{venir}: \emph{vengo}, \emph{vengas}, {\em venga}, \emph{vengamos}, \emph{vengáis}, \emph{vengan}. \end{itemize} 

\item Entre la 1.ª pers.\ pl. del presente de indicativo y del pretérito indefinido de los verbos regulares de la 1.ª y 3.ª conj.: {\em amamos}, \emph{cantamos}, \emph{conseguimos}, etc. \item Otros casos: \emph{amase}, \emph{amasen}, \emph{amases} ({\em amar}, \emph{amasar}); \emph{fui}, \emph{fuiste}, \ldots (\emph{ir} i \emph{ser}), \emph{ven} (\emph{ver} i \emph{venir}), etc. \end{enumerate} 

\item Homógrafos verbo conjugado--preposición: \emph{bajo}, {\em cabe}, \emph{entre}, \emph{para}, \emph{sobre}. 

\item Homógrafos adjetivo--preposición: \emph{bajo}. 

\item Homógrafos sustantivo--preposición: \emph{ante}, \emph{sobre}. 

\item Homógrafos verbo conjugado--determinante: \emph{uno}, \emph{una}, \emph{unas} (\emph{unir}) 

\item Homógrafos verbo conjugado--adverbio: \emph{así} (\emph{asir}), \emph{fuera} (\emph{ser}, \emph{ir}), \emph{ arriba} (\emph{arribar}), {\em adelante} (\emph{adelantar}), \emph{cerca} (\emph{cercar}). 

\item Homógrafos adjetivo--adverbio: \emph{mucho}, \emph{poco}, {\em fuerte}\ldots 

\item Homógrafos sustantivo--adverbio: \emph{antes}, \emph{tanto}, \emph{mal}, \emph{bien}\ldots 

\item Homógrafos adjetivo--sustantivo: \emph{complejo}, {\em impreso}, \emph{derecho}\ldots 

\item Homógrafos determinante--pronombre \emph{la}, \emph{los}, \emph{las}, \emph{lo} (en ``lo que'', ``lo grande'') 

\item Otros homógrafos: \emph{como} (conjunción y forma de \emph{comer}), \emph{ora} (conjunción y forma de \emph{orar}), \emph{bien} (conjunción, sustantivo y adverbio) 

\end{enumerate} 

\subsection{Divergencias de traducción} 

Imaginemos que hemos podido segmentar el texto español y que hemos resuelto correctamente las ambigüedades léxicas; si todavía decidimos hacer la traducción palabra por palabra, nos encontraremos que hay ciertas construcciones para las cuales la traducción no es correcta, puesto que las palabras catalanas no se corresponden palabra por palabra con las españolas. Veamos cuáles son algunos de los problemas: 

\begin{description} \item[Concordancia de género y número:] A veces el género y el número de una palabra varían del español al catalán. La dificultad para un sistema de traducción automática aparece a la hora de propagar el género y el número del núcleo de un sintagma a los modificadores que tengan que concordar con él: \emph{su único amparo} $\rightarrow$ \emph{la seua única empara}; \emph{un buen postre} $\rightarrow$ \emph{unes bones postres}. Los problemas aumentan si la concordancia se debe producir entre sintagmas distantes: \emph{el calor producido por el motor ha resultado ser nefasto} $\rightarrow$ \emph{la calor produïda pel motor ha resultat ser nefasta}. Un problema similar lo presenta el establecimiento (opcional) de la concordancia del participio, inexistente en español en situaciones como por ejemplo \emph{todavía no la hemos estudiado con profundidad} $\rightarrow$ \emph{encara no l'hem estudiada amb profunditat}. \item[El artículo neutro:] El español posee el llamado {\em artículo neutro}, que no tiene correspondencia en catalán estándar (\emph{lo que me dijiste} $\rightarrow$ \emph{el que em vas dir}); presentan especial dificultad las construcciones usadas para expresar la abstracción o la intensidad: \emph{recibirá el informe lo más pronto posible} $\rightarrow$ \emph{recibirá el informe el més aviat possible}; \emph{me asusta lo grande que es} $\rightarrow$ \emph{m'espanta com és de gran}. \item[Los posesivos:] A veces, el catalán usa artículos determinados y construcciones con el pronombre débil \emph{en} donde el español usa posesivos: \emph{cuando hagas cosas así debes valorar sus consecuencias} $\rightarrow$ \emph{quan faces coses així n'has de valorar les conseqüències}. \item[Los relativos:] El principal problema aparece cuando se quieren traducir oraciones que contienen el relativo posesivo \emph{cuyo}, inexistente en catalán, donde lo más sencillo es usar una construcción con {\em qual}, que, además, presenta un esquema de concordancia diferente (\emph{qual} debe concordar con el antecedente, mientras {\em cuyo} concuerda con el nombre que lo sigue): \emph{el contribuyente cuyos informes hemos solicitado llegará tarde} $\rightarrow$ \emph{el contribuent els informes del qual hem sol·licitat arribarà tard} (véase el final del apartado~\ref{s3:transyn}). \item[Los pronombres débiles:] Los principales problemas se encuentran: en la traducción de \emph{lo}, ya que puede corresponder en catalán a alguna forma del pronombre masculino singular {\em lo} o a alguna forma del pronombre neutro \emph{ho}; en la traducción de \emph{se}, el cual corresponde normalmente al reflexivo catalán \emph{se} pero en las combinaciones españolas \emph{se la}, \emph{se lo}, etc. puede corresponder a veces a alguna forma de \emph{li} o \emph{els}, y en el hecho que el español no tiene equivalentes de los pronombres catalanes adverbiales \emph{en} y \emph{hi} (\emph{me $\emptyset$ dio uno} $\rightarrow$ \emph{me'n va donar un}); \emph{$\emptyset$ había dos salidas} $\rightarrow$ \emph{hi havia dues eixides}; \emph{no $\emptyset$ $\emptyset$ dejó una} $\rightarrow$ \emph{no n'hi va deixar cap}). \item[Régimen preposicional:] Hay diferencias notables entre los regímenes preposicionales español y catalán: las preposiciones españolas delante de \emph{que} completivo no aparecen en catalán (\emph{el hecho de que me hable} $\rightarrow$ \emph{el fet que em parle}); algunas preposiciones no son posibles en catalán delante de infinitivo (\emph{el juego consiste en ganar\ldots } $\rightarrow$ \emph{el joc consisteix a guanyar\ldots }), etc. \end{description} 

\section{Experiencias de TA español--catalán} \label{se:ETACC} 

En esta sección se describen brevemente cuatro experiencias de traducción automática del español al catalán: SALT, el traductor español--catalán de Lucy Software, el traductor de \textit{El Periódico de Cataluña} y Automatictrans, interNOSTRUM, y, con más detalle, Apertium. 

\subsection{SALT, de la Generalitat Valenciana} 

El programa SALT (la versión actual es la 4.0) lleva el nombre del antiguo \emph{Servicio de Asesoramiento Lingüístico y Traducción} de la Conselleria de Cultura, Educación y Ciencia (ahora Conselleria de Educación, Investigación, Cultura y Deporte) de la Generalitat Valenciana; se trata de un programa que se ejecuta en los sistemas operativos Windows, GNU/Linux y MacOS. El desarrollo del programa lo inició a finales de los noventa un equipo de programadores dirigido por Rafael Pinter bajo la dirección lingüística de Josep Lacreu, en aquel momento responsable de este servicio. Inicialmente, la disponibilidad del programa fue más bien reducida y su lanzamiento se retrasó por las discusiones en cuanto al estándar de valenciano que debía producir; actualmente se puede descargar gratuitamente de varios servidores de Internet\footnote{como por ejemplo \url{http://www.ceice.gva.es/polin/val/salt/apolin_salt4.htm} y  \url{https://www.softcatala.org/wiki/rebost:Salt}} y también lo distribuyen los servicios de normalización lingüística de algunas universidades. SALT 4.0 se ejecuta como una extensión de los procesadores de textos LibreOffice y Openoffice.org, traduce textos en español a la variante valenciana del catalán y está concebido también como una ayuda a las personas que quieren empezar a generar documentos en valenciano (entre otras herramientas, incluye diccionarios y guías de consulta completísimas). La Academia Valenciana de la Lengua declaró \emph{oficiales} ``los contenidos'' del programa SALT 2 (acuerdo de 20 de mayo del 2002). \footnote{En el año 2000, la empresa Autotrad de Valencia lanzó el programa Ara. El gerente de la empresa era Rafael Pinter, responsable informático de SALT. Ara era básicamente una versión bastante mejorada de la primera versión de SALT, con una apariencia muy similar pero con algunas diferencias: p.e., producía textos en catalán oriental estándar, podía dialogar con la persona usuaria en español y en catalán, y permitía programar tareas de traducción que se ejecutaban sin necesidad de que la persona usuaria las atendiera. El coste (en 2004) era de 45 euros por licencia. El sitio web de la empresa\footnote{\url{http://www.ara-autotrad.es}} no parece funcionar correctamente, y es posible que el programa ya no se esté comercializando.} 

\subsection{El traductor español--catalán de Lucy Software} 

El sistema de traducción automática español--catalán de Lucy Software, originalmente desarrollado por la empresa Incyta de Cornellà en colaboración con la Universitat Autònoma de Barcelona es un sistema de transferencia sintáctica estándar (véase el apartado~\ref{s3:transyn}), heredero del sistema METAL de la empresa Siemens. Su desarrollo fue pasando de una empresa a otra: en la actualidad lo desarrolla la empresa Lucy Software y lo distribuye la empresa Incyta, S.L. El programa se puede usar en Internet\footnote{\url{http://www.lucysoftware.com/catala/traduccio-automatica/kwik-translator-/}} y los resultados son de gran calidad. 

\subsection{El traductor de \emph{El Periódico de Cataluña} y AutomaticTrans} \label{ss:ePdC} 

Una experiencia interesante \citep{fiteperiodico} de traducción español--catalán para la diseminación es la edición bilingüe del diario \emph{El Periódico de Cataluña};\footnote{Disponible por Internet: \url{http://www.elperiodico.es}.} el texto original ---en español la mayor parte de las veces--- se traduce usando un sistema de traducción automática basado en corpus combinado con técnicas similares a las \emph{memorias de traducción} (véase el capítulo~\ref{se:memtrad}) y después es revisado por los redactores de cierre del mismo periódico antes de ser publicado. 

Un programa similar (y, según la información de la que disponemos, de origen común) al usado por \emph{El Periódico de Cataluña} se denominaba antes AutomaticTrans y ahora probablemente lo comercializa la empresa AT Language Solutions.\footnote{\url{https://www.at-languagesolutions.com/}} 

\subsection{interNOSTRUM} 

Un equipo de investigadores de la Universitat d'Alacant, financiado por la extinta Caja de Ahorros del Mediterráneo y por la misma Universidad, desarrolló entre 1998 y 2006 bajo la dirección de uno de los autores de este libro un sistema de traducción automática español--catalán llamado \textsf{interNOSTRUM} \citep{canals01a,canals01b}. El objetivo del proyecto era desarrollar un sistema de traducción automática del español a las variantes estándares del catalán y el sistema inverso correspondiente. Durante el último decenio, ha sido uno de los sistemas de traducción automática más usados en Internet. 

La versión actual de {\sf interNOSTRUM} (que estuvo accesible de forma gratuita a través de la URI \url{http://www.internostrum.com} y que en noviembre de 2016 todavía estaba accesible a través de la dirección \url{http://torsimany.ua.es/index.php}) genera, casi instantáneamente, borradores de traducciones al catalán listas para ser corregidas (posteditadas). 

{\sf interNOSTRUM} traduce textos en formatos ANSI, HTML y RTF del español al catalán oriental y a la inversa y permite la navegación traducida por Internet (es decir, permite la traducción instantánea de los documentos que se vayan visitando sin tener que invocar explícitamente el traductor). 

El traductor estaba escrito para ejecutarse sobre el sistema operativo GNU/Linux y es todavía accesible, como ya se ha dicho, a través de un servidor de Internet.\footnote{A pesar de que ya no se mantiene: también hay disponible una versión para servidores basados en el sistema operativo Windows.} Se trata de un sistema de transferencia morfológica avance como el descrito en el apartado~\ref{s3:STMorf}; el diseño del sistema es muy similar al del sistema Apertium que se describe más abajo en la sección~\ref{ss:apertium}, del cual es precursor. 

\subsection{Apertium} \label{ss:apertium} 

Apertium\footnote{\url{www.apertium.org}} \citep{forcada2011apertium}, iniciado en la Universitat d'Alacant en 2004, es una plataforma de software libre o de código fuente abierto que permite construir sistemas de traducción automática de transferencia morfológica avanzada (como los del apartado~\ref{s3:STMorf}); que sea libre o de código fuente abierto quiere decir que se puede descargar y copiar libremente pero también que programadores y lingüistas pueden modificar el software, los diccionarios, las reglas, etc. y distribuir versiones modificadas, dado que además del ejecutable del software que necesitamos para usarlo, se  distribuye el código fuente, es decir, la forma del software que permite a los expertos modificarlo. 

El primer sistema que se construyó sobre la plataforma Apertium fue el sistema español--catalán (en la actualidad hay disponibles en Apertium más de 40 sistemas de traducción automática diferentes). 

Apertium se puede usar gratuitamente en linea a través de muchas \emph{webs}\footnote{Por ejemplo: \url{http://www.apertium.org}, \url{http://apertium.ua.es}, \url{http://apertium.uoc.edu}, \url{http://politraductor.upv.es} y \url{http://aplica.prompsit.com}.}  pero también se puede instalar localmente. La versión completa ---por ejemplo para montar un servidor para un entorno de producción--- se instala sobre ordenadores con sistema operativo GNU/Linux, pero hay otras muchas versiones que funcionan sin necesidad de conexión a Internet, como por ejemplo: \begin{itemize} \item Una aplicación para el sistema operativo Android, \emph{Apertium offline translator};\footnote{\url{https://play.google.com/store/apps/details?id=org.apertium.android}} \item Una aplicación de sobremesa, \emph{apertium-caffeine}\footnote{\url{http://wiki.apertium.org/wiki/Apertium-Caffeine}} para GNU/Linux, Windows o MacOS (requiere que se haya instalado Java); \item Una extensión para el programa de traducción asistida OmegaT, denominada \emph{apertium-omegat}.\footnote{\url{http://wiki.apertium.org/wiki/Apertium-OmegaT}} \end{itemize} 

\begin{figure} {\scriptsize\sf \setlength{\tabcolsep}{0.5mm} \begin{center} texto LO \\ \(\downarrow\)\\ \framebox{Desformateador} \\ \(\downarrow\)\\ \framebox{\parbox{6cm}{ \begin{flushright} \vspace{-3ex} Análisis \vspace{-5ex} \end{flushright} \begin{center} \(\downarrow\)\\ \framebox{Analizador morfológico} \\ \(\downarrow\)\\ \framebox{Desambiguador categorial} \\ \(\downarrow\)\\ \vspace{-3ex} \end{center} }}\\ \(\downarrow\)\\ \framebox{\parbox{6cm}{ \begin{flushright} \vspace{-3ex} Transferencia \vspace{-5ex} \end{flushright} \begin{center} \(\downarrow\)\\ \framebox{Módulo de transferencia léxica} \\ \(\downarrow\)\\ \framebox{Módulo de selección léxica} \\ \(\downarrow\)\\ \framebox{Módulo de transferencia estructural} \\ \(\downarrow\) \vspace{-3ex} \end{center} }}\\ \(\downarrow\)\\ \framebox{\parbox{6cm}{ \begin{flushright} \vspace{-3ex} Generación \vspace{-5ex} \end{flushright} \begin{center} \(\downarrow\)\\ \framebox{Generador morfológico} \\ \(\downarrow\)\\ \framebox{Post-generador} \\ \(\downarrow\)\\ \vspace{-3ex} \end{center} }} \\ \(\downarrow\)\\ \framebox{Re-formateador} \\ \(\downarrow\)\\ texto LM \\ \end{center} } \caption{Arquitectura más común de los sistemas de traducción automática basados en Apertium.} \label{fg:modules} \end{figure} 

\paragraph{El diseño de Apertium} Como se ha dicho más arriba, los sistemas de traducción automática basados en la arquitectura Apertium son todos sistemas de transferencia morfológica avanzada. La arquitectura de un sistema de traducción concreto basado en Apertium es flexible: dependiendo de la lengua origen y la lengua meta, se pueden seleccionar módulos diferentes. La figura~\ref{fg:modules} representa la configuración más común de un sistema de traducción automática basado en Apertium: la construcción del texto de entrada se va haciendo etapa por etapa, como en la cadena de montaje de una fábrica de automóviles. El sistema español--catalán sigue la arquitectura de la figura~\ref{fg:modules} pero la versión disponible en 2016 no tiene (todavía) módulo de selección léxica. 

Apertium separa efectivamente el \emph{motor de traducción} (que consiste en módulos genéricos, comunes a todas los pares lengua origen--lengua meta) y los \emph{datos lingüísticos} específicos de un par lengua origen--lengua meta, tal como se discute en la p.~\pageref{pg:separacio}. Esto permite que las personas expertas que desarrollan datos lingüísticos (diccionarios, reglas) para un sistema determinado no tengan que preocuparse de cómo está programado el motor de traducción. 

Los siguientes párrafos describen algunos de los módulos con más detalle. 

\paragraph{Subprogramas basados en técnicas de estados finitos:} Los módulos de \emph{análisis morfológico}, \emph{transferencia léxica}, \emph{generación morfológica} y {\em postgeneración} están basados en \emph{transductores de estados finitos}, similares a los descritos en el cuadro ``para saber más'' del epígrafe~\ref{s3:anmor}. Esta tecnología permite velocidades de procesamiento del orden de 10.000 palabras por segundo en equipos estándares, velocidades que prácticamente no dependen del tamaño de los diccionarios. Los \emph{transductores de estados finitos} usados en Apertium leen la entrada símbolo a símbolo; cada vez que se lee una letra cambian de estado y van produciendo, también letra a letra, una o más salidas. \begin{description} \item[Analizador morfológico:] El analizador morfológico se genera automáticamente a partir de un \emph{diccionario morfológico} de la lengua origen (LO), el cual contiene los lemas, los paradigmas de flexión y las conexiones entre ellos. La entrada son las formas superficiales del texto y la salida, formas léxicas consistentes en lema, categoría léxica e información de flexión. \item[Transferencia léxica]: El subprograma de consulta del diccionario bilingüe se genera automáticamente a partir de un fichero que contiene las correspondencias bilingües. La entrada es la forma léxica de la LO y la salida, la forma léxica o formas léxicas correspondientes en la lengua meta (LM). \item[Generación morfológica:] El generador morfológico hace la operación inversa al analizador morfológico pero con formas de la LM y se genera automáticamente a partir de un diccionario morfológico de la LM. \item[Postgeneración:] Las formas superficiales que están implicadas en procesos de apostrofación y guionado (pronombres átonos, artículos, algunas preposiciones, etc.) activan este subprograma, que normalmente se encuentra inactivo. El postgenerador se genera a partir de reglas sencillas de apostrofación, guionado y combinación de pronombres átonos. \end{description} Cómo ya se ha discutido más arriba, la división de un texto en palabras presenta algunos aspectos no triviales; se mencionan dos: las \emph{locuciones} (o \emph{giros}) y los pronombres enclíticos. 

\paragraph{Locuciones y giros:} Hay numerosas locuciones y giros que se pueden tratar como \emph{unidades léxicas multipalabra} y que se van incorporando a los diccionarios morfológicos de las dos lenguas y al diccionario bilingüe: \begin{itemize} \item \emph{con cargo a} $\rightarrow$ \emph{a càrrec de} \item \emph{por adelantado} $\rightarrow$ \emph{per endavant}, \emph{a la bestreta} \item \emph{el abajo firmante} $\rightarrow$ \emph{el sotasignat} \item \emph{{\bf echar} de menos} $\rightarrow$ \emph{{\bf trobar} a faltar} \end{itemize} En el último ejemplo, el giro no es invariable sino que tiene un elemento que se flexiona (en negritas). 

\paragraph{Pronombres enclíticos:} El subprograma de análisis morfológico también es capaz de resolver las combinaciones de verbos y pronombres débiles enclíticos en español, las cuales presentan variaciones ortográficas como por ejemplo cambios de acentuación o pérdida de consonantes: \begin{itemize} \item \emph{d\'{a}melo} = \emph{da} $+$ \emph{me} $+$ \emph{lo} $\rightarrow$ \emph{dóna} $+$ \emph{me} $+$ \emph{lo} = \emph{dóna-me'l} \item \emph{pong\'{a}monos} = \emph{pongamos} $+$ \emph{nos} $\rightarrow$ {\em posem} $+$ \emph{nos} = \emph{posem-nos}. \end{itemize} 

El sistema Apertium trata estos dos problemas con el analizador morfológico, el cual es capaz de decidir cuando un grupo de palabras se tiene que tratar conjuntamente o por separado. 

\paragraph{El módulo de desambiguación léxica categorial:} Este programa se encarga de decidir, cuando el analizador morfológico entrega, para una palabra homógrafa, más de una forma léxica, cuál es la forma léxica más adecuada en el contexto. Los desambiguadores léxicos categoriales combinan (véase el apartado~\ref{s3:reshom}) reglas de base lingüística que permiten eliminar algunas formas léxicas y modelos estadísticos, entrenados sobre un corpus de referencia, que asignan una probabilidad a cada posible desambiguación de la frase que contiene palabras con ambigüedad categorial: la desambiguación más probable (la más verosímil) es la elegida. 

\paragraph{El módulo de transferencia estructural:} A pesar del gran parecido entre el español y el catalán, hay divergencias gramaticales considerables (véase el capítulo~\ref{se:PdTACC}): \begin{itemize} \item perífrasis modales: \emph{tienen que firmar} $\rightarrow$ \emph{han de firmar}; \item cambios de género y número: \emph{la deuda contraída} $\rightarrow$ \emph{el deute contret} (masc.); \item caída de preposiciones: \emph{la intención de que el cliente} $\rightarrow$ \emph{la intenció $\emptyset$ que el client}; \item construcciones relativas: \emph{la cuenta cuyo titular es} $\rightarrow$ \emph{el compte el titular del qual és}. \end{itemize} Estas divergencias se deben tratar con las reglas gramaticales oportunas, muy similares a las que se discuten en el apartado~\ref{s3:STMorf}: la solución se basa en la detección y el tratamiento de secuencias predefinidas de categorías léxicas (denominadas \emph{patrones}), es decir, un tipo de sintagmas rudimentarios, como por ejemplo {\bf art}--{\bf nom} o {\bf art}--{\bf nom}--{\bf adj}. Las secuencias consideradas por el módulo  forman el \emph{catálogo} de patrones. El funcionamiento del subprograma se basa en un esquema patrón--acción: \begin{itemize} \item Lee el texto (analizado y ya desambiguado) de izquierda a derecha, categoría léxica a categoría léxica. \item Busca, en la posición actual de la frase, el patrón más largo que concuerda con un patrón de su catálogo (por ejemplo, si en la posición actual se lee ``una señal inequívoca\ldots'', elige {\bf art}--{\bf nom}--{\bf adj} en vez de {\bf art}--{\bf nom}). \item Opera sobre este patrón (propagación de género y número, reordenamiento, cambios léxicos) siguiendo las reglas asociadas a él. \item Continúa inmediatamente detrás del patrón tratado (no vuelve a visitar las palabras sobre las cuales ha operado). \end{itemize} Cuando no se detecta ningún patrón en la posición actual, se traduce literalmente una palabra y se vuelve a iniciar el proceso. Los fenómenos ``a la larga'' como la concordancia sujeto--predicado son algo más difíciles de tratar; se usan variables de \emph{estado}, una especie de \emph{memoria } que recuerda cierta información a lo largo del proceso. 

El subprograma de tratamiento de patrones se genera automáticamente a partir de un fichero de reglas que especifica los patrones y las acciones asociadas. Este es muy probablemente el subprograma más lento (unos pocos miles de palabras por segundo). 

\section{Cuestiones y ejercicios} Estos ejercicios pueden servir para repasar los conceptos tratados en este apéndice. \begin{enumerate} \item¿Cuál de estas tres tareas es más difícil en un sistema de traducción automática español--catalán? \begin{enumerate} \item Decidir la traducción del pronombre español \emph{se} (puede ser \emph{se}, \emph{li} o \emph{els}). \item Detectar las formas de \emph{tener que} y traducirlas por \emph{haver de}. \item Hacer el análisis morfológico de verbos seguidos de enclíticos como por ejemplo \emph{estudiémonoslos} o \emph{dándoselo}. \end{enumerate} 

\item Indica cuál de estas tres es la fuente más importante de homografía (ambigüedad léxica categorial) del español: \begin{enumerate} \item Las coincidencias de algunas formas de algunos nombres y de algunos adjetivos con ciertas formas conjugadas de algunos verbos. \item Las coincidencias de algunas formas de nombres con preposiciones. \item Las coincidencias de algunas formas de nombres con adverbios. \end{enumerate} 

\item El catalán no tiene ninguna construcción equivalente al \emph{cuyo} español. En traducción automática del español al catalán, una alternativa interesante es poner primero el sintagma nominal que sigue al \emph{cuyo} y después, una forma de \emph{del qual} que concuerde con el antecedente. ¿Se puede hacer siempre correctamente esta operación en un sistema de traducción automática que no haga análisis sintáctico? \begin{enumerate} \item Sí, basta con hacer el análisis morfológico. \item No, porque hay que determinar bien la longitud del sintagma nominal que sigue a \emph{cuyo} para poder poner \emph{del qual} en la posición correcta. \item No, porque \emph{cuyo} no tiene un equivalente morfológico en español. \end{enumerate} 

% < QT12
\end{enumerate} 

\section{Soluciones} \begin{enumerate} \item (a) \item (a) \item (b) 

% < SL12
\end{enumerate} 