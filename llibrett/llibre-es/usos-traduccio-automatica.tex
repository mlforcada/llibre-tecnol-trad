\chapter[Traducción automática y aplicaciones]{La traducción automática y sus aplicaciones} \label{se:UTA} \label{se:TiTA} 

% dos etiquetes per si cal agarrar-ho tot.
Una de las aplicaciones más importantes de la informática a la traducción es la \emph{traducción automática} (TA). Pero antes de considerar la automatización de la traducción y sus aplicaciones sería bueno que nos paramos un poco para discutir qué quiere decir exactamente la palabra \emph{traducción}. Más adelante, en la sección~\ref{ss:humaut}, discutiremos sobre la relación entre traducción humana y traducción automática. 

\section{¿Qué es la traducción?} \label{ss:trad} 

Para empezar, se tiene que tener en cuenta que la palabra \emph{traducción} es ambigua\footnote{Como otros muchos sustantivos acabados en \emph{-ción}} porque se puede referir al \emph{proceso} de traducir o al \emph{producto} (resultado) de este proceso. 

\citet{sager93b}\footnote{Los conceptos de esta sección están tomados casi íntegramente de esta obra.} empieza su definición diciendo que, como proceso, se puede denominar traducción a ``un rango de actividades humanas deliberadas, que se hacen como resultado de instrucciones recibidas de un tercero, y que consisten en la producción de textos en una lengua meta (LM), basada, entre otras cosas, en la modificación de un texto en una lengua origen (LO) para hacerlo adecuado a un propósito nuevo'', pero todavía no describe la naturaleza de la modificación. 

Como producto, una \emph{traducción} se puede identificar como tal porque es un documento (en LM) derivado de otro documento en otro idioma (LO), y que mantiene una cierta similitud de contenido con éste. 

Se pueden decir todavía más cosas sobre la traducción: \begin{itemize} \item las traducciones suelen estar escritas en un sublenguaje particular (registro, especialidad, etc.) de la comunidad lingüística de la LM, basado en un sublenguaje paralelo de la LO; \item los documentos y las traducciones correspondientes se pueden clasificar en tipos\footnote{Por ejemplo, \emph{carta comercial}, \emph{edicto municipal}, \emph{comentario editorial}, \emph{manual técnico informático} o compilación \emph{de poemas}. } y esta tipología afecta a la traducción; \item la traducción se ve afectada por elementos extralingüísticos porque, normalmente, los documentos son entidades que unen la expresión lingüística con la no lingüística; \item las traducciones tienen un \emph{receptor} o \emph{lector}; una traducción, como acto comunicativo tiene que ser considerado, además de la intención de la traducción, las expectativas de los lectores, que resultan de su trasfondo cultural y de sus necesidades comunicativas, y que influyen en la recepción del texto traducido; \item la traducción siempre tiene una motivación: la superación de barreras comunicativas; por ello, se ha creado una profesión. \end{itemize} 

Se puede profundizar un poco más en la definición de traducción que hemos considerado más arriba, revisando definiciones existentes (algunas tomadas de \citealt{sager93b}): \begin{itemize} \item \citet[p.~19]{nida59b}: ``La traducción consiste en producir en la LM el equivalente natural más cercano del mensaje en la LO, primero en cuanto al significado y después en cuanto al estilo.'' Sager dice que, más \emph{que natural} (en sentido absoluto) habría que decir \emph{adecuado} (a la tarea concreta). Esta definición introduce dos de las tres dimensiones básicas de un documento escrito (original o traducido): el \emph{contenido} (significado) y la \emph{forma} (estilo), pero olvida el \emph{propósito}. \item \citet{flamand83b}: traducir es representar con precisión (fidelidad al autor) un mensaje en LO en una forma auténtica y correcta de la LM, adaptada al contenido y al receptor (fidelidad al lector)''. El problema de esta definición es la indefinición del concepto de \emph{fidelidad}. \item \citet{jakobson66b}: ``Traducción es la interpretación de signos verbales por medio de otra lengua''. Esta definición evita el concepto de \emph{equivalencia} e introduce el de \emph{interpretación} como conjunto de procesos cognitivos que tienen lugar en la mente del traductor. \item En el \emph{Diccionari de la Llengua Catalana}\footnote{Editorial Enciclopèdia Catalana, 7a ed., 1987-} se define {\em traducción} como la ``reproducción del contenido de un texto o de un enunciado oral, formulado en una lengua, en formas propias de otra lengua'' (y \emph{traducir} como ``escribir o decir en una lengua aquello que ha sido escrito o dicho en otra''). La definición incluye, por lo tanto, el tratamiento y la producción de mensajes no textuales (orales). \item \citet{alcaraz97b} definen traducción como ``expresión de un enunciado en la lengua de llegada [lengua meta] que sea equivalente al de la lengua de partida [lengua origen]''; queda por definir la noción de \emph{equivalencia}, que los mismos autores definen así: ``la posesión del mismo valor por parte de los enunciados de la lengua de partida y de la de llegada''; la equivalencia puede ser {\em semántica}, \emph{estilística} y \emph{textual}. \end{itemize} 

Para acabar este apartado, conviene mencionar algunos procesos a los que no llamaremos \emph{traducción} en el contexto de este libro: \begin{itemize} \item la adaptación de textos antiguos a la forma moderna de un idioma; \item la traducción de palabras y frases cuando se enseña un idioma nuevo; \item la interpretación (de mensajes hablados); \item la codificación (en Morse, etc.). \end{itemize} 

\section{Traducción automática} \label{ss:TA} 

%\subsection{Definició}
La traducción automática (TA) se puede definir como el proceso (o el producto) de traducir un texto informatizado\footnote{Llamaremos \emph{texto informatizado} a un fichero que contiene un texto codificado en un formato conocido (véase el capítulo~\ref{se:EPT}), y que puede ser editado con un editor o con un procesador de textos adecuado.} en una lengua origen a un texto informatizado en una lengua meta mediante el uso de un programa de ordenador. Normalmente se reserva la denominación \emph{traducción automática} para la completamente automática; cuando se produce intervención humana se habla de {\em traducción asistida por ordenador} o de \emph{traducción semiautomática}. El resultado de la traducción automática es normalmente un producto bastante diferente al de la traducción profesional, y en la mayoría de los casos no se puede usar en su lugar tal y como está; por eso, este capítulo está dedicado a analizar las diversas modalidades de interacción entre personas y máquinas en traducción. 

Es necesario hacer una aclaración sobre el tratamiento de los textos informatizados. Cuando los programas de traducción automática y semiautomática tienen que tratar documentos estructurados (como los discutidos en los epígrafes~\ref{s3:SGML} y \ref{s3:RTF}) tienen que ser capaces de identificar las partes de los documentos que corresponden a los textos que se tienen que traducir, separándolas de las etiquetas de formato. Normalmente, los programas tienen un módulo inicial que podríamos denominar \emph{desformateador} y un módulo final que podríamos denominar \emph{reformateador} y que restituye las etiquetas de formato de manera que el formato y la estructura del documento se conservan tanto como sea posible. En general, estas operaciones se pueden considerar básicamente independientes del mismo proceso de traducción ---como haremos en este libro---, pero hay programas más avanzados que son incluso capaces de usar la información de las etiquetas de formato como contexto para elegir una traducción cuando hay más de una alternativa. 

Las referencias que se han hecho en el epígrafe~\ref{ss:trad} a propósito o motivación de la traducción y a la tipología de los documentos que tienen que ser traducidos son también muy importantes a la hora de analizar la traducción automática. 

\paragraph{Sobre el nombre en otras lenguas.} En inglés, la traducción automática se denomina \emph{machine translation} y se abrevia MT, paralelamente al alemán, que usa la denominación \emph{maschinelle übersetzung}; en estas dos lenguas se expresa la noción de automatismo mediante la referencia a una {\em máquina}. En cambio, en francés se habla, como en catalán o en español, de {\em traduction automatique}. Otras lenguas, como por ejemplo el neerlandés, usan una palabra compuesta con la palabra \emph{ordenador}: \emph{Computervertaling}. 

\begin{persabermes}{la historia de la traducción automática} La mayor parte de lo discutido en este apartado se ha extraído de los trabajos de John Hutchins, especialmente de \cite{hutchins1995} y \cite{hutchins2001}. El Dr.\ John Hutchins es considerado el historiador de la traducción automática, y hasta hace poco mantenía activamente el archivo \url{www.mt-archive.info}, en donde se pueden encontrar reproducciones facsímiles de muchos artículos de los inicios de esta disciplina. 

\paragraph{Los pioneros, hasta 1954:} La traducción mediante máquinas es una ambición humana desde hace siglos que no se hizo realidad hasta el siglo XX. No hacía mucho que se había creado el primer ordenador, cuando ya se empezó a pensar en la posibilidad de usarlos para traducir lenguajes humanos. 

A pesar de que en el decenio que va de 1930 a 1940 hubo algunos trabajos precursores, es a principios de los años cincuenta cuando empieza realmente la investigación en TA en muchas universidades de todo el mundo, especialmente en los Estados Unidos de América. Los recursos de hardware, software y lenguajes de programación eran demasiado reducidos y la primera aproximación fue la traducción palabra por palabra basada en diccionario con algunas reglas sencillas de reordenamiento (a veces erróneamente llamada \emph{traducción directa}) y similar a los sistemas de traducción indirecta por transferencia morfológica avanzada (véase el apartado~\ref{s3:STMorf}). Esta carencia de recursos hizo que los primeros objetivos fueran muy modestos y, así, los primeros investigadores se concentraron en el desarrollo de lenguajes controlados (véase el apartado~\ref{ss:llecon}) y en la ayuda humana en tareas de preedición y postedición (véase~el apartado \ref{ss:preedposted}); estaba bastante claro que los sistemas reales no podrían producir más que traducciones de muy baja calidad. En 1952 se celebró en los Estados Unidos el primer congreso sobre TA donde se definieron las líneas fundamentales a seguir. 

\paragraph{El decenio del optimismo, 1954--1966:} La primera demostración pública de un sistema de TA fue desarrollada por IBM y la Universidad Georgetown en el año 1954. Se tradujo al inglés un conjunto de 49 frases en ruso usando un diccionario de nada más que 250 palabras y 6 reglas gramaticales; estas lenguas fueron elegidas por razones geopolíticas para los primeros sistemas de TA. A pesar de que los resultados no eran demasiados buenos, el público y la industria creyeron que en unos años se podrían conseguir traducciones automáticas de calidad de documentos científicos y técnicos. Esta idea se reforzó por el hecho de que empezaron a aparecer mejoras significativas en el hardware, los primeros lenguajes de programación y muchas mejoras en la lingüística formal (especialmente en el área de la sintaxis). El entusiasmo hizo que se financiaran un gran número de proyectos entre la mitad de la década de los 50 y la mitad de la década de los 60, proyectos dentro de los que nació la mayor parte de las técnicas actuales, como la traducción indirecta por transferencia o la traducción por interlingua (véase el capítulo \ref{se:TdTA}). 

El objetivo era el desarrollo de sistemas perfectos. Había que reducir al mínimo la intervención humana en el proceso de TA, hasta lograr la independencia total y una calidad comparable a la de los humanos. Prácticamente nadie consideró cómo se podría sacar provecho a un sistema imperfecto: ---con excepciones contadas: \cite{masterman67b} se estudió la utilidad de la traducción \emph{palabra por palabra} como \emph{pidgin}, es decir, como lengua de contacto, en comparación con una traducción \emph{nativa}. ¿Para qué pensar en ello si pronto se dispondría de sistemas perfectos? Los traductores se sintieron amenazados. Sin embargo, algunas voces se pronunciaron en contra del perfeccionismo dominante y defendieron una aproximación al problema a más a largo plazo y la construcción de sistemas que hicieran un uso efectivo de la interacción persona-máquina. 

Un decenio después, los avances eran escasos y el futuro próximo no parecía poder mejorar la situación. Muchos investigadores empezaban a encontrar barreras de todo tipo, especialmente semánticas, que parecían demasiado difíciles de superar y que exigían métodos más complejos. La Academia Nacional de las Ciencias de los Estados Unidos publicó en 1966 el informe ALPAC (Automatic Language Processing Advisory Committee) en el cual se recomendaba que los numerosos recursos que se dedicaban a la investigación en TA se utilizaran para tareas menos ambiciosas y más básicas relacionadas con el procesamiento del lenguaje natural y con el desarrollo de herramientas de apoyo para los traductores como por ejemplo diccionarios automáticos. La conclusión era que sólo después de conocer las raíces del problema, podría estudiarse la construcción de un sistema de TA real. El informe aseguraba que la TA era más lenta y menos exacta que la realizada por humanos, además de ser el doble de cara, y que no había ningún indicio de la obtención en el futuro más o menos inmediato de un sistema de TA útil. El informe hizo que se redujera significativamente el número de personas que se dedicaban a la TA y que los laboratorios empezaran a trabajar en lo que se conoció como lingüística computacional. 

\paragraph{Desde el informe ALPAC (1966) hasta los ochenta:} El informe acabó casi virtualmente con la investigación en TA en los Estados Unidos (también tuvo un impacto negativo en los proyectos desarrollados en el resto del mundo) y durante muchos años la TA fue vista como un auténtico fracaso. Aún así, algunos grupos continuaron trabajando en Canadá y en Europa y aparecieron los primeros sistemas que funcionaban; en 1970 el sistema Systran empezó a ser usado por la USAF (United States Air Force) y en 1976 por la Comisión de la Comunidad Europea. También en 1976 aparece Metéo, desarrollado por la Universidad de Montréal, que traduce al francés informes meteorológicos. Por esta época, además, los sistemas de TA empiezan a ser demandados por empresas y administraciones y no sólo para traducir textos científicos y técnicos. 

Desde el informe ALPAC el campo sufrió una redefinición progresiva hacia una concepción de la TA como un proceso en el cual los traductores humanos juegan un papel básico, y empiezan a desarrollarse herramientas de traducción pensando en esta intervención. 

Las principales corrientes dentro de la TA desde los años 70 son, por lo tanto: herramientas de apoyo a la traducción para traductores, sistemas de TA con intervención humana e investigación teórica hacia un sistema completamente automático de traducción. 

\paragraph{Principios de los ochenta:} En la década de 1980 aparecen nuevos sistemas de TA en todo el mundo con expectativas más reales y el interés en la TA resurge. Son especialmente importantes los resultados obtenidos en varias empresas como Xerox, donde se elimina casi por completo la postedición (véase la pag.~\pageref{pg:homografia}) gracias al control de la lengua origen (véase el apartado~\ref{ss:llecon}); esto permite la traducción sencilla de los manuales técnicos en inglés de la compañía a un gran número de idiomas (francés, alemán, italiano, español, portugués y lenguas escandinavas). 

Durante esta década los esfuerzos se centran en la traducción indirecta, con representaciones intermedias o sin ellas (como la interlingua; véase el apartado~\ref{ss:interlingua}), mediante análisis morfológicos y sintácticos y, a veces, conocimiento semántico. Los proyectos más notables son GETA-Ariane (Grenoble), SUSY (Saarbrücken), Mu (Kyoto), DLT (Utrecht), Rosetta (Eindhoven), el proyecto de la Universidad Carnegie-Mellon (Pittsburgh) y dos proyectos internacionales: Eurotra, financiado por la Comunidad Europea, y el proyecto japonés CICC, con participantes en China, Indonesia y Tailandia. 

Eurotra es uno de los proyectos de traducción más conocidos del decenio de 1980. Su objetivo era la construcción de un sistema de transferencia multilingüe que permitiera la traducción entre todas las lenguas de la Comunidad Europea. A pesar de que se esperaba que la traducción resultante fuera de gran calidad, todavía se preveía una gran cantidad de postedición. El proyecto, que se abandonó en el año 1992, no fue capaz de entregar un sistema de TA funcional, pero estimuló la investigación sobre tecnologías lingüísticas en toda Europa. 

En estos años se consolida la idea de que los sistemas de TA no son para traductores; un traductor necesita herramientas que le facilitan el trabajo: diccionarios, bases de datos terminológicas (véase el capítulo~\ref{se:basesdades}), sistemas de comunicación, memorias de traducción (véase el capítulo~\ref{se:memtrad}), etc. De hecho, actualmente, la postedición no se encarga siempre a traductores (muchos de los cuales no consideran esto como parte de su trabajo), sino a personas que se presume que tiene formación específica en postedición. 

Todo el mundo acepta ya en este decenio la importancia de los lenguajes controlados y los sublenguajes en la TA, como ya habían defendido los precursores de la TA durante el decenio de los cincuenta. 

El sistema comercial más sofisticado de los 80 es Metal (1988), financiado por Siemens y que traduce del alemán al inglés. Se trata básicamente de un sistema por transferencia, indicado para la traducción de documentos relacionados con el procesamiento de datos y las telecomunicaciones. 

A finales de la década de 1980 empieza la aplicación de técnicas de inteligencia artificial al procesamiento del lenguaje humano (sistemas expertos y sistemas basados en conocimiento diseñados para entender los textos). 

\paragraph{Principios de los noventa:} Todos los sistemas de TA de los ochenta, tanto los de transferencia como los de interlingua, funcionan básicamente a partir de reglas lingüísticas. Pero en la década de 1990 aparecen nuevas estrategias conocidas como métodos basados en corpus. Los métodos basados en corpus se pueden dividir en dos grupos: estadísticos y basados en ejemplos. 

Los métodos estadísticos ya fueron considerados en los años sesenta, pero pronto fueron descartados porque los resultados obtenidos no eran aceptables. Pero el descubrimiento de nuevas técnicas hizo posible proyectos como Candide, de IBM. Candide usa métodos estadísticos para el análisis y la generación, pero ninguna regla lingüística. Los trabajos en IBM utilizaron el corpus de textos en inglés y francés resultantes de las sesiones del Parlamento de Canadá. El método consiste en alinear en primer lugar las frases, los grupos de palabras y las palabras en los dos textos y calcular después la probabilidad que una palabra del texto origen corresponda a uno o más palabras del texto meta con la cual ha sido alineada. 

Los métodos basados en ejemplos (véase el capítulo~\ref{se:memtrad}) se aprovechan también de la existencia de grandes corpus de textos traducidos (por eso también se les llama basados en memoria). La idea fundamental es que el proceso de traducción se puede hacer a menudo consultando traducciones anteriores e identificando frases o grupos de palabras en el corpus ya traducido. Para poder llevar a cabo la traducción es necesario que los textos del corpus hayan sido alineados previamente (mediante métodos estadísticos o métodos basados en reglas). 

A pesar de que la gran innovación de los noventa fueron los métodos descritos antes, la investigación y el desarrollo de los sistemas clásicos también continuó: por ejemplo, el proyecto Eurolang, basado en el sistema de transferencia Metal, puede traducir del inglés al francés, alemán, italiano y español, y viceversa. En los últimos 10 años, uno de los campos con más investigaciones ha sido el de traducción del habla, una idea que evidentemente ha estado presente desde hace décadas, pero que sólo ahora se puede materializar parcialmente. El objetivo no es obtener un sistema de traducción perfecta, sino un sistema adecuado para aplicaciones con lenguajes, dominios y usuarios restringidos. Los principales son los desarrollados en ATR, CMU y el proyecto Verbmobil. 

Una característica importante de los inicios de la década de 1990 es la aparición de las primeras aplicaciones prácticas para traductores: herramientas de apoyo a la traducción, diccionarios y bases de datos terminológicas, procesadores de texto multilingües, acceso a glosarios y terminologías electrónicas, herramientas de comunicación (escáners, OCRs, Internet; véanse los capítulos~\ref{se:Internet} y~\ref{se:EPT}) o herramientas para entornos restringidos. La combinación de algunas de estas herramientas en un software concreto es el que se conoce como \emph{estaciones de trabajo para traductores}; por ejemplo, el Translation Manager de IBM, recientemente liberado como software libre/de código fuente abierto con el nombre OpenTM2, \url{http://www.opentm2.org}, o el Translator Workbench de Trados, ahora llamado SDL Trados Studio. La mayor parte de estas estaciones de trabajo están disponibles para ordenadores personales. 

\paragraph{Desde finales de los noventa hasta la actualidad:} La TA y las herramientas de apoyo a la traducción son cada vez más usadas por las grandes empresas y por las administraciones, principalmente para la traducción de documentación técnica. 

A lo largo de los últimos años, con la generalización del uso de Internet, se han desarrollado servicios de traducción disponibles en linea, como por ejemplo, Google Translate, \url{http://translate.google.com} o Bing Translator, \url{http://translator.bing.com}, de uso muy común por parte del público en general para la \emph{asimilación} (véase el apartado~\ref{s3:assim}) de contenidos web escritos en otras lenguas e incluso para la traducción de carteles y textos fotografiados con la cámara del teléfono móvil. 

Desde sus inicios, casi toda la investigación y casi todos los sistemas comerciales de TA se han centrado en los principales idiomas internacionales: inglés, francés, español, japonés, ruso, etc. Todavía queda mucho por hacer con las otras lenguas del mundo; con excepciones como el proyecto Apertium (\url{http://www.apertium.org}), que ofrece traducción automática para lenguas menos centrales, como por ejemplo el gallego, el occitano y el bretón. 

En el momento de escribir estas líneas (diciembre de 2015), la mayor parte de los sistemas de traducción automática se basan en una evolución de la \emph{traducción automática estadística} iniciada en IBM durante la década de 1990; se suele hablar de \emph{traducción automática estadística basada en segmentos}, en inglés \emph{phrase-based statistical machine translation}. Esta hegemonía se debe, en gran parte, a la disponibilidad de software libre/de código fuente abierto para \emph{entrenar} y aplicar sistemas de traducción automática, como por ejemplo Moses (\url{http://statmt.org/moses}). Incluso hay empresas como KantanMT (\url{http://kantanmt.com}) que construyen sistemas a medida para sus clientes usando simplemente un navegador. 

En los últimos años se está investigando una nueva modalidad de traducción basada en el llamado \emph{aprendizaje profundo}, en inglés \emph{deep learning}, que usa métodos de un campo de la inteligencia artificial denominado \emph{redes neuronales}, y los resultados empiezan a ser, en pruebas de laboratorio, comparables a los mejores disponibles. 

%\mbox{}% per evitar un problema de formatatge en el "persabermes".
\end{persabermes} 

\section{Utilidad de la traducción automática} \label{ss:UTA} La traducción automática produce resultados que normalmente no pueden sustituir directamente los producidos por profesionales de la traducción (véase el capítulo~\ref{se:ASTA}). Por ejemplo, en muchos casos es difícil conseguir que el ordenador sepa elegir la interpretación correcta entre las posibles interpretaciones de un enunciado ambiguo como \begin{center} \emph{Los soldados dispararon a los niños. Los vi caer.} \end{center} dado que esto requiere el uso de cantidades enormes de conocimiento enciclopédico sobre el funcionamiento del ``mundo real''. En este caso, el sistema tiene que saber: que los disparos hieren gravemente o matan a las personas que los reciben y que la condición de herido grave o muerto es incompatible con mantenerse derecho, y que, por todo esto, la interpretación más probable es que cayeron los niños, no los soldados. 

% Com a conseqüència de problemes com aquest i d'altres de
% naturalesa diversa, en moltes aplicacions, la traducció produïda per
% un bon programa s'ha de considerar com un esborrany que ha de ser
% revisat abans de la publicació.
% Però, per posar-nos en l'altre extrem, ha de quedar clar que la
% traducció automàtica pot ser molt útil en aquelles situacions en què
% la traducció completa per part de professionals de la traducció siga
% impracticable o impossible econòmicament. Algunes d'aquestes
% situacions són:
% \begin{itemize}
% \item La traducció automàtica de correu electrònic entre les persones
%   d'un grup de treball internacional amb la finalitat d'agilitzar les
%   comunicacions; la traducció immediata de documents durant la
%   \emph{navegació} per Internet (de fet, hi ha programes especialment
%   dissenyats per a aquesta finalitat, com ara \emph{Google Translate}
%   o \emph{Bing Translator}), o la traducció automàtica de converses
%   electròniques interactives (teclat i pantalla, \emph{chat}).
% \item La traducció i el manteniment de totes les versions de tots els
%   manuals tècnics d'una família de productes (per exemple, els manuals
%   de manteniment d'una gamma d'automòbils).
% \end{itemize}
Muchas de las aplicaciones de la traducción automática se pueden dividir en dos grandes grupos: la \emph{asimilación} de información (cuando una persona usa la traducción automática para obtener información a partir de un documento escrito en otra lengua) y la \emph{diseminación} ---también llamada \emph{difusión}--- de información (cuando una persona usa la traducción automática para producir documentos que tienen que ser distribuidos a más de un usuario). La traducción automática, a pesar de ser muy diferente de las traducciones hechas por profesionales competentes, puede ser una herramienta muy útil en estos dos grupos de aplicaciones. 

\subsection{Asimilación} \label{s3:assim} En situaciones de \emph{asimilación} de información no parece necesaria una traducción gramaticalmente correcta y similar al texto que produciría una persona nativa, sino más bien una traducción rápida y razonablemente inteligible. Se debe tener en cuenta que hay características de los textos nativos que pueden no ser necesarias para la comprensión. Por ejemplo, un texto puede ser inteligible aunque no concuerden los adjetivos con los nombres o incluso aunque se hayan eliminado los artículos (\emph{A amigo mio le gustan chicas mayores}), o el orden de las palabras no sea gramatical (\emph{La guerra evitar no podremos}).\footnote{De hecho, se podrían diseñar los sistemas de traducción automática para que no se \emph{preocuparan} por estos asuntos menores.} 

Una de las primeras aplicaciones de la traducción automática en los EE.UU. fue el llamado \emph{screening} o exploración de documentos para decidir cuáles eran relevantes y merecían una atención más detallada: se quería tener acceso a la información tecnológica presente en documentos de la Unión Soviética. Los usos civiles del \emph{screening} han superado actualmente el uso tradicional, el militar. En el caso del \emph{screening}, incluso una traducción incompleta además de incorrecta (por ejemplo, sólo de las palabras terminológicas) puede ser de gran utilidad. Otros ejemplos de uso de la traducción automática para la \emph{asimilación} son: \begin{itemize} \item La traducción automática del correo electrónico entre las personas de un grupo de trabajo internacional con la finalidad de agilizar las comunicaciones. \item La traducción inmediata de documentos durante la \emph{navegación} por Internet (de hecho, hay programas diseñados especialmente para esta finalidad, como \emph{Google Translate} o \emph{Bing Translator}). \item La traducción automática de \emph{conversaciones electrónicas} interactivas (usando el teclado y la pantalla de ordenadores conectados entre sí; \emph{chat}) entre personas que hablan dos idiomas diferentes. Las limitaciones de la traducción automática se pueden compensar con preguntas o diciendo las cosas de otro modo hasta que los dos interlocutores se entiendan (es decir, mediante una \emph{negociación}). \item La traducción de despachos de prensa en otros idiomas. \end{itemize} 

Es importante indicar que en casi todas las situaciones de asimilación el papel del traductor profesional es inexistente, dado que el trabajo es de naturaleza muy diferente, y el uso de un traductor profesional seria muy caro y muy lento. 

Por rudimentario que sea un sistema de traducción automática, puede ser muy útil en tareas de asimilación. Una de las aproximaciones más simples a la TA es la llamada \emph{traducción palabra por palabra}, en la que el programa identifica cada palabra, la busca en un diccionario bilingüe y la sustituye por una traducción aproximada (véase también la pág.\ref{pg:mpm}. A modo de ejemplo, considerad el siguiente texto en tok pisin\footnote{Lengua de contacto que se habla en Papúa Nueva Guinea y que tiene 50.000 hablantes que la hablan como primera lengua y más de dos millones de hablantes que la hablan como segunda lengua.} (el texto está tomado de \citealt{lyovin97b}): \begin{quote}{\sl Long taim bifo, wanpela ailan, draipela pik i save stap ya, na em i save kaikai ol man. Em i save kaikai ol man nau; wanpela taim, wanpela taim nau ol man go tokim bikpela man bilong ol, bos bilong ol, ol i go tokim em nau, em i tok: ``Orait yumi mas painim nupela ailan''. } \end{quote} Si cogemos un diccionario y traducimos el texto palabra por palabra, tomando la primera traducción posible en cada caso ---puede haber más de una---, se obtiene el texto siguiente:\footnote{Es posible que ya os hayáis dado cuenta de que el tok pisin tiene mucho vocabulario tomado del inglés, como lengua de contacto que es.} \begin{quote}{\sl En tiempo pasado, un isla, enorme cerdo - soler vivir mencionado y él - soler comer más-de un hombre. Él - soler comer más-de un hombre entonces; un tiempo, un tiempo entonces, más-de un hombre ir hablar gran hombre en más-de uno, ninguno en más-de uno, más-de un - ir hablar él entonces, él - decir: ``Muy-bien, vosotros-y-yo haber-de encontrar nueva isla''.} \end{quote} Y ahora, ¿verdad que se entiende un poquito más? Una traducción más idiomática podría ser: \begin{quote}{\sl Hace mucho tiempo, en una cierta isla, vivía un gran cerdo y se solía comer a la gente. Se solía comer a la gente, y una vez, la gente fue y dijo a su gran hombre, a su jefe, fue y hablaron con él. Él dijo: ``Muy bien, tenemos que encontrar una nueva isla''.} \end{quote} El orden de las palabras no es muy diferente en tok pisin y en español y esto hace que la traducción palabra por palabra sea basta leíble. En cambio, si el texto original está en vasco, las cosas no son tan sencillas. El texto, prácticamente ininteligible para quien no sepa vasco: \begin{quote}{\sl Baazkaria bukatu ondoren Koldo egunkarira joan zen eta Teoren foto bat hartu zuen. Gero, egunkariaren ale zaharrak irakurri zituen, boxeo txapelketako berriak aztertzeko. Boxealarien izenak apuntatu zituen.} \end{quote} se puede traducir palabra por palabra cómo: \begin{quote}{\sl La-comida acabada después Koldo al-diario ido era y de-Teo foto una tomado lo había. Después, del-diario número los-viejos leído los-había, boxeo del-campeonato las-noticias por-a-examinar. De los-boxeadores los nombres apuntado los-había.} \end{quote} que es mucho más difícil de leer que el resultado de traducir el texto en tok pisin palabra por palabra. Una traducción idiomática posible es: \begin{quote}{\sl Después de comer Koldo fue al diario y tomó una foto de Teo. Después, leyó [los] números viejos del diario para examinar las noticias del campeonato de boxeo. Apuntó los nombres de los boxeadores.} \end{quote} Fijaos que incluso en este caso tan desfavorable el texto traducido palabra por palabra da bastantes pistas sobre el significado del texto original. 

En los últimos años, especialmente desde que se ha generalizado el acceso público en Internet, se observa una tendencia a incorporar sistemas de traducción automática como uno de los componentes de sistemas más grandes de comunicación. Esta aplicación de la TA para la asimilación se puede ver \emph{en chats} bilingües, o en los sistemas que traducen las páginas \emph{web} según vamos visitándolas siguiendo enlaces; en estos sistemas, la TA no se invoca explícitamente, sino implícitamente cuando usamos el servicio. 

\subsection{Diseminación} En situaciones de \emph{diseminación} de la información hay que revisar el borrador de traducción producido por el traductor automático y hacer las modificaciones oportunas para convertirla en una traducción adecuada al propósito de las traducciones. Para minimizar las modificaciones a hacer en la traducción automática, puede ser útil restringir la lengua de origen (no permitir todas las realizaciones posibles, ni todo el léxico, ni todos los registros) a un lenguaje que pueda ser traducido automáticamente con el mínimo posible de problemas, es decir, con el mínimo esfuerzo de postedición, o al menos, con un esfuerzo aceptable para un revisor.\footnote{Es decir, cuando la revisión no es más costosa que rehacer toda la traducción a mano.} Esto es especialmente importante cuando se trata de traducir manuales técnicos a varios idiomas. Las restricciones se pueden expresar bajo la forma de mensajes interactivos dirigidos a la persona que prepara el documento original.\footnote{Véase el apartado~\ref{ss:llecon}, donde se discute un concepto muy relacionado, el de \emph{lenguaje controlado}.} 

La traducción automática para la diseminación es especialmente eficiente cuando sólo se traducen textos pertenecientes a una parte muy reducida y muy regulada del idioma en cuestión (un \emph{sublenguaje}). Un ejemplo es Méteo, el sistema que desde 1982 hasta 2001 producía en Canadá informes meteorológicos simultáneos en francés e inglés. 

%\section{CAT, HAMT i MAHT} 
\section{Traducción semiautomática} \label{se:cat} Muchas situaciones de traducción automática se pueden clasificar como situaciones de traducción asistida por ordenador (en inglés \emph{computer-aided translation}; CAT), también llamada a veces \emph{traducción semiautomática}. El término \emph{computer-aided translation} se usa normalmente para referirse al entorno de software que permite la traducción profesional con el apoyo de bases de datos léxicas (véase el epígrafe~\ref{ss:bdterm}), y de las sugerencias de traducción provenientes de memorias de traducción (véase el capítulo~\ref{se:memtrad}), e incluso, de la traducción automática. 

Para precisar mejor qué queremos decir con esto de ``asistida por ordenador'', es necesario considerar las nociones de traducción humana asistida por una máquina (en inglés \emph{machine-aided human translation}; MAHT), y traducción automática asistida por un humano (en inglés \emph{human-aided machine translation}; HAMT), que establecen las dos situaciones básicas de interacción entre una persona y un ordenador a la hora de hacer la traducción. Los párrafos siguientes dan algunos ejemplos. 

\paragraph{MAHT:} El usuario (un traductor competente o un profesional independiente) utiliza diccionarios bilingües, tesauros o \emph{thesaurus}, conjugadores y declinadores, correctores ortográficos, sintácticos y de estilo, y formularios o modelos de documentos, como ayuda mientras produce una traducción de manera manual usando un procesador de textos. Otras herramientas ---de uso común entre varios traductores, y accesibles normalmente como recursos remotos--- pueden ser las bases de datos terminológicas y las bases de datos léxicas multilingües (véase el epígrafe~\ref{ss:bdterm}), o las memorias de traducción (véase el capítulo~\ref{se:memtrad}). 

\paragraph{HAMT:} Un programa de traducción automática pregunta al usuario cuando tiene más de una posible traducción para una palabra o para una frase. Esta y otras situaciones de \emph{negociación} del texto original con el usuario del sistema implican una interacción que también puede ayudar a preparar un texto más correcto, es decir, a \emph{preeditarlo} (véase el apartado \ref{ss:preedposted}) para que pueda ser traducido automáticamente. En otras ocasiones, el programa puede analizar la estructura profunda de la frase y presentar las posibles interpretaciones al autor, para que resuelva alguna posible ambigüedad. En estos sistemas interactivos, hay que tener en cuenta dos factores: el primero, que un sistema que pregunta demasiado no es cómodo de usar (no es \emph{ergonómico}); y el segundo, que puede suceder que el usuario sea monolingüe, circunstancia que cambia mucho la naturaleza de la interacción entre el programa y el usuario. Los usuarios de este tipo de sistemas se podrían clasificar en tres grandes grupos: traductores ocasionales, traductores profesionales individuales y traductores profesionales que trabajan para empresas de traducción. 

\section{Automatización del proceso de traducción} \label{ss:preedposted} A la hora de abordar la automatización del proceso de traducción hay que hacer un análisis de los costes de traducción para estimar el ahorro en recursos (como por ejemplo tiempos y dinero) que se producirá con la introducción de la traducción automática. El capítulo \ref{se:ASTA} se centra en la evaluación de los sistemas de traducción automática y el análisis de costes de traducción; en este apartado discutiremos las diferentes tareas y opciones para automatizar el proceso de traducción. 

\subsection{Postedición} \label{ss:postedicio} La \emph{postedición} es la modificación \emph{mínima} de una traducción generada por ordenador para \emph{hacerla adecuada a un \textbf{propósito} bien definido}: el texto meta producido por el sistema se refina o revisa ({\em postedita}) para que sea gramaticalmente correcto o esté escrito de acuerdo con un registro determinado. 

A la hora de posteditar tenemos que evitar hacer cambios \emph{preferenciales} (esta solución adecuada ``me gusta más'' que esta otra que también es adecuada). Los cambios estilísticos se tienen que hacer estrictamente cuando, de no hacerse, la traducción resultante no cumpliría con el propósito para el cual fue encargada. Las modificaciones pueden ser: \emph{borrados} de una palabra que sobra, \emph{sustituciones} de una palabra por otra, o \emph{inserciones} de una palabra que falta. Tienen que ser las \emph{mínimas} necesarias: si hay más de una edición posible, hay que elegir la que se haga con el mínimo de modificaciones necesarias. 

Tenemos que tener en cuenta que la persona posteditora, además de conocer la lengua meta y ser capaz de convertir el texto en bruto a una forma genuina en esta lengua (es decir, además de ser profesional de la traducción), tiene que ser una verdadera especialista en postedición, que conoce el sistema de traducción automática y cuáles son los errores más típicos. Así, la tarea de postedición es mucho más eficiente, puesto que al conocer el origen y la causa de los errores se hace más fácil y rápida la corrección. 

En una primera aproximación, la postedición será conveniente cuando $$\textbf{coste}\left(\mbox{\begin{tabular}{c}traducción automática\\ +\\ postedición\end{tabular}}\right) < \textbf{coste}(\mbox{traducción profesional}). $$ \label{pg:cost} Hay que comprobar que la fórmula anterior se cumple, aunque sea a largo plazo, antes de elegir una estrategia de traducción para la diseminación basada en la postedición.\footnote{Para un análisis de costes más detallado, véase el apartado~\ref{ss:costdetall}.} 

\subsection{Preedición} \label{ss:preedicio} La \emph{preedición} consiste en preparar o adaptar (\emph{preeditar}) el texto origen para facilitar su traducción y mejorar el comportamiento del sistema de traducción automática, reduciendo la necesidad de postedición de la traducción en bruto. Esto se consigue, por ejemplo, eliminando la ambigüedad del texto,\footnote{Por ejemplo, en inglés técnico, la palabra \emph{replace} presenta una \emph{ambigüedad léxica} (véase el apartado~\ref{ss:amblex}), puesto que puede querer decir \emph{exchange} (reemplazar) o \emph{put back} (volver a colocar).}, evitando el uso de la voz pasiva, reduciendo el uso de oraciones subordinadas o usando frases cortas y completas sintácticamente y semánticamente.\footnote{\citet{kohl08} ofrece indicaciones para escribir textos en inglés para una audiencia global, de forma que los textos sean más fáciles de entender para los no nativos y más fáciles de traducir manualmente y automáticamente.} La preedición del texto origen se puede hacer también para marcar partes del texto que no tienen que ser traducidas, como por ejemplo una cita, o que tienen que ser tratadas de manera especial por no ser frases completas, como un título. 

La preedición suele ser tanto más conveniente cuanto mayor es el número de lenguas a las que se traduce el texto preeditado porque un cambio en el texto origen puede ahorrar tantas postediciones como lenguas de llegada tengamos. 

En resumen, hay tres modalidades básicas de interacción entre las personas y los programas de traducción automática: \begin{itemize} \item la preedición (preparación del texto \emph{antes} de la traducción automática), \item la postedición (corrección del texto \emph{después} de la traducción automática) y la interacción \item de la persona con el sistema de traducción automática durante el proceso de traducción. \end{itemize} 

\com{Tengo en la libreta un esquema en el que aparece todo el proceso de HAMT: preedición, postedición, interacción, etc. y vendría bien colocarlo aquí.} 

\subsection{Lenguajes controlados} \label{ss:llecon} Cuando la traducción automática se usa para la diseminación de documentos técnicos de temática homogénea, puede ser interesante hacer que los documentos originales estén escritos usando un léxico estándar sin ambigüedades semánticas y siguiendo unas reglas sintácticas y de estilo bien determinadas, es decir, en un \emph{lenguaje controlado} \citep{wojcik96uno,arnold94b,o2003controlling} diseñado de forma que el resultado de la traducción automática pueda ser usado directamente para publicarlo con el mínimo posible de postedición. 

Un \emph{lenguaje controlado} es ``un subconjunto del lenguaje natural definido con precisión, por un lado restringido en cuanto al léxico, a la gramática y al estilo, y por otro, posiblemente extendido con terminología y construcciones gramaticales específicas de un dominio'' \citep{huijsen98u}. 

Un lenguaje controlado tiene a menudo asociado un conjunto de programas de apoyo que ayudan a evaluar y escribir documentos que cumplan las restricciones. Quien escribe en un lenguaje controlado normalmente usa un editor de textos inteligente que realiza las siguientes tareas: \begin{itemize} \item Comprobar el cumplimiento de las restricciones: \begin{itemize} \item terminológicas (como el caso de la palabra \emph{replace} mencionado más arriba; para eso, puede ser útil acceder a una base de datos terminológica, como las mencionadas en el capítulo \ref{se:basesdades}); \item sintácticas (por ejemplo, haciendo el análisis sintáctico de las oraciones y detectando las ambigüedades estructurales, véase el apartado~\ref{ss:ambest}), y \item de estilo (por ejemplo, especificando cuál debe ser el formato de las fechas o de las horas). \end{itemize} \item Emitir un mensaje de error, lo más informativo posible, cuando se detecte una violación de las especificaciones del lenguaje. \item Proponer a la persona usuaria formas alternativas válidas al texto erróneo. \end{itemize} Como se puede ver, los desarrollos técnicos hechos alrededor del diseño de un lenguaje controlado se relacionan con muchos conceptos que se tratan en este libro. 

Un ejemplo histórico de lenguaje controlado que fue utilizado para mejorar los resultados de la traducción automática ---concretamente, los obtenidos con un sistema también histórico llamado Weidner MicroCat--- es PACE (\emph{Perkins Approved Clear English}), el lenguaje controlado usado durante los años ochenta y parte de los noventa por la empresa de ingeniería Perkins Engines para facilitar la traducción automática de los manuales que describen las características y el mantenimiento de estos motores \citep{newton92b,douglas96p}. Uno de los principios de PACE es ``una palabra, un significado", es decir, se establecen restricciones léxicas claras a través de un diccionario, cosa que simplifica el diseño de los diccionarios del sistema de traducción automática. Además del léxico, PACE también especifica la sintaxis \cite[secció 8.3]{arnold94b}. Otros ejemplos de lenguajes controlados son el \emph{ScaniaSwedish} usado por la firma de camiones y autobuses Scania \citep{almqvist96p}, o el {\em Caterpillar Technical English} de la compañía de maquinaria de excavación Caterpillar. 

También hay lenguajes controlados que no están específicamente diseñados para la traducción automática, como el inglés simplificado ({\em Simplified English}) del AECMA (Asociación Europea de Industrias Aeroespaciales), que se caracteriza por ``una sintaxis sencilla, un número limitado de palabras, un número limitado de significados bien definidos por palabra (normalmente uno), y un número limitado de categorías léxicas\footnote{Las \emph{categorías léxicas} (o simplemente \emph{categorías}) son conjuntos de palabras que tienen la misma función sintáctica; hay categorías \emph{mayores}, \emph{léxicas} o \emph{de clase abierta} (substantivo, adjetivo, verbo, etc.) que crecen cuando se añade léxico nuevo a la lengua y categorías \emph{menores}, \emph{gramaticales} o \emph{de clase cerrada} (artículos, conjunciones, etc.), que no crecen y contienen palabras con función gramatical. La sintaxis se define normalmente, no en términos de palabras, sino en términos de categorías léxicas.}\label{pg:catgra} por palabra (normalmente una)", con ``el objetivo de producir textos breves y no ambiguos" \citep{AECMA07u}. 

Algunas de las ventajas del uso de lenguajes controlados \citep{schwitten07uno} se pueden resumir como sigue: \begin{itemize} \item los textos son más sencillos e inteligibles; \item el mantenimiento de los documentos es más fácil; \item se simplifica el tratamiento computacional de los documentos, en particular la traducción automática. \end{itemize} En cuanto a las desventajas, podemos decir que: \begin{itemize} \item el diseño de un lenguaje controlado no es nada trivial: hay que estudiar con profundidad corpus de textos pertenecientes al dominio y tomar decisiones difíciles; \item el poder de expresión de un lenguaje controlado es siempre más restringido; \item la escritura de textos en un lenguaje controlado es más lenta; \item es necesaria una inversión adicional de tiempo en el aprendizaje del lenguaje controlado por parte de los autores. \end{itemize} Las dos últimas desventajas se pueden reducir si se dota a los autores de herramientas informáticas, como por ejemplo de un editor de textos inteligente que les ayude a escribir en el lenguaje controlado. 

Por último, hay que dejar claro que el uso de un lenguaje controlado es una alternativa a la preedición de los textos, pero que no elimina por completo la necesidad de postedición o, al menos, de revisión de las traducciones en bruto. 

\section{Cuestiones y ejercicios} \begin{enumerate} \item(*) Elegid un idioma cualquiera que conozcáis bien, $L$. Seguramente $L$ tenga palabras polisémicas que en otra lengua $L'$ tienen más de una traducción, según el sentido que se tome. Elegid tres palabras de $L$ que tengan este problema y describid cómo los trataríais en un lenguaje controlado basado en $L$. Las reglas que formuléis para los autores que escriban en el lenguaje controlado tienen que estar escritas en $L$ y no tienen que contener referencias a otras lenguas. 

\item En los sistemas de traducción automática, la preedición... \begin{enumerate} \item ... reduce la cantidad de postedición. \item ... es una alternativa a la postedición, que elimina completamente esta última fase. \item ... imposibilita el uso del sistema para tareas de diseminación de información. \end{enumerate} 

\item Indica en cuál de estas situaciones de traducción automática son menos cruciales la gramaticalidad o naturalidad lingüística de la traducción. \begin{enumerate} \item Joan usa Web Translator mientras navega por las páginas de Internet de la Universität Mainz para saber qué asignatura da el profesor Karl-Hans Lehninger y cuáles son sus intereses investigadores. \item Joan usa Web Translator para hacer una versión en alemán de su página Web. \item El personal de IBM traduce patentes europeas para detectar posibles avances en corrección de errores de comunicaciones digitales. \end{enumerate} 

\item Imaginad que podemos elegir entre dos sistemas de traducción automática diferente $t_A$ y $t_B$ para traducir manuales de televisores del inglés al francés, y que se tiene que diseñar un inglés controlado para minimizar la postedición. Las reglas del inglés controlado, ¿pueden depender del sistema de TA elegido? \begin{enumerate} \item No, porque los lenguajes controlados se tienen que diseñar independientemente de los sistemas de TA. \item Sí, porque en cada caso se tienen que evitar problemas diferentes. \item No, porque la lengua meta de los dos sistemas es la misma. \end{enumerate} 

\item Indica cuál de estas situaciones de traducción automática es de \emph{asimilación} de información: \begin{enumerate} \item Narciso usa el programa traductor del inglés al español Spanish Assistant para leer los documentos electrónicos que encuentra en Internet sobre la influencia del euskera sobre el gascón. \item Joan usa Web Translator para hacer una versión en alemán de su página Web antes de publicarla en Internet. \item La empresa Into the Wind traduce automáticamente su catálogo de dos tipos de cometas a varias lenguas. \end{enumerate} 

\item Muchas veces, la preedición la hace el autor cuando interacciona con el programa de traducción automática. ¿Es posible diseñar un sistema de preedición interactiva para autores monolingües? \begin{enumerate} \item Sí. \item No. Para preeditar correctamente hay que conocer el idioma de destino. \item Sólo para ciertos idiomas con estructura gramatical sencilla como el inglés. \end{enumerate} 

\item ¿Cuál de las siguientes \emph{no} es una ventaja de los lenguajes controlados? \begin{enumerate} \item Se evita la necesidad de que una persona interaccione con el programa de traducción automática para resolver ambigüedades durante la traducción. \item Los textos meta resultantes son mucho más cortos. \item Los textos origen se hacen más inteligibles. \end{enumerate} 

\item ¿Por qué es necesaria la preedición en los sistemas de traducción automática? \begin{enumerate} \item Para evitar construcciones o frases difíciles de traducir. \item Para que el formato quede más agradable a la vista. \item Es una alternativa a la postedición. \end{enumerate} 

\item Imaginad que un traductor profesional cobra 0,05 euros por palabra de texto traducido y que un corrector de textos cobra 0,10 euros por palabra de texto corregido. Imaginad que tenemos un sistema de traducción automática que nos cuesta unos 0,03 euros por palabra traducida y que produce un 10\% de palabras incorrectas en las traducciones. ¿Conviene adoptarlo y contratar el corrector o es mejor contratar el traductor profesional? (si no sabéis calcularlo en general, haced los cálculos con un texto de, por ejemplo, 1000 palabras). 

\item La traducción automática instantánea de páginas \emph{web} durante la navegación es un caso de traducción automática{\ldots} \begin{enumerate} \item {\ldots} con preedición. \item {\ldots} para la diseminación. \item {\ldots} para la asimilación. \end{enumerate} 

\item El ``control'' de los lenguajes controlados{\ldots} \begin{enumerate} \item {\ldots} se refiere tanto a la terminología como a la sintaxis. \item {\ldots} solo puede referirse a la sintaxis. \item {\ldots} sólo puede referirse a la terminología. \end{enumerate} 

\item Cuando se usan para la traducción, los lenguajes controlados restringen directamente{\ldots} \begin{enumerate} \item {\ldots} la lengua meta. \item {\ldots} la lengua origen. \item {\ldots} tanto la lengua origen como la lengua meta. \end{enumerate} 

\item Si un anglohablante usa el traductor automático--inglés de \verb|babelfish.altavista.com| para leer en línea el diario brasileño \emph{O Globo}, está usando la traducción automática para un propósito\ldots \begin{enumerate} \item \ldots de asimilación de información. \item \ldots de diseminación. \item \ldots para el que no está pensada. \end{enumerate} 

\item ¿Cuál es la alternativa estándar a la preedición en un en torno de producción masiva de documentación multilingüe? \begin{enumerate} \item El uso de un lenguaje controlado \item El uso de un sistema de interlingua. \item La postedición sistemática \end{enumerate} 

\item Fran consulta a través de Internet la base de datos terminológica IATE (véase el apartado~\ref{ss:bdterm}) cuando traduce dossieres antiglobalización del inglés al neerlandés. ¿En cuál de las tres situaciones siguientes se encuentra? \begin{enumerate} \item Traducción automática asistida por la persona \item Traducción humana asistida por la máquina \item Usa un lenguaje controlado \end{enumerate} 

\item Si enviamos un documento HTML a un servidor de traducción automática y después posteditamos el resultado para que sea una traducción aceptable del original antes de publicarla, estamos usando la traducción automática{\ldots} \begin{enumerate} \item {\ldots} con memoria de traducción. \item {\ldots} para la diseminación. \item {\ldots} para la asimilación. \end{enumerate} 

\item La adopción de un lenguaje controlado en una situación de traducción de documentos de una lengua a muchas lenguas para la diseminación es, en el proceso completo, una alternativa a la{\ldots} \begin{enumerate} \item {\ldots}postedición repetitiva de los documentos meta. \item {\ldots}la preedición repetitiva de los documentos origen. \item {\ldots}la traducción de fragmentos ya traducidos anteriormente. \end{enumerate} 

\item Un sistema que sugiere mejoras en el estilo de un documento se puede considerar como {\ldots} \begin{enumerate} \item {\ldots} HAMT. \item {\ldots} MAHT. \item {\ldots} un sistema de traducción automática ergonómico. \end{enumerate} 

\item Una inventora monolingüe consulta documentos web traducidos a su lengua para descubrir si su nuevo invento ha sido patentado antes. Si la traducción se hace mediante un sistema automático, ¿qué uso está haciendo? \begin{enumerate} \item Asimilación; más concretamente para aquello que se llama \emph{screening}. \item Diseminación. \item Postedición, ya que el idioma del documento cambia para que pueda ser entendido. \end{enumerate} 

\item El programa de la Generalitat Valenciana SALT 4.0 traduce textos del español a la variedad valenciana del catalán y pregunta esporádicamente a la persona usuaria qué equivalente es más adecuado para algunas palabras ambiguas difíciles. Esta es una situación de{\ldots} \begin{enumerate} \item {\ldots} postedición. \item {\ldots} traducción automática asistida por la persona. \item {\ldots} traducción humana asistida por el ordenador. \end{enumerate} 

\item Queremos posteditar un texto traducido automáticamente mirando tan poco como sea posible el texto original. ¿Nos ayuda conocer cuáles son las palabras homógrafas (véase la p.~\pageref{pg:homografia}) más comunes de la lengua origen? \begin{enumerate} \item No, porque las palabras homógrafas del texto origen no afectan al texto meta en bruto. \item No, porque solo estamos mirando el texto meta. \item Sí, porque son una fuente muy importante de errores especialmente difíciles de corregir si no se conoce qué ha pasado. \end{enumerate} 

\item Una persona está escribiendo un documento en lengua origen que después será traducido automáticamente a más de una lengua meta y el sistema que usa para escribir le avisa cuando teclea una palabra que dará problemas de traducción ---y le sugiere alternativas--- o cuando escribe una estructura que será difícil de traducir. Esta es una situación\ldots \begin{enumerate} \item \ldots de preedición. \item \ldots de aplicación de un lenguaje controlado. \item \ldots de postedición. \end{enumerate} 

\item ¿Cuál de las siguientes situaciones es absurda en traducción automática? \begin{enumerate} \item La postedición en una aplicación de asimilación. \item La postedición en una aplicación de diseminación. \item El uso de un lenguaje controlado en una aplicación de diseminación. \end{enumerate} 

\item Solo una de estas tres afirmaciones es correcta. ¿Cuál? \begin{enumerate} \item Los lenguajes controlados definen reglas de postedición. \item El uso de un lenguaje controlado elimina completamente la necesidad de postedición. \item Cuando se aplican las reglas de un lenguaje controlado, el texto resultante es gramaticalmente aceptable, pero se evitan construcciones y palabras que dan problemas. \end{enumerate} 

\item Un sistema de traducción automática hipotético del ruso al español produce texto que es básicamente correcto excepto por el hecho de que no genera ni artículos determinados (\emph{el}, \emph{la}, \emph{los}, \emph{las}) ni indeterminados (\emph{un}, \emph{una}, \emph{unos}, \emph{unas}). ¿Qué diríais de este sistema? \begin{enumerate} \item Que es especialmente adecuado para la asimilación, pero no tanto para la diseminación porque los artículos son más del 10\% del texto. \item Que es especialmente adecuado para la diseminación, ya que los artículos son palabras muy poco frecuentes en el texto y por lo tanto, no será necesaria mucha postedición. \item Que no es útil ni para la asimilación ni para la diseminación. \end{enumerate} 

\item ¿Se puede posteditar sin mirar el texto original? \begin{enumerate} \item Sí. \item En general, no. Quien postedita produce una traducción. Por lo tanto, tiene que estar seguro de que el resultado es traducción del texto original. \item Si el texto es técnico, se puede hacer sin mirar. En otro caso, hay que mirar siempre el texto original. \end{enumerate} 

\item El uso de un lenguaje controlado hace que \ldots\begin{enumerate} \item \ldots la escritura sea más rápida. \item \ldots el estilo del documento resultante sea más homogéneo. \item \ldots el poder de expresión del idioma sea más grande. \end{enumerate} 

\item Cuando posteditamos un texto encontramos una palabra que el traductor automático no ha sabido traducir y nos la ofrece en la lengua origen. Sin embargo, este error no ha afectado a la traducción del resto de la oración. ¿Qué debemos que hacer? \begin{enumerate} \item Preeditar el texto original completo sustituyendo la palabra por un sinónimo que sí reconozca el traductor automático y volver a traducir todo el texto. \item Probar a traducir todo el texto con otro traductor automático. \item Corregirla y seguir posteditando. \end{enumerate} 

\item Si en una fábrica de frigoríficos se usan sistemas de traducción automática para traducir a otras muchas lenguas los manuales de los numerosos modelos que se fabrican (y que son muy similares entre ellos), la solución más eficiente para evitar errores de traducción es{\ldots} \begin{enumerate} \item \ldots regular la manera en la que los autores escriben los manuales. \item \ldots posteditar todas las traducciones. \item \ldots preeditar los manuales antes de traducirlos. \end{enumerate} 

\item A la hora de preeditar un texto para traducirlo automáticamente conviene \ldots\begin{enumerate} \item \ldots usar frases cortas. \item \ldots usar la forma pasiva. \item \ldots usar oraciones subordinadas. \end{enumerate} 

\item La postedición de la traducción realizada por un traductor automático es siempre necesaria \ldots\begin{enumerate} \item \ldots para usarla con finalidades de diseminación. \item \ldots para usarla con finalidades de asimilación. \item\ldots cuando se ha realizado también preedición. \end{enumerate} \end{enumerate} 

\section{Soluciones} \begin{enumerate} \item(*) \label{pr:escondite} Por ejemplo, si $L$ es el español, palabras como {\em escondite} pueden referirse a un lugar donde esconderse (1) o a un juego (2) (en $L'$=catalán, \emph{amagatall} (1) y \emph{fet}, {\em amagar}, \emph{fet a amagar} o \emph{conillets a amagar} (2)). En el lenguaje controlado, se podría evitar el primer significado proponiendo a los autores que usaran la palabra alternativa {\em escondrijo}. Las reglas se podrían formular como sigue en español: {\sl \begin{description} \item[escondite] úsese sólo en el sentido de ``juego del escondite''; úsese \emph{escondrijo} si se quiere indicar el lugar donde se esconde alguna persona o cosa. \item[registro] úsese sólo en el sentido de ``transcripción'', ``inscripción'' u ``oficina de registro''; úsese \emph{inspección} cuando se refiera, por ejemplo a la investigación detallada de un local por parte de la policía. \item[explotar] úsese sólo en el sentido de ``aprovechar económicamente''; úsese \emph{estallar} en el sentido de ``deflagrar'' (una bomba, etc.) o ``reventar'' (un globo, etc.). \end{description} } \item (a) \item (a) \item (b) \item (a) 

% 5
\item (a). Las preguntas se pueden plantear como en el problema~\ref{pr:escondite}. \item (b) \item (a) \item

\begin{description} \item [Solución 1 ($n$ palabras):] El traductor profesional traduce un texto de $n$ palabras por 0,05$\times n$ euros. El sistema de traducción automática lo traduce por 0,03$\times n$ y corregirlo cuesta 0,10$\times($10$/$100$)\times n$, es decir, 0,01$\times n$ euros porque sólo 10 de cada 100 palabras son incorrectas. Por lo tanto, el sistema semiautomático cuesta sólo $($0,03$+$0,01$)\times n = $0,04$\times n$ euros, frente a los 0,05$\times n$ euros del traductor profesional. 

\item[Solución 2 (1000 palabras):] El traductor profesional traduce un texto de 1000 palabras por 0,05$\times$1000$=$50 euros. El sistema de traducción automática lo traduce por 0,03$\times$1000$=$30 euros, y corregirlo cuesta 10 euros, porque en 1000 palabras hay 1000$\times $10$/$100$=$100 palabras incorrectas y corregir cada una cuesta 0,10 euros: 0,10$\times$100$=$10. Por lo tanto, el sistema semiautomático cuesta sólo 40 euros, frente a los 50 del traductor profesional. \end{description} \item (c) 

% 10
\item (a) \item (b) \item (a) \item (a) \item (b) 

% 15
\item (b) \item (b) \item (b) \item (a) \item (b) 

% 20
\item (c) \item (b) \item (a) \item (c) \item (a) 

% 25
\item (b) \item (b) \item (c) \item (a) \item (a) 

% 30
\item (a) \end{enumerate} 

