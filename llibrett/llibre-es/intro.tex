\chapter{Introducción} 

Estas páginas cubren la mayor parte de los contenidos\footnote{Hay contenidos ---mejor dicho, habilidades--- que se aprenden como parte de las sesiones de laboratorio y que no figuran en este documento.} de la asignatura \emph{Tecnologías de la Traducción} que cursará el alumnado de segundo curso del grado en Traducción e Interpretación de la Universitat d'Alacant; también pueden ser útiles para asignaturas similares en otras universidades (por eso se  ha incluido material más avanzado que no se estudia en Tecnologías de la Traducción). La lectura de este manual ---que puede incluso contener algún error no detectado--- no puede nunca sustituir el estudio de otros libros sobre la materia, algunos de los cuales se citan en este texto y se listan en la bibliografía. 

Los contenidos de este manual se pueden dividir en dos partes: la primera presenta algunos conceptos básicos de la informática (capítulo~\ref{se:OiP}) y de Internet (capítulo~\ref{se:Internet}), sobre la entrada y el procesamiento de textos (capítulo~\ref{se:EPT}) y sobre las bases de datos (capítulo~\ref{se:basesdades}); la segunda es una introducción a algunos aspectos generales de la traducción automática (capítulos~\ref{se:TiTA} a \ref{se:ASTA}) y a la traducción asistida por ordenador con memorias de traducción (capítulo~\ref{se:memtrad}). Finalmente, un apéndice discute la problemática de la traducción español--catalán y algunos de los sistemas existentes para este par de lenguas; esta información puede servir como ilustración en un caso concreto de lo que se ha estudiado sobre traducción automática. Los contenidos de esta tercera parte son, por lo tanto, complementarios. 

Este manual se puede mejorar mucho, e iremos publicando versiones nuevas. 
%   Heus ací una llista d'algunes de les coses que sabem que ens queden
%   a fer:\todo{Repassar aquesta llista}
% \begin{itemize}
% \item Potser s'ha de millorar una miqueta el capítol que tracta sobre
%   conceptes bàsics de la informàtica.\footnote{En aquest sentit, volem
%     agrair les contribucions de Raül Canals i Marote, qui a més ha
%     trobat i corregit algunes errades del text.}  També hi falten
%   referències bibliogràfiques.
% \item La descripció dels conceptes de fitxer i directori és molt curta
%   i cal ampliar-la, amb gràfics si és possible.
% \item Cal explicar com funcionen els diversos tipus d'impressora.
% \item Faltaria una secció que descriga recursos per a traductors que
%   es poden trobar en Internet: diccionaris, traductors, lliçons
%   introductòries (\emph{tutorials}), etc.
% \com{Juan Antonio?}
% \item Cal explicar altres modalitats d'accés domèstic a Internet
%   (ADSL, cable). \com{En marxa}
% \item Hi ha capítols sobre temes de molt interés que encara són massa
%   curts (per exemple, els capítols~\ref{se:basesdades} i
%   \ref{se:memtrad}); en tots dos falten esquemes i gràfics que
%   expliquen millor els conceptes i els processos.
%   \jacom{Fet!}
% \item S'ha de millorar la descripció de les tècniques de resolució de
%   l'ambigüitat lèxica de transferència (polisèmia), tot donant un
%   exemple de xarxa semàntica.
% \item Potser cal millorar l'apartat sobre resolució de l'ambigüitat
%   estructural pura.
% \item Cal ampliar l'apartat sobre llenguatges controlats.
%   \jacom{L'he augmentat una mica.}
% \item Potser s'ha d'ampliar la descripció dels sistemes de traducció
%   automàtica basats en transferència semàntica i en interlingua.
% \item Cal fer més èmfasi sobre les aplicacions de codi obert i en
%   concret a la possibilitat d'una estació de treball de traducció
%   basada en elles: GNU/Linux (p.ex., Ubuntu Linux), OpenOffice.org,
%   Mozilla, el programa de traducció assistida OmegaT i l'alineador
%   bitext2tmx, etc.\
%\item Cal descriure la plataforma de traducció automàtica de codi
%  obert Apertium, i en tot cas explicar interNOSTRUM com a precursor.
% \item Cal actualitzar les descripcions de sistemes de traducció
%   automàtica espanyol--català que hi ha al final del llibre.
% \item Cal ampliar el nombre de qüestions i exercicis d'alguns
%   capítols (alguns, de fet, no en tenen), recuperant els d'exàmens,
%   etc.  \com{Aquesta faena està en marxa} \jacom{Feta!}
% \end{itemize}
Además, seguro que hay errores que deben corregirse. El texto está abierto, por supuesto, a sugerencias y a correcciones que lo hagan más útil, tanto para el alumnado de la asignatura como para otras personas que quieran saber sobre el tema. De hecho, aprovechamos para dar las gracias a todas las personas (alumnado, profesorado, etc.) que, con sus comentarios críticos, han ido mejorando este texto.\footnote{Este libro está basado en una obra anterior, \protect\citep{forcada09b}, usada para la licenciatura en Traducción e Interpretación.} Son demasiada gente para mencionarlos a todos, pero no queremos acabar sin agradecer las aportaciones de Raül Canals y Marote, que corrigió errores de versiones anteriores e hizo aportaciones en la parte de conceptos básicos de la informática, de Gema Ramírez Sánchez, particularmente en el capítulo de memorias de traducción, y de Sandra Montserrat, en la discusión sobre divergencias lingüísticas español--catalán del apéndice. 

Los ficheros fuente (\LaTeX, .eps, etc.) necesarios para volver a generar el libro están disponibles en un \emph{repositorio} público,\footnote{Repositorio GitHub: \url{https://github.com/mlforcada/llibre-tecnol-trad}} de forma que, si lo deseáis, los podéis modificar para generar un texto nuevo y publicarlo vosotros, pero siempre de acuerdo con las condiciones de la versión 3 de la Licencia General Pública de GNU\footnote{Descrita en \url{http://www.gnu.org/licenses/gpl-3.0.txt}} o de la licencia Creative Commons Reconocimiento-CompatirIgual 4.0 Internacional.\footnote{Descrita en \url{http://creativecommons.org/licenses/by-sa/4.0/}} Estas licencias os obligan a publicar cualquier trabajo derivado de este con la misma licencia. Así garantizamos que nuestro trabajo está siempre accesible para cualquier persona que lo considere útil para la enseñanza o el estudio personal. Si este es vuestro caso, os estaremos muy agradecidos si nos mandáis un mensaje de correo electrónico diciéndonos para qué asignatura lo estáis usando a \url{fsanchez@dlsi.ua.es}. 

Este manual es una traducción; podéis descargar la versión original en catalán desde \url{http://rua.ua.es/dspace/handle/10045/53085}; si detectáis alguna errata o error de traducción os estaremos muy agradecidos si nos lo hacéis saber mandando un mensaje de correo electrónico a \url{fsanchez@dlsi.ua.es}.

