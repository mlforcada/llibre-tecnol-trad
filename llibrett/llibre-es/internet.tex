\chapter{Internet} \label{se:Internet} \com{ Lista de cosas a hacer del capítulo: \begin{itemize} \item Definir \emph{buscadores} y \emph{directorios} y diferenciarlos claramente. Explicar que lo que se teclea en un buscador son condiciones y que sólo nos proporciona las páginas que hay en un índice. Decir que nos entrega páginas con listas de URI y una breve descripción (\emph{snippet}) del documento o de la parte del documento donde aparecen las palabras. \end{itemize} } 

Una de las herramientas informáticas básicas que se encuentran al alcance de las personas que se dedican profesionalmente a la traducción es Internet. Internet permite básicamente tres tipos de uso: \begin{description} \item[Como medio de comunicación:] Internet permite la comunicación y el intercambio de archivos (véase el apartado~\ref{se:fitxers}) con clientes o proveedores, la participación en foros de profesionales, la realización de consultas, etc. \item[Como fuente de documentación:] Además de contener textos de muchas clases que pueden servir de ejemplo o inspiración a la hora de hacer traducciones, se pueden encontrar enciclopedias, diccionarios, glosarios, memorias de traducción (véase el capítulo~\ref{se:memtrad}), y otras muchas fuentes de documentación. \item[Como repositorio de software de asistencia a la traducción:] Muchos de los programas específicos de asistencia a la traducción están disponibles en Internet, como por ejemplo los sistemas de traducción automática (véase el capítulo~\ref{se:TdTA}) en línea o los programas de concordancias bilingües.\footnote{Programas de concordancias bilingües disponibles en Internet: Linguee (\url{http://www.linguee.es/}); Reverso Contexto (\url{http://context.reverso.net/}).} El acceso puede ser a través de un navegador, o a través otros programas que tengamos instalados localmente en nuestro ordenador.\footnote{Usando protocolos bien especificados, normalmente a través de API, \emph{Application Program Interfaces} o \emph{interfaces de programación de aplicaciones}.} \end{description} 

\section{¿Qué es Internet?} 

Se denomina \emph{Internet} a un conjunto de ordenadores, distribuidos en todo el mundo e interconectados mediante un protocolo estándar (el \emph{protocolo de Internet} o IP) de forma que los recursos presentes en unos ordenadores (normalmente, información) están disponibles para ser usados por los usuarios de otros ordenadores. Se dice que los ordenadores de Internet forman una \emph{red}, en la cual los nodos o nudos son los ordenadores y los hilos, las conexiones. Las conexiones pueden ser de naturaleza muy diversa (líneas telefónicas, fibra óptica, enlaces de radio terrestres o por satélite, etc.), pero el protocolo de Internet está diseñado de forma que la naturaleza de la conexión no sea relevante para el usuario ni para los programas de aplicación que hacen uso de estas conexiones. Otros nombres que se usan en vez de \emph{Internet} son \emph{World Wide Web} o \emph{WWW} (``telaraña de alcance mundial'') o simplemente \emph{web} (``telaraña''), femenino en español (\emph{la web}). 

\section{Números IP} Cada nodo (cada ordenador) de la red Internet tiene un {\em número IP} único, el cual se compone de 4 bytes (4 enteros del 0 al 255) separados por puntos, como por ejemplo {\tt 192.168.5.5}. Los enteros iniciales se usan para designar grandes subredes, mientras que los finales se usan para designar redes más pequeñas, y dentro de éstas, ordenadores concretos (en esto recuerdan a los números de teléfono: dos abonados próximos normalmente comparten las cifras iniciales). 

\section{Nombres} Como recordar números IP no es fácil, normalmente se usan \emph{nombres} o \emph{direcciones} para referirse a las máquinas; algunos de los ordenadores de la red (llamados \emph{servidores de nombres}) se encargan de traducir los nombres a números IP. Por ejemplo, un nombre podría ser \texttt{altea.dlsi.ua.es}, donde \texttt{altea} se refiere a una máquina concreta del Departamento de Lenguajes y Sistemas Informáticos (\texttt{dlsi}) de la Universitat d'Alacant (\texttt{ua}), que se encuentra en España (\texttt{es}); este orden es el inverso al de los números IP (en esto los nombres se asemejan a las direcciones postales: primero se da el más concreto y al final el país). 

La tabla~\ref{tb:pais} da algunos ejemplos de indicativos de países. A veces, el último componente de un nombre no se corresponde con el indicativo de un país, sino que indica la naturaleza del lugar; antiguamente había que sobreentender que se trataba de un ordenador situado físicamente en los Estados Unidos de América, pero esto ya no es necesariamente así. Estos indicativos aparecen en la tabla~\ref{tb:tipus}. En otros países ({\tt .uk}, {\tt .nz}, {\tt .za}) se usan indicativos similares ({\tt .co}(mercial), {\tt .ac}(académico), etc.) antes del indicativo de país (por ejemplo, {\tt www.shef.ac.uk} es la Universidad de Sheffield). 

\begin{table} \begin{center} \begin{tabular}{l|l} \hline\hline {\sc Indicativo} &{\sc País} \\\hline {\tt .es} &España \\ {\tt .fr} &Francia \\ {\tt .pt} &Portugal \\ {\tt .it} &Italia \\ {\tt .uk} &Reino Unido \\ {\tt .ru} &Rusia \\ {\tt .za} &Suráfrica \\ {\tt .ie} &Irlanda \\ {\tt .tv} &Tuvalu \\ {\tt .to} &Tonga \\ {\tt .nu} &Niue \\ {\tt .fm} &Estados Federados de Micronesia \\ \hline

\end{tabular} \end{center} \caption{Indicativos de Internet de algunos países. Fijaos que algunos indicativos (\texttt{.tv}, \texttt{.fm}, etc.) se usan para aplicaciones no estrictamente relacionadas con estos países.} \label{tb:pais} \end{table} 

\begin{table} \begin{center} \begin{tabular}{l|l} \hline\hline {\sc Indicativo} &{\sc Tipo} \\\hline {\tt .gob} &gubernamental \\ {\tt .mil} &militar \\ {\tt .com} &comercial \\ {\tt .org} &organización sin ánimo de lucro \\ {\tt .edu} &institución educativa \\ {\tt .info} &webs informativas \\ {\tt .cat} &cultura y lengua catalanas (patrocinado por la fundación puntCat) \\ {\tt .eus} &cultura y lengua vascas (patrocinado por la fundación PuntuEus) \\ {\tt .museum} &museos (patrocinado por MuseDoma) \\ \hline\end{tabular} \end{center} \caption{Algunos indicativos de Internet usados originalmente en los Estados Unidos de América y más recientemente en todo el mundo, algunos de ellos patrocinados por determinadas instituciones.} \label{tb:tipus} \end{table} 

\begin{persabermes}{servidores de nombres} Podríamos hacer una analogía entre la relación entre los nombres y los números IP de los ordenadores de Internet y los nombres y los números de teléfono de la agenda de nuestro móvil. Cuando telefoneamos a una persona, normalmente lo hacemos buscando su nombre en la agenda, y pocas veces lo hacemos por el número, pero para hacer la llamada es necesario el número. Cuando accedemos a ordenadores de Internet, lo hacemos de manera similar: accedemos por el nombre y no por el número IP, el cual es indispensable para hacer la conexión. Pero, en contraste con la agenda de nuestro móvil, es impracticable tener todos los nombres y los números IP correspondientes a todos los ordenadores del mundo en nuestro ordenador. Por eso se usan \emph{servidores de nombres}: ordenadores a los cuales nuestro ordenador se conecta por el número IP y a los cuales puede preguntar por el número IP correspondiendo a un nombre. Los \emph{servidores de nombres} se organizan de forma que se distribuyen la información de manera jerárquica haciendo uso del \emph{sistema de nombres de dominio} (en inglés, \emph{domain name system} o DNS). 

Por ejemplo, cuando queremos conectarnos a \url{cercador.dlsi.ua.es}, el servidor de nombres de nuestro proveedor de acceso a Internet ve que el nombre acaba en \texttt{.es} y pregunta al servidor de nombres que se encarga de este \emph{dominio}; este servidor ve que el elemento anterior es \texttt{.ua} y pregunta al servidor de nombres de la Universitat d'Alacant (UA), y éste, a su vez, pregunta al servidor de nombres del Departamento de Lenguajes y Sistemas Informáticos, puesto que el elemento anterior es \texttt{.dlsi}. Este último, finalmente, entrega el número IP del ordenador llamado \texttt{cercador} al servidor de la UA, y éste al servidor del dominio geográfico \texttt{.es}, que lo entrega a nuestro proveedor de servicios de Internet y éste, a su vez, a nuestro ordenador para hacer la conexión. Por eso, cuando navegamos, la primera conexión tarda más: se está \emph{resolviendo} el nombre que hemos tecleado. Una vez resuelto, nuestro ordenador se guarda la IP durando un tiempo para evitar preguntar de nuevo. Además, para reducir el tráfico en Internet los proveedores de servicios de Internet y los servidores de nombres consultados también se guardan temporalmente esta información de forma que no siempre se desencadena el proceso completo de consultas descrito. \end{persabermes} 

\section{Identificadores de recursos} Los servicios y los documentos concretos presentes en un ordenador (un servidor de Internet) que los hace disponibles se pueden designar mediante su \emph{identificador uniforme de recursos} o, más comúnmente, \emph{URI} (del inglés \emph{uniform resource identifier}).\footnote{La denominación más usual era URL, \emph{uniform resource locator} o localizador uniforme de recursos, que todavía se usa profusamente, aunque no todos los URI son URL.} El URI es, por lo tanto, una expresión que identifica o localiza uniformemente un servicio o documento (un recurso) de cualquiera de los que se ofrecen en Internet. 

Un URI tiene generalmente tres partes, aunque se dan algunas variaciones y una de ellas no es obligatoria: \begin{description} \item[esquema:] indica la clase de recurso y cómo lo tiene que usar el ordenador solicitante (o \emph{cliente}). \item[autoridad:] identifica por su nombre o número IP el ordenador (\emph{servidor}) donde está el recurso. \item[trayectoria:] (opcional) da información sobre la localización del servicio o documento dentro del ordenador servidor (muchas veces similar a las \emph{trayectorias} de los ficheros, p.~\pageref{pg:fitxer}). \end{description} Por ejemplo, el URI \begin{center} \url{http://www.canalcuina.tv/concurs/sms/index.html} \end{center} se refiere a un documento de \emph{hipertexto} ---un documento de texto que contiene enlaces que permiten acceder directamente a otros hipertextos relacionados--- compatible con el esquema \texttt{http} (\emph{hypertext transfer protocol} o protocolo de transferencia de hipertextos) situado en el ordenador \texttt{www.canalcuina.tv} (de la empresa ficticia Canal Cocina, posiblemente perteneciente al mundo de la televisión\footnote{Aunque, como se muestra en la tabla~\ref{tb:pais}, el indicativo designa un estado del Pacífico denominado Tuvalu.}.), y, dentro de este, en el directorio \texttt{concurso}, subdirectorio \texttt{sms}. El fichero que contiene el hipertexto se denomina \texttt{index.html}, donde las siglas HTML corresponden a \emph{hypertext markup language}, nombre del lenguaje o sistema de marcas más usado para dar formato a los hipertextos (véase el apartado~\ref{ss:formats}). 

El esquema {\tt https://} es similar al esquema {\tt http://} pero incorpora, además, mecanismos para transmitir con seguridad información encriptada (cifrada). Muchos de los servidores de Internet encargados de manipular información privada usan este esquema. 

Los URI no sólo sirven para enlazar hipertextos: el URI \url{mailto:anton@dlsi.ua.es} sirve para enviar correo electrónico (\texttt{mailto}) al usuario que tiene la dirección de correo electrónico \url{anton@dlsi.ua.es}. Otros esquemas son \texttt{rtsp://}, \emph{real-time streaming protocolo}, para enlazar contenido como por ejemplo vídeo, audio, etc. en tiempo real, o {\tt ftp://}, \emph{file transfer protocol}, usado, cada vez menos, para descargar (transferir) ficheros para guardarlos en nuestro ordenador\label{pg:ftp}. 

\section{Navegadores} \label{ss:navegadors} Los programas navegadores se conocen también por otros nombres: \emph{browsers}, \emph{exploradores}, etc. (véase también la pág.~\pageref{pg:navegadors}). Son programas que permiten acceder de manera sencilla a los documentos o servicios de Internet en ordenadores conectados a esta red; entre otras cosas, los navegadores interpretan los hipertextos escritos en HTML y los presentan a la persona usuaria en el formato que indican las marcas, de forma que los enlaces a otros hipertextos queden claramente destacados y sean \emph{activos}, es decir, que respondan a un clic del ratón \emph{saltando} al hipertexto o recurso enlazado; además, los navegadores pueden \emph{ejecutar} automáticamente otros programas de aplicación para poder abrir el recurso correspondiente si no es un hipertexto. 

Los navegadores más usados son \emph{Firefox} (un programa libre y de código fuente abierto desarrollado por centenares de colaboradores en todo el mundo), \emph{Chrome} (el navegador de la compañía Google, el cual tiene una versión libre y de código fuente abierto denominada \emph{Chromium}), \emph{Microsoft Internet Explorer} (incorporado en el sistema operativo Windows), \emph{Safari} (el cual forma parte del sistema operativo MacOS), y otros como \emph{Opera}, etc. 

\com{No me gusta que el material de navegadores esté repartido y repetido entre dos capítulos. Unir? } 

\section{Buscadores} Uno de los recursos de Internet más útiles son los \emph{buscadores}. Se trata de páginas \emph{web} que permiten buscar documentos de Internet; se tiene que teclear una o más palabras, además de otras \emph{condiciones de búsqueda} opcionales, como por ejemplo que los documentos estén en una lengua determinada o en un servidor determinado, y entregan los URIs de los documentos que cumplen estas condiciones, enlaces a estos documentos y un pequeño resumen o recorte (denominado \emph{snippet}) del contenido de las páginas deseadas.\footnote{Algunos buscadores, como por ejemplo \emph{Google}, modifican los enlaces que llevan a los resultados, de forma que no llevan directamente sino que pasan primero por el servidor del buscador, para conocer las preferencias de la gente y mejorar así la relevancia de los resultados, o incluso para establecer un perfil de cada persona usuaria. Esto hace que algunas personas se planteen el uso de buscadores que no hagan este \emph{seguimiento} o \emph{tracking}.} 

Ejemplos de búsquedas: \begin{itemize} \item \texttt{megamondrío}: documentos que contengan la palabra \emph{megamondrío} y quizás también formas del mismo como por ejemplo el plural \emph{megamondríos}, el compuesto \emph{mega-mondrío}, o la forma sin acento \emph{megamondrio}. \item \texttt{megamondrío síngulo}: documentos que contengan estas dos palabras o variantes. \item \texttt{megamondrío síngulo site:ua.es}: documentos que contengan estas dos palabras o variantes y que estén en documentos cuyo URI acabe en \emph{ua.es}. \item \texttt{megamondrío síngulo filetype:pdf}: documentos PDF que contengan estas dos palabras. \end{itemize} 

Tiene que quedar claro que los buscadores realmente \emph{no buscan} documentos en Internet sino que consultan \emph{índices} que han ido construyendo a partir de los documentos visitados. Por lo tanto, puede haber documentos que los buscadores no encuentren porque nunca los han visitado. Por la misma razón, también puede pasar que los buscadores entregan resultados correspondientes a páginas que ya no existen. 

Uno de los buscadores más populares a la hora de escribir estas líneas es \emph{Google} (\url{http://www.google.com}); pero también hay otros como \emph{Duckduckgo!}, \emph{StartPage}, etc. La mayor parte de estos buscadores tienen interfaces de uso en muchas lenguas. 

\section{Correo electrónico} \label{ss:correue} Uno de los servicios más usados de Internet es el correo electrónico (en inglés \emph{electronic mail} o \emph{e-mail}), que nos permite enviar mensajes (textos informatizados) a usuarios de otros ordenadores. Los mensajes pueden contener, además del texto mismo del mensaje, ficheros \emph{anexos} (o \emph{adjuntos}, en inglés \emph{attachments}) como por ejemplo imágenes, documentos, mensajes reenviados, etc. 

Las direcciones de correo electrónico tienen dos partes, separadas por el carácter ``\texttt{@}'', que se suele pronunciar \emph{at} (en inglés, por parte de los informáticos más viejos) o \emph{arroba}. La primera parte (o \emph{parte local}) es frecuentemente el identificador de una persona y la segunda (la \emph{parte de dominio}) suele ser el nombre de un ordenador (o de un grupo de ordenadores que comparten un mismo nombre). Por ejemplo, una dirección electrónica válida podría ser \begin{center} \texttt{marty.mcfly@backtothefuture.info} \end{center} A veces, podemos usar nuestra dirección de correo electrónico para identificarnos a la hora de acceder a algunos de los servicios que se ofrecen en Internet, como por ejemplo los servicios de redes sociales (véase el apartado~\ref{ss:xarsessocials}). 

Una dirección electrónica puede también identificar una lista de usuarios (\emph{alias}) o una lista de distribución (la cual envía una copia de cada mensaje que recibe a todos los inscritos en la lista). En ambos casos, si mandamos un mensaje, lo reciben todos los inscritos, de forma que se puede usar para establecer, por ejemplo, foros de discusión.\footnote{Por ejemplo, la lista de distribución sobre traducción automática MT-List, mantenida por la EAMT (\emph{European Association for Machine Translation}, asociación europea para la traducción automática), tiene la dirección \texttt{mt-list@eamt.org}; para formar parte de la lista hay que subscribirse en la URI \url{http://lists.eamt.org/mailman/listinfo/mt-list}. Si se manda un mensaje a \texttt{mt-list@eamt.org} lo reciben todos los subscritos. Otra lista de interés, Tradumàtica, sobre tecnologías de la traducción, permite suscripción a través de \url{https://listserv.rediris.es/cgi-bin/wa?A0=TRADUMATICA}.} 

% \com{No sé si mereix la pena dir alguna cosa sobre l'abús del mot
% \emph{e-mail} en sentits com \emph{adreça}, \emph{programa} o
% \emph{missatge} quan es refereix al \emph{servei}. Ho faig a classe
% cada any però no sé si ací...}
Para leer, escribir o enviar los mensajes de correo electrónico, se usan \emph{programas gestores de correo electrónico}, como por ejemplo Thunderbird, Outlook, etc. También es muy frecuente acceder al correo electrónico desde cualquier lugar usando un navegador, a través de un servicio llamado \emph{webmail}; son comunes los \emph{webmails} gratuitos (GMail, Yahoo Mail, Microsoft Outlook); cada alumna o alumno de la Universitat d'Alacant tiene, por serlo, una dirección de correo de la forma \emph{xxxx}\texttt{@alu.ua.es}, accesible a través de la sección \emph{webmail} de la página web de la Universidad. 

\section{Mensajería instantánea y chat} \label{ss:missinst} La mensajería instantánea y el chat (en inglés \emph{chat}) permiten ---como en el caso del correo electrónico, a través de programas especializados o \emph{webs} accesibles con un navegador--- una comunicación escrita muy rápida (``en tiempo real''), consistente en mensajes normalmente cortos ---opcionalmente con anexos como por ejemplo fotografías, contactos---, de forma que el resultado es similar al de una conversación, pero por escrito, \footnote{Aunque se está popularizando el uso de \emph{notas de voz}, archivos de audio grabados y que se envían como anexos.} cosa que permite un registro de comunicación muy informal que, de hecho, ha dado lugar a una lengua muy diferenciada tanto de la oral como de la escrita. 

Con la generalización del uso de los móviles inteligentes o \emph{smartphones}, han aparecido muchas aplicaciones de este tipo, como por ejemplo Telegram, Whatsapp, Line, etc., que usan como identificador el número de teléfono. 

Las conversaciones pueden ser entre dos personas, o entre un \emph{grupo} de personas, a veces reunidos en una \emph{sala}, con intereses comunes. Las personas que participan en un \emph{chat} de estos últimos, pueden a veces elegir un alias o mote y ``entrar'' a la sala, o estar siempre conectadas al grupo, y ``conversar'' por escrito públicamente. Desde el grupo o la sala se pueden establecer ``conversaciones aparte'' (en privado) cuando hace falta con alguna persona concreta. 

Entre los servicios de mensajería instantánea comercial más populares están los asociados a redes sociales como por ejemplo Facebook o Tuenti, u otros asociados a otras aplicaciones como por ejemplo Google Hangouts o Skype. 

\section{Servicios de red social} \label{ss:xarsessocials} En la actualidad, una de las aplicaciones más frecuentes de Internet son los \emph{servicios de red social}, comúnmente denominados simplemente \emph{redes sociales}. Son plataformas informáticas que usan Internet (a través de un navegador y frecuentemente también a través de programas de aplicación específicos, muy populares para teléfonos móviles) para construir redes de personas que comparten intereses u objetivos. Algunos ejemplos: \begin{description} \item[Facebook] permite a cada persona publicar informaciones sobre su \emph{estado}, incluyendo fotos, y enviarse mensajes; el estado puede ser visible para todo el mundo o sólo para personas \emph{amigas}. \item[Google+] se puede ver como la respuesta de Google a Facebook; en Google+ el concepto básico es el del \emph{círculo}. \item[Twitter] es una red social que se basa en mensajes de menos de 140 caracteres que pueden llevar adjuntos fotos, enlaces, etc. \item[Instagram] hace énfasis en la posibilidad de compartir fotografías y vídeos. \item[LinkedIn] sirve para construir redes relacionadas con la actividad profesional. \end{description} Hay otros muchos de alcance mundial (Pinterest, Reddit, Tumblr, etc.) y algunos particulares de determinadas áreas geográficas (como por ejemplo VK en los países donde se habla ruso). 

% \section{\emph{News}}
% Aquest és un servei similar a les llistes de distribució i, per
% tant, també serveix per a constituir grups de discussió o de
% notícies (\emph{newsgroups}) sobre un tema; els missatges que s'hi
% envien hi queden guardats i qui vulga saber si hi ha novetats en un
% grup de discussió determinat només s'ha de connectar al
% \emph{servidor de notícies} més pròxim, seleccionar el grup i llegir
% els missatges nous. Els grups de notícies tenen noms compostos per
% diversos mots separats per punts, com ara {\tt comp.ai.neural-nets}.
\section{El acceso en Internet} \label{ss:adaI} 

\subsection{Acceso doméstico} Para acceder a Internet desde casa, hace falta, por un lado, darse de alta con algún proveedor de servicios de Internet (ISP, \emph{Internet service provider}) ---algunos proveedores ofrecen, además del acceso a Internet, televisión digital y telefonía convencional--- y de otro, tener un \emph{módem} adecuado al tipo de conexión (módem ADSL, módem de cable, etc.). Normalmente los módems que nos venden los proveedores de servicios de Internet son al mismo tiempo módem y encaminador (en inglés \emph{router}) para permitir que conectemos más de un dispositivo, normalmente a través de una conexión inalámbrica Wi-Fi, formando una red local (en inglés \emph{local area network}, LAN). La figura \ref{fg:accesinternet} muestra un esquema del acceso doméstico a Internet desde varios dispositivos. 

\begin{figure} \centering

\includegraphics[scale=0.35]{connex-internet} \caption{Esquema de acceso doméstico a Internet desde varios dispositivos conectados a un módem-encaminador.} \label{fg:accesinternet} \end{figure} 

Nuestro proveedor de servicios de Internet asigna un número IP público a nuestro encaminador (\emph{router}), de forma que forme parte de Internet y pueda facilitar el acceso a todos los dispositivos de nuestra red local (clientes) a todos los servicios y documentos disponibles en cualquier máquina (servidora) de Internet. A los dispositivos de nuestra red local el encaminador les asigna un número IP privado al que sólo tienen acceso los ordenadores que forman parte de esta red local; todos los dispositivos de esta red local se conectan a Internet usando el mismo número IP, el número IP público asignado a nuestro encaminador. 

En la mayoría de los casos, el número IP público que nuestro proveedor de acceso a Internet asigna a nuestro encaminador es temporal y va cambiando, por eso los ordenadores que tenemos en casa no pueden actuar como servidores. 

En la actualidad, en los domicilios del País Valencià hay básicamente tres modalidades de acceso a Internet, todas por precios que oscilan alrededor de los 30--40 euros/mes: \begin{description} \item[ADSL:] (véase el glosario de la sección~\ref{ss:OiPgloss}) el módem hace la conexión a través de los hilos de telefonía convencional ya instalados en las casas; en la actualidad se consiguen velocidades de bajada de algunos Mb/s y de subida normalmente inferiores a 1 Mb/s. \item[Cable:] el módem hace la conexión a través de un cable coaxial, tecnología que se usaba tradicionalmente para televisión, con velocidades y precios similares al ADSL. \item[Fibra:] el módem hace la conexión a través de un cable de fibra óptica (láser); esta tecnología es la más reciente y permite velocidades de bajada de decenas o incluso centenares de Mb/s y velocidades de subida superiores al Mb/s. \end{description} 

\begin{persabermes}{los antiguos módems telefónicos} Hasta los primeros años del decenio del 2000, la mayor parte de los domicilios particulares y las pequeñas empresas se conectaban a Internet mediante la línea telefónica, pero usando una tecnología mucho más rudimentaria que requería hacer una llamada telefónica al número del proveedor de acceso a Internet, llamada que tenía que durar el tiempo de conexión, independientemente de la cantidad de datos que se transfirieran. La llamada se podía pagar por minutos o con planes que permitían conexiones ilimitadas en horario no comercial que se denominaban \emph{tarifa plana}. El módem modulaba y desmodulaba señales similares a las que se envían cuando se hace una llamada de voz. Durante la conexión, la línea quedaba ocupada y no se podían hacer ni recibir llamadas. Las velocidades eran muy bajas, de decenas de kb/s. \end{persabermes} 

\subsection{Acceso móvil} Cada vez nos conectamos menos a Internet desde casa, y más desde nuestros dispositivos móviles como por ejemplo los \emph{smartphones}. Si estamos en el radio de alcance de una red Wi-Fi a la cual tenemos acceso (como la de nuestra casa o la que nos ofrece la Universidad), nuestro dispositivo móvil se conectará en Internet a través de Wi-Fi. Si no, tendremos que usar los \emph{datos móviles} que comercializa nuestro proveedor de telefonía móvil a través de su red celular. En la actualidad se está haciendo la transición de la tecnología llamada de tercera generación o 3G (que se muestra a veces también como una ``H'' en la pantalla), que permite velocidades de conexión de unos pocos Mb/s a la de cuarta generación o 4G, que permite velocidades mucho más rápidas (a veces, cuando la cobertura no es buena, nuestro móvil recurre a tecnologías más antiguas y más lentas, como por ejemplo EDGE\footnote{\emph{Enhanced Data Rates for GSM Evolution} ``Tasas de datos mejoradas para la evolución del GSM''} ---que permite centenares de kb/s y se muestra como una ``E'' en la pantalla--- o GPRS\footnote{\emph{General Packet Radio Service} ``Servicio general de paquetes [de datos] por radio''} ---que permite decenas de kb/s y se muestra como una ``G'' en la pantalla). En la actualidad, podemos comprar paquetes de datos y pagar unos 5--7 euros por GB, o una cuota mensual de decenas de euros y tener datos ilimitados. 

\section{Cuestiones y ejercicios} \begin{enumerate} 

\item La primera parte de un URI (identificador uniforme de recursos) especifica \ldots\begin{enumerate} \item \ldots el esquema de acceso. \item \ldots el nombre del servidor. \item \ldots el directorio donde se encuentra el servicio. \end{enumerate} 

\item Después del esquema de acceso, un URI (identificador uniforme de recursos) especifica \begin{enumerate} \item la velocidad de transferencia. \item el nombre del servidor. \item el directorio donde se encuentra el servicio. \end{enumerate} 

\item ¿Qué es ``\url{http://www.tharaka.org.ke/nkoru}''? \begin{enumerate} \item Un URI. \item Una dirección de correo electrónico. \item El nombre de un fichero local de nuestro ordenador. \end{enumerate} 

\item Los números IP se componen de 4 números del 0 al 255 separados por puntos. ¿Cuántos bits son necesarios para almacenar un número IP? \begin{enumerate} \item 16 \item 32 \item 4 \end{enumerate} 

\item Cuándo en un navegador no se indica el esquema de un URI, ¿qué esquema se sobreentiende? \begin{enumerate} \item \texttt{http://} \item \texttt{mailto:} \item El esquema de Internet \end{enumerate} 

\item ¿Qué se puede decir de los números IP de dos máquinas que se encuentran en la misma subred? \begin{enumerate} \item No se  puede decir nada: los números IP pueden no tener nada que ver. \item Que tienen en común los primeros bytes. \item Que tienen en común los últimos bytes. \end{enumerate} 

\item Sakurako se conecta a Internet por vía telefónica con un ordenador que tiene un procesador de 1000~MHz, 128 megabytes de RAM y un módem de 57600 bits por segundo. Ahmed tiene un ordenador con un procesador de 500~MHz y 128~megabytes de RAM y ha contratado un módem de cable de 128 kilobits por segundo para conectarse a Internet. Sakurako quiere convencer a Ahmed de que ella se descarga los ficheros MP3 más rápido que él, pero Ahmed le dice que en iguales condiciones él tarda menos en bajarse los ficheros, a veces la mitad de tiempo. ¿Quién tiene razón? \begin{enumerate} \item Ahmed \item Los dos se bajan los archivos en el mismo tiempo porque los dos ordenadores tienen la misma RAM. \item Sakurako \end{enumerate} 

\item Si la máquina \url{fictici.deconya.ua.es} tiene el número IP 232.111.22.33, ¿cuál de los tres número IP siguientes es más probable que corresponda a la máquina \url{fals.deconya.ua.es}? \begin{enumerate} \item 230.111.22.33 \item 232.111.22.13 \item 67.15.22.99 \end{enumerate} 

\item ¿Cómo se indica en Internet dónde está un recurso concreto? \begin{enumerate} \item Mediante un URI. \item Mediante un hiperenlace. \item Mediante una etiqueta HTML. \end{enumerate} 

\item ¿Cuál de los siguientes es un número IP válido? \begin{enumerate} \item 64.128.64 \item 255.256.111.1 \item 111.255.111.111 \end{enumerate} 

\item En un número IP, ¿qué parte es igual para dos ordenadores conectados a la misma red local? \begin{enumerate} \item La parte inicial. \item La parte final. \item Puede no coincidir nada porque los números IP se asignan aleatoriamente. \end{enumerate} 

\item En nuestra casa hemos tenido teléfono toda la vida y ahora estamos pensando en conectarnos a Internet mediante ADSL. ¿Tenemos que hacer alguna instalación adicional en casa? \begin{enumerate} \item No, pero nos quedaremos sin teléfono y sólo tendremos Internet. \item Sí, seguro que los técnicos vendrán a pasar cables por todas partes. \item No. Y disfrutaremos de teléfono e Internet a la vez. \end{enumerate} 

\item Cuando realizamos una búsqueda en Internet, el buscador \ldots\begin{enumerate} \item \ldots visita en ese momento los diferentes documentos de Internet y recopila aquellos que satisfacen el criterio de búsqueda. \item \ldots consulta un índice que ha construido previamente al visitar los diferentes documentos de Internet. \item \ldots consulta un índice que ha construido previamente y también visita en ese momento los documentos de Internet por si hubiera alguno nuevo que no existía cuando creó el índice. \end{enumerate} 

\item Si al buscar un recurso en Internet el buscador nos dice que no hay documentos que satisfagan el criterio de búsqueda \ldots\begin{enumerate} \item \ldots podemos estar seguros de que no existe en todo Internet ningún documento que lo satisfaga. \item \ldots podría darse el caso de que un documento de reciente creación satisfaga el criterio de búsqueda pero no haya sido visitado por el buscador. \item Las otras dos respuestas son erróneas. \end{enumerate} 

\item Dado el URI \url{http://edu.gob.es/educacion/universidades.html}, indica cuál de las siguientes afirmaciones es falsa: \begin{enumerate} \item El recurso \texttt{universidades.html} está alojado en un ordenador cuyo nombre es \texttt{edu.gob.es}. \item La ruta de acceso al recurso es \texttt{educacion/universidades.html}. \item El ordenador donde se aloja el recurso se llama \texttt{http://edu.gob.es}. \end{enumerate} \end{enumerate} 

\section{Soluciones} \begin{enumerate} \item (a) \item (b) \item (a) \item (b): Cada número del 0 al 255 se puede almacenar en 8 bits ($2^8=256$) y hay cuatro: $8\times 4=32$. \item (a) 

%5
\item (b) \item (a): 57.600 b/s son 57.600 / 1024 = 56,25 kb/s. \item (b) \item (a) \item (c) 

%10
\item (a) \item (c) \item (b) \item (b) \item (c) 

% 15
\end{enumerate} 