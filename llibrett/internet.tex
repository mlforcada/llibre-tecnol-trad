\chapter{Internet}
\label{se:Internet}
\com{
Llista de coses a fer del capítol:
\begin{itemize}
\item Definir \emph{buscadors} i \emph{directoris} i diferenciar-los
  clarament. Explicar que el que es tecleja en un buscador són
  condicions i que només ens dóna les pàgines que són en un índex. Dir
  que ens entrega pàgines amb llistes d'URIs i una breu descripció
  (\emph{snippet}) del document o de la part del document on apareixen
  les paraules.
\end{itemize}
}

Una de les eines informàtiques bàsiques que es troben a l'abast de les
persones que es dediquen professionalment a la traducció és
Internet. Internet permet bàsicament tres tipus d'ús:\todo{Per a millorar, tot afegint-hi referències creuades!}
\begin{description}
\item[Com a mitjà de comunicació:] Internet permet la comunicació
  i l'intercanvi d'arxius amb clients o proveïdors, la participació en fòrums de
  professionals, la realització de consultes, etc.
\item[Com a font de documentació:] A més de contenir textos de moltes
  classes que poden servir d'exemple o inspiració a l'hora de fer
  traduccions, s'hi poden trobar enciclopèdies, diccionaris,
  glossaris, memòries de traducció, i moltes altres fonts de
  documentació.
\item[Com a rebost de programari d'assistència a la traducció:] Molts
  dels programes específics d'assistència a la traducció estan
  disponibles a Internet, com ara els sistemes de traducció automàtica
  en línia o els programes de concordances bilingües. L'accés pot ser
  a través d'un navegador, o a través d'altres programes que tinguem
  instal·lats localment en el nostre ordinador.\footnote{Usant
    protocols ben especificats, normalment a través d'API,
    \emph{Application Program Interfaces} o \emph{interfícies de
      programació d'aplicacions}.}
\end{description}



\section{Què és Internet?}

Anomenem \emph{Internet}  un conjunt d'ordinadors, distribu\"{\i}ts
arreu del món i interconnectats mitjançant un protocol
estàndard (el \emph{protocol d'Internet} o IP) de manera que els recursos
presents en uns ordinadors (normalment, informació) estan
disponibles per a ser usats pels usuaris d'altres ordinadors. Es diu que
els ordinadors d'Internet formen una \emph{xarxa}, en la qual els nodes
o nusos són els ordinadors i els fils, les connexions; les connexions
poden ser de naturalesa molt diversa (línies telefòniques,
fibra òptica, enllaços de ràdio terrestres o per
satèl$\cdot$lit, etc.), però el protocol d'Internet 
està dissenyat de manera que la naturalesa de la connexió no siga
rellevant per a l'usuari. Altres noms que s'usen
(més recentment) en comptes d'\emph{Internet} són \emph{World Wide Web} o
\emph{WWW} (``teranyina d'abast mundial'') o simplement \emph{web}
(``teranyina''), masculí en català (\emph{el web}).

\section{Números IP}
Cada node (cada ordinador) de la xarxa Internet té un {\em
número IP} únic, el qual es compon de 4 octets (4 enters del 0
al 255) separats per punts, com ara {\tt 192.168.5.5}. Els enters
inicials s'usen per a designar grans subxarxes, mentre que els finals
s'usen per a designar xarxes més menudes, i dins d'aquestes, ordinadors
concrets (en això recorden els números de telèfon:
dos abonats pròxims normalment comparteixen les xifres inicials).

\section{Noms}
Com que recordar números IP no és fàcil, normalment s'usen
\emph{noms} o \emph{adreces} per a referir-se a les màquines; alguns
dels ordinadors de la xarxa (anomenats \emph{servidors de noms})
s'encarreguen de traduir els noms a números IP.  Per exemple, un
nom podria ser {\tt altea.dlsi.ua.es}, on {\tt altea} es refereix a
una màquina concreta del Departament de Llenguatges i Sistemes
Informàtics ({\tt dlsi}) de la Universitat d'Alacant ({\tt ua}),
que es troba a Espanya ({\tt es}) (en això els noms s'assemblen a
les adreces postals: primer es dóna el més concret i al final
el país; aquest ordre és l'invers al dels números IP).

\todo{Comprovar el \emph{Per saber més}}
\begin{persabermes}{servidors de noms}
  Podríem fer una analogia entre la relació entre els noms i els
  números IP dels ordinadors d'Internet i els noms i els números de
  telèfon de l'agenda del nostre mòbil. Quan telefonem a una persona,
  normalment ho fem buscant el seu nom en l'agenda, i poques vegades
  ho fem pel número, però per fer la telefonada és necessari el
  número. Quan accedim a ordinadors d'Internet, ho fem de manera
  similar: accedim pel nom i no pel número IP, el qual és
  indispensable per a fer la connexió. Però, en contrast amb l'agenda
  del nostre mòbil, seria impracticable tenir tots els noms i els
  números IP corresponents a tots els ordinadors del món en el nostre
  ordinador. Per això s'usen \emph{servidors de noms}: ordinadors als
  quals el nostre ordinador es connecta pel número IP i als quals pot
  preguntar per l'IP corresponent a un nom. Els \emph{servidors de
    noms} s'organitzen de manera que es distribueixen la
  informació. Per exemple, quan volem connectar a
  \url{cercador.dlsi.ua.es}, el primer servidor de noms veu que el nom
  acaba en \texttt{.es} i pregunta al servidor de noms que s'encarrega
  d'aquest \emph{domini}; aquest servidor veu que l'element anterior
  és \texttt{.ua} i pregunta al servidor de noms de la Universitat
  d'Alacant (UA), i aquest, al seu torn, pregunta al servidor de noms
  del Departament de Llenguatges i Sistemes Informàtics (DLSI), ja que
  l'element anterior és \texttt{.dlsi}. Aquest últim, finalment,
  entrega el número IP de l'ordinador anomenat \texttt{cercador} al
  servidor de la UA, i aquest al servidor del domini geogràfic
  \texttt{.es}, qui l'entrega al nostre ordinador per a fer la
  connexió. Per això, quan naveguem, la primera connexió tarda més:
  s'està \emph{resolent} el nom que hem teclejat. Una vegada resolt,
  el nostre ordinador es guarda l'IP durant un temops per evitar
  preguntar de nou.
\end{persabermes}

La taula~\ref{tb:pais} dóna alguns exemples d'indicatius de
pa\"{\i}sos. De vegades, l'últim component d'un nom no es correspon
amb l'indicatiu d'un país, sinó que indica la naturalesa del lloc;
antigament calia sobreentendre que es tractava d'un ordinador situat
físicament als Estats Units d'Amèrica, però ara això ja no és
necessàriament així. Aquests indicatius apareixen en la
taula~\ref{tb:tipus}. En altres pa\"{\i}sos ({\tt .uk}, {\tt .nz},
{\tt .za}) s'usen indicatius similars ({\tt .co}(mercial), {\tt
  .ac}(adèmic), etc.) davant de l'indicatiu de país (per exemple, {\tt
  www.shef.ac.uk} és la Universitat de Sheffield).

\begin{table}
\begin{center}
\begin{tabular}{l|l}
\hline\hline
{\sc Indicatiu} & {\sc País} \\\hline
{\tt .es} & Espanya \\
{\tt .fr} & França \\
{\tt .pt} & Portugal \\
{\tt .it} & Itàlia \\
{\tt .uk} & Regne Unit \\
{\tt .ru} & Rússia \\
{\tt .za} & Sud-àfrica \\
{\tt .ie} & Irlanda \\
{\tt .tv} & Tuvalu \\
{\tt .to} & Tonga \\
{\tt .nu} & Niue \\
{\tt .fm} & Estats Federats de Micronèsia \\
\hline
\end{tabular}
\end{center}
\caption{Indicatius d'Internet d'alguns pa\"{\i}sos. Fixeu-vos que
  alguns indicatius (\texttt{.tv}, \texttt{.fm}, etc.) s'usen per a
  aplicacions no estrictament relacionades amb aquestos països.}
\label{tb:pais}
\end{table}

\begin{table}
\begin{center}
\begin{tabular}{l|l}
\hline\hline
{\sc Indicatiu} & {\sc Tipus} \\\hline
{\tt .gov} & governamental \\
{\tt .mil} & militar \\
{\tt .com} & comercial \\
{\tt .org} & organització no lucrativa \\
{\tt .edu} & institució educativa \\
{\tt .info} & webs informatives \\
{\tt .cat} & cultura i llengua catalanes (patrocinat per la fundació puntCat) \\
{\tt .eus} & cultura i llengua basques (patrocinat per la fundació PuntuEus) \\
{\tt .museum} & museus (patrocinat per MuseDoma) \\
\hline\end{tabular}
\end{center}
\caption{Alguns indicatius Internet usats originalment als Estats
  Units d'Amèrica i més recentment arreu del
  món, alguns d'ells patrocinats per determinades institucions.}
\label{tb:tipus}
\end{table}

\section{Correu electrònic}

Un dels serveis més usats d'Internet és el correu electrònic (en
anglés \emph{electronic mail} o \emph{e-mail}), que ens permet enviar
missatges (textos informatitzats que poden contenir, a més, fitxers
adjunts o \emph{attachments}) a usuaris d'altres ordinadors.  Les
adreces de correu electrònic tenen dues parts, separades pel caràcter
``{\tt @}'', que se sol pronunciar \emph{at} (en anglés, per part dels
informátics més vells), {\em arrova}, \emph{rova} o, per què no,
\emph{ensa\"{\i}mada}. La primera part (o \emph{part local}) és
freqüentment l'identificador d'una persona i la segona (la \emph{part
  de domini}) sol prendre la forma de nom d'un ordinador (o d'un grup
d'ordinadors que comparteixen un nom). Per exemple, una adreça
electrònica vàlida podria ser
\begin{center}
{\tt marty.mcfly@backtothefuture.info}
\end{center}
Altres vegades, una adreça electrònica identifica una llista
d'usuaris (\emph{àlies}) o una llista de distribució (la qual
envia una còpia de cada missatge que rep a tots els inscrits en la
llista). En els dos casos, si hi enviem un missatge, el reben tots els
inscrits, de manera que es pot usar per a establir, per exemple,
fòrums de discussió.\footnote{Per exemple, la llista de
  distribució sobre traducció automàtica MT-List, mantinguda per l'EAMT
  (\emph{European Association for Machine Translation}, associació
  europea per a la traducció automàtica), té l'adreça
  \texttt{mt-list@eamt.org}; per a formar part de la llista cal
  subscriure's en l'URI
  \url{http://lists.eamt.org/mailman/listinfo/mt-list}. Si
  s'envia un missatge a \texttt{mt-list@eamt.org} el reben tots els
  subscrits. Una altra llista d'interés, Tradumàtica, sobre tecnologies de la traducció, permet subscripció a través de \url{https://listserv.rediris.es/cgi-bin/wa?A0=TRADUMATICA}.}
  \label{pg:annex} Els missatges poden contenir, a més del text mateix
  del missatge, fitxers \emph{annexos} (o \emph{adjunts}), en anglés {\em
    attachments}) com ara imatges,  documents, missatges reenviats, etc.

% \com{No sé si mereix la pena dir alguna cosa sobre l'abús del mot
%   \emph{e-mail} en sentits com \emph{adreça}, \emph{programa} o
%   \emph{missatge} quan es refereix al \emph{servei}. Ho faig a classe
%   cada any però no sé si ací...}

Per llegir, escriure o enviar els missatges de correu electrònic,
s'usen \emph{programes gestors de correu electrònic}, com ara
Thunderbird, Outlook, etc.  També és molt freqüent, accedir al correu
electrònic des de qualsevol lloc usant un navegador, a través d'un
servei anomenat \emph{webmail}; són comuns els \emph{webmails}
gratuïts (GMail, Yahoo Mail, MicroSoft Outlook); cada alumna o alumne
de la Universitat d'Alacant té, per ser-ho, una adreça de correu de la
forma \emph{xxx}\texttt{@alu.ua.es}, accessible a través de la secció
\emph{webmail} del servidor de la Universitat.
\texttt{http://www.ua.es/va/webmail/}.\footnote{Els professors de
  Tecnologies de la Traducció només considerem vàlids els missatges
  que provenen d'aquestes adreces oficials, però preferim que useu el
  servei de tutoria electrònica de la Universitat.}

\section{Missatgeria instantània i xat}

La missatgeria instantània i el xat (en anglés \emph{chat}) permeten
---com en el cas del correu electrònic, a través de programes
especialitzats o \emph{webs} accessibles amb un navegador--- una
comunicació escrita molt ràpida (``en temps real''), consistent en
missatges normalment curts ---opcionalment amb annexos com ara
fotografies, contactes---, de manera que el resultat és similar al
d'una conversa, però per escrit,\footnote{Encara que s'està
  popularitzant l'ús de \emph{notes de veu}, arxius d'audio
  enregistrats i que s'envien com a annexos.} cosa que permet un registre
de comunicació molt informal que, de fet, ha donat lloc a una llengua molt
diferenciada tant de l'oral com de l'escrita.

Amb la generalització de l'ús dels mòbils intel·ligents o
\emph{smartphones}, han aparegut moltes aplicacions d'aquest tipus,
com ara Telegram, Whatsapp, Line, etc., que usen com a identificador
el número de telèfon. 

Les converses poden ser entre dues persones, o agregades un
\emph{grup} o \emph{sala}, amb interessos comuns.  Les persones que
participen en un \emph{xat} d'aquests últims, poden de vegades elegir
un àlies o malnom i ``entrar'' a la sala, o estar sempre connectades
al grup, i ``conversar'' per escrit públicament. Des del grup o la
sala es poden establir ``converses a part'' (a banda) quan cal amb
alguna persona concreta.

Entre els serveis de missatgeria instantània comercial més populars
estan els associats a xarxes socials com ara Facebook o Tuenti, o altres
associats a altres aplicacions com ara Google Hangouts o Skype.

\section{Serveis de xarxa social}

Una de les aplicacions més freqüents d'Internet són en l'actualitat
els \emph{serveis de xarxa social}, comunament anomenats simplement
\emph{xarxes socials}. Són plataformes informàtiques que usen Internet
(a través d'un navegador i freqüentment també a través de programes
d'aplicació específics, molt populars per a telèfons mòbils) per a
construir xarxes de persones que comparten interessos u
objectius. Alguns exemples:
\begin{description}
\item[Facebook] permet a cada persona publicar informacions sobre el seu \emph{estat}, incloent fotos, i enviar-se missatges; l'estat pot ser visible per a tothom o per a persones \emph{amigues}.
\item[Google+] es pot veure com la resposta de Google a Facebook; en Google+ el concepte bàsic és el del \emph{cercle}.
\item[Twitter] és una xarxa social que es basa en missatges de menys de 140 caràcters que poden dur adjuntes fotos, enllaços, etc.
\item[Instagram] fa èmfasi en la possibilitat de compartir fotografies i videos.
\item[LinkedIn] serveix per a construir xarxes relacionades amb
  l'activitat professional.
\end{description}
Hi ha molts altres d'abast mundial (Pinterest, Reddit, Tumblr, etc.) i alguns particulars de determinades àrees geogràfiques (com ara VK en els països on es parla rus).
%\com{Ací potser caldria dir molt més però per ara va bé.}

% \section{\emph{News}}

% Aquest és un servei similar a les llistes de
% distribució i, per tant, també serveix per a constituir grups
% de discussió o de notícies (\emph{newsgroups}) sobre un tema;
% els missatges que s'hi envien hi queden guardats i qui vulga saber
% si hi ha novetats en un grup de discussió determinat només s'ha
% de connectar al \emph{servidor de notícies} més pròxim,
% seleccionar el grup i llegir els missatges nous. Els grups de
% notícies tenen noms compostos per diversos mots separats per punts,
% com ara {\tt comp.ai.neural-nets}.




\section{Els identificadors}

Els serveis i els documents concrets presents en un ordinador que els
fa disponibles (un servidor d'Internet) es poden designar mitjançant
el seu \emph{identificador uniforme de recursos} o, més comunament,
\emph{URI} (de l'anglés \emph{uniform resource
  identifier}).\footnote{La denominació més usual era URL,
  \emph{uniform resource locator} o localitzador uniforme de recursos,
  que encara s'usa profusament, encara que no tots els URI son URL.}
L'URI és, per tant, una expressió que identifica o localitza
uniformement un servei o document (un recurs) de qualsevol dels que
s'ofereixen en Internet.

Un URI té generalment dues parts, encara que s'hi donen algunes
variacions:
\begin{itemize}
\item l'esquema, que indica la classe de recurs i com l'ha d'usar
      l'ordinador sol$\cdot$licitant (o \emph{client})
\item el nom del recurs, que sovint està format per:
  \begin{itemize}
  \item el nom de l'ordinador servidor o el seu número IP
  \item (opcionalment) informació sobre la localització del
    servei o document dins de l'ordinador servidor (moltes vegades
    similar a les \emph{trajectòries} dels fitxers, p.~\pageref{pg:fitxer})
  \end{itemize}
\end{itemize}
Per exemple, l'URI
\begin{center}
\url{http://www.canalcuina.tv/concurs/sms/index.html}
\end{center}
es refereix a un document d'\emph{hipertext} ---un
  document de text que conté enllaços que permeten accedir
  directament a altres hipertextos relacionats--- compatible amb 
l'esquema {\tt http} (\emph{hypertext transfer protocol} o protocol de
transferència d'hipertextos) situat en l'ordinador {\tt
  www.canalcuina.tv} (de l'empresa fictícia Canal Cuina,
  possiblement pertanyent al món de la televisió\footnote{Encara que,
  com es mostra en la taula~\ref{tb:pais}, l'indicatiu designa un estat
  del Pacífic anomenat Tuvalu}), i, dins d'aquest, 
  en el directori {\tt concurs},
subdirectori {\tt sms}. El fitxer que conté l'hipertext s'anomena
{\tt index.html} (on les sigles HTML corresponen a \emph{hypertext
  markup language}, nom del llenguatge o sistema de marques més usat
per a donar format als hipertextos, descrit en
l'epígraf~\ref{ss:formats}).
%\com{Cal millorar la definició d'hipertext ---distingint clarament el
%  cas base (text sense enllaços) i la recursió--- i explicar millor qué és un
%  enllaç: que té text i adreça, etc.}


L'esquema {\tt https://} és similar a l'esquema {\tt http://} però
incorpora, a més, mecanismes per a transmetre amb seguretat informació
encriptada (xifrada). Molts dels servidors d'Internet encarregats de
manipular informació privada usen aquest esquema.

Els URI no només serveixen per enllaçar hipertextos: l'URI
\url{mailto:anton@dlsi.ua.es} serveix per a enviar correu electrònic
({\tt mailto}) a l'usuari que té l'adreça de correu electrònic
\url{anton@dlsi.ua.es}. Altres esquemes són \texttt{rtsp://},
\emph{real-time streaming protocol}, per a enllaçar contingut com ara
vídeo, audio, etc. en temps real, o {\tt ftp://}, \emph{file transfer
  protocol}, usat, cada vegada menys, per a descarregar (transferir)
fitxers per a guardar-los en el nostre ordinador\label{pg:ftp}.

\section{Navegadors}
\label{ss:navegadors}
Els programes navegadors es coneixen també per altres noms: {\em
  browsers} (fullejadors), \emph{exploradors}, etc. (vegeu també la
pàg.~\pageref{pg:navegadors}).  Són programes que permeten accedir de
manera senzilla als documents o serveis d'Internet en ordinadors
connectats a aquesta xarxa; entre altres coses, els navegadors
interpreten els hipertextos escrits en HTML i els presenten a la
persona usuària en el format que indiquen les marques, de manera que
els enllaços a altres hipertextos queden clarament destacats i siguen
\emph{actius}, és a dir, que responguen a un clic del ratolí
\emph{saltant} a l'hipertext o recurs enllaçat; a més, els navegadors
poden \emph{llançar} automàticament altres programes d'aplicació per a
poder obrir el recurs corresponent si no és un hipertext. Els
navegadors més usats són \emph{Firefox} (un programa lliure i de codi
font obert desenvolupat per centenars de col·laboradors arreu del
món), \emph{Chrome} (el navegador de la companyia Google, el qual té
una versió lliure i de codi font obert anomenada \emph{Chromium}),
\emph{Microsoft Internet Explorer} (incorporat en el sistema operatiu
Windows), \emph{Safari} (el qual forma part del sistema operatiu
MacOS), i d'altres com \emph{Opera}, etc.
% Alguns navegadors encara van empaquetats amb un programa de correu
% electrònic i un redactor de pàgines \emph{web} incorporats, encara que
% la tendència actual és a separar aquestes tres aplicacions en
% programes diferents.
 
\com{No m'agrada que el material de navegadors estiga repartit i
  repetit entre dos capítols. Unir? }

\subsection{Buscadors}

Un dels recursos d'Internet més útils són els \emph{buscadors} o
\emph{cercadors}. Es tracta de pàgines \emph{web} que permeten buscar
documents d'Internet; s'hi ha de teclejar una o més paraules, a més
d'altres \emph{condicions de recerca} opcionals, com ara que els
documents hagen d'estar en una llengua determinada o en un servidor
determinat, i entreguen els URIs dels documents que compleixen
aquestes condicions, enllaços a aquests documents i un petit resum o
retall (anomenat \emph{snippet}) del contingut de les pàgines
desitjades.\footnote{Alguns buscadors, com ara \emph{Google},
  modifiquen els enllaços que porten als resultats, de manera que no
  hi porten directament sinó passant primerament pel servidor del
  buscador, per tal de conéixer les preferències de la gent i millorar
  així la rellevància dels resultats, però fins i tot poden establir
  un perfil de cada persona usuària. Això fa que algunes persones es
  plantegen l'ús de buscadors que no facen aquest \emph{seguiment} o
  \emph{tracking}.}

Exemples de recerques:
\begin{itemize}
\item \texttt{megamòndria} : documents que continguen el mot \emph{megamòndria} i potser formes del mateix com ara el plural \emph{megamòndries}, el compost \emph{mega-mòndria}, o la forma sense accent \emph{megamondria}.
\item \texttt{megamòndria síngula} : documents que continguen aquests dos mots o variants.
\item \texttt{megamòndria síngula site:ua.es} : documents que
continguen aquests dos mots o variants i que estiguen en documents amb
URIs que acaben en \emph{ua.es}.
\item \texttt{megamòndria síngula filetype:pdf} : documents PDF que continguen aquests dos mots.
\end{itemize}

Ha de quedar clar que els buscadors realment \emph{no busquen}
documents en Internet sinó que consulten \emph{índexs} que han anat
construint a partir dels documents que van visitant. Per tant, pot
haver-hi documents que els buscadors no troben perquè mai no els han
visitat. Per la mateixa raó, també pot passar que els buscadors
entreguen resultats corresponents a pàgines que ja no existeixen.

Un dels buscadors més populars a l'hora d'escriure aquestes línies és
\emph{Google} (\url{http://www.google.com}); però també hi ha altres
com \emph{Duckduckgo!}, \emph{StartPage}, etc. La major part d'aquests
buscadors tenen interfícies d'ús en moltes llengües.


\section{L'accés a Internet}
\label{ss:adaI}

\subsection{Accés domèstic}
Per a accedir a Internet des de casa, cal, d'una banda, un mòdem
adequat (que normalment permet que hi connectem més d'un dispositiu a
través d'una connexió sense fils Wi-Fi), i d'altra, haver-se donat
d'alta amb algun proveïdor d'Internet (ISP, \emph{Internet service
  provider}); alguns proveïdors ofereixen, a més de l'accés a
Internet, televisió digital i telefonia convencional.

En la majoria dels casos, el proveïdor d'Internet assigna un número IP
temporal al nostre ordinador domèstic, de manera que forme part
d'Internet i puga accedir com a client a tots els serveis i documents
disponibles en qualsevol màquina de la xarxa. El número IP temporal va
canviant, per això l'ordinador de casa no pot actual com a servidor.

En l'actualitat, en els domicilis del País Valencià hi ha bàsicament
tres modalitats d'accés a Internet, totes per preus que oscil·len al
voltant dels 30--40 euros/mes :
\begin{description}
\item[ADSL] (vegeu el glossari de la secció~\ref{ss:OiPgloss}): el
  mòdem fa la connexió a través dels fils de telefonia convencional ja
  instal·lats a les cases, i en l'actualitat s'aconsegueixen
  velocitats de baixada d'alguns Mb/s i de pujada normalment inferiors
  a 1 Mb/s.
\item[Cable]: el mòdem fa la connexió a través d'un cable coaxial,
  tecnologia que s'usava tradicionalment per a televisió, amb
  velocitats i preus similars a l'ADSL.
\item[Fibra]: el mòdem fa la connexió a través d'un feix de fibra
  òptica (làser); aquesta tecnologia és la més recent i permet
  velocitats de baixada de desenes o fins i tot centenars de Mb/s i
  velocitats de pujada superiors al Mb/s.
\end{description}

\begin{persabermes}{els antics mòdems telefònics}
  Fins als primers anys del decenni del 2000, la major part dels
  domicilis particulars i les petites empreses es connectaven a
  Internet mitjançant la línia telefònica, però usant una tecnologia
  molt més rudimentària, que requeria fer una telefonada al número del
  proveïdor d'Internet d'Internet, que havia de durar el temps de
  connexió, independentment de la quantitat de dades que s'hi
  transferiren. La telefonada es podia pagar per minuts o amb plans
  que permetien connexions il·limitades en horari no comercial que
  s'anomenaven \emph{tarifa plana}. El mòdem modulava i desmodulava
  senyals similars als que s'envien quan es fa una telefonada de
  veu. Durant la connexió, la línia quedava ocupada i no es podien fer
  ni rebre telefonades. Les velocitats eren molt baixes, de desenes de
  kb/s.
\end{persabermes}


\subsection{Accés mòbil}
Com més va, menys ens connectem a Internet des de casa, i més des dels
nostres dispositius mòbils com ara els \emph{smartphones}. Si som a
l'abast d'una xarxa Wi-Fi a la qual tenim accés (com la de la nostra
casa o la que ens ofereix la Universitat), el nostre dispositiu mòbil
es connectarà a Internet a través de Wi-Fi. Si no, haurem d'usar les
\emph{dades mòbils} que comercialitza el nostre proveïdor de telefonia
mòbil a través de la seua xarxa cel·lular. En l'actualitat s'està fent
la transició de la tecnologia anomenada de tercera generació o 3G (que
es mostra de vegades també com una ``H'' en la pantalla), que permet
velocitats de connexió d'uns pocs Mb/s a la de quarta generació o 4G,
que permet velocitats molt més ràpides (de vegades, quan la cobertura
no és bona, el nostre mòbil recorre a tecnologies més antigues i més
lentes, com ara EDGE\footnote{\emph{Enhanced Data Rates for GSM
    Evolution} ``Taxes de dades millorades per a l'evolució del GSM''}
(que permet centenars de kb/s i es mostra com una ``E'' en la
pantalla) o GPRS\footnote{\emph{General Packet Radio Service} ``Servei
  general de paquets [de dades] per ràdio''} (que permet desenes de
kb/s i es mostra com una ``G'' en la pantalla). En l'actualitat, podem
comprar paquets de dades i pagar uns 5--7 euros per GB, o una quota mensual
de desenes d'euros i tenir dades il·limitades.


\section{Exercicis i qüestions}
\begin{enumerate}

\item La primera part d'un URI (identificador uniforme de recursos)
especifica
     \begin{enumerate}
     \item l'esquema d'accés.
     \item el nom del servidor.
     \item el directori on es troba el servei.
     \end{enumerate}


\item Després de l'esquema d'accés, un URI (identificador uniforme de
  recursos) especifica
     \begin{enumerate}
     \item la velocitat de transferència.
     \item el nom del servidor.
     \item el directori on es troba el servei.
     \end{enumerate}

\item Què es ``\url{http://www.tharaka.org.ke/nkoru}''?
      \begin{enumerate}     
      \item Un URI.
      \item Una adreça de correu electrònic.
      \item El nom d'un fitxer local del nostre ordinador.
      \end{enumerate}

    \item Els números IP es componen de 4 números del 0 al 255
      separats per punts. Quants bits són necessaris per a
      emmagatzemar un número IP?
\begin{enumerate}
\item 16
\item 32
\item 4
\end{enumerate} 


\item 
   Quan en un navegador no s'indica l'esquema d'un URI, quin
   esquema se sobreentén?
   
\begin{enumerate}
\item \verb|http://|
\item \verb|mailto:|
\item L'esquema d'Internet
\end{enumerate}

\item 
   Què es pot dir dels números IP de dues màquines que es 
   troben en la mateixa subxarxa?
   
\begin{enumerate}
\item No se'n pot dir res: els números IP poden no tenir res a veure.
\item Que tenen en comú els primers octets.
\item Que tenen en comú els últims octets.
\end{enumerate}

\item Sakurako es connecta a Internet per via telefònica amb un
   ordinador que té un processador de 1000~MHz, 128 megaoctets
   (\emph{megabytes}) de RAM i un mòdem de 57600 bits per segon.
   Ahmed té un ordinador amb un processador de 500~MHz i 128~megaoctets
   de RAM i ha contractat un mòdem de cable de 128 kilobits per segon
   per a connectar-se a Internet. Sakurako vol convéncer Ahmed que ella
   es descarrega els fitxers MP3 més ràpid que ell, però Ahmed li diu
   que en les mateixes condicions ell tarda menys a baixar-se els
   fitxers, de vegades la meitat de temps.  Qui té raó?
   
\begin{enumerate}
\item Ahmed
\item Els dos es baixen els arxius en el mateix temps perquè els
      dos ordinadors tenen la mateixa RAM.
\item Sakurako
\end{enumerate}

\item 
   Si la màquina \verb|fictici.deconya.ua.es| té el número IP
   \verb|232.111.22.33| quin dels tres número IP següents és més
   probable que corresponga a la màquina \verb|fals.deconya.ua.es|?
\begin{enumerate}
\item 230.111.22.33
\item 232.111.22.13
\item 67.15.22.99
\end{enumerate}

\item Com s'indica en Internet on és un recurs concret?   
\begin{enumerate}
\item Mitjançant un URI.
\item Mitjançant un hiperenllaç.
\item Mitjançant una etiqueta HTML.
\end{enumerate}


\item Quin dels següents és un número IP vàlid?  
  \begin{enumerate}
  \item 64.128.64
  \item 255.256.111.1
  \item 111.255.111.111
  \end{enumerate}

\item En un número IP, quina part és igual per a dos ordinadors connectats en la mateixa xarxa local?
  \begin{enumerate}
  \item La part inicial.
  \item La part final.
  \item Pot no coincidir-hi res perquè els números IP s'assignen aleatòriament.
  \end{enumerate}

\item
A la nostra casa hem tingut telèfon tota la vida i ara estem pensant a conectar-nos a Internet mitjançant ADSL. Hem de fer cap instal·lació addicional a casa? 

\begin{enumerate}
\item No, però ens quedarem sense telèfon i només hi tindrem Internet.
\item Sí, de segur que els tècnics vindran a passar cables arreu la casa.
\item No. I gaudirem de telèfon i Internet alhora.
\end{enumerate}


% < QT3

\end{enumerate}

\section{Solucions}
\begin{enumerate}
\item (a)
\item (b)
\item (a)
\item (b). Cada número del 0 al 255 es pot emmagatzemar en 8 bits
  ($2^8=256$) i n'hi ha quatre: $8\times 4=32$.
\item (a)
\item (b)
\item (a)
\item (b)
\item (a)
\item (c)
\item (a)
\item (c)

% < SL3

\end{enumerate}
